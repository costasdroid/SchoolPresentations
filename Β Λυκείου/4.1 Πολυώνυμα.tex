\documentclass{../presentation}

\title{Πολυώνυμα}
\subtitle{Βασικά}
\author[Λόλας]{Κωνσταντίνος Λόλας}
\date{}

\begin{document}

\begin{frame}
  \titlepage
\end{frame}

\section{Θεωρία}
\begin{frame}{Πού βρισκόμαστε?}
  Πιάνουμε πολυώνυμα
  \begin{enumerate}
    \item<1-> Ορισμοί
    \item<2-> Μορφές
    \item<3-> Πράξεις
    \item<4-> Ρίζες
    \item<5-> Παραγοντοποίηση
    \item<6-> Πρόσημο
  \end{enumerate}
\end{frame}

\begin{frame}{Παλιά, Ξινά σταφύλλια}
  \begin{block}{Μονώνυμο του $x$}<1->
    Καλούμε \emph{μονώνυμο} κάθε παράσταση της μορφής $α\cdot x^ν$, όπου $α\in \mathbb{R}$ και $ν\in \mathbb{N}$
  \end{block}
  \begin{block}{Πολυώνυμο του $x$}<2->
    Καλούμε \emph{πολυώνυμο} κάθε παράσταση της μορφής
    $$α_νx^ν+α_{ν-1}x^{ν-1}+\cdots+α_1x+α_0$$
    με $ν\in\mathbb{N}$ και $α_i\in\mathbb{R}$, $0\le i \le ν$
  \end{block}
\end{frame}

\begin{frame}{Παλιά, Ξινά σταφύλλια}
  \begin{block}{Ορισμοί στα πολυώνυμα}
    \begin{itemize}
      \item Όροι: Τα μονώνυμα $α_ix^i$, $0\le i \le ν$
      \item Συντελεστές: Οι πραγματικοί $α_i$, $0\le i \le ν$
      \item Σταθερός όρος: Το $α_0$
      \item Σταθερό πολυώνυμο: το πολυώνυμο $α_0$
      \item Μηδενικό πολυώνυμο: το πολυώνυμο με όλα τα $α_i=0$, $0\le i \le ν$
      \item Βαθμός πολυωνύμου: ο μεγαλύτερος εκθέτης από τους μη μηδενικούς όρους
    \end{itemize}

  \end{block}
\end{frame}

\begin{frame}{Ισότητα}
  \begin{block}{Ισότητα Πολυωνύμων}
    Δύο πολυώνυμα

    $α_νx^ν+α_{ν-1}x^{ν-1}+\cdots+α_1x+α_0$ και $β_κx^κ+β_{κ-1}x^{κ-1}+\cdots+β_1x+β_0$ με $ν\ge κ$

    είναι \emph{ίσα} όταν

    $α_0=β_0$, $α_0=β_0$, \ldots, $α_κ=β_κ$ και $β_{κ+1}=β_{κ+2}=\cdots=β_ν=0$
  \end{block}
\end{frame}

\begin{frame}{Πολυώνυμο = Συνάρτηση, άρα...}
  \begin{block}{Αριθμητική Τιμή}
    Έστω ένα πολυώνυμο $P(x)=α_νx^ν+α_{ν-1}x^{ν-1}+\cdots+α_1x+α_0$

    \emph{Αριθμητική τιμή} ή απλά \emph{τιμή} του είναι κάθε αριθμός $P(ρ)$ με $ρ\in\mathbb{R}$

    $$P(ρ)=α_νρ^ν+α_{ν-1}ρ^{ν-1}+\cdots+α_1ρ+α_0$$
  \end{block}

  \begin{block}{Ρίζα}
    Έστω ένα πολυώνυμο $P(x)=α_νx^ν+α_{ν-1}x^{ν-1}+\cdots+α_1x+α_0$

    \emph{Ρίζα} του είναι κάθε $ρ\in\mathbb{R}$ με $P(ρ)=0$
  \end{block}
\end{frame}

\begin{frame}{Πράξεις}
  \begin{block}{}
    \begin{itemize}
      \item<1-> Πρόσθεση
      \item<2-> Αφαίρεση
      \item<3-> Πολλαπλασιασμός
      \item<4-> Διαίρεση!!!!!
    \end{itemize}
  \end{block}
  \onslide<5-> Τι γίνεται με τους βαθμούς?
\end{frame}

\begin{frame}[noframenumbering]
  Στο moodle θα βρείτε τις ασκήσεις που πρέπει να κάνετε, όπως και αυτή τη παρουσίαση
\end{frame}

\section{Ασκήσεις}

\begin{frame}[noframenumbering]
  \vfill
  \centering
  \begin{beamercolorbox}[sep=8pt,center,shadow=true,rounded=true]{title}
    \usebeamerfont{title}Ασκήσεις
  \end{beamercolorbox}
  \vfill
\end{frame}

\begin{askisi}
  Δίνεται το πολυώνυμο $P(x)=x^4-3x+2$
  \begin{enumerate}
    \item<1-> Να βρείτε την τιμή του πολυωνύμου για $x=2$
    \item<2-> Να εξετάσετε ποιοι από τους αριθμούς $1$ και $-1$ είναι ρίζες του

  \end{enumerate}

  % \hyperlink{Λύση1}{\beamerbutton{Λύση}}
\end{askisi}

\begin{askisi}
  Να βρείτε τις τιμές των $α$ και $β$ για τις οποίες η αριθμητική τιμή του πολυωνύμου
  $$P(x)=αx^3+βx+1$$
  για $x=2$ είναι $5$ και ο αριθμός $1$ είναι ρίζα του

  % \hyperlink{Λύση2}{\beamerbutton{Λύση}}
\end{askisi}

\begin{askisi}
  Δίνονται τα πολυώνυμα $P(x)=x^3-x$ και $Q(x)=2x-1$. Να βρείτε τα παρακάτω πολυώνυμα και τον βαθμό τους:
  \begin{enumerate}
    \item<1-> $R(x)=2P(x)-3Q(x)$
    \item<2-> $H(x)=P(x)\cdot Q(x)$
    \item<3-> $Φ(x)=\left( Q(x) \right)^2 $
  \end{enumerate}

  % \hyperlink{Λύση3}{\beamerbutton{Λύση}}
\end{askisi}

\begin{askisi}
  Δίνεται το πολυώνυμο $P(x)=(x+1)λ^2+(x^2-1)λ-x^2-x+2$.
  \begin{enumerate}
    \item<1-> Να βρείτε τους συντελεστές του πολυωνύμου $P(x)$ και το σταθερό όρο του
    \item<2-> Να δείξετε ότι ο σταθερός όρος του είναι μη μηδενικός για κάθε $λ\in\mathbb{R}$
    \item<3-> Να βρείτε τις τιμές του $λ$, για τις οποίες το πολυώνυμο $P(x)$ είναι σταθερό πολυώνυμο
  \end{enumerate}

  % \hyperlink{Λύση4}{\beamerbutton{Λύση}}
\end{askisi}

\begin{askisi}
  Να βρείτε τιμές του $μ$, για τις οποίες το πολυώνυμο

  $$P(x)=(μ^3-1)x^3+(μ^2-μ)x+|μ|-1$$

  είναι το μηδενικό πολυώνυμο

  % \hyperlink{Λύση5}{\beamerbutton{Λύση}}
\end{askisi}

\begin{askisi}
  Να βρείτε τις τιμές των πραγματικών αριθμών $α$, $β$ και $γ$ για τις οποίες τα πολυώνυμα
  $$P(x)=(α-2)x^2-3x \text{ και } Q(x)=(β-1)x+γ-α$$
  είναι ίσα

  % \hyperlink{Λύση6}{\beamerbutton{Λύση}}
\end{askisi}

\begin{askisi}
  Να βρείτε τους πραγματικούς αριθμούς $α$, $β$, $γ$ και $δ$ για τους οποίους το πολυώνυμο $f(x)=3x^2-2x+5$ παίρνει τη μορφή:
  $$f(x)=αx(x^2-1)+βx^2+γx+δ$$

  %\hyperlink{Λύση7}{\beamerbutton{Λύση}}
\end{askisi}

\begin{askisi}
  Έστω το πολυώνυμο $P(x)=(λ^2-4)x-λ+2$. Να βρείτε τις τιμές του $λ$, για τις οποίες:
  \begin{enumerate}
    \item<1-> Το $P(x)$ είναι 1ου βαθμού
    \item<2-> Το $P(x)$ είναι μηδενικού βαθμού
    \item<3-> Δεν ορίζεται βαθμός του $P(x)$
  \end{enumerate}

  %\hyperlink{Λύση8}{\beamerbutton{Λύση}}
\end{askisi}

\begin{askisi}
  Να βρείτε πολυώνυμο 2ου βαθμού με ρίζες τους αριθμούς $-1$, $0$ και να ισχύει $P(1)=4$

  %\hyperlink{Λύση9}{\beamerbutton{Λύση}}
\end{askisi}

\begin{askisi}
  Να βρείτε το βαθμό του πολυωνύμου
  $$P(x)=(λ^3-4λ)x^3+(λ^2-2λ)x-λ+2$$
  για τις διάφορες τιμές του $λ\in\mathbb{R}$

  %\hyperlink{Λύση10}{\beamerbutton{Λύση}}
\end{askisi}

\begin{askisi}
  Να βρείτε πολυώνυμο $P(x)$ για το οποίο ισχύει:

  $$(x^2+1)P(x)=2x^3-x(x-2)-1$$

  %\hyperlink{Λύση11}{\beamerbutton{Λύση}}
\end{askisi}

\begin{askisi}
  Αν το πολυώνυμο $P(x)$ έχει ρίζα το $1$, να δείξετε ότι το πολυώνυμο

  $$Q(x)=P(x^2-3)+(x-2)P(3x)$$

  έχει ρίζα το $2$

  %\hyperlink{Λύση12}{\beamerbutton{Λύση}}
\end{askisi}

\begin{askisi}
  Να βρείτε το πολυώνυμο $P(x)$ για το οποίο ισχύει
  $$P(x+1)=x^3-3x+1$$

  %\hyperlink{Λύση13}{\beamerbutton{Λύση}}
\end{askisi}

\begin{askisi}
  Να βρείτε πολυώνυμο $P(x)$, για το οποίο ισχύει
  $$\left[ P(x) \right]^2-P(x)=x^2+x$$
  %\hyperlink{Λύση14}{\beamerbutton{Λύση}}
\end{askisi}

\end{document}
