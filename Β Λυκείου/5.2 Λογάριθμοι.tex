\documentclass[greek]{beamer}
%\usepackage{fontspec}
\usepackage{amsmath,amsthm}
\usepackage{unicode-math}
\usepackage{xltxtra}
\usepackage{graphicx}
\usetheme{CambridgeUS}
\usecolortheme{seagull}
\usepackage{hyperref}
\usepackage{ulem}
\usepackage{xgreek}

\usepackage{pgfpages}
\usepackage{tikz}
%\setbeameroption{show notes on second screen}
%\setbeameroption{show only notes}

\setsansfont{Calibri}

\usepackage{multicol}

\usepackage{appendixnumberbeamer}

\usepackage{polynom}

\usepackage{pgffor}

\setbeamercovered{transparent}
\beamertemplatenavigationsymbolsempty

\title{Εκθετική Λογαριθμική Συνάρτηση}
\subtitle{Λογάριθμοι}
\author[Λόλας]{Κωνσταντίνος Λόλας}
\date{}

\begin{document}

\begin{frame}
 \titlepage
\end{frame}

\section{Θεωρία}
\begin{frame}{Σταμάταααααααα}
 Τελειώνουμε!
\end{frame}

\begin{frame}{Γιατί καινούρια έννοια?}
 \begin{itemize}
  \item<1-> Οι εκθέτες είναι δύσκολοι στον χειρισμό
  \item<2-> Οι εκθέτες είναι δύσκολοι στην εύρεση
  \item<3-> Οι τεράστιοι αριθμοί είναι δυσβάσταχτοι
 \end{itemize}
\end{frame}

\begin{frame}{Ας κατανοήσουμε την ανάγκη!}
 \begin{itemize}
  \item<1-> Δεν γνωρίζουμε τις ρίζες των πραγματικών, αλλά γράφουμε $\sqrt{2}$
  \item<2-> Δεν γνωρίζουμε τα ημίτονα των τόξων, αλλά γράφουμε $ημ(\dfrac{π}{7})$
 \end{itemize}
 \onslide<3-> Γιατί να μην υπάρχει λοιπόν το $2^x=5$? Περιγράψτε το!
\end{frame}

\begin{frame}[label=Θεωρία1]{Λογάριθμοι λοιπόν}
 \begin{block}{Ορισμός}
  $$\log_ax=y\iff a^y=x$$
  Για κάθε $a>0$, $a\ne 1$ και $x>0$
 \end{block}
 \begin{exampleblock}{Παραλλαγές}
  \begin{itemize}
   \item $\log_{10}=\log$
   \item $\log_e=\ln$
  \end{itemize}
 \end{exampleblock}
 \hyperlink{Άσκηση1}{\beamerbutton{Ασκήσεις}}
\end{frame}

\begin{frame}{Πάμε για ιδιότητες}
 Από τον ορισμό έχουμε άμεσα...
 \begin{block}{Ιδιότητες}
  \begin{itemize}
   \item<1-> $\log_aa=1$
   \item<2-> $\log_a1=0$
   \item<3-> $\log_aa^x=x$
   \item<4-> $a^{\log_ax}=x \onslide<5-> \impliedby \left( \log_ax=y\iff a^y=x \right) $
  \end{itemize}
 \end{block}
\end{frame}

\begin{frame}[label=Ιδιότητες]{Και λίγες ακόμα}
 \begin{block}{Ιδιότητες}
  \begin{itemize}
   \item<1-> $\log_α\left( θ_1\cdot θ_2 \right)=\log_αθ_1+\log_αθ_2 $
   \item<2-> $\log_α\dfrac{θ_1}{θ_2}=\log_αθ_1-\log_αθ_2 $
   \item<3-> $\log_αθ^κ=κ\log_αθ$
  \end{itemize}
 \end{block}
 \hyperlink{Απόδειξη1}{\beamerbutton{Απόδειξη}}
\end{frame}

\begin{frame}{Σχεδόν Τελειώσαμε!}
 Έμεινε μόνο ο ορισμός της συνάρτησης $\ln x$ και η γραφική της παράσταση!
\end{frame}

\section{Ασκήσεις}
\subsection{Άσκηση 1}
\begin{frame}[label=Άσκηση1,t]{Εξάσκηση 1}
 Να βρείτε για ποιες τιμές του $x$ ισχύει:
 \begin{enumerate}
  \item<1-> $\log_2x=3$
  \item<2-> $\log_3x=-2$
  \item<3-> $\log_4x=\dfrac{1}{2}$
  \item<4-> $\ln x=2$
  \item<5-> $\log x+1=0$
  \item<6-> $\log_x9=2$
 \end{enumerate}

 % \hyperlink{Λύση1}{\beamerbutton{Λύση}}
\end{frame}

\subsection{Άσκηση 2}
\begin{frame}[label=Άσκηση2,t]{Εξάσκηση 2}
 Να υπολογίσετε τις παραστάσεις
 \begin{enumerate}
  \item<1-> $\ln e^3$
   \item<2-> $\log\dfrac{1}{100}$
    \item<3-> $e^{\ln 3}$
     \item<4-> $\ln \sqrt[3]{e^2}$
      \item<5-> $\log_{\frac{1}{3}}\sqrt{3}$
       \item<6-> $\log_8\sqrt{2}$
        \item<7-> $\ln^2e^3$
 \end{enumerate}

 \hyperlink{Θεωρία1}{\beamerbutton{Συνέχεια Θεωρίας}}
 % \hyperlink{Λύση2}{\beamerbutton{Λύση}}
\end{frame}

\subsection{Άσκηση 3}
\begin{frame}[label=Άσκηση3,t]{Εξάσκηση 3}
 Να αποδείξετε ότι:
 \begin{enumerate}
  \item<1-> $\log 4+\log 20-3\log 2=1$
  \item<2-> $\ln 2e-\dfrac{1}{2}\ln 4=1$
  \item<3-> $\log\sqrt{8}+\dfrac{3}{2}\log 5$
 \end{enumerate}

 % \hyperlink{Λύση3}{\beamerbutton{Λύση}}
\end{frame}

\subsection{Άσκηση 4}
\begin{frame}[label=Άσκηση4,t]{Εξάσκηση 4}
 Να υπολογίσετε τις τιμές των παραστάσεων:
 \begin{enumerate}
  \item<1-> $5^{1+3\log_5 2}$
  \item<2-> $100^{\frac{1}{2}-\frac{3}{2}\log 2}$
  \item<3-> $\left( \dfrac{1}{e} \right)^{3-\ln \sqrt{2}} $
 \end{enumerate}

 % \hyperlink{Λύση4}{\beamerbutton{Λύση}}
\end{frame}

\appendix

\section{Αποδείξεις}
\begin{frame}[label=Απόδειξη1,t]{Απόδειξη Ιδιοτήτων λογαρίθμων}

 \begin{itemize}
  \item<1-> $\log_α\left( θ_1\cdot θ_2 \right)=\log_αθ_1+\log_αθ_2 $
  \item<3-> $\log_α\dfrac{θ_1}{θ_2}=\log_αθ_1-\log_αθ_2 $
  \item<4-> $\log_αθ^κ=κ\log_αθ$
 \end{itemize}
 \only<1-3>{Θέτουμε $\log_αθ_1=x_1\implies α^{x_1}=θ_1$ και $\log_αθ_2=x_2\implies α^{x_2}=θ_2$ και έχουμε}
 \only<2>{$$\log_α\left( θ_1\cdot θ_2 \right)=\log_α\left( α^{x_1}α^{x_2} \right)=\log_α α^{x_1+x_2} =x_1+x_2=\log_αθ_1+\log_αθ_2$$}

 \only<3>{$$\log_α\dfrac{θ_1}{θ_2}=\log_α \dfrac{α^{x_1}}{α^{x_2}}=\log_α α^{x_1-x_2} =x_1-x_2=\log_αθ_1-\log_αθ_2$$}

 \only<4-5>{Θέτουμε $\log_αθ=x\implies α^{x}=θ$ και έχουμε}
 \only<5>{$$\log_αθ^κ=\log_α{(α^x)}^κ=\log_αα^{κx} =κx=κ\log_αθ$$}

 \hyperlink{Ιδιότητες}{\beamerbutton{Πίσω στη θεωρία}}
\end{frame}


% \section{Λύσεις Ασκήσεων}
% \begin{frame}
%  \tableofcontents
% \end{frame}
%
% \subsection{Άσκηση 1}
% \begin{frame}[label=Λύση1]
%  Με θεώρημα ενδιαμέσων τιμών. Η συνάρτηση είναι συνεχής στο $[10,11]$ με $f(10)=1024$ και $f(11)=2048$. Αφού $2023\in (1024,2048)$ υπάρχει $x_0$...
%
%  \hyperlink{Άσκηση1}{\beamerbutton{Πίσω στην άσκηση}}
% \end{frame}
%
% \subsection{Άσκηση 2}
% \begin{frame}[label=Λύση2]
%  Με Bolzano ή με μέγιστης ελάχιστης τιμής και ΘΕΤ.
%
%  \begin{gather*}
%   f(3)<f(2)<f(1) \\
%   3f(3)<f(1)+f(2)+f(3)<3f(1) \\
%   f(3)<\dfrac{f(1)+f(2)+f(3)}{3}<f(1)
%  \end{gather*}
%
%  \hyperlink{Άσκηση2}{\beamerbutton{Πίσω στην άσκηση}}
% \end{frame}
%
% \subsection{Άσκηση 3}
% \begin{frame}[label=Λύση3]
%  Προφανές ελάχιστο στα $x_1=1$ και $x_2=3$. Ως συνεχής στο $[1,3]$ έχει σίγουρα ΚΑΙ μέγιστο στο $(1,3)$
%
%  \hyperlink{Άσκηση3}{\beamerbutton{Πίσω στην άσκηση}}
% \end{frame}
%
% \subsection{Άσκηση 4}
% \begin{frame}[label=Λύση4]
%  Η συνάρτηση `απόστασης` $f(x)-x$ είναι ορισμένη στο κλειστό διάστημα και έχει σίγουρα μέγιστο
%
%  \hyperlink{Άσκηση4}{\beamerbutton{Πίσω στην άσκηση}}
% \end{frame}
%
% \subsection{Άσκηση 5}
% \begin{frame}[label=Λύση5]
%  Όμοια με την Άσκηση 2
%
%  \hyperlink{Άσκηση5}{\beamerbutton{Πίσω στην άσκηση}}
% \end{frame}
%
% \subsection{Άσκηση 6}
% \begin{frame}[label=Λύση6]
%  \begin{enumerate}
%   \item Είναι γνησίως αύξουσα άρα $(f(+\infty),f(-\infty))$
%   \item Προφανώς $[f(0),f(1)]$...
%  \end{enumerate}
%
%  \hyperlink{Άσκηση6}{\beamerbutton{Πίσω στην άσκηση}}
% \end{frame}

\end{document}
