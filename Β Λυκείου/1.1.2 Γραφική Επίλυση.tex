\documentclass{../presentation}

\title{Συστήματα}
\subtitle{Γραφική Επίλυση}
\author[Λόλας]{Κωνσταντίνος Λόλας}

\begin{document}

\begin{frame}
  \titlepage
\end{frame}

\section{Θεωρία}
\begin{frame}
\end{frame}

\section{Ασκήσεις}
\exercises

\begin{askisi}
  Να σχεδιάσετε τις ευθείες που παριστάνουν οι παρακάτω εξισώσεις και να βρείτε τον συντελεστή διεύθυνσης (εφόσον ορίζεται).
  \begin{enumerate}[<+->]
    \item $2x-y-1=0$
    \item $y=3$
    \item $x=-2$
  \end{enumerate}
\end{askisi}

\begin{askisi}
  Να βρείτε την εξίσωση της ευθείας $ε:y=(α-β)x+α$, που φαίνεται στο σχήμα.
\end{askisi}

\begin{askisi}
  Δίνονται οι ευθείες $ε_1:3x+6y=5$ και $ε_2:x+2y=α$, όπου $α\in\mathbb{R}$.
  \begin{enumerate}[<+->]
    \item Να βρείτε τους συντελεστές διεύθυνσης των $ε_1$ και $ε_2$.
    \item Υπάρχουν τιμές του $α$ για τις οποίες οι $ε_1$ και $ε_2$ τέμνονται;
    \item Για ποιες τιμές του $α$ οι ευθείες:
          \begin{enumerate}
            \item είναι παράλληλες;
            \item ταυτίζονται;
          \end{enumerate}
  \end{enumerate}
\end{askisi}

\begin{askisi}
  Να λύσετε γραφικά το σύστημα: $$\begin{cases} 2x-y=0 \\ x+y=3 \end{cases}$$
\end{askisi}

\begin{askisi}
  Να λύσετε γραφικά το σύστημα: $$\begin{cases} 2x+y=1 \\ 2x-y=3 \end{cases}$$
\end{askisi}

\begin{askisi}
  Να λύσετε γραφικά το σύστημα: $$\begin{cases} 3x-y=2 \\ 6x-2y=4 \end{cases}$$
\end{askisi}

\begin{askisi}
  \begin{enumerate}
    \item Να βρείτε τις εξισώσεις των ευθειών $ε_1$ και $ε_2$ που φαίνονται στο σχήμα.
    \item Να βρείτε το σημείο τομής των $ε_1$ και $ε_2$.
  \end{enumerate}
\end{askisi}

\begin{askisi}
  Να δείξετε ότι οι ευθείες $ε:y=λx-2$ και $ζ:4x+λy-λ=0$ τέμνονται για κάθε $λ\in\mathbb{R}$
\end{askisi}

\begin{askisi}
  Δίνονται οι ευθείες $ε:(λ-1)x+λy=λ$ και $ζ:x+λy=2$.
  \begin{enumerate}[<+->]
    \item Να βρείτε τις τιμές του $λ$ για τις οποίες οι $ε$ και $ζ$ είναι παράλληλες.

          Αν $λ=-2$, τότε:
    \item Να βρείτε το εμβαδό $Ε$ του τριγώνου που ορίζει η ευθεία $ζ$ με τους άξονες.
    \item Να υπολογίσετε την απόσταση των σημείων που η ευθεία $ζ$ τέμνει τους άξονες.
  \end{enumerate}
\end{askisi}

\moodle

\end{document}
