\documentclass{../presentation}

\title{Τριγωνομετρία}
\subtitle{Τριγωνομετρικές Συναρτήσεις}
\author[Λόλας]{Κωνσταντίνος Λόλας}
\date{}

\begin{document}

\begin{frame}
  \titlepage
\end{frame}

\section{Θεωρία}
\begin{frame}{Γεύση από το μέλλον (part 2)}
  Τριγωνομετρικές συναρτήσεις
  \begin{enumerate}
    \item<1-> νέα ιδιότητα συναρτήσεων (περίοδος)
    \item<2-> γραφική παράσταση $ημx$, $συνx$, $εφx$ και $σφx$
    \item<3-> γραφική παράσταση σύνθετων τριγωνομετρικών συναρτήσεων
  \end{enumerate}
\end{frame}

\begin{frame}{Καταιγίδα πριν την λιακάδα}
  Γνωστή έννοια με άγνωστα μαθηματικά... Περίοδος
  \begin{enumerate}
    \item<1-> περιστροφής γης γύρω από τον εαυτό της $T\approx 24$ ώρες
    \item<2-> λεπτοδείκτη $60$ λεπτά
    \item<3-> σελήνης γύρω από τη γη $\approx 27$ ημέρες...
  \end{enumerate}
\end{frame}

\begin{frame}{Λίιιιιιγο μαθηματικά}
  \begin{block}{Ορισμός}<1->
    Μία συνάρτηση $f$ με πεδίο ορισμού το $Α$ λέγεται \emph{περιοδική}, όταν υπάρχει αριθμός $Τ$ τέτοιος ώστε για κάθε $x\in Α$ να ισχύει:
    \begin{itemize}
      \item $x+Τ\in Α$, $x-Τ\in Α$

            και

      \item $f(x+Τ)=f(x-Τ)=f(x)$
    \end{itemize}
    Ο πραγματικός αριθμός $Τ$ λέγεται \emph{περίοδος} της συνάρτησης $f$
  \end{block}

  \begin{block}{Ορισμός (εξτρά)}<2->
    Θεμελιώδης περίοδος είναι το μικρότερο θετικό $Τ$
  \end{block}

\end{frame}

\begin{frame}{Δεν κρατιέστε!}
  Γνωστές περιοδικές συναρτήσεις
  \begin{itemize}
    \item<1-> $ημx$ με περίοδο \onslide<2-> $Τ=2π$ \onslide<3-> ή $Τ=4π$
    \item<4-> $συνx$ με περίοδο \onslide<5-> το ίδιο με πριν
    \item<6-> $εφx$ με περίοδο \onslide<7-> $π$
  \end{itemize}
\end{frame}

\begin{frame}{Γραφικές yeah!}
  \begin{itemize}
    \item<1-> Ας δούμε την γραφική παράσταση της $ημx$ στο \href{https://www.geogebra.org/m/jvvwgvru}{Geogebra}
    \item<2-> Για το $συνx$ σιγά μην ξανακάνουμε το ίδιο
    \item<3-> Ας δούμε την γραφική παράσταση της $εφx$ στο \href{https://www.geogebra.org/m/qmk7cmh2}{Geogebra}
  \end{itemize}
\end{frame}

\section{Ασκήσεις}
\begin{askisi}
  Να δείξετε ότι η συνάρτηση $f(x)$ έχει περίοδο $Τ=π$
\end{askisi}

\begin{askisi}
  Να εξετάσετε αν οι παρακάτω συναρτήσεις είναι άρτιες ή περιττές
  \begin{enumerate}
    \item<1-> $f(x)=xημ\frac{1}{x}-3συνx$
    \item<2-> $f(x)=\frac{ημx}{x^2}$
  \end{enumerate}
\end{askisi}

\begin{askisi}
  Να συγκρίνετε τους αριθμούς
  \begin{enumerate}
    \item<1-> $ημ\frac{π}{5}$, $ημ\frac{π}{7}$
    \item<2-> $συν\frac{π}{5}$, $συν\frac{π}{7}$
    \item<3-> $εφ\frac{π}{5}$, $εφ\frac{π}{7}$
    \item<4-> $σφ\frac{π}{5}$, $σφ\frac{π}{7}$
  \end{enumerate}
\end{askisi}

\begin{askisi}
  \begin{enumerate}
    \item<1-> Να δείξετε ότι αν, $0<x<\frac{π}{6}$, τότε $2ημx-1<0$
    \item<2-> Να δείξετε ότι αν, $x\in(\frac{3π}{4},π)$, τότε $εφx+1>0$
  \end{enumerate}
\end{askisi}

\begin{askisi}
  Σε ένα σύστημα αξόνων, να σχεδιάσετε τη γραφική παράσταση της συνάρτησης $f(x)=ημx$ και στη συνέχεια, τη γραφική παράσταση της συνάρτησης $g(x)=1+ημx$
\end{askisi}

\begin{askisi}
  Να κάνετε τη γραφική παράσταση της συνάρτησης $f(x)=|ημx|$ στο διάστημα $[0,2π]$
\end{askisi}

\begin{askisi}
  Στις παρακάτω συναρτήσεις:
  \begin{itemize}
    \item Να εξετάσετε αν είναι περιοδικές και να βρείτε την περίοδό τους
    \item Να βρείτε τη μέγιστη και ελάχιστη τιμή τους
          \begin{enumerate}
            \item<1-> $f(x)=3ημ2x$
            \item<2-> $f(x)=-2συν\frac{x}{3}$
            \item<3-> $f(x)=εφ\frac{πx}{2}$
          \end{enumerate}
  \end{itemize}
\end{askisi}

\begin{askisi}
  Να παραστήσετε γραφικά τη συνάρτηση $f(x)=ημ\frac{x}{2}$
\end{askisi}

\begin{askisi}
  Να παραστήσετε γραφικά τη συνάρτηση $f(x)=3συν2x$ σε διάστημα πλάτους μιας περιόδου
\end{askisi}

\begin{askisi}
  Να παραστήσετε γραφικά τη συνάρτηση $f(x)=ημ(π-x)+συν(\frac{π}{2}-x)$
\end{askisi}

\begin{askisi}
  Να δείξετε ότι η συνάρτηση $f(x)=2ημx+συν\frac{x}{2}$ έχει περίοδο $T=4π$
\end{askisi}

\begin{askisi}
  Αν η συνάρτηση $f(x)=(2-α)ημβx$, $α>2$ και $β>0$ έχει περίοδο το $\frac{π}{2}$ και μέγιστη τιμή το $3$, να βρείτε τα $α$ και $β$
\end{askisi}

\begin{askisi}
  Έστω η συνάρτηση $f(x)=1-3ημ2x$
  \begin{enumerate}
    \item<1-> Ποιά είναι η περίοδός της
    \item<2-> Ποιά είναι η μέγιστη και ποια η ελάχιστη τιμή της συνάρτησης αυτής
    \item<3-> Να σχεδιάσετε τη γραφική παράσταση της $f$ σε διάστημα πλάτους μιας περιόδου
  \end{enumerate}
\end{askisi}

\begin{askisi}
  Η θερμοκρασία σε βαθμούς Κελσίου μιας ημέρας σε ένα χώρο περιγράφεται κατά προσέγγιση από τη συνάρτηση $Θ=10ημ\frac{πt}{12}$, όπου $t$ ο χρόνος σε ώρες.
  \begin{enumerate}
    \item<1-> Πόση είναι η μέγιστη μεταβολή της θερμοκρασίας κατά τη διάρκεια ενός 24ώρου?
    \item<2-> Να κάνετε τη γραφική παράσταση της συνάρτησης για $0\le t \le 24$
    \item<3-> Να βρείτε γραφικά ποιες χρονικές στιγμές η θερμοκρασία ήταν:
          \begin{enumerate}
            \item<3-> $0^{\circ}$C
            \item<4-> κάτω από $0^{\circ}$C
          \end{enumerate}
  \end{enumerate}
\end{askisi}

\begin{askisi}
  Δίνεται η συνάρτηση $f(x)=\frac{ημx}{x}$, $x\in\left[ \frac{π}{2},π \right) $
  \begin{enumerate}
    \item<1-> Να μελετήσετε τη συνάρτηση $f$ ως προς τη μονοτονία
    \item<2-> Αν $α$, $β\in\left[ \frac{π}{2},π \right) $ και $α<β$, να δείξετε ότι $\frac{α}{β}<\frac{ημα}{ημβ}$
    \item<3-> Να βρείτε την τιμή $f(\frac{π}{2})$
    \item<4-> Να δείξετε ότι $\frac{ημx}{x}<\frac{π}{2}$ για κάθε $x\in\left[ \frac{π}{2},π \right) $
  \end{enumerate}
\end{askisi}

\section{}
\begin{frame}
  Στο moodle θα βρείτε τις ασκήσεις που πρέπει να κάνετε, όπως και αυτή τη παρουσίαση
\end{frame}

% \appendix
% \section{Λύσεις Ασκήσεων}
% \begin{frame}
%  \tableofcontents
% \end{frame}

\end{document}
