\documentclass[greek]{beamer}
%\usepackage{fontspec}
\usepackage{amsmath,amsthm}
\usepackage{unicode-math}
\usepackage{xltxtra}
\usepackage{graphicx}
\usetheme{CambridgeUS}
\usecolortheme{seagull}
\usepackage{hyperref}
\usepackage{ulem}
\usepackage{xgreek}

\usepackage{pgfpages}
\usepackage{tikz}
\usepackage{tkz-tab}
%\setbeameroption{show notes on second screen}
%\setbeameroption{show only notes}

\setsansfont{Noto Serif}

\usepackage{multicol}

\usepackage{appendixnumberbeamer}

\setbeamercovered{transparent}
\beamertemplatenavigationsymbolsempty

\title{Τριγωνομετρία}
\subtitle{Τριγωνομετρικές Συναρτήσεις}
\author[Λόλας]{Κωνσταντίνος Λόλας}
\date{}

\begin{document}

\begin{frame}
 \titlepage
\end{frame}

\section{Θεωρία}
\begin{frame}{Γεύση από το μέλλον (part 2)}
 Τριγωνομετρικές συναρτήσεις
 \begin{enumerate}
  \item<1-> νέα ιδιότητα συναρτήσεων (περίοδος)
  \item<2-> γραφική παράσταση $ημx$, $συνx$, $εφx$ και $σφx$
  \item<3-> γραφική παράσταση σύνθετων τριγωνομετρικών συναρτήσεων
 \end{enumerate}
\end{frame}

\begin{frame}{Καταιγίδα πριν την λιακάδα}
 Γνωστή έννοια με άγνωστα μαθηματικά... Περίοδος
 \begin{enumerate}
  \item<1-> περιστροφής γης γύρω από τον εαυτό της $T\approx 24$ ώρες
  \item<2-> λεπτοδείκτη $60$ λεπτά
  \item<3-> σελήνης γύρω από τη γη $\approx 27$ ημέρες...
 \end{enumerate}
\end{frame}

\begin{frame}{Λίιιιιιγο μαθηματικά}
 \begin{block}{Ορισμός}<1->
  Μία συνάρτηση $f$ με πεδίο ορισμού το $Α$ λέγεται \emph{περιοδική}, όταν υπάρχει αριθμός $Τ$ τέτοιος ώστε για κάθε $x\in Α$ να ισχύει:
  \begin{itemize}
   \item $x+Τ\in Α$, $x-Τ\in Α$

         και

   \item $f(x+Τ)=f(x-Τ)=f(x)$
  \end{itemize}
  Ο πραγματικός αριθμός $Τ$ λέγεται \emph{περίοδος} της συνάρτησης $f$
 \end{block}

 \begin{block}{Ορισμός (εξτρά)}<2->
  Θεμελιώδης περίοδος είναι το μικρότερο θετικό $Τ$
 \end{block}

\end{frame}

\begin{frame}{Δεν κρατιέστε!}
 Γνωστές περιοδικές συναρτήσεις
 \begin{itemize}
  \item<1-> $ημx$ με περίοδο \onslide<2-> $Τ=2π$ \onslide<3-> ή $Τ=4π$
  \item<4-> $συνx$ με περίοδο \onslide<5-> το ίδιο με πριν
  \item<6-> $εφx$ με περίοδο \onslide<7-> $π$
 \end{itemize}
\end{frame}

\begin{frame}{Γραφικές yeah!}
 \begin{itemize}
  \item<1-> Ας δούμε την γραφική παράσταση της $ημx$ στο \href{https://www.geogebra.org/m/jvvwgvru}{Geogebra}
  \item<2-> Για το $συνx$ σιγά μην ξανακάνουμε το ίδιο
  \item<3-> Ας δούμε την γραφική παράσταση της $εφx$ στο \href{https://www.geogebra.org/m/qmk7cmh2}{Geogebra}
 \end{itemize}
\end{frame}

\section{Ασκήσεις}
\subsection{Άσκηση 1}
\begin{frame}[label=Άσκηση1]{Εξάσκηση 1}
 Να δείξετε ότι η συνάρτηση $f(x)$ έχει περίοδο $Τ=π$

 % \hyperlink{Λύση1}{\beamerbutton{Λύση}}
\end{frame}

\subsection{Άσκηση 2}
\begin{frame}[label=Άσκηση2]{Εξάσκηση 2}
 Να εξετάσετε αν οι παρακάτω συναρτήσεις είναι άρτιες ή περιττές
 \begin{enumerate}
  \item<1-> $f(x)=xημ\frac{1}{x}-3συνx$
  \item<2-> $f(x)=\frac{ημx}{x^2}$
 \end{enumerate}

 %\hyperlink{Λύση2}{\beamerbutton{Λύση}}
\end{frame}

\subsection{Άσκηση 3}
\begin{frame}[label=Άσκηση3]{Εξάσκηση 3}
 Να συγκρίνετε τους αριθμούς
 \begin{enumerate}
  \item<1-> $ημ\frac{π}{5}$, $ημ\frac{π}{7}$
  \item<2-> $συν\frac{π}{5}$, $συν\frac{π}{7}$
  \item<3-> $εφ\frac{π}{5}$, $εφ\frac{π}{7}$
  \item<4-> $σφ\frac{π}{5}$, $σφ\frac{π}{7}$
 \end{enumerate}

 %\hyperlink{Λύση3}{\beamerbutton{Λύση}}
\end{frame}

\subsection{Άσκηση 4}
\begin{frame}[label=Άσκηση4]{Εξάσκηση 4}
 \begin{enumerate}
  \item<1-> Να δείξετε ότι αν, $0<x<\frac{π}{6}$, τότε $2ημx-1<0$
  \item<2-> Να δείξετε ότι αν, $x\in(\frac{3π}{4},π)$, τότε $εφx+1>0$
 \end{enumerate}

 %\hyperlink{Λύση4}{\beamerbutton{Λύση}}
\end{frame}

\subsection{Άσκηση 5}
\begin{frame}[label=Άσκηση5]{Εξάσκηση 5}
 Σε ένα σύστημα αξόνων, να σχεδιάσετε τη γραφική παράσταση της συνάρτησης $f(x)=ημx$ και στη συνέχεια, τη γραφική παράσταση της συνάρτησης $g(x)=1+ημx$

 %\hyperlink{Λύση5}{\beamerbutton{Λύση}}
\end{frame}

\subsection{Άσκηση 6}
\begin{frame}[label=Άσκηση6]{Εξάσκηση 6}
 Να κάνετε τη γραφική παράσταση της συνάρτησης $f(x)=|ημx|$ στο διάστημα $[0,2π]$

 %\hyperlink{Λύση6}{\beamerbutton{Λύση}}
\end{frame}

\subsection{Άσκηση 7}
\begin{frame}[label=Άσκηση]{Εξάσκηση 7}
 Στις παρακάτω συναρτήσεις:
 \begin{itemize}
  \item Να εξετάσετε αν είναι περιοδικές και να βρείτε την περίοδό τους
  \item Να βρείτε τη μέγιστη και ελάχιστη τιμή τους
        \begin{enumerate}
         \item<1-> $f(x)=3ημ2x$
         \item<2-> $f(x)=-2συν\frac{x}{3}$
         \item<3-> $f(x)=εφ\frac{πx}{2}$
        \end{enumerate}
 \end{itemize}

 %\hyperlink{Λύση}{\beamerbutton{Λύση}}
\end{frame}

\subsection{Άσκηση 8}
\begin{frame}[label=Άσκηση8]{Εξάσκηση 8}
 Να παραστήσετε γραφικά τη συνάρτηση $f(x)=ημ\frac{x}{2}$

 %\hyperlink{Λύση8}{\beamerbutton{Λύση}}
\end{frame}

\subsection{Άσκηση 9}
\begin{frame}[label=Άσκηση9]{Εξάσκηση 9}
 Να παραστήσετε γραφικά τη συνάρτηση $f(x)=3συν2x$ σε διάστημα πλάτους μιας περιόδου

 %\hyperlink{Λύση9}{\beamerbutton{Λύση}}
\end{frame}

\subsection{Άσκηση 10}
\begin{frame}[label=Άσκηση10]{Εξάσκηση 10}
 Να παραστήσετε γραφικά τη συνάρτηση $f(x)=ημ(π-x)+συν(\frac{π}{2}-x)$

 %\hyperlink{Λύση10}{\beamerbutton{Λύση}}
\end{frame}

\subsection{Άσκηση 11}
\begin{frame}[label=Άσκηση11]{Εξάσκηση 11}
 Να δείξετε ότι η συνάρτηση $f(x)=2ημx+συν\frac{x}{2}$ έχει περίοδο $T=4π$

 %\hyperlink{Λύση11}{\beamerbutton{Λύση}}
\end{frame}

\subsection{Άσκηση 12}
\begin{frame}[label=Άσκηση12]{Εξάσκηση 12}
 Αν η συνάρτηση $f(x)=(2-α)ημβx$, $α>2$ και $β>0$ έχει περίοδο το $\frac{π}{2}$ και μέγιστη τιμή το $3$, να βρείτε τα $α$ και $β$

 %\hyperlink{Λύση12}{\beamerbutton{Λύση}}
\end{frame}

\subsection{Άσκηση 13}
\begin{frame}[label=Άσκηση13]{Εξάσκηση 13}
 Έστω η συνάρτηση $f(x)=1-3ημ2x$
 \begin{enumerate}
  \item<1-> Ποιά είναι η περίοδός της
  \item<2-> Ποιά είναι η μέγιστη και ποια η ελάχιστη τιμή της συνάρτησης αυτής
  \item<3-> Να σχεδιάσετε τη γραφική παράσταση της $f$ σε διάστημα πλάτους μιας περιόδου
 \end{enumerate}

 %\hyperlink{Λύση13}{\beamerbutton{Λύση}}
\end{frame}

\subsection{Άσκηση 14}
\begin{frame}[label=Άσκηση14]{Εξάσκηση 14}
 Η θερμοκρασία σε βαθμούς Κελσίου μιας ημέρας σε ένα χώρο περιγράφεται κατά προσέγγιση από τη συνάρτηση $Θ=10ημ\frac{πt}{12}$, όπου $t$ ο χρόνος σε ώρες.
 \begin{enumerate}
  \item<1-> Πόση είναι η μέγιστη μεταβολή της θερμοκρασίας κατά τη διάρκεια ενός 24ώρου?
  \item<2-> Να κάνετε τη γραφική παράσταση της συνάρτησης για $0\le t \le 24$
  \item<3-> Να βρείτε γραφικά ποιες χρονικές στιγμές η θερμοκρασία ήταν:
   \begin{enumerate}
    \item<3-> $0^{\circ}$C
    \item<4-> κάτω από $0^{\circ}$C
   \end{enumerate}
 \end{enumerate}

 %\hyperlink{Λύση14}{\beamerbutton{Λύση}}
\end{frame}

\subsection{Άσκηση 15}
\begin{frame}[label=Άσκηση15]{Εξάσκηση 15}
  Δίνεται η συνάρτηση $f(x)=\frac{ημx}{x}$, $x\in\left[ \frac{π}{2},π \right) $
\begin{enumerate}
  \item<1-> Να μελετήσετε τη συνάρτηση $f$ ως προς τη μονοτονία
  \item<2-> Αν $α$, $β\in\left[ \frac{π}{2},π \right) $ και $α<β$, να δείξετε ότι $\frac{α}{β}<\frac{ημα}{ημβ}$
  \item<3-> Να βρείτε την τιμή $f(\frac{π}{2})$
  \item<4-> Να δείξετε ότι $\frac{ημx}{x}<\frac{π}{2}$ για κάθε $x\in\left[ \frac{π}{2},π \right) $
\end{enumerate}
  %\hyperlink{Λύση15}{\beamerbutton{Λύση}}
\end{frame}

\section{}
\begin{frame}
 Στο moodle θα βρείτε τις ασκήσεις που πρέπει να κάνετε, όπως και αυτή τη παρουσίαση
\end{frame}

% \appendix
% \section{Λύσεις Ασκήσεων}
% \begin{frame}
%  \tableofcontents
% \end{frame}

\end{document}
