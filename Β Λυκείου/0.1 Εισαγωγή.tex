\documentclass{presentation}

\title{Άλγεβρα}
\subtitle{Μια εισαγωγή}
\author[Λόλας]{Κωνσταντίνος. Λόλας}

\begin{document}

\begin{frame}
  \titlepage
\end{frame}

\section{Θεωρία}
\begin{frame}{Με τι θα σας πρίξω}
  \begin{enumerate}
    \item Κεφάλαιο 1. Συστήματα
          \only<2>{
            \begin{itemize}
              \item ΜΟΝΟ γραμμικά συστήματα
            \end{itemize}
          }
    \item Κεφάλαιο 2. Ιδιότητες συναρτήσεων
          \only<3>{
            \begin{itemize}
              \item Ακρότατα
              \item Μονοτονία
              \item Συμμετρίες
              \item Μετατοπίσεις
            \end{itemize}
          }
    \item Κεφάλαιο 3. Τριγωνομετρία
          \only<4>{
            \begin{itemize}
              \item Τριγωνομετρικοί αριθμοί
              \item Υπολογισμός τριγωνομετρικών αριθμών > 90$^\circ$
              \item Τριγωνομετρικές ταυτότητες
              \item Τριγωνομετρικές εξισώσεις
              \item Τριγωνομετρικές συναρτήσεις
            \end{itemize}
          }
    \item Κεφάλαιο 4. Πολυώνυμα
          \only<5>{
            \begin{itemize}
              \item Βασικοί ορισμοί
              \item Διαίρεση πολυωνύμων
              \item Θεωρήματα
              \item Ρίζες πολυωνύμων
              \item Ανισώσεις
            \end{itemize}
          }
    \item Κεφάλαιο 5. Εκθετικές και λογαριθμικές συναρτήσεις
          \only<6>{
            \begin{itemize}
              \item Βασικοί ορισμοί
              \item Ιδιότητες
              \item Εξισώσεις και ανισώσεις
              \item Γραφικές παραστάσεις
            \end{itemize}
          }
  \end{enumerate}

\end{frame}

\begin{frame}{Σε τι θα σας φανώ χρήσιμος}
  \begin{itemize}
    \item Πράξεις (προτεραιότητα, παραγοντοποίηση, κλπ)
    \item Ταυτότητες
    \item Εξισώσεις και ανισώσεις (πρώτου και δευτέρου βαθμού)
    \item Εισαγωγή στη Γ Λυκείου
  \end{itemize}
\end{frame}

\begin{frame}{Τι θα χρειαστείτε}
  \begin{itemize}[<+->]
    \item Όρεξη
    \item Χρόνο
    \item Υπομονή
    \item Μολύβι - Στυλό
  \end{itemize}
\end{frame}

\begin{frame}[noframenumbering]
  Στο moodle θα βρείτε τις ασκήσεις που πρέπει να κάνετε, όπως και αυτή τη παρουσίαση
\end{frame}

\section{Ασκήσεις}

\begin{frame}[noframenumbering]
  \vfill
  \centering
  \begin{beamercolorbox}[sep=8pt,center,shadow=true,rounded=true]{title}
    \usebeamerfont{title}Ασκήσεις
  \end{beamercolorbox}
  \vfill
\end{frame}

\begin{askisi}
  Στο διπλανό σχήμα φαίνεται η γραφική παράσταση μιας συνάρτησης $f$
  \begin{enumerate}
    \item Να βρείτε τις θέσεις ακροτάτων και τα ακρότατα της $f$\pause
    \item Να δείξετε ότι $-1\le f(x) \le 3$ για κάθε $x\in[-2,2]$\pause
    \item Να δείξετε ότι $f(α)-f(β)\le 4$, $α$, $β\in[-2,2]$ \pause
    \item Να λύσετε
          \begin{enumerate}
            \item Την εξίσωση $f(x)=1$ \pause
            \item Την ανίσωση $f(x)>-1$
          \end{enumerate}
  \end{enumerate}

\end{askisi}

\begin{frame}
  Στο moodle θα βρείτε τις ασκήσεις που πρέπει να κάνετε, όπως και αυτή τη παρουσίαση
\end{frame}

\end{document}
