\documentclass{../presentation}

\title{Συστήματα}
\subtitle{Γραμμικά}
\author[Λόλας]{Κωνσταντίνος Λόλας}
\date{}

\begin{document}

\begin{frame}
  \titlepage
\end{frame}

\section{Θεωρία}
\begin{frame}{Σύστημα?}
  Πρώτα εξίσωση, μετά σύστημα
  \begin{itemize}[<+->]
    \item \textbf{Εξίσωση} είναι μια ισότητα αλγεβρικών παραστάσεων.
    \item \textbf{Λύση} μιας εξίσωσης είναι η τιμή της μεταβλητής που την καθιστά αληθή.
    \item \textbf{Σύστημα} είναι δύο ή περισσότερες εξισώσεις με σκοπό να λυθούν ταυτόχρονα.
    \item \textbf{Λύση} ενός συστήματος είναι ένα σύνολο τιμών των μεταβλητών που ικανοποιεί όλες τις εξισώσεις του συστήματος.
  \end{itemize}
\end{frame}

\begin{frame}{Τύποι συστημάτων}
  \begin{itemize}[<+->]
    \item \textbf{Γραμμικό σύστημα} είναι το σύστημα στο οποίο όλες οι εξισώσεις είναι γραμμικές ως προς τις μεταβλητές τους.
    \item \textbf{Μη γραμμικό σύστημα} είναι το σύστημα όπου τουλάχιστον μία εξίσωση δεν είναι γραμμική.
    \item \textbf{Ομογενές σύστημα} είναι το σύστημα όπου όλες οι εξισώσεις του έχουν μηδενικό όρο.
    \item \textbf{Μη ομογενές σύστημα} είναι το σύστημα όπου τουλάχιστον μία εξίσωση έχει μη μηδενικό όρο.
  \end{itemize}
\end{frame}

\begin{frame}{Τύποι εξισώσεων}
  \begin{itemize}[<+->]
    \item \textbf{πολυωνύμων}
    \item \textbf{εκθετικών}
    \item \textbf{λογαριθμικών}
    \item \textbf{τριγωνομετρικών}
    \item \textbf{διαφορικών}
    \item \textbf{διαφορών}
  \end{itemize}
\end{frame}

\begin{frame}
  \begin{block}{Γραμμικό σύστημα}
    Ένα σύστημα του οποίου οι εξισώσεις έχουν αγνώστους το πολύ στον πρώτο βαθμό.
  \end{block}
  \begin{block}{Κανονική μορφή γραμμικού συστήματος}
    $$
      \begin{cases}
        a_{11}x_1+a_{12}x_2+\cdots+a_{1n}x_n=b_1 \\
        a_{21}x_1+a_{22}x_2+\cdots+a_{2n}x_n=b_2 \\
        \vdots                                   \\
        a_{m1}x_1+a_{m2}x_2+\cdots+a_{mn}x_n=b_m
      \end{cases}
    $$
  \end{block}

\end{frame}

\begin{frame}{Επίλυση}
  \begin{itemize}[<+- | alert@+>]
    \item Αντικατάσταση
    \item Αντίθετων συντελεστών
  \end{itemize}
\end{frame}

\section{Ασκήσεις}
\exercises

\begin{askisi}
  Δίνεται η εξίσωση $x+4y=1$.
  \begin{enumerate}
    \item Να εξετάσετε αν τα ζεύγη $(-2,1)$, $(1,-2)$ είναι λύσεις της εξίσωσης.
    \item Να βρείτε την τιμή του $α$, για την οποία το ζεύγος $(α,α-1)$ είναι λύση της εξίσωσης.
    \item Να λύσετε την εξίσωση.
  \end{enumerate}
\end{askisi}

\begin{askisi}
  Να λύσετε το σύστημα
  $$\begin{cases}
      2x+y=4 \\
      5x+2y=10
    \end{cases}$$
\end{askisi}

\begin{askisi}
  Να λύσετε το σύστημα
  $$\begin{cases}
      3x-5y=8 \\
      4x+7y=-3
    \end{cases}$$
\end{askisi}

\begin{askisi}
  Να λύσετε τα συστήματα
  \begin{multicols}{2}
    \begin{enumerate}[<+->]
      \item
            $\begin{cases}
                x+2y=1 \\
                2x+4y=0
              \end{cases}$
      \item
            $\begin{cases}
                x(y+1)-y(x-3)=6 \\
                \dfrac{x}{3}+y=2
              \end{cases}$
    \end{enumerate}
  \end{multicols}
\end{askisi}

\begin{askisi}
  Να προσδιορίσετε το πλήθος των λύσεων των συστημάτων
  \begin{multicols}{3}
    \begin{enumerate}[<+->]
      \item
            $\begin{cases}
                3x+y=5 \\
                4x+3y=10
              \end{cases}$
      \item
            $\begin{cases}
                3x-2y=10 \\
                6x-4y=5
              \end{cases}$
      \item
            $\begin{cases}
                2x+4y=2 \\
                x+2y=1
              \end{cases}$
    \end{enumerate}
  \end{multicols}
\end{askisi}

\begin{askisi}
  Εκατόν τριάντα έξι τουρίστες μεταφέρθηκαν με 10 λεωφορεία των 12 θέσεων και 16 θέσεων. Πόσα ήταν τα λεωφορεία κάθε τύπου?
\end{askisi}

\section{}
\begin{frame}
  Στο moodle θα βρείτε τις ασκήσεις που πρέπει να κάνετε, όπως και αυτή τη παρουσίαση
\end{frame}

\end{document}
