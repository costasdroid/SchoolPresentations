\documentclass[greek]{beamer}
%\usepackage{fontspec}
\usepackage{amsmath,amsthm}
\usepackage{unicode-math}
\usepackage{xltxtra}
\usepackage{graphicx}
\usetheme{CambridgeUS}
\usecolortheme{seagull}
\usepackage{hyperref}
\usepackage{ulem}
\usepackage{xgreek}

\usepackage{pgfpages}
\usepackage{tikz}
\usepackage{tkz-tab}
%\setbeameroption{show notes on second screen}
%\setbeameroption{show only notes}

\setsansfont{Noto Serif}

\usepackage{multicol}

\usepackage{appendixnumberbeamer}

\setbeamercovered{transparent}
\beamertemplatenavigationsymbolsempty

\title{Τριγωνομετρία}
\subtitle{Τριγωνομετρικές Ταυτότητες}
\author[Λόλας]{Κωνσταντίνος Λόλας}
\date{}

\begin{document}

\begin{frame}
 \titlepage
\end{frame}

\section{Ασκήσεις}
\subsection{Άσκηση 1}
\begin{frame}[label=Άσκηση1]{Εξάσκηση 1}
Αν $συνω=-\frac{4}{5}$ και $\frac{π}{2}<ω<π$, να βρείτε τους άλλους τριγωνομετρικούς αριθμούς της γωνίας $ω$ σε rad

 % \hyperlink{Λύση1}{\beamerbutton{Λύση}}
\end{frame}

\subsection{Άσκηση 2}
\begin{frame}[label=Άσκηση2]{Εξάσκηση 2}
Αν $σφω=-\frac{5}{12}$ και $270^{\circ}<ω<360{\circ}$, να βρείτε τους άλλους τριγωνομετρικούς αριθμούς της γωνίας $ω$ σε rad

 %\hyperlink{Λύση2}{\beamerbutton{Λύση}}
\end{frame}

\subsection{Άσκηση 3}
\begin{frame}[label=Άσκηση3]{Εξάσκηση 3}
Αν ισχύει $2συν^2x+5ημx-4=0$ και $0<x<\frac{π}{2}$, να βρείτε το $ημx$

 %\hyperlink{Λύση3}{\beamerbutton{Λύση}}
\end{frame}

\subsection{Άσκηση 4}
\begin{frame}[label=Άσκηση4]{Εξάσκηση 4}
Να εξετάσετε αν υπάρχουν τιμές του $x$ για τις οποίες ισχύει συγχρόνως $ημx=\frac{2}{3}$ και $συνx=\frac{1}{3}$
 %\hyperlink{Λύση4}{\beamerbutton{Λύση}}
\end{frame}

\subsection{Άσκηση 5}
\begin{frame}[label=Άσκηση5]{Εξάσκηση 5}
Να αποδείξετε ότι $\frac{ημx}{1+συνx}+\frac{1+συνx}{ημx}=\frac{2}{ημx}$

 %\hyperlink{Λύση5}{\beamerbutton{Λύση}}
\end{frame}

\subsection{Άσκηση 6}
\begin{frame}[label=Άσκηση6]{Εξάσκηση 6}
 Να δείξετε ότι $εφ^2x-ημ^2x=εφ^2x\cdot ημ^2x$

 %\hyperlink{Λύση6}{\beamerbutton{Λύση}}
\end{frame}

\subsection{Άσκηση 7}
\begin{frame}[label=Άσκηση]{Εξάσκηση 7}
 Να δείξετε ότι $\frac{συνθ}{1+εφθ}-\frac{ημθ}{1+σφθ}=συνθ-ημθ$

 %\hyperlink{Λύση}{\beamerbutton{Λύση}}
\end{frame}

\subsection{Άσκηση 8}
\begin{frame}[label=Άσκηση8]{Εξάσκηση 8}
 Να δείξετε ότι $\frac{1-εφθ}{1+εφθ}=\frac{σφθ-1}{σφθ+1}$

 %\hyperlink{Λύση8}{\beamerbutton{Λύση}}
\end{frame}

\subsection{Άσκηση 9}
\begin{frame}[label=Άσκηση9]{Εξάσκηση 9}
 Αν ισχύει $3ημθ+4συνθ=5$, να υπολογίσετε την τιμή της παράστασης $Α=5συνθ-4εφθ$

 %\hyperlink{Λύση9}{\beamerbutton{Λύση}}
\end{frame}

\subsection{Άσκηση 10}
\begin{frame}[label=Άσκηση10]{Εξάσκηση 10}
 Να αποδείξετε ότι:
 \begin{itemize}
   \item \begin{itemize}
     \item $ημ^4x+συν^4x=1-2ημ^2x\cdot συν^2x$
       \item $ημ^6x+συν^6x=1-3ημ^2x\cdot συν^2x$
   \end{itemize}
   \item η παράσταση
   $$Α=3(ημ^4x+συν^4x)-2(ημ^6x+συν^6x)$$
   είναι ανεξάρτητη του $x$
 \end{itemize}

 %\hyperlink{Λύση10}{\beamerbutton{Λύση}}
\end{frame}

\subsection{Άσκηση 11}
\begin{frame}[label=Άσκηση11]{Εξάσκηση 11}
 Αν $0<x<π$, να δείξετε ότι $\sqrt{\frac{1-συνx}{1+συνx}}-\frac{1+συνx}{1-συνx}=-2σφx$

 %\hyperlink{Λύση11}{\beamerbutton{Λύση}}
\end{frame}

\section{}
\begin{frame}
 Στο moodle θα βρείτε τις ασκήσεις που πρέπει να κάνετε, όπως και αυτή τη παρουσίαση
\end{frame}

% \appendix
% \section{Λύσεις Ασκήσεων}
% \begin{frame}
%  \tableofcontents
% \end{frame}

\end{document}
