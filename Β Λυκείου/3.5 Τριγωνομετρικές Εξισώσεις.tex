\documentclass{../presentation}

\title{Τριγωνομετρία}
\subtitle{Τριγωνομετρικές Εξισώσεις}
\author[Λόλας]{Κωνσταντίνος Λόλας}
\date{}

\begin{document}

\begin{frame}
  \titlepage
\end{frame}

\section{Θεωρία}
\begin{frame}{Εξισώσεις! Πάλι?}
  Μέχρι στιγμής τι είδους εξισώσεις λύναμε?
  \begin{enumerate}
    \item<1-> πολυωνυμικές 1ου βαθμού
    \item<2-> πολυωνυμικές 2ου βαθμού
    \item<3-> όσες ανάγονται σε πολυωνυμικές
    \item<4-> ΑΥΤΑ
  \end{enumerate}
\end{frame}

\begin{frame}{Γιατί?????}
  Γιατί ειδικά οι τριγωνομετρικές?
  \begin{enumerate}
    \item<1-> δεν απομονώνεται το $x$
    \item<2-> έχουν σχεδόν πάντα άπειρες??? λύσεις
    \item<3-> και άλλα πολλά που δεν θυμάμαι...
  \end{enumerate}
\end{frame}

\begin{frame}{Λίιιιιιγο μαθηματικά}
  \begin{enumerate}
    \item<1-> Φτιάξτε άξονες
    \item<2-> Διαλέξτε μία τιμή στον άξονα των ημιτώνων
    \item<3-> Βρείτε μία γωνία που να έχει αυτό το ημίτονο
    \item<4-> Βρείτε δεύτερη γωνία που να έχει αυτό το ημίτονο
    \item<5-> Βρείτε τρίτη γωνία που να έχει αυτό το ημίτονο
  \end{enumerate}
  \onslide<6-> Συμπέρασμα

\end{frame}

\begin{frame}{Εξίσωση 1}
  \begin{block}{$ημx=ημθ$}
    Η εξίσωση $ημx=ημθ$ έχει λύσεις
    $$x=2κπ+θ$$
    και
    $$x=2κπ+π-θ$$
    όπου $κ\in\mathbb{Z}$
  \end{block}
\end{frame}

\begin{frame}{Λίιιιιιγο ακόμα μαθηματικά}
  \begin{enumerate}
    \item<1-> Φτιάξτε άξονες
    \item<2-> Διαλέξτε μία τιμή στον άξονα των συνημιτώνων
    \item<3-> Βρείτε μία γωνία που να έχει αυτό το συνημίτονο
    \item<4-> Βρείτε δεύτερη γωνία που να έχει αυτό το συνημίτονο
    \item<5-> Βρείτε τρίτη γωνία που να έχει αυτό το συνημίτονο
  \end{enumerate}
  \onslide<6-> Συμπέρασμα
\end{frame}

\begin{frame}{Εξίσωση 2}
  \begin{block}{$συνx=συνθ$}
    Η εξίσωση $συνx=συνθ$ έχει λύσεις
    $$x=2κπ+θ$$
    και
    $$x=2κπ-θ$$
    όπου $κ\in\mathbb{Z}$
  \end{block}
\end{frame}

\begin{frame}{Λίιιιιιγο ακόμα μαθηματικά υπόσχομαι!}
  \begin{enumerate}
    \item<1-> Φτιάξτε άξονες
    \item<2-> Διαλέξτε μία τιμή στον άξονα των εφαπτομένων
    \item<3-> Βρείτε μία γωνία που να έχει αυτή την εφαπτόμενη
    \item<4-> Βρείτε δεύτερη γωνία που να έχει αυτή την εφαπτόμενη
    \item<5-> Βρείτε τρίτη γωνία που να έχει αυτή την εφαπτόμενη
  \end{enumerate}
  \onslide<6-> Συμπέρασμα
\end{frame}

\begin{frame}{Εξίσωση 3}
  \begin{block}{$εφx=εφθ$}<1->
    Η εξίσωση $εφx=εφθ$ έχει λύσεις
    $$x=κπ+θ$$
    όπου $κ\in\mathbb{Z}$
  \end{block}
  \onslide<2-> και όμοια γιατί το υποσχέθηκα

  \begin{block}{$σφx=σφθ$}<3->
    Η εξίσωση $σφx=σφθ$ έχει λύσεις
    $$x=κπ+θ$$
    όπου $κ\in\mathbb{Z}$
  \end{block}
\end{frame}

\begin{frame}
  Άρα
  \begin{enumerate}
    \item<1-> τα φέρνουμε πάντα στην μορφή $ημ=ημ$ ή $συν=συν$...
    \item<2-> αν δεν είναι
          \begin{itemize}
            \item<3-> ανάγουμε στο 1ο τεταρτημόριο ή
            \item<4-> χρησιμοποιούμε τριγωνομετρικές (και όχι μόνο) ταυτότητες ή
            \item<5-> θέτουμε για να διευκολύνουμε και...
            \item<6-> γενικά σκεφτόμαστε
          \end{itemize}

          \onslide<7-> και πάμε ξανά στο βήμα 1

    \item<8-> ΜΟΝΟ τότε διώχνουμε τους "προστάτες"
  \end{enumerate}
\end{frame}

\section{Ασκήσεις}
\begin{askisi}
  Να λύσετε τις εξισώσεις
  \begin{enumerate}
    \item<1-> $ημx=\frac{\sqrt{3}}{2}$
    \item<2-> $2συνx-1=0$
    \item<3-> $εφx-\sqrt{3}=0$
    \item<4-> $\sqrt{3}σφx-1=0$
  \end{enumerate}

\end{askisi}

\begin{askisi}
  Να λύσετε τις εξισώσεις
  \begin{enumerate}
    \item<1-> $ημx=-\frac{1}{2}$
    \item<2-> $\sqrt{2}συνx+1=0$
    \item<3-> $εφx+1=0$
    \item<4-> $(\sqrt{2}συνx+1)(εφx+1)=0$
  \end{enumerate}

\end{askisi}

\begin{askisi}
  Να λύσετε τις εξισώσεις
  \begin{enumerate}
    \item<1-> $(ημx-1)(συνx+1)=0$
    \item<2-> $ημx\cdot συνx=ημx$
    \item<3-> $1+ημx-συνx-ημx\cdot συνx=0$
  \end{enumerate}

\end{askisi}

\begin{askisi}
  \begin{enumerate}
    \item<1-> Να λύσετε την ανίσωση: $ημx<\sqrt{1}{2}$, στο διάστημα $Δ=(0,\frac{π}{2})$
    \item<2-> Να κάνετε τον πίνακα προσήμων της συνάρτησης $f(x)=2συνx-1$, $x\in [0,π]$
  \end{enumerate}

\end{askisi}

\begin{askisi}
  Να λύσετε τις εξισώσεις
  \begin{enumerate}
    \item<1-> $εφ2x=1$
    \item<2-> $συν(x-\frac{π}{3})=\frac{1}{2}$
    \item<3-> $ημ3x=ημ2x$
  \end{enumerate}

\end{askisi}

\begin{askisi}
  Να λύσετε την εξίσωση $2ημ^2x-1=0$

\end{askisi}

\begin{askisi}
  Να λύσετε την εξίσωση $2ημ^2x+3συνx=0$

\end{askisi}

\begin{askisi}
  Να λύσετε την εξίσωση $ημ2x-συνx=0$

\end{askisi}

\begin{askisi}
  Να λύσετε την εξίσωση $ημx-\sqrt{3}συνx=0$

\end{askisi}

\begin{askisi}
  Να λύσετε την ανίσωση $ημx-συνx>0$ sto di;asthma $Δ=(0,\frac{π}{2})$

\end{askisi}

\begin{askisi}
  Να κάνετε τον πίνακα προσήμων της συνάρτησης $f(x)=ημx+συνx$, $x\in (0,π)$

\end{askisi}

\begin{askisi}
  Να λύσετε την εξίσωση $εφx-ημx=1-ημx\cdot εφx$

\end{askisi}

\begin{askisi}
  Να λύσετε την εξίσωση $\sqrt{2}ημx+1=0$ στο διάστημα $[-π,π]$

\end{askisi}

\begin{askisi}
  Να λύσετε την εξίσωση $ημ(συν(x))=0$

\end{askisi}

\begin{askisi}
  Δίνεται η συνάρτηση $f(x)=xημx$
  \begin{enumerate}
    \item<1-> Να δείξετε ότι η συνάρτηση $f$ είναι άρτια
    \item<2-> Να βρείτε τα κοινά σημεία της $C_f$ με τον άξονα $x'x$ και την ευθεία $y=x$
    \item<3-> Να δείξετε ότι $-|x|\le f(x) \le |x|$
  \end{enumerate}
\end{askisi}

\begin{askisi}
  Να λύσετε την εξίσωση $\sqrt{x^2+1}=συνx$

\end{askisi}

\section{}
\begin{frame}
  Στο moodle θα βρείτε τις ασκήσεις που πρέπει να κάνετε, όπως και αυτή τη παρουσίαση
\end{frame}

% \appendix
% \section{Λύσεις Ασκήσεων}
% \begin{frame}
%  \tableofcontents
% \end{frame}

\end{document}
