\documentclass[greek]{beamer}
%\usepackage{fontspec}
\usepackage{amsmath,amsthm}
\usepackage{unicode-math}
\usepackage{xltxtra}
\usepackage{graphicx}
\usetheme{CambridgeUS}
\usecolortheme{seagull}
\usepackage{hyperref}
\usepackage{ulem}
\usepackage{xgreek}

\usepackage{pgfpages}
\usepackage{tikz}
\usetikzlibrary{mindmap,trees}\usetikzlibrary{shapes.geometric, arrows}

  \tikzstyle{startstop} = [rectangle, rounded corners,
  minimum width=3cm,
  minimum height=1cm,
  text centered,
  draw=black,
  fill=red!30]
  \tikzstyle{decision} = [diamond,
  minimum width=3cm,
  minimum height=1cm,
  text centered,
  draw=black,
  fill=green!30]
  \tikzstyle{arrow} = [thick,->,>=stealth]
  \tikzstyle{process} = [rectangle,
  minimum width=3cm,
  minimum height=1cm,
  text centered,
  text width=3cm,
  draw=black,
  fill=orange!30]
\usepackage{tkz-tab}
%\setbeameroption{show notes on second screen}
%\setbeameroption{show only notes}

\setsansfont{Noto Serif}

\usepackage{multicol}

\usepackage{appendixnumberbeamer}

\usepackage{polynom}

\usepackage{pgffor}

\setbeamercovered{transparent}
\beamertemplatenavigationsymbolsempty

\title{Εκθετική και Λογαριθμική Συνάρτηση}
\subtitle{Εκθετική Συνάρτηση}
\author[Λόλας]{Κωνσταντίνος Λόλας}
\date{}

\begin{document}

\begin{frame}
 \titlepage
\end{frame}

\section{Θεωρία}
\begin{frame}{Ώρα ιστορίας}
 Ξέρουμε όλοι τι σημαίνει
 \begin{itemize}
  \item<1-> αριθμός υψωμένος σε φυσικό αριθμό
  \item<2-> αριθμός υψωμένος σε ακέραιο αριθμό
  \item<3-> μάθαμε στην Α Λυκείου, αριθμό υψωμένο σε ρητό
  \item<4-> πάμε για αριθμό υψωμένο σε πραγματικό!
 \end{itemize}
 \onslide<5> και επειδή σίγουρα δεν θυμάστε...
\end{frame}

\begin{frame}{Βασική επανάληψη}
 Ιδιότητες δυνάμεων με $α\in \mathbb{R}$ και $κ$, $λ\in\mathbb{N}$
 \begin{itemize}
  \item<1-> $α^{κ+λ}=α^κ\cdot α^λ$
  \item<2-> $α^{κ-λ}=\dfrac{α^κ}{α^λ}$
  \item<3-> $\left( α^κ \right)^λ=α^{κ\cdot λ}$
  \item<4-> $α^κ\cdot β^κ=(α\cdot β)^κ$
  \item<5-> $\dfrac{α^κ}{β^κ}=\left(   \dfrac{α}{β}\right)^κ$
 \end{itemize}
 \begin{block}{Ορισμός}
  Για κάθε $α\ge 0$ και $κ$, $λ\in\mathbb{N}$
  $$α^{\frac{κ}{λ}}=\sqrt[λ]{α^κ}$$
 \end{block}
\end{frame}

\begin{frame}{Πώς υπολογίζουμε λοιπόν}
 \begin{itemize}
  \item<1-> Φυσικός: $α^x=\overbrace{α\cdot α \cdots α}^{\text{x φορές}}$
  \item<2-> Ακέραιος: $α^{-x}=\dfrac{1}{a^x}$
  \item<3-> Ρητός: $α^{\frac{x}{y}}=\sqrt[y]{α^x}$, μόνο για $α\ge 0$
  \item<4-> Άρρητος: ΤΣΟΥΚ! Μόνο με υπολογιστή, προσεγγιστικά!
 \end{itemize}
\end{frame}

\begin{frame}{Επιστροφή στο σήμερα}
 Ιδιότητες δυνάμεων με $α$, $β$ \emph{θετικοί πραγματικοί} και $x$, $x_1$, $x_2\in\mathbb{N}$
 \begin{itemize}
  \item $α^{x_1+x_2}=α^{x_1}\cdot α^{x_2}$
  \item $α^{x_1-x_2}=\dfrac{α^{x_1}}{α^{x_2}}$
  \item $\left( α^{x_1} \right)^{x_2}=α^{x_1\cdot x_2}$
  \item $α^x\cdot β^x=(α\cdot β)^x$
  \item $\dfrac{α^x}{β^x}=\left(   \dfrac{α}{β}\right)^x$
 \end{itemize}
\end{frame}

\begin{frame}{Με τι θα ασχοληθούμε}
 Συνάρτησεις $f(x)=a^x$ και εξισώσεις με άγνωστους εκθέτες!!!!
\end{frame}

\begin{frame}{Πάμε!}
 \begin{block}{Ορισμός}
  Εκθετική συνάρτηση ονομάζεται κάθε $f:\mathbb{R}\to\mathbb{R}$ με $f(x)=a^x$, $α>0$, $a\ne 1$
 \end{block}
\end{frame}

\begin{frame}{Παιχνίδι Geogebra}
 Ας μελετήσουμε την εκθετική συνάρτηση $f(x)=a^x$ για κάθε $a$ \href{https://www.geogebra.org/m/tuzgusfm}{\beamergotobutton{Geogebra}}
 \begin{itemize}
  \item<1-> Για ποιά $a$ ορίζεται ως εκθετική?
  \item<2-> Τι γίνεται με την μονοτονία
  \item<3-> Τι γίνεται με τα ακρότατα
  \item<4-> Τι γίνεται με τα $-\infty$ και $+\infty$
  \item<5-> Υπάρχει ένα σταθερό σημείο για όλες?
  \item<6-> Σύνολο τιμών?
 \end{itemize}
\end{frame}

\begin{frame}{Ταξίδι στο μέλλον}
 \begin{enumerate}
  \item<1-> Μια γνησίως μονότονη συνάρτηση πιάνει οποιαδήποτε πραγματική τιμή, το πολύ μία φορά!
  \item<2-> Αν $x_1\ne x_2$ τότε $f(x_1)\ne f(x_2)$
  \item<3-> Αν $f(x_1)=f(x_2)$ τότε $x_1=x_2$
 \end{enumerate}
 \onslide<4-> Μια τέτοια συνάρτηση θα ονομάζεται \emph{ένα προς ένα}.
\end{frame}

\begin{frame}{Άρα για εξισώσεις!}
 \begin{itemize}
  \item<1-> Αν μπορούμε να έχουμε $a^x=a^y$ τότε $x=y$
  \item<2-> Αν δεν μπορούμε, ίσως δεν μπορούμε να λύσουμε άμεσα ως προς $x$. Ίσως
   \begin{itemize}
    \item<3-> Θέτουμε
    \item<4-> Μετασχηματίζουμε
    \item<4-> ...
   \end{itemize}
 \end{itemize}
\end{frame}

\section{Ασκήσεις}
\subsection{Άσκηση 1}
\begin{frame}[label=Άσκηση1,t]{Εξάσκηση 1}
 Να απλοποιήσετε τις παραστάσεις
 \begin{enumerate}
  \item<1-> $\dfrac{2}{\sqrt[3]{4}}$
  \item<2-> $\dfrac{x}{\sqrt[4]{x^3}}$, $x>0$
 \end{enumerate}

 % \hyperlink{Λύση1}{\beamerbutton{Λύση}}
\end{frame}

\subsection{Άσκηση 2}
\begin{frame}[label=Άσκηση2,t]{Εξάσκηση 2}
 Να παραστήσετε γραφικά τις συναρτήσεις:
 \begin{enumerate}
  \item<1-> $f(x)=e^x+1$
  \item<2-> $f(x)=e^{x-1}$
  \item<3-> $f(x)=e^{|x|}$
 \end{enumerate}

 % \hyperlink{Λύση2}{\beamerbutton{Λύση}}
\end{frame}

\subsection{Άσκηση 3}
\begin{frame}[label=Άσκηση3,t]{Εξάσκηση 3}
 Έστω η συνάρτηση $f(x)=(a-1)^x$. Να βρείτε τις τιμές του $α$ για τις οποίες η συνάρτηση $f$:
 \begin{enumerate}
  \item<1-> ορίζεται σε όλο το $\mathbb{R}$
  \item<2-> είναι γνησίως αύξουσα στο $\mathbb{R}$
  \item<3-> είναι γνησίως φθίνουσα στο $\mathbb{R}$
 \end{enumerate}

 % \hyperlink{Λύση3}{\beamerbutton{Λύση}}
\end{frame}

\subsection{Άσκηση 4}
\begin{frame}[label=Άσκηση4,t]{Εξάσκηση 4}
 Έστω η συνάρτηση $f(x)=e^x+x-1$.
 \begin{enumerate}
  \item<1-> Να εξετάσετε τη συνάρτηση $f$ ως προς τη μονοτονία
  \item<2-> Να βρείτε τη τιμή $f(0)$ και στη συνέχεια να λύσετε την εξίσωση $f(x)=0$
  \item<3-> Να λύσετε την ανίσωση $e^x+x<1$
  \item<4-> Αν $α<β$, να δείξετε ότι $e^α-e^β<β-α$
 \end{enumerate}

 % \hyperlink{Λύση4}{\beamerbutton{Λύση}}
\end{frame}

\subsection{Άσκηση 5}
\begin{frame}[label=Άσκηση5,t]{Εξάσκηση 5}
 Να κάνετε το πίνακα προσήμων της συνάρτησης $f(x)=e^x-1+\dfrac{x}{x^2+1}$

 % \hyperlink{Λύση5}{\beamerbutton{Λύση}}
\end{frame}

\subsection{Άσκηση 6}
\begin{frame}[label=Άσκηση6,t]{Εξάσκηση 6}
 Δίνονται οι συναρτήσεις $f(x)=x^{\frac{2}{3}}$ και $g(x)=\sqrt[3]{x^2}$
 \begin{enumerate}
  \item<1-> Να βρείτε το πεδίο ορισμού των συναρτήσεων
  \item<2-> Να βρείτε το ευρύτερο δυνατό υποσύνολο του $\mathbb{R}$ στο οποίο ισχύει $f(x)=g(x)$
  \item<3-> Να γράψετε τη συνάρτηση $g$ σε μορφή δύναμης
  \item<4-> Έστω η συνάρτηση $h(x)=\dfrac{g(x)}{x}$, $x\ne 0$. Να δείξετε ότι:
   $$h(x)=\begin{cases}
     \dfrac{1}{\sqrt[3]{x}},   & x>0 \\
     -\dfrac{1}{\sqrt[3]{-x}}, & x<0
    \end{cases}$$
 \end{enumerate}

 % \hyperlink{Λύση6}{\beamerbutton{Λύση}}
\end{frame}

\subsection{Άσκηση 7}
\begin{frame}[label=Άσκηση7,t]{Εξάσκηση 7}
 Να λύσετε τις εξισώσεις
 \begin{enumerate}
  \item<1-> $3^x=9$
  \item<2-> $2^{x-1}=\dfrac{1}{8}$
  \item<3-> $3^x=\sqrt{3}$
  \item<4-> $\left( \dfrac{3}{2} \right)^{2x-1}=\dfrac{8}{27}$
 \end{enumerate}

 %\hyperlink{Λύση7}{\beamerbutton{Λύση}}
\end{frame}

\subsection{Άσκηση 8}
\begin{frame}[label=Άσκηση8,t]{Εξάσκηση 8}

 Να λύσετε τις εξισώσεις
 \begin{enumerate}
  \item<1-> $e^x-e=0$
  \item<2-> $e^{3x-2}=1$
  \item<3-> $e^{-x}-\sqrt{e}=0$
  \item<4-> $e^x-e^{-x}=0$
  \item<5-> $e^x+1=0$
 \end{enumerate}

 %\hyperlink{Λύση8}{\beamerbutton{Λύση}}
\end{frame}

\subsection{Άσκηση 9}
\begin{frame}[label=Άσκηση9,t]{Εξάσκηση 9}
 Να λύσετε τις εξισώσεις
 \begin{enumerate}
  \item<1-> $2^{x+1}+4\cdot 2^{x-1}=4$
  \item<2-> $9^x+3^{x+1}-4=0$
 \end{enumerate}

 %\hyperlink{Λύση9}{\beamerbutton{Λύση}}
\end{frame}

\subsection{Άσκηση 10}
\begin{frame}[label=Άσκηση10,t]{Εξάσκηση 10}
 Να λύσετε τις εξισώσεις
 \begin{enumerate}
  \item<1-> $e^x-2\cdot e^{-x}+1=0$
  \item<2-> $3\cdot 2^x-2\cdot 3^x=0$
 \end{enumerate}

 %\hyperlink{Λύση10}{\beamerbutton{Λύση}}
\end{frame}

\subsection{Άσκηση 11}
\begin{frame}[label=Άσκηση11,t]{Εξάσκηση 11}
 Να λύσετε τις ανισώσεις
 \begin{enumerate}
  \item<1-> $3^x<9$
  \item<2-> $\left( \dfrac{2}{3} \right)^x > \dfrac{8}{27} $
  \item<3-> $e^x-1<0$
 \end{enumerate}

 %\hyperlink{Λύση11}{\beamerbutton{Λύση}}
\end{frame}

\subsection{Άσκηση 12}
\begin{frame}[label=Άσκηση12,t]{Εξάσκηση 12}
 Να λύσετε την ανίσωση $5^x+5^{1-x}<6$

 %\hyperlink{Λύση12}{\beamerbutton{Λύση}}
\end{frame}

\subsection{Άσκηση 13}
\begin{frame}[label=Άσκηση13,t]{Εξάσκηση 13}

 Να αποδείξετε ότι:
 \begin{enumerate}
  \item<1-> $e^x+e^{-x}-2\ge 0$, για κάθε $x\in \mathbb{R}$
  \item<2-> $e^{x^2}-1\ge 0$ για κάθε $x\in\mathbb{R}$
  \item<3-> $e^x-1>0$ για κάθε $x>0$
  \item<4-> $e^{-x}-1<0$ για κάθε $x>0$
 \end{enumerate}

 %\hyperlink{Λύση13}{\beamerbutton{Λύση}}
\end{frame}

\subsection{Άσκηση 14}
\begin{frame}[label=Άσκηση14,t]{Εξάσκηση 14}
 Να κάνετε τον πίνακα προσήμων των συναρτήσεων
 \begin{enumerate}
  \item<1-> $f(x)=e-e^x$
  \item<2-> $f(x)=\dfrac{e^x-e^2}{x-1}$
 \end{enumerate}

 %\hyperlink{Λύση14}{\beamerbutton{Λύση}}
\end{frame}

\subsection{Άσκηση 15}
\begin{frame}[label=Άσκηση15,t]{Εξάσκηση 15}
 Να λύσετε τα συστήματα
 \begin{enumerate}
  \item<1-> $\begin{cases}
     4^{x-1}\cdot 2^{y-2}=8 \\
     3^{x-1}\cdot 3^{y-3}=1
    \end{cases}$
  \item<2-> $\begin{cases}
     3^{x}- 5^{y}=4 \\
     9\cdot 3^{-x}+ 5^{y}=6
    \end{cases}$
 \end{enumerate}

 %\hyperlink{Λύση15}{\beamerbutton{Λύση}}
\end{frame}

\subsection{Άσκηση 16}
\begin{frame}[label=Άσκηση16,t]{Εξάσκηση 16}
 Να λύσετε τις εξισώσεις
 \begin{enumerate}
  \item<1-> $2^x+\sqrt{2^{x+4}}-5=0$
  \item<2-> $2\cdot 5^{x-2}+2^x-12\cdot 5^{x-3}-3\cdot 2^{x-3}=0$
 \end{enumerate}

 %\hyperlink{Λύση16}{\beamerbutton{Λύση}}
\end{frame}

\subsection{Άσκηση 17}
\begin{frame}[label=Άσκηση17,t]{Εξάσκηση 17}
  Να λύσετε την ανίσωση $8^x+4^x-2<0$

  %\hyperlink{Λύση17}{\beamerbutton{Λύση}}
\end{frame}

\subsection{Άσκηση 17}
\begin{frame}[label=Άσκηση17,t]{Εξάσκηση 17}
  \begin{enumerate}
    \item<1-> Να αποδείξετε ότι $e^x>ημx$ για κάθε $x>0$
    \item<2-> Να λύσετε την εξίσωση $e^x=συνx$ στο διάστημα $[0,+\infty)$
  \end{enumerate}

  %\hyperlink{Λύση17}{\beamerbutton{Λύση}}
\end{frame}

\subsection{Άσκηση 18}
\begin{frame}[label=Άσκηση18,t]{Εξάσκηση 18}
  Αν η ημιζωή ενός ραδιενεργού υλικού είναι $t_0$ χρόνια, να δείξετε ότι η συνάρτηση που εκφράζει την εκθετική απόσβεση αυτού είναι $Q(t)=Q_0\cdot 2^{-\frac{t}{t_0}}$

  %\hyperlink{Λύση18}{\beamerbutton{Λύση}}
\end{frame}

\subsection{Άσκηση 19}
\begin{frame}[label=Άσκηση19,t]{Εξάσκηση 19}
  Αν η ημιζωή ενός ραδιενεργού υλικού είναι 10 χρόνια και η αρχική ποσότητα είναι $20$ γραμμάρια, τότε:
  \begin{enumerate}
    \item<1-> Να βρείτε τη συνάρτηση που εκφράζει την εκθετική απόσβεση αυτού
    \item<2-> Να υπολογίσετε την ποσότητ που θα έχει απομείνει μετά από $20$ χρόνια
    \item<3-> Να βρείτε μετά από πόσα χρόνια θα έχουν απομείνει $\dfrac{5}{256}$ γραμμάρια του ραδιενεργού υλικού
  \end{enumerate}

  %\hyperlink{Λύση19}{\beamerbutton{Λύση}}
\end{frame}


% \appendix
%
% \section{Αποδείξεις}
% \begin{frame}[label=Απόδειξη1]{Απόδειξη μονοτονίας συνάρτησης}
%  Θα δείξουμε ότι για κάθε $x_1<x_2\in Δ \implies f(x_1)<f(x_2)$.
%
%  \onslide<1-> Στο $[x_1,x_2]$ είναι παραγωγίσιμη, άρα θα ισχύει το ΘΜΤ
%
%  \onslide<2-> Υπάρχει $ξ\in Δ$ ώστε $f'(ξ)=\dfrac{f(x_2)-f(x_1)}{x_2-x_1}$.
%
%  \onslide<3-> Αλλά $f'(x)>0$ για κάθε $x\in Δ$
%
%  \onslide<4-> Άρα $f'(ξ)=\dfrac{f(x_2)-f(x_1)}{x_2-x_1}>0$
%
%  \onslide<5-> $f(x_2)>f(x_1)$
%
%  \hyperlink{Θεώρημα1}{\beamerbutton{Πίσω στη θεωρία}}
% \end{frame}


% \section{Λύσεις Ασκήσεων}
% \begin{frame}
%  \tableofcontents
% \end{frame}
%
% \subsection{Άσκηση 1}
% \begin{frame}[label=Λύση1]
%  Με θεώρημα ενδιαμέσων τιμών. Η συνάρτηση είναι συνεχής στο $[10,11]$ με $f(10)=1024$ και $f(11)=2048$. Αφού $2023\in (1024,2048)$ υπάρχει $x_0$...
%
%  \hyperlink{Άσκηση1}{\beamerbutton{Πίσω στην άσκηση}}
% \end{frame}
%
% \subsection{Άσκηση 2}
% \begin{frame}[label=Λύση2]
%  Με Bolzano ή με μέγιστης ελάχιστης τιμής και ΘΕΤ.
%
%  \begin{gather*}
%   f(3)<f(2)<f(1) \\
%   3f(3)<f(1)+f(2)+f(3)<3f(1) \\
%   f(3)<\dfrac{f(1)+f(2)+f(3)}{3}<f(1)
%  \end{gather*}
%
%  \hyperlink{Άσκηση2}{\beamerbutton{Πίσω στην άσκηση}}
% \end{frame}
%
% \subsection{Άσκηση 3}
% \begin{frame}[label=Λύση3]
%  Προφανές ελάχιστο στα $x_1=1$ και $x_2=3$. Ως συνεχής στο $[1,3]$ έχει σίγουρα ΚΑΙ μέγιστο στο $(1,3)$
%
%  \hyperlink{Άσκηση3}{\beamerbutton{Πίσω στην άσκηση}}
% \end{frame}
%
% \subsection{Άσκηση 4}
% \begin{frame}[label=Λύση4]
%  Η συνάρτηση `απόστασης` $f(x)-x$ είναι ορισμένη στο κλειστό διάστημα και έχει σίγουρα μέγιστο
%
%  \hyperlink{Άσκηση4}{\beamerbutton{Πίσω στην άσκηση}}
% \end{frame}
%
% \subsection{Άσκηση 5}
% \begin{frame}[label=Λύση5]
%  Όμοια με την Άσκηση 2
%
%  \hyperlink{Άσκηση5}{\beamerbutton{Πίσω στην άσκηση}}
% \end{frame}
%
% \subsection{Άσκηση 6}
% \begin{frame}[label=Λύση6]
%  \begin{enumerate}
%   \item Είναι γνησίως αύξουσα άρα $(f(+\infty),f(-\infty))$
%   \item Προφανώς $[f(0),f(1)]$...
%  \end{enumerate}
%
%  \hyperlink{Άσκηση6}{\beamerbutton{Πίσω στην άσκηση}}
% \end{frame}

\end{document}
