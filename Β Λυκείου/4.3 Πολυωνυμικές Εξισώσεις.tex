\documentclass{../presentation}

\title{Πολυώνυμα}
\subtitle{Πολυωνυμικές Εξισώσεις}
\author[Λόλας]{Κωνσταντίνος Λόλας}
\date{}

\begin{document}

\begin{frame}
  \titlepage
\end{frame}

\section{Θεωρία}
\begin{frame}{Ότι μάθατε μάθατε, τώρα παιχνίδι!}
  Θα ασχοληθούμε με εξισώσεις!
  \begin{itemize}
    \item<1-> 1ου βαθμού
    \item<2-> 2ου βαθμού
    \item<3-> $ν$ βαθμού με $ν\ge 3$
  \end{itemize}
  \only<4>{χωρίς να μάθουμε τίποτα καινούριο!!!}
\end{frame}

\begin{frame}{Τα άχρηστα που έχουμε μάθει?}
  Κάποτε κάποιος σας είπε ότι η παραγοντοποίηση είναι σημαντική!
  \begin{itemize}
    \item<1-> για 1ου βαθμού? \only<2> {δεν χρειάζεται!}
    \item<2-> για 2ου βαθμού? \only<3> {μόνο όταν μπορούμε!}
    \item<3-> $ν$ βαθμού με $ν\ge 3$? \only<4> {ίσως ο μοναδικός τρόπος}
  \end{itemize}
\end{frame}

\begin{frame}{1ου βαθμού}
  \begin{tikzpicture}
    [level distance=2cm,
      level 1/.style={sibling distance=5cm},
      level 2/.style={sibling distance=5cm},
      edge from parent/.style={->,draw}]

    \node[circle, draw] {$αx+β=0$}
    child {
        node [rectangle,draw] {$x=-\dfrac{β}{α}$, Μοναδική}
        edge from parent
        node[left] {$α\ne 0$}
      }
    child{
        node {$0=β$}
        child {
            node [rectangle,draw] {Ταυτότητα, Άπειρες}
            edge from parent
            node[left] {$β=0$}
          }
        child {
            node [rectangle,draw] {Αδύνατη, Καμία}
            edge from parent
            node[right] {$β\ne 0$}
          }
        edge from parent
        node[right] {$α=0$}
      };
  \end{tikzpicture}
\end{frame}

\begin{frame}{2ου βαθμού}
  \centering
  \begin{tikzpicture}
    [level distance=3cm,
      edge from parent/.style={->,draw}]

    \node[rectangle, draw]{$αx^2+βx+γ=0$}
    child[grow=down] {
        node {$αx^2+γ=0$}
        edge from parent
        node[left] {$β= 0$}
      }
    child[grow=south east] {
        node{$αx^2+βx=0$}
        edge from parent
        node[right] {$γ=0$}
      }
    child[grow=east] {
        node{Ταυτότητα}
      }
    child[grow=north east] {
        node{Vietta}
      }
    child[grow=north] {
        node{Διακρίνουσα}
      }
    child[grow=north west] {
        node{Ομαδοποίηση}
      }
    ;
  \end{tikzpicture}
\end{frame}

\begin{frame}{3ου+ ΟΕΟ?}
  Θα ήταν υπέροχο αν μπορούσαμε να το παραγοντοποιήσουμε...
  \begin{itemize}
    \item<1-> Σε 1 πρώτου και 1 δευτέρου?
    \item<2-> Σε 3 πρώτου βαθμού?
  \end{itemize}
\end{frame}

\begin{frame}{4ου ΟΕΟ?}
  Αν μπορούσαμε να το παραγοντοποιήσουμε...
  \begin{itemize}
    \item<1-> Σε 1 πρώτου και 1 τρίτου...?
    \item<2-> Σε 2 πρώτου βαθμού και 1 δευτέρου?
    \item<3-> Σε 2 δευτέρου?
  \end{itemize}
\end{frame}

\begin{frame}{Δηλαδή}
  Μακάρι να πετυχαίναμε ρίζες (και μάλιστα ακέραιες)!
  \begin{block}{Ακέραιες Ρίζες}<2->
    Αν ένα πολυώνυμο με ακέραιους συντελεστές έχει ακέραια ρίζα, τότε η ρίζα αυτή διαιρεί την σταθερά του πολυωνύμου
  \end{block}
  \only<3>
  {$$P(ρ)=0$$ $$α_νρ^ν+α_{ν-1}ρ^{ν-1}+\cdots+α_1ρ+α_0=0$$ $$a_0=-α_νρ^ν-α_{ν-1}ρ^{ν-1}-\cdots-α_1ρ $$ $$a_0=ρ(-α_νρ^{ν-1}-α_{ν-1}ρ^{ν-2}-\cdots-α_1)$$}

\end{frame}

\begin{frame}{Γενικά!}

  \begin{tikzpicture}[node distance=2cm]
    \node (start) [startstop] {\shortstack{Πολυωνυμική\\ Εξίσωση}};
    \node (dec1) [decision, right of=start, xshift=2.2cm] {\tiny Παραγοντοποίηση?};
    \node (pro1) [process, below of=dec1, yshift=-0.5cm] {$P(x)Q(x)=0$};
    \node (dec2) [decision, right of=dec1, xshift=2cm] {Ακέραιες?};
    \node (pro2) [process, above of=dec2, yshift=0.7cm] {Βρίσκω
      $P(x):(x-ρ)$};
    \node (stop) [startstop, below of=dec2, yshift=-0.5cm] {Δεν λύνεται!};

    \draw [arrow] (start) -- (dec1);
    \draw [arrow] (dec1) -- node[anchor=east] {ναι} (pro1);
    \draw [arrow] (dec1) -- node[anchor=north] {όχι} (dec2);
    \draw [arrow] (pro1) -| (start);
    \draw [arrow] (dec2) -- node[anchor=east] {ναι} (pro2);
    \draw [arrow] (pro2) -| (start);
    \draw [arrow] (dec2) -- node[anchor=east] {όχι} (stop);
  \end{tikzpicture}

  Για κάθε, ΜΑ ΚΑΘΕ πολυώνυμο!
\end{frame}

\section{Ασκήσεις}
\begin{askisi}
  Να λύσετε τις εξισώσεις
  \begin{enumerate}
    \item<1-> $x^4=8x$
    \item<2-> $x^3-2x^2-9x+18=0$
  \end{enumerate}

  % \hyperlink{Λύση1}{\beamerbutton{Λύση}}

\end{askisi}

\begin{askisi}
  Να βρείτε τις ακέραιες ρίζες της εξίσωσης $x^3-5x+2=0$

  % \hyperlink{Λύση2}{\beamerbutton{Λύση}}

\end{askisi}

\begin{askisi}
  Να δείξετε ότι η εξίσωση $x^3+3x+2=0$ δεν έχει ακέραιες ρίζες.

  % \hyperlink{Λύση3}{\beamerbutton{Λύση}}

\end{askisi}

\begin{askisi}
  Να λύσετε την εξίσωση $x^3-3x+2=0$

  % \hyperlink{Λύση4}{\beamerbutton{Λύση}}

\end{askisi}

\begin{askisi}
  Να λύσετε την εξίσωση $2x^4-3x^3-17x^2+27x-9=0$

  % \hyperlink{Λύση5}{\beamerbutton{Λύση}}

\end{askisi}

\begin{askisi}
  Να λύσετε την εξίσωση $x^6-7x^2-6=0$

  % \hyperlink{Λύση6}{\beamerbutton{Λύση}}

\end{askisi}

\begin{askisi}
  Να βρείτε τα κοινά σημεία της γραφικής παράστασης της συνάρτησης $f(x)=-x^3-x+2$ με τον άξονα $x'x$

  %\hyperlink{Λύση7}{\beamerbutton{Λύση}}

\end{askisi}

\begin{askisi}
  Να βρείτε τα κοινά σημεία των γραφικών παραστάσεων των συναρτήσεων $$f(x)=x^3+9$$ και $$g(x)=5x^2-3x$$

  %\hyperlink{Λύση8}{\beamerbutton{Λύση}}

\end{askisi}

\begin{askisi}
  Αν η εξίσωση $x^3-(λ+2)x^2+2λx-1=0$, $λ\in\mathbb{Z}$ έχει ακέραια ρίζα, να βρείτε το $λ$ και μετά να λύσετε την εξίσωση.

  %\hyperlink{Λύση9}{\beamerbutton{Λύση}}

\end{askisi}

\begin{askisi}
  Να λύσετε την εξίσωση $(3x+1)^8-15(3x+1)^4-16=0$

  %\hyperlink{Λύση10}{\beamerbutton{Λύση}}

\end{askisi}

\begin{askisi}
  Να λύσετε την εξίσωση $(x^2-x-1)^2-6(x^2-x-3)-7=0$

  %\hyperlink{Λύση11}{\beamerbutton{Λύση}}

\end{askisi}

\begin{askisi}
  Να λύσετε την εξίσωση $6x^4+5x^3-38x^2+5x+6=0$

  %\hyperlink{Λύση12}{\beamerbutton{Λύση}}

\end{askisi}


% \appendix
%
% \section{Αποδείξεις}
% \begin{frame}[label=Απόδειξη1]{Απόδειξη μονοτονίας συνάρτησης}
%  Θα δείξουμε ότι για κάθε $x_1<x_2\in Δ \implies f(x_1)<f(x_2)$.
%
%  \onslide<1-> Στο $[x_1,x_2]$ είναι παραγωγίσιμη, άρα θα ισχύει το ΘΜΤ
%
%  \onslide<2-> Υπάρχει $ξ\in Δ$ ώστε $f'(ξ)=\dfrac{f(x_2)-f(x_1)}{x_2-x_1}$.
%
%  \onslide<3-> Αλλά $f'(x)>0$ για κάθε $x\in Δ$
%
%  \onslide<4-> Άρα $f'(ξ)=\dfrac{f(x_2)-f(x_1)}{x_2-x_1}>0$
%
%  \onslide<5-> $f(x_2)>f(x_1)$
%
%  \hyperlink{Θεώρημα1}{\beamerbutton{Πίσω στη θεωρία}}
% \end{frame}


% \section{Λύσεις Ασκήσεων}
% \begin{frame}
%  \tableofcontents
% \end{frame}
%
% \subsection{Άσκηση 1}
% \begin{frame}[label=Λύση1]
%  Με θεώρημα ενδιαμέσων τιμών. Η συνάρτηση είναι συνεχής στο $[10,11]$ με $f(10)=1024$ και $f(11)=2048$. Αφού $2023\in (1024,2048)$ υπάρχει $x_0$...
%
%  \hyperlink{Άσκηση1}{\beamerbutton{Πίσω στην άσκηση}}
% \end{frame}
%
% \subsection{Άσκηση 2}
% \begin{frame}[label=Λύση2]
%  Με Bolzano ή με μέγιστης ελάχιστης τιμής και ΘΕΤ.
%
%  \begin{gather*}
%   f(3)<f(2)<f(1) \\
%   3f(3)<f(1)+f(2)+f(3)<3f(1) \\
%   f(3)<\dfrac{f(1)+f(2)+f(3)}{3}<f(1)
%  \end{gather*}
%
%  \hyperlink{Άσκηση2}{\beamerbutton{Πίσω στην άσκηση}}
% \end{frame}
%
% \subsection{Άσκηση 3}
% \begin{frame}[label=Λύση3]
%  Προφανές ελάχιστο στα $x_1=1$ και $x_2=3$. Ως συνεχής στο $[1,3]$ έχει σίγουρα ΚΑΙ μέγιστο στο $(1,3)$
%
%  \hyperlink{Άσκηση3}{\beamerbutton{Πίσω στην άσκηση}}
% \end{frame}
%
% \subsection{Άσκηση 4}
% \begin{frame}[label=Λύση4]
%  Η συνάρτηση `απόστασης` $f(x)-x$ είναι ορισμένη στο κλειστό διάστημα και έχει σίγουρα μέγιστο
%
%  \hyperlink{Άσκηση4}{\beamerbutton{Πίσω στην άσκηση}}
% \end{frame}
%
% \subsection{Άσκηση 5}
% \begin{frame}[label=Λύση5]
%  Όμοια με την Άσκηση 2
%
%  \hyperlink{Άσκηση5}{\beamerbutton{Πίσω στην άσκηση}}
% \end{frame}
%
% \subsection{Άσκηση 6}
% \begin{frame}[label=Λύση6]
%  \begin{enumerate}
%   \item Είναι γνησίως αύξουσα άρα $(f(+\infty),f(-\infty))$
%   \item Προφανώς $[f(0),f(1)]$...
%  \end{enumerate}
%
%  \hyperlink{Άσκηση6}{\beamerbutton{Πίσω στην άσκηση}}
% \end{frame}

\end{document}
