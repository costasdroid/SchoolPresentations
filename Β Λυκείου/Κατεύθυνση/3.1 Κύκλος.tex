\documentclass[greek]{beamer}
\usepackage{amsmath,amsthm} % needed for mathematics
\usepackage{unicode-math}
\usepackage{xltxtra}
\usepackage{graphicx}
\usetheme{CambridgeUS}
\usecolortheme{seagull}
\usepackage{hyperref}
\usepackage{ulem} % underline words package
\usepackage{xgreek}

\usepackage{pgfpages} 
\usepackage{tikz} % package for shapes and more
%\setbeameroption{show notes on second screen}
%\setbeameroption{show only notes}

\setsansfont{Calibri} % it is said that Calibri is the proper font for reading difficulties

\usepackage{multicol} % package for two or more columns

\usepackage{appendixnumberbeamer} % remove page numbering in appendix

\usepackage{polynom} % polynomial divisions package

\usepackage{pgffor} % macros

\setbeamercovered{highly dynamic}
\setbeamertemplate{navigation symbols}{}

\newcounter{askisi} % enviroment for exercises
\newenvironment{askisi}
{
 \refstepcounter{askisi}\par
 \subsection{Άσκηση \theaskisi}
 \begin{frame}[label=Άσκηση\theaskisi,t]{Εξάσκηση \theaskisi}
}{
 \end{frame}
}
 
\newcounter{lisi} % enviroment for solutions
\newenvironment{lisi}
{
 \refstepcounter{lisi}\par
 \subsection{Άσκηση \thelisi}
 \begin{frame}[label=Λύση\thelisi,t]{Λύση \thelisi}
}{
 \end{frame} 
}

\title{Κύκλος}
\subtitle{Εξίσωση Κύκλου}
\author[Λόλας]{Κωνσταντίνος Λόλας}
\date{}

\begin{document}

\begin{frame}
  \titlepage
\end{frame}

\section{Θεωρία}
\begin{frame}{Κωνικές Τομές}
  Ώρα για παιχνίδι!
  \href{https://www.geogebra.org/m/pCg8NFVT}{Geogebra}
\end{frame}

\begin{frame}{Όλες? Κανονικά ναι, αλλά!}
  Θα ασχοληθούμε με
  \begin{itemize}
    \item Κύκλο
    \item Έλλειψη
    \item Παραβολή
    \item Υπερβολή
  \end{itemize}
\end{frame}

\begin{frame}{Ναι, αλλά τι?}
  Για κάθε κωνική τομή, θα...
  \begin{itemize}
    \item<1-> ορίσουμε τον γεωμετρικό τόπο
    \item<2-> ονομάσουμε στοιχεία
    \item<3-> βρίσκουμε στοιχεία τους
    \item<4-> ψάξουμε ιδιότητες
    \item<5-> την αναγνωρίζουμε από γενική εξίσωση
  \end{itemize}
\end{frame}

\begin{frame}{Κύκλος ολέ!}
  \begin{block}{Ορισμός}
    Κύκλος ονομάζεται ο γεωμετρικός τόπος των σημείων του επιπέδου, που ισαπέχουν από ένα σταθερό σημείο. Το σημείο ονομάζεται κέντρο του κύκλου και η απόσταση, ακτίνα.
  \end{block}
\end{frame}

\begin{frame}{Πάμε!}
  \onslide<1-> Έστω $A(x,y)$ το οποιοδήποτε σημείο του κύκλου, $ρ$ η ακτίνα και $Κ(x_0,y_0)$ το κέντρο του

  \onslide<2-> Θα πρέπει $|ΟΑ|=ρ$
  \begin{align*}
    \sqrt{(x-x_0)^2+(y-y_0)^2}=ρ \\
    (x-x_0)^2+(y-y_0)^2=ρ^2
  \end{align*}

  \onslide<3-> Τι, αυτό ήταν?

\end{frame}

\begin{frame}{Ο Μοναδιαίος!}
  Κέντρο $Κ(0,0)$ και ακτίνα $ρ=1$, άρα...
  $$(x-0)^2+(y-0)^2=1^2$$
  $$x^2+y^2=1$$
\end{frame}

\begin{frame}{Τι έμεινε?}
  \begin{itemize}
    \item ιδιότητες
    \item εφαπτομένη
    \item γενική εξίσωση
  \end{itemize}
\end{frame}

\begin{frame}{Ιδιότητες κύκλου}
  \begin{itemize}
    \item<1-> Η εφαπτομένη του σε ένα σημείο, είναι κάθετη στην \emph{εκεί} ακτίνα
    \item<2-> Το κέντρο του κύκλου που ορίζεται από $3$ σημεία είναι το σημείο τομής των μεσοκαθέτων
    \item<3-> Το απόστημα είναι μεσοκάθετο της χορδής
    \item<4-> Το κέντρο απέχει από την εφαπτόμενη όσο η ακτίνα
    \item<5-> Ένα σημείο είναι έξω, μέσα ή ανήκει όταν...
    \item<6-> Δύο κύκλοι μεταξύ τους...
    \item<7-> Κύκλος και ευθεία...
    \item<8-> ...
  \end{itemize}

\end{frame}

\begin{frame}[label=Εφαπτόμενη]{Εφαπτόμενη}
  Αν την κατασκευάζετε δεν χρειάζεστε τύπο!

  Μόνο για τον μοναδιαίο κύκλο, η εφαπτόμενη από το σημείο του $(x_1,y_1)$ είναι η
  $$x\cdot x_1+y\cdot y_1=1$$
  Ή στη γενική περίπτωση (εκτός ύλης σαν τύπος)
  $$(x-x_0)(x_1-x_0)+(y-y_0)(y_1-y_0)=ρ^2$$

  \hyperlink{Απόδειξη}{\beamerbutton{Απόδειξη}}
\end{frame}

\begin{frame}[label=ΓενικήΕξίσωση]{Όχι πάντα στην έτοιμη μορφή}

  \begin{block}{Γενική εξίσωση κύκλου}
    Κάθε εξίσωση
    $$x^2+y^2+Αx+Βy+Γ=0\text{ με }Α^2+Β^2-4Γ>0$$
    παριστάνει κύκλο και κάθε κύκλος έχει εξίσωση αυτής της μορφής.

    Το κέντρο είναι το $Κ=(-\dfrac{Α}{2},-\dfrac{Β}{2})$ και η ακτίνα $ρ=\dfrac{\sqrt{Α^2+Β^2-4Γ}}{2}$
  \end{block}

  \hyperlink{Απόδειξη2}{\beamerbutton{Απόδειξη $\rightarrow$}}

  \hyperlink{Απόδειξη3}{\beamerbutton{Απόδειξη $\leftarrow$}}
\end{frame}

\begin{frame}[noframenumbering]
  Στο moodle θα βρείτε τις ασκήσεις που πρέπει να κάνετε, όπως και αυτή τη παρουσίαση
\end{frame}

\section{Ασκήσεις}

\begin{frame}[noframenumbering]
  \vfill
  \centering
  \begin{beamercolorbox}[sep=8pt,center,shadow=true,rounded=true]{title}
    \usebeamerfont{title}Ασκήσεις
  \end{beamercolorbox}
  \vfill
\end{frame}

\begin{askisi}
  Να βρείτε την εξίσωση του κύκλου $C$ που έχει κέντρο την αρχή των αξόνων και:
  \begin{enumerate}
    \item<1-> έχει ακτίνα $ρ=2$
    \item<2-> διέρχεται από το σημείο $Α(-3,4)$
    \item<3-> εφάπτεται της ευθείας $ε:3x+4y-10=0$
  \end{enumerate}

  % \hyperlink{Λύση1}{\beamerbutton{Λύση}}
\end{askisi}

\begin{askisi}
  Δίνεται ο κύκλος $C:x^2+y^2=25$. Να βρείτε:
  \begin{enumerate}
    \item<1-> την εξίσωση της χορδής του κύκλου $C$ που έχει μέσον το σημείο $Α(2,4)$
    \item<2-> το μήκος της παραπάνω χορδής
  \end{enumerate}

  % \hyperlink{Λύση2}{\beamerbutton{Λύση}}
\end{askisi}

\begin{askisi}
  Να βρείτε την εφαπτομένη του κύκλου $C:x^2+y^2=25$ στο σημείο του:
  \begin{enumerate}
    \item<1-> $Α(3,4)$
    \item<2-> $Β(-4,μ)$, $μ>0$
  \end{enumerate}

  % \hyperlink{Λύση3}{\beamerbutton{Λύση}}
\end{askisi}

\begin{askisi}
  Να βρείτε την εξίσωση του κύκλου που έχει:
  \begin{enumerate}
    \item<1-> κέντρο το σημείο $Κ(3,-2)$ και ακτίνα $ρ=\sqrt{7}$
    \item<2-> διάμετρο το τμήμα $ΑΒ$ με άκρα τα σημεία $Α(1,-2)$ και $Β(-5,6)$
  \end{enumerate}

  % \hyperlink{Λύση4}{\beamerbutton{Λύση}}
\end{askisi}

\begin{askisi}
  Να βρείτε την εξίσωση του κύκλου $C_1$ που είναι ομόκεντρος με τον κύκλο
  $$C_2:(x-3)^2+(y-2)^2=7$$
  και εφάπτεται της ευθείας $ε:4x-3y+2=0$

  % \hyperlink{Λύση5}{\beamerbutton{Λύση}}
\end{askisi}

\begin{askisi}
  Να βρείτε την εξίσωση του κύκλου που εφάπτεται της ευθείας $ε_1:4x+3y=12$ στο σημείο $Α(3,0)$ και το κέντρο του ανήκει στην ευθεία $ε_2:y=x-1$

  % \hyperlink{Λύση6}{\beamerbutton{Λύση}}
\end{askisi}

\begin{askisi}
  Δίνεται τρίγωνο $ΑΒΓ$ με $Α(-5,-1)$, $Β(1,-3)$ και $Γ(7,3)$. Να βρείτε την εξίσωση του περιγεγραμμένου κύκλου του τριγώνου $ΑΒΓ$.

  %\hyperlink{Λύση7}{\beamerbutton{Λύση}}
\end{askisi}

\begin{askisi}
  Να βρείτε ττο κέντρο και την ακτίνα του κύκλου
  $$x^2+y^2-2x+6y-6=0$$

  %\hyperlink{Λύση8}{\beamerbutton{Λύση}}
\end{askisi}

\begin{askisi}
  Δίνεται η εξίσωση
  $$(2x-1)^2+2y(2y+1)=0$$
  Να δείξετε ότι η εξίσωση παριστάνει κύκλο και μετά να βρείτε το κέντρο και την ακτίνα του

  %\hyperlink{Λύση9}{\beamerbutton{Λύση}}
\end{askisi}

\begin{askisi}
  Δίνεται η εξίσωση:
  $$x^2+y^2-2λx+4λy+5λ^2-λ+1=0,λ\in\mathbb{R}$$
  \begin{enumerate}
    \item<1-> Να βρείτε τις τιμές του $λ$, ώστε η εξίσωση να παριστάνει κύκλο
    \item<2-> Να δείξετε ότι τα κέντρα των κύκλων που ορίζονται από την εξίσωση, βρίσκονται σε ευθεία
  \end{enumerate}

  %\hyperlink{Λύση10}{\beamerbutton{Λύση}}
\end{askisi}

\begin{askisi}
  Δίνεται η εξίσωση:
  $$x^2+y^2-2xημθ-2yσυνθ+ημθ-3=0,θ\in\mathbb{R}$$
  \begin{enumerate}
    \item<1-> Να δείξετε ότι η εξίσωση παριστάνει κύκλο για κάθε $θ\in\mathbb{R}$
    \item<2-> Να δείξετε ότι τα κέντρα των κύκλων ανήκουν στο μοναδιαίο κύκλο
  \end{enumerate}

  %\hyperlink{Λύση11}{\beamerbutton{Λύση}}
\end{askisi}

\begin{askisi}
  Δίνονται οι κύκλοι
  $$C_1:(x-3)^2+(y-2)^2=5^2$$
  και
  $$C_2:(x+1)^2+y^2=3^2$$
  \begin{enumerate}
    \item<1-> Να βρείτε την εξίσωση της εφαπτομένης $ε$ του κύκλου $C_1$ στο σημείο του $Α(-1,5)$
    \item<2-> Να αποδείξετε ότι η $ε$ εφάπτεται και του κύκλου $C_2$
  \end{enumerate}

  %\hyperlink{Λύση12}{\beamerbutton{Λύση}}
\end{askisi}

\begin{askisi}
  Να βρείτε την εξίσωση της εφαπτομένης του κύκλου $C:(x-1)^2+y^2=2$, που είναι παράλληλη στην ευθεία $ζ:y=x+1$

  %\hyperlink{Λύση13}{\beamerbutton{Λύση}}
\end{askisi}

\begin{askisi}
  Να βρείτε τις εφαπτομένες του κύκλου $C:x^2+y^2-4y+3=0$ που διέρχονται από το σημείο $Α(-1,0)$

  %\hyperlink{Λύση14}{\beamerbutton{Λύση}}
\end{askisi}

\begin{askisi}
  Από το σημείο $Μ(4,3)$ φέρνουμε τις εφαπτόμενες στον κύκλο $C:x^2+y^2=2$. Αν $Α$, $Β$ είναι τα σημεία επαφής, να βρείτε:
  \begin{enumerate}
    \item<1-> την ευθεία $ΑΒ$
    \item<2-> την απόσταση της αρχής των αξόνων από την ευθεία $ΑΒ$
  \end{enumerate}

  %\hyperlink{Λύση15}{\beamerbutton{Λύση}}
\end{askisi}

\begin{askisi}
  Δίνονται οι κύκλοι
  $$C_1:x^2+y^2=4$$
  και
  $$C_2:(x-5)^2+y^2=25$$
  Να βρείτε τις κοινές εφαπτόμενες των κύκλων $C_1$ και $C_2$

  %\hyperlink{Λύση16}{\beamerbutton{Λύση}}
\end{askisi}

\begin{askisi}
  Να βρείτε τη σχετική θέση του κύκλου $C:x^2+y^2=4$ ως προς:
  \begin{enumerate}
    \item<1-> το σημείο $Α(1,3)$
    \item<2-> την ευθεία $ε:3x+4y-5=0$
    \item<3-> τον κύκλο $C_1:(x-3)^2+y^2=1$
  \end{enumerate}

  %\hyperlink{Λύση17}{\beamerbutton{Λύση}}
\end{askisi}

\begin{askisi}
  Δίνονται οι κύκλοι
  $$C_1:x^2+y^2-6x+8y=0$$
  και
  $$C_2:x^2+y^2-8x-6y+16=0$$
  \begin{enumerate}
    \item<1-> Να δείξετε ότι οι κύκλοι $C_1$ και $C_2$ τέμνονται
    \item<2-> Να βρείτε την κοινή χορδή των κύκλων
  \end{enumerate}

  %\hyperlink{Λύση18}{\beamerbutton{Λύση}}
\end{askisi}

\begin{askisi}
  Δίνονται οι κύκλοι
  $$C_1:x^2+y^2+2x+6y+1=0$$
  και
  $$C_2:x^2+y^2-4x-2y+1=0$$
  \begin{enumerate}
    \item<1-> Να δείξετε ότι οι κύκλοι $C_1$ και $C_2$ εφάπτονται εξωτερικά
    \item<2-> Να βρείτε το σημείο επαφής των δύο κύκλων
    \item<3-> Βρείτε την κοινή εσωτερική εφαπτόμενη των κύκλων
  \end{enumerate}

  %\hyperlink{Λύση19}{\beamerbutton{Λύση}}
\end{askisi}

\begin{askisi}
  Δίνεται η οικογένεια κύκλων
  $$C_λ:x^2+y^2-4λx+2λy-5=0,λ\in\mathbb{R}$$
  \begin{enumerate}
    \item<1-> Να δείξετε ότι όλοι οι κύκλοι που ορίζονται από την εξίσωση, διέρχονται από δύο σταθερά σημεία
    \item<2-> Να βρείτε την κοινή χορδή όλων των κύκλων που ορίζονται από την εξίσωση
  \end{enumerate}

  %\hyperlink{Λύση20}{\beamerbutton{Λύση}}
\end{askisi}

\begin{askisi}
  Δίνεται η εξίσωση $x^2+y^2-2λx-2λy+4λ-2=0,λ\in\mathbb{R}$
  \begin{enumerate}
    \item<1-> Να βρείτε τις τιμές του $λ$, για τις οποίες η εξίσωση παριστάνει κύκλο
    \item<2-> Να δείξετε ότι όλοι οι παραπάνω κύκλοι διέρχονται από ένα σταθερό σημείο, το οποίο και να βρείτε
  \end{enumerate}

  %\hyperlink{Λύση21}{\beamerbutton{Λύση}}
\end{askisi}

\begin{askisi}
  Να βρείτε την εξίσωση του κύκλου $C$, όταν ισχύουν:
  \begin{itemize}
    \item η ευθεία $ε:y=-2x$ τέμνει τον κύκλο στα σημεία $Α(3,1)$ και $Β$
    \item ο κύκλος $C$ διέρχεται από το σημείο $Γ(-1,0)$
    \item $\overrightarrow{ΓΑ}\cdot\overrightarrow{ΓΒ}=0$
  \end{itemize}

  %\hyperlink{Λύση22}{\beamerbutton{Λύση}}
\end{askisi}

\begin{askisi}
  Δίνεται ο κύκλος $C:(x-2)^2+(y-1)^2=9$. Να βρείτε την εξίσωση της χορδής του κύκλου που διέρχεται από το σημείο $Α(4,2)$ και έχει μήκος $2\sqrt{5}$

  %\hyperlink{Λύση23}{\beamerbutton{Λύση}}
\end{askisi}

\begin{askisi}
  Να δείξετε ότι οι εφαπτόμενες που φέρνουμε στον κύκλο $C:x^2+y^2=5$ από το σημείο $Α(1,3)$ είναι κάθετες

  %\hyperlink{Λύση24}{\beamerbutton{Λύση}}
\end{askisi}

\begin{askisi}
  Δίνονται τα σημεία $Α(2,0)$ και $Β(-2,4)$. Να βρείτε το γεωμετρικό τόπο των σημείων $Μ$, για τα οποία ισχύει:
  \begin{enumerate}
    \item<1-> $\overrightarrow{ΜΑ}\perp\overrightarrow{ΜΒ}$
    \item<2-> $\widehat{ΑΜΒ}=90^{\circ}$
  \end{enumerate}

  %\hyperlink{Λύση25}{\beamerbutton{Λύση}}
\end{askisi}

\begin{askisi}
  Να βρείτε το γεωμετρικό τόπο των σημείων $Μ$, των οποίων το τετράγωνο της απόστασης από το σημείο $Α(0,1)$ είναι ίσο με το διπλάσιο της απόστασης από την ευθεία $ε:y=\dfrac{3}{2}$

  %\hyperlink{Λύση26}{\beamerbutton{Λύση}}
\end{askisi}

\begin{askisi}
  Να βρείτε που κινείται το σημείο $Μ(3+2ημθ,1+2συνθ)$, $θ\in [0,2π)$

  %\hyperlink{Λύση27}{\beamerbutton{Λύση}}
\end{askisi}

\begin{askisi}
  Δίνεται ο κύκλος $C:(x-1)^2+y^2=4$. Αν το σημείο $Α$ κινείται στον κύκλο $C$ με κέντρο το $Κ$, να βρείτε που κινείται το σημείο $Μ$, για το οποίο ισχύει:
  $$\overrightarrow{ΜΑ}=3\overrightarrow{ΚΑ}$$

  %\hyperlink{Λύση28}{\beamerbutton{Λύση}}
\end{askisi}

\begin{askisi}
  Δίνεται ο κύκλος $C:(x-2)^2+(y-1)^2=9$
  \begin{enumerate}
    \item<1-> Να βρείτε τη μέγιστη απόσταση που μπορούν να απέχουν δύο σημεία του κύκλου $C$
    \item<2-> Να βρείτε τη σχετική θέση του σημείου $Α(1,2)$ ως προς τον κύκλο και μετά τη μέγιστη και την ελάχιστη απόσταση του σημείου $Α$ από ένα σημείο του κύκλου
    \item<3-> Να βρείτε τη σχετική θέση της ευθείας $ε:3x+4y+18=0$ ως προς τον κύκλο $C$ και μετά τη μέγιστη και την ελάχιστη απόσταση ενός σημείου του κύκλου $C$ από την ευθεία $ε$
  \end{enumerate}

  %\hyperlink{Λύση29}{\beamerbutton{Λύση}}
\end{askisi}

\begin{askisi}
  Δίνονται οι κύκλοι:
  $$C_1:(x-1)^2+(y-2)^2=4$$
  $$C_2:(x-7)^2+(y-10)^2=9$$
  και
  $$C_3:(x-3)^2+(y-7)^2=64$$
  \begin{enumerate}
    \item<1-> Να βρείτε τη σχετική θέση των κύκλων $C_1$ και $C_2$ και μετά να βρείτε τη μέγιστη και την ελάχιστη απόσταση που απέχει ένα σημείο του $C_1$ από ένα σημείο του $C_2$
    \item<2-> Να βρείτε τη σχετική θέση των κύκλων $C_2$ και $C_3$ και μετά να βρείτε τη μέγιστη απόσταση ποτ απέχει ένα σημείο του $C_2$ από ένα σημείο του $C_3$
  \end{enumerate}

  %\hyperlink{Λύση30}{\beamerbutton{Λύση}}
\end{askisi}

\appendix

\section{Αποδείξεις}
\begin{frame}[label=Απόδειξη]{Απόδειξη εφαπτομένης κύκλου}
  Ο κύκλος με κέντρο $Κ(x_0,y_0)$ και ακτίνα $ρ$ έχει εξίσωση $(x-x_0)^2+(y-y_0)^2=ρ^2$ και έστω το σημείο του $Μ(x_1,y_1)$. Έστω $Α(x,y)$ τυχαίο σημείο της εφαπτόμενης

  Θα ισχύει $ΜΑ\perp ΚΜ$
  \begin{gather*}
    (x-x_1,y-y_1)\perp (x_1-x_0,y_1-y_0) \\ \pause
    (x-x_1,y-y_1)\cdot (x_1-x_0,y_1-y_0)=0 \\ \pause
    (x-x_1)(x_1-x_0)+(y-y_1)(y_1-y_0)=0 \\ \pause
    (x-x_0+x_0-x_1)(x_1-x_0)+(y-y_1+y_1-y_0)(y_1-y_0)=0 \\ \pause
    (x-x_0)(x_1-x_0)-(x_1-x_0)^2+(y-y_0)(y_1-y_0)-(y_1-y_0)^2=0 \\ \pause
    (x-x_0)(x_1-x_0)+(y-y_0)(y_1-y_0)-ρ^2=0 \\ \pause
    (x-x_0)(x_1-x_0)+(y-y_0)(y_1-y_0)=ρ^2
  \end{gather*}

  \hyperlink{Εφαπτόμενη}{\beamerbutton{Πίσω στη θεωρία}}
\end{frame}

\begin{frame}[label=Απόδειξη2]{Απόδειξη γενικής εξίσωσης κύκλου}
  $$x^2+y^2+Αx+Βy+Γ=0$$ \pause
  \begin{gather*}
    (x-x_0)^2+(y-y_0)^2=ρ^2 \\ \pause
    x^2-2xx_0+x_0^2+y^2-2yy_0+y_0^2-ρ^2=0 \\ \pause
    x^2+y^2-2xx_0-2yy_0+x_0^2+y_0^2-ρ^2=0
  \end{gather*} \pause
  Άρα $Α=-2x_0$, $Β=-2y_0$ και $Γ=x_0^2+y_0^2-ρ^2$

  \hyperlink{ΓενικήΕξίσωση}{\beamerbutton{Πίσω στη θεωρία}}
\end{frame}

\begin{frame}[label=Απόδειξη3]{Απόδειξη γενικής εξίσωσης κύκλου}
  $$(x-x_0)^2+(y-y_0)^2=ρ^2$$ \pause
  \begin{gather*}
    x^2+y^2+Αx+Βy+Γ=0 \\ \pause
    x^2+Αx+y^2+Βy+Γ=0 \\ \pause
    x^2+Αx+\dfrac{Α^2}{4}+y^2+Βy+\dfrac{Β^2}{4}=\dfrac{Α^2}{4}+\dfrac{Α^2}{4}-Γ \\ \pause
    (x-\dfrac{Α}{2})^2+(y-\dfrac{Β}{2})^2=\dfrac{Α^2+Β^2-4Γ}{4}
  \end{gather*} \pause
  Άρα $x_0=-\dfrac{Α}{2}$, $y_0=-\dfrac{Β}{2}$, $ρ=\dfrac{\sqrt{Α^2+Β^2-4Γ}}{2}$

  \hyperlink{ΓενικήΕξίσωση}{\beamerbutton{Πίσω στη θεωρία}}
\end{frame}

\end{document}
