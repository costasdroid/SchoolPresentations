\documentclass[greek]{beamer}
%\usepackage{fontspec}
\usepackage{amsmath,amsthm}
\usepackage{unicode-math}
\usepackage{xltxtra}
\usepackage{graphicx}
\usetheme{CambridgeUS}
\usecolortheme{seagull}
\usepackage{hyperref}
\usepackage{ulem}
\usepackage{xgreek}

\usepackage{pgfpages} 
\usepackage{tikz}
%\setbeameroption{show notes on second screen}
%\setbeameroption{show only notes}

\setsansfont{Calibri}

\usepackage{multicol}

\usepackage{appendixnumberbeamer}

\usepackage{polynom}

\usepackage{pgffor}


\setbeamercovered{transparent}
\beamertemplatenavigationsymbolsempty

\title{Διανύσματα}
\subtitle{Εσωτερικό Γινόμενο Διανυσμάτων}
\author[Λόλας]{Κωνσταντίνος Λόλας}
\institute[$10^ο$ ΓΕΛ]{$10^ο$ ΓΕΛ Θεσσαλονίκης}
\date{}

\begin{document}

\begin{frame}
      \titlepage
\end{frame}
\begin{frame}{Τι, δεν τελειώσαμε με τα διανύσματα?}
      Αν προσέχατε μιλήσαμε για:
      \begin{enumerate}
            \item τι είναι
            \item μέτρο
            \item άθροισμα - διαφορά
            \item πολλαπλασιασμό με αριθμό
            \item ισότητα
            \item παραλληλία
                  \only<2>{\item τι έμεινε...}
                  \only<3>{\item γωνίες, καθετότητα...}
      \end{enumerate}
      \only<2>{
            \begin{figure}
                  \centering
                  \includegraphics[width=0.4 \textwidth]{"../images/whatelse"}
            \end{figure}
      }
\end{frame}

\begin{frame}{Νέος κόσμος}
      Κάθε διανυσματικός χώρος (?), εφοδιάζεται με ένα δικό του εσωτερικό γινόμενο. Για εμάς, να ναι καλά ο Ευκλείδης!
      \begin{block}{Ευκλείδειο Εσωτερικό Γινόμενο}
            Έστω $\vec{α}=(x_α,y_α)$ και $\vec{β}=(x_β,y_β)$. Ορίζουμε εσωτερικό γινόμενο την πράξη $\mathbb{R^2}\cdot\mathbb{R^2}\to\mathbb{R}$:
            $$\vec{α}\cdot\vec{β}=x_αx_β+y_αy_β$$
      \end{block}
      Ή απλά για τους noobάδες: \emph{εσωτερικό γινόμενο}
\end{frame}

\begin{frame}{Ναι αλλά πού είναι η γωνία?}
      \begin{figure}
            \centering
            \includegraphics[width=0.8 \textwidth]{"../images/missing"}
      \end{figure}
\end{frame}

\begin{frame}{Για δυνατούς λύτες}
      Αν μάθατε τον τύπο των συνημιτόνων:
      \begin{figure}
            \centering
            \includegraphics[width=0.35 \textwidth]{"../images/dotproductproof"}
      \end{figure}
      \begin{align*}
            |AB|^2                           & =|OA|^2+|OB|^2-2|OA|\cdot|ΟΒ|\cdot συνθ                             \\
            \sqrt{(x_2-x_1)^2+(y_2-y_1)^2}^2 & =\sqrt{x_1^2+y_1^2}^2+\sqrt{x_2^2+y_2^2}^2-2|OA|\cdot|ΟΒ|\cdot συνθ \\
            -2(x_1x_2+y_1y_2)                & =-2|OA|\cdot|ΟΒ|\cdot συνθ                                          \\
            x_1x_2+y_1y_2                    & =|OA|\cdot|ΟΒ|\cdot συνθ
      \end{align*}
\end{frame}

\begin{frame}{Τι, δεύτερος τύπος?}
      Στο βιβλίο τον μαθαίνετε ως πρώτο (και stick to it)
      \begin{block}{Ευκλείδειο Εσωτερικό Γινόμενο}
            Έστω $\vec{α}$ και $\vec{β}$ δύο μη μηδενικά διανύσματα. Ορίζουμε εσωτερικό γινόμενο την πράξη
            $$\vec{α}\cdot\vec{β} = |\vec{α}|\cdot|\vec{β}|\cdot συνθ$$
      \end{block}
      \only<2->{
            Problems, problems, problems!

            Τι γίνεται αν $\vec{α}=\vec{0}$ ή $\vec{β}=\vec{0}$?
      }

\end{frame}

\begin{frame}{Και ναι ήρθε η ώρα!}
      Απλά από τον προηγούμενο τύπο $\vec{α}\cdot\vec{β} = |\vec{α}|\cdot|\vec{β}|\cdot συνθ$
      \begin{itemize}
            \item<1-> $\vec{α}\upuparrows\vec{β}\iff \vec{α}\cdot\vec{β} = |\vec{α}|\cdot|\vec{β}|$
            \item<2-> $\vec{α}\updownarrows\vec{β}\iff \vec{α}\cdot\vec{β} = -|\vec{α}|\cdot|\vec{β}|$
            \item<3-> $\vec{α}\perp\vec{β}\iff \vec{α}\cdot\vec{β} = 0$
            \item<4-> $\vec{α}^2=|\vec{α}|^2$
                  \only<5> {και το αστέρι μας...}
            \item<6-> $συνθ=\dfrac{\vec{α}\cdot\vec{β}}{|\vec{α}|\cdot|\vec{β}|}$
      \end{itemize}
\end{frame}

\begin{frame}{Και λίγο ιδιότητες της νέας πράξης}
      Η απoδείξεις είναι απλές...
      \begin{itemize}
            \item<1-> $\vec{α}(\vec{β}+\vec{γ})=\vec{α}\vec{β}+\vec{α}\vec{γ}$ (Επιμεριστική)
            \item<2-> $\vec{α}\vec{β}=\vec{β}\vec{α}$ (Αντιμεταθετική)
            \item<3-> $λ(\vec{α}\vec{β})=\vec{α}(λ\vec{β})=λ\vec{α}\vec{β}$ (Προσεταιριστική με πραγματικό)
            \item<4-> Γενικά $\vec{α}(\vec{β}\vec{γ})\ne (\vec{α}\vec{β})\vec{γ}$, αλλά γιατί?
      \end{itemize}
\end{frame}

\begin{frame}
      \begin{figure}
            \centering
            \includegraphics[width=0.5 \textwidth]{"../images/relief"}
      \end{figure}
\end{frame}

\begin{frame}
      Στο moodle θα βρείτε τις ασκήσεις που πρέπει να κάνετε, όπως και αυτή τη παρουσίαση
\end{frame}

\section{Ασκήσεις}

\subsection{Άσκηση 1}
\begin{frame}[label=Άσκηση1,t]{Εξάσκηση 1}
      Δίνονται δύο διανύσματα $\vec{α}$, $\vec{β}$ για τα οποία ισχύουν $|\vec{α}|=2$, $|\vec{β}|=1$ και $\widehat{(\vec{α }, \vec{β })}=\dfrac{\pi}{3}$. Να υπολογίσετε:
      \begin{enumerate}
            \item<1-> $\vec{α}\vec{β}$
            \item<2-> $\vec{α}^2$
      \end{enumerate}
      %\hyperlink{Λύση1}{\beamerbutton{Λύση}}
\end{frame}

\subsection{Άσκηση 2}
\begin{frame}[label=Άσκηση2,t]{Εξάσκηση 2}
      Αν $\vec{α}\updownarrows\vec{β}$, $|\vec{α}|=3$ και $\vec{α}\vec{β}=-6$, να βρείτε το $\vec{β}$
      %\hyperlink{Λύση2}{\beamerbutton{Λύση}}
\end{frame}

\subsection{Άσκηση 3}
\begin{frame}[label=Άσκηση3,t]{Εξάσκηση 3}
      Αν $\vec{α}\perp\vec{β}$, $\vec{α}\updownarrows\vec{γ}$ και $|\vec{α}|=|\vec{γ}|=1$, να υπολογίσετε την τιμή της παράστασης
      $$(\vec{α}\vec{β})^{2023}+|\vec{α}\vec{γ}|$$
      %\hyperlink{Λύση3}{\beamerbutton{Λύση}}
\end{frame}

\subsection{Άσκηση 4}
\begin{frame}[label=Άσκηση4,t]{Εξάσκηση 4}
      Αν $\vec{α}=(2,3)$ και $\vec{β}=(4,5)$ να υπολογίσετε τα:
      \begin{enumerate}
            \item<1-> $\vec{α}\vec{β}$
            \item<2-> $\vec{α}^2$
      \end{enumerate}
      %\hyperlink{Λύση4}{\beamerbutton{Λύση}}
\end{frame}

\subsection{Άσκηση 5}
\begin{frame}[label=Άσκηση5,t]{Εξάσκηση 5}
      Αν τα διανύσματα $\vec{α}=(x-1,-4)$, $\vec{β}=(6,x)$ είναι κάθετα, να βρείτε το $x$

      %\hyperlink{Λύση5}{\beamerbutton{Λύση}}
\end{frame}

\subsection{Άσκηση 6}
\begin{frame}[label=Άσκηση6,t]{Εξάσκηση 6}
      Δίνονται τα διανύσματα $\vec{α}=(1,3)$, $\vec{β}=(1,-2)$ και $\vec{γ}=(4,-3)$. Να βρείτε διάνυσμα $\vec{v}=κ\vec{α}+λ\vec{β}$, ώστε να είναι κάθετο στο $\vec{γ}$ και να έχει μέτρο 5.

      %\hyperlink{Λύση6}{\beamerbutton{Λύση}}
\end{frame}

\subsection{Άσκηση 7}
\begin{frame}[label=Άσκηση7,t]{Εξάσκηση 7}
      Δίνονται δύο διανύσματα $\vec{α}$, $\vec{β}$ για τα οποία ισχύουν $|\vec{α}|=2$, $|\vec{β}|=3$ και $\widehat{(\vec{α }, \vec{β })}=\dfrac{2\pi}{3}$. Να υπολογίσετε:
      \begin{enumerate}
            \item<1-> $(\vec{α}-2\vec{β})(3\vec{α}+\vec{β})$
            \item<2-> $(3\vec{α}-\vec{β})^2$
      \end{enumerate}

      %\hyperlink{Λύση7}{\beamerbutton{Λύση}}
\end{frame}

\subsection{Άσκηση 8}
\begin{frame}[label=Άσκηση8,t]{Εξάσκηση 8}
      Δίνονται τα διανύσματα $\vec{α}=(-2,1)$, $\vec{β}=(3,1)$ και $\vec{γ}=(0,-3)$. Να βρείτε τα:

      \begin{enumerate}
            \item<1-> $(3\vec{α})(-\vec{γ})$
            \item<2-> $\vec{α}(\vec{β}-2\vec{γ})$
            \item<3-> $(|\vec{γ}\vec{α}|)\vec{β}$
      \end{enumerate}

      %\hyperlink{Λύση8}{\beamerbutton{Λύση}}
\end{frame}

\subsection{Άσκηση 9}
\begin{frame}[label=Άσκηση9,t]{Εξάσκηση 9}
      Δίνονται δύο κάθετα διανύσματα $\vec{α}$ και $\vec{β}$ με $|\vec{α}|=\sqrt{2}$, $|\vec{β}|=\sqrt{3}$. Να δείξετε ότι τα διανύσματα $\vec{v}=\vec{α}-2\vec{β}$ και $\vec{u}=3\vec{α}+\vec{β}$ είναι κάθετα.
      %\hyperlink{Λύση9}{\beamerbutton{Λύση}}
\end{frame}

\subsection{Άσκηση 10}
\begin{frame}[label=Άσκηση10,t]{Εξάσκηση 10}
      Αν $|\vec{α}|=2$, $|\vec{β}|=3$ και $\widehat{(\vec{α }, \vec{β })}=\dfrac{5\pi}{6}$. Να βρείτε το μέτρο του διανύσματος
      $$\vec{v}=2\vec{α}-3\vec{β}$$
      %\hyperlink{Λύση10}{\beamerbutton{Λύση}}
\end{frame}

\subsection{Άσκηση 11}
\begin{frame}[label=Άσκηση11,t]{Εξάσκηση 11}
      Δίνονται δύο διανύσματα $\vec{α}$, $\vec{β}$ για τα οποία ισχύουν $|\vec{α}|=1$, $|\vec{β}|=\sqrt{2}$ και $\widehat{(\vec{α }, \vec{β })}=45^{\circ}$ και το παραλληλόγραμμο $ΑΒΓΔ$ με $\widehat{ΑΒ}=λ\vec{α}+\vec{β}$ και $\widehat{ΑΔ}=\vec{α}+\vec{β}$. Να βρείτε το $λ$ ώστε το μήκος της διαγωνίου $ΑΓ$ να είναι $2$.
      %\hyperlink{Λύση11}{\beamerbutton{Λύση}}
\end{frame}

\subsection{Άσκηση 12}
\begin{frame}[label=Άσκηση12,t]{Εξάσκηση 12}
      \begin{enumerate}
            \item<1-> Αν $\vec{v}$ και $\vec{u}$ δύο διανύσματα, να αποδείξετε ότι
                  $$|\vec{v}+\vec{u}|^2+|\vec{v}-\vec{u}|^2=2|\vec{v}|^2+2|\vec{u}|^2$$
            \item<2-> Αν για τα διανύσματα $\vec{v}$ και $\vec{u}$ ισχύουν $|\vec{v}|=1$, $|\vec{u}|=2$ και $|\vec{v}+\vec{u}|=3$, να βρείτε το $|\vec{v}-\vec{u}|=3$
      \end{enumerate}
      %\hyperlink{Λύση12}{\beamerbutton{Λύση}}
\end{frame}

\subsection{Άσκηση 13}
\begin{frame}[label=Άσκηση13,t]{Εξάσκηση 13}
      Αν $|\vec{α}|=1$, $|\vec{β}|=2$ και $|\vec{α}-\vec{β}|=\sqrt{7}$. Να βρείτε τη γωνία των διανυσμάτων $\widehat{(\vec{α }, \vec{β })}$
      %\hyperlink{Λύση13}{\beamerbutton{Λύση}}
\end{frame}

\subsection{Άσκηση 14}
\begin{frame}[label=Άσκηση14,t]{Εξάσκηση 14}
      Δίνονται τα σημεία $Α(-3,2)$ και $Β(1,-2)$. Να βρείτε σημείο $Μ$ του άξονα $y'y$, ώστε το τρίγωνο $ΜΑΒ$ να είναι ισοσκελές με βάση την $ΑΒ$
      %\hyperlink{Λύση14}{\beamerbutton{Λύση}}
\end{frame}

\subsection{Άσκηση 15}
\begin{frame}[label=Άσκηση15,t]{Εξάσκηση 15}
      Δίνονται τα διανύσματα $\vec{α}$ και $\vec{β}$, όπου το $\vec{α}$ είναι μοναδιαίο, $|\vec{β}|=2$ και η γωνία των διανυσμάτων $\vec{α}$ και $\vec{α}$ είναι $120^{\circ}$. Να βρείτε την γωνία των διανυσμάτων $\vec{δ}$ και $\vec{α}$, όπου $\vec{δ}=2\vec{α}-\vec{β}$.
      %\hyperlink{Λύση15}{\beamerbutton{Λύση}}
\end{frame}

\subsection{Άσκηση 16}
\begin{frame}[label=Άσκηση16,t]{Εξάσκηση 16}
      Αν $|\vec{α}|=2$, $|\vec{β}|=1$ και $\widehat{(\vec{α }, \vec{β })}=\dfrac{2\pi}{3}$ και $\vec{α}+2\vec{β}+\vec{γ}=\vec{0}$, να υπολογίσετε:
      \begin{enumerate}
            \item<1-> το μέτρο του $\vec{γ}$
            \item<2-> την τιμή της παράστασης $\vec{α}\vec{γ}+\vec{β}\vec{γ}$
      \end{enumerate}
      %\hyperlink{Λύση16}{\beamerbutton{Λύση}}
\end{frame}

\subsection{Άσκηση 17}
\begin{frame}[label=Άσκηση17,t]{Εξάσκηση 17}
      Αν για τα μη μηδενικά διανύσματα $\vec{α}$, $\vec{β}$ και $\vec{γ}$, ισχύουν
      $$\dfrac{|\vec{α}|}{2}=\dfrac{2|\vec{β}|}{3}=\dfrac{|\vec{γ}|}{5} \text{ και } \vec{α}-2\vec{β}+\vec{γ}=\vec{0}$$
      να αποδείξετε ότι:
      \begin{enumerate}
            \item<1-> $\vec{α}=-\dfrac{4}{3}\vec{β}$
            \item<2-> $\vec{β}\parallel \vec{γ}$
      \end{enumerate}
      %\hyperlink{Λύση17}{\beamerbutton{Λύση}}
\end{frame}

\subsection{Άσκηση 18}
\begin{frame}[label=Άσκηση18,t]{Εξάσκηση 18}
      Έστω τα διανύσματα $\vec{α}$, $\vec{β}$ με $|\vec{α}|=2$, $|\vec{β}|=1$ και $\widehat{(\vec{α }, \vec{β })}=\dfrac{2\pi}{3}$
      \begin{enumerate}
            \item<1-> Να αποδείξετε ότι: $\vec{α}-2\vec{β}\ne \vec{0}$
            \item<2-> Να βρείτε διάνυσμα $\vec{x}$, ώστε:
                  \begin{itemize}
                        \item $\vec{x}\parallel (\vec{α}-2\vec{β})$ και
                        \item $\vec{β}\perp (\vec{α}-\vec{x})$
                  \end{itemize}
      \end{enumerate}
      %\hyperlink{Λύση18}{\beamerbutton{Λύση}}
\end{frame}

\subsection{Άσκηση 19}
\begin{frame}[label=Άσκηση19,t]{Εξάσκηση 19}
      Αν $\vec{α}=(x,3)$ και $\vec{β}=(1,7)$, να βρείτε την τιμή του $x$ ώστε $\widehat{(\vec{α }, \vec{β })}=\dfrac{\pi}{4}$
      %\hyperlink{Λύση19}{\beamerbutton{Λύση}}
\end{frame}

\subsection{Άσκηση 20}
\begin{frame}[label=Άσκηση20,t]{Εξάσκηση 20}
      Δίνονται τα σταθερά σημεία $Α$ και $Β$. Να βρείτε το γεωμετρικό τόπο των σημείων $Μ$ του επιπέδου, για τα οποία ισχύει $|\overrightarrow{ΜΑ}-2\overrightarrow{ΜΒ}|=\sqrt{\overrightarrow{ΜΑ}^2+4\overrightarrow{ΜΒ}^2}$
      %\hyperlink{Λύση20}{\beamerbutton{Λύση}}
\end{frame}

\subsection{Άσκηση 21}
\begin{frame}[label=Άσκηση21,t]{Εξάσκηση 21}
      Δίνεται τρίγωνο $ΑΒΓ$ με $(ΑΒ)=6$. Να βρείτε το γεωμετρικό τόπο των σημείων $Μ$ του επιπέδου, για τα οποία ισχύει:
      $$\overrightarrow{ΜΑ}\cdot \overrightarrow{ΜΓ}=7+\overrightarrow{ΜΑ}\cdot \overrightarrow{ΒΓ}$$
      %\hyperlink{Λύση21}{\beamerbutton{Λύση}}
\end{frame}

\end{document}