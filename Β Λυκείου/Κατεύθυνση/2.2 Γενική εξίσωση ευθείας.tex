\documentclass[greek]{beamer}
%\usepackage{fontspec}
\usepackage{amsmath,amsthm}
\usepackage{unicode-math}
\usepackage{xltxtra}
\usepackage{graphicx}
\usetheme{CambridgeUS}
\usecolortheme{seagull}
\usepackage{hyperref}
\usepackage{ulem}
\usepackage{xgreek}

\usepackage{pgfpages}
\usepackage{tikz}
\usepackage{tkz-tab}
%\setbeameroption{show notes on second screen}
%\setbeameroption{show only notes}

\setsansfont{Noto Serif}

\usepackage{multicol}

\usepackage{appendixnumberbeamer}

\setbeamercovered{transparent}
\beamertemplatenavigationsymbolsempty

\title{Ευθείες}
\subtitle{Γενική Εξίσωση Ευθείας}
\author[Λόλας]{Κωνσταντίνος Λόλας}
\date{}

\begin{document}

\begin{frame}
 \titlepage
\end{frame}

\section{Θεωρία}
\begin{frame}{Αχχχχ! Μεγαλώνουμε}
 Μέχρι στιγμής καμαρώνουμε τις ευθείες σε μία μορφή
 $$y=αx+β$$
 \onslide<2-> Αν και όχι πάντα (π.χ. $x=a$)
\end{frame}

\begin{frame}{One equation to rule them all?}
 Τι γνωρίζουμε?
 \begin{enumerate}
  \item<1-> γραμμή ονομάζουμε οποιαδήποτε εξίσωση με τουλάχιστον μία μεταβλητή
  \item<2-> έχουμε δύο περιπτώσεις (λόγω κλίσης $λ$)
  \item<3-> άρα...
 \end{enumerate}
 \onslide<4-> γιατί να μην έχουμε ΜΙΑ εξίσωση και ας μην έχουμε $λ$.
 \centering
 \only<5->{
  \includegraphics[width=0.4\textwidth]{"../images/duh.jpeg"}}

\end{frame}

\begin{frame}{YEAHHHHH!}
 Θα μας έκανε κάτι τέτοιο?
 $$Αx+Βy+Γ=0 \text{, με } Α^2+Β^2\ne 0$$
 \centering
 \includegraphics[width=0.4\textwidth]{"../images/feelsgood.png"}
\end{frame}

\begin{frame}{Check 1!}
 Υπάρχει η $y=αx+β$ στην $Αx+Βy+Γ=0 \text{, με } Α^2+Β^2\ne 0$?
 \onslide<2-> Φυσικά, αρκεί $Β\ne 0$
 \begin{align*}
  Αx+Βy+Γ=0 \\
  Βy=-Αx-Γ  \\
  y=-\frac{Α}{Β}x-\frac{Γ}{Α}
 \end{align*}
 \only<3>{
  \centering
  \includegraphics[width=0.2\textwidth]{"../images/h1.png"}}
\end{frame}

\begin{frame}{Check 2!}
 Υπάρχει η $x=α$ στην $Αx+Βy+Γ=0 \text{, με } Α^2+Β^2\ne 0$?
 \onslide<2-> Φυσικά, αρκεί $Β=0$ και $Α\ne 0$
 \begin{align*}
  Αx+0y+Γ=0 \\
  Αx=Γ      \\
  x=\frac{Γ}{Α}
 \end{align*}
 \only<3>{
  \centering
  \includegraphics[width=0.2\textwidth]{"../images/h1.png"} \includegraphics[width=0.2\textwidth]{"../images/h2.png"}}
\end{frame}

\begin{frame}{\scalebox{-1}[1]{Check 1!}}
 Γράφεται η $y=αx+β$ στην $Αx+Βy+Γ=0 \text{, με } Α^2+Β^2\ne 0$?
 \onslide<2-> Φυσικά
 \begin{align*}
  y=αx+β \\
  αx-1y+β=0
 \end{align*}
 \only<3>{
  \centering
  \includegraphics[width=0.2\textwidth]{"../images/h1.png"} \includegraphics[width=0.2\textwidth]{"../images/h2.png"} \includegraphics[width=0.2\textwidth]{"../images/h3.png"}}
\end{frame}

\begin{frame}{\scalebox{-1}[1]{Check 2!}}
 Γράφεται η $x=α$ στην $Αx+Βy+Γ=0 \text{, με } Α^2+Β^2\ne 0$?
 \onslide<2-> Φυσικά
 \begin{align*}
  x=α \\
  1x+0y-α=0
 \end{align*}
 \only<3>{
  \centering
  \includegraphics[width=0.2\textwidth]{"../images/h1.png"} \includegraphics[width=0.2\textwidth]{"../images/h2.png"} \includegraphics[width=0.2\textwidth]{"../images/h3.png"} \includegraphics[width=0.2\textwidth]{"../images/h4.png"}}
\end{frame}

\begin{frame}{Μα γιατί να ασχοληθούμε???}
 Μπορούμε να βρίσκουμε άμεσα το παράλληλο στην ευθεία διάνυσμα
 \begin{block}{Το παράλληλο 1}
  Αν $Β\ne 0$ γράφεται ως εξής $y=-\frac{Α}{Β}x-\frac{Γ}{Α}$, άρα ένα διάνυσμα παράλληλό της είναι το $$(-Β,Α)$$
 \end{block}
 \begin{block}{Το παράλληλο 2}
  Αν $Β= 0$ και $Α\ne 0$ τότε ένα παράλληλο είναι το $(0,Α)$ (γιατί?) άρα το $$(-Β,Α)$$
 \end{block}
\end{frame}

\begin{frame}{Γιατί όχι και κάθετα?}
 Αφού η ευθεία είναι παράλληλη στο $(-Β,Α)$
 \begin{block}{Το κάθετο}
  η ευθεία $Αx+Βy+Γ=0$ είναι κάθετη στο διάνυσμα $(Α,Β)$
 \end{block}

\end{frame}

\begin{frame}{Από εδώ και πέρα?}
 Παρατηρήσεις:
 \begin{enumerate}
  \item<1-> ξανά τις ασκήσεις από άλλη σκοπιά
  \item<2-> θα ξέρουμε κατ' ευθείαν την παράλληλη σε διάνυσμα ευθεία
  \item<3-> θα ξέρουμε κατ' ευθείαν την κάθετη σε διάνυσμα ευθεία
  \item<4-> θα ξέρουμε κατ' ευθείαν τις συμμετρικές ως προς άξονες ευθείες
 \end{enumerate}
\end{frame}

\section{Ασκήσεις}
\subsection{Άσκηση 1}
\begin{frame}[label=Άσκηση1]{Εξάσκηση 1}
 Δίνεται η ευθεία $ε:2x+3y-6=0$. Να βρείτε:
 \begin{enumerate}
  \item<1-> την ευθεία $ζ$ που είναι παράλληλη στην ευθεία $ε$ και διέρχεται από το σημείο $Α(-1,2)$
  \item<2-> τα σημεία τομής της ευθείας $ζ$ με τους άξονες
 \end{enumerate}

 % \hyperlink{Λύση1}{\beamerbutton{Λύση}}
\end{frame}

\subsection{Άσκηση 2}
\begin{frame}[label=Άσκηση2]{Εξάσκηση 2}
 Να αποδείξετε ότι οι ευθείες
 $$ε_1:x-3y+2=0 \qquad ε_2:2x-y-1=0 \qquad ε_3:5x-3y-2=0$$
 διέρχονται από το ίδιο σημείο

 %\hyperlink{Λύση2}{\beamerbutton{Λύση}}
\end{frame}

\subsection{Άσκηση 3}
\begin{frame}[label=Άσκηση3]{Εξάσκηση 3}
 Δίνεται η εξίσωση:
 $$(λ^2-1)x+(λ^2-λ)y+λ+1=0,λ\in\mathbb{R}$$
 Να βρείτε για ποιες τιμές του $λ$ η εξίσωση παριστάνει:
 \begin{enumerate}
  \item<1-> ευθεία
  \item<2-> ευθεία παράλληλη στον άξονα $x'x$
  \item<3-> ευθεία παράλληλη στον άξονα $y'y$
 \end{enumerate}

 %\hyperlink{Λύση3}{\beamerbutton{Λύση}}
\end{frame}

\subsection{Άσκηση 4}
\begin{frame}[label=Άσκηση4]{Εξάσκηση 4}
 Δίνονται οι ευθείες:
 \begin{itemize}
  \item $ε_1:(μ-1)x-(μ-2)y-μ=0$
  \item $ε_2:(μ-2)x-(μ+1)y-3=0$
 \end{itemize}
 Να βρείτε το $μ$ ώστε:
 \begin{enumerate}
  \item<1-> οι ευθείες $ε_1$ και $ε_2$ να τέμνονται
  \item<2-> $ε_1 \parallel ε_2$
  \item<3-> $ε_1 \perp ε_2$
 \end{enumerate}

 %\hyperlink{Λύση4}{\beamerbutton{Λύση}}
\end{frame}

\subsection{Άσκηση 5}
\begin{frame}[label=Άσκηση5]{Εξάσκηση 5}
 Να βρείτε την οξεία γωνία των ευθειών
 $$ε_1:y=(-2+\sqrt{3})x$$
 και
 $$ε_2:y=-x$$
 %\hyperlink{Λύση5}{\beamerbutton{Λύση}}
\end{frame}

\subsection{Άσκηση 6}
\begin{frame}[label=Άσκηση6]{Εξάσκηση 6}
 Να βρείτε τις ευθείες που διέρχονται από το σημείο $Ρ(1,-1)$ και σχηματίζουν με την ευθεία $η:x+y-1=0$ οξεία γωνία ίση με $45^{\circ}$

 %\hyperlink{Λύση6}{\beamerbutton{Λύση}}
\end{frame}

\subsection{Άσκηση 7}
\begin{frame}[label=Άσκηση]{Εξάσκηση 7}
 Να αποδείξετε ότι όλες οι ευθείες που ορίζονται από την εξίσωση:
 $$ε_λ:(λ+1)x+(λ-1)y+2λ=0 \text{, όπου } λ\in\mathbb{R}$$
 διέρχονται από το ίδιο σημείο $Α$ και στη συνέχεια, να βρείτε εκείνη την ευθεία $ε$ που ορίζεται από την εξίσωση αυτή και είναι κάθετη στην ευθεία $ζ:y=2x$
 %\hyperlink{Λύση}{\beamerbutton{Λύση}}
\end{frame}

\subsection{Άσκηση 8}
\begin{frame}[label=Άσκηση8]{Εξάσκηση 8}
 Δίνεται η εξίσωση: $x^2-3y^2-2x+1=0$
 \begin{enumerate}
   \item<1-> Να αποδείξετε ότι παριστάνει δύο ευθείες $ε_1$ και $ε_2$ συμμετρικές ως προς τον άξονα $x'x$
   \item<2-> Να βρείτε την οξεία γωνία που σχηματίζουν οι ευθείες $ε_1$ και $ε_2$
 \end{enumerate}

 %\hyperlink{Λύση8}{\beamerbutton{Λύση}}
\end{frame}

\section{}
\begin{frame}
 Στο moodle θα βρείτε τις ασκήσεις που πρέπει να κάνετε, όπως και αυτή τη παρουσίαση
\end{frame}

% \appendix
% \section{Λύσεις Ασκήσεων}
% \begin{frame}
%  \tableofcontents
% \end{frame}

\end{document}
