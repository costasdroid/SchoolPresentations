\documentclass[greek]{beamer}
%\usepackage{fontspec}
\usepackage{amsmath,amsthm}
\usepackage{unicode-math}
\usepackage{xltxtra}
\usepackage{graphicx}
\usetheme{CambridgeUS}
\usecolortheme{seagull}
\usepackage{hyperref}
\usepackage{ulem}
\usepackage{xgreek}

\usepackage{pgfpages}
\usepackage{tikz}
\usepackage{tkz-tab}
%\setbeameroption{show notes on second screen}
%\setbeameroption{show only notes}

\setsansfont{Noto Serif}

\usepackage{multicol}

\usepackage{appendixnumberbeamer}

\setbeamercovered{transparent}
\beamertemplatenavigationsymbolsempty

\title{Ευθεία}
\subtitle{Εξίσωση Ευθείας}
\author[Λόλας]{Κωνσταντίνος Λόλας}
\date{}

\begin{document}

\begin{frame}
 \titlepage
\end{frame}

\section{Θεωρία}
\begin{frame}{Το μεγάλο ταξίδι}
 \begin{itemize}
  \item<1-> Ορισμός
  \item<2-> Εξίσωση
  \item<3-> Γενική Εξίσωση, to rule them all!
  \item<4-> Ελάχιστος συνδυασμός με διανύσματα, shame!
  \item<5-> Εύρεση εξίσωσης από κάθε περίπτωση, brace yourselfs!
  \item<6-> 2 νέοι τύποι (απόστασης και εμβαδού)
 \end{itemize}
\end{frame}

\begin{frame}{Γνωστά ή Άγνωστα νερά?}
 Λέξεις κλειδιά
 \begin{itemize}
  \item Κλίση
  \item Συντελεστής διεύθυνσης
  \item $εφθ$
  \item $α$
  \item Σημεία
  \item Παραλληλία
  \item Καθετότητα
  \item Σημεία τομής...
 \end{itemize}
 είναι μερικά που θυμάμαι!
\end{frame}

\begin{frame}{Γραμμές, γραμμές παντού}
 \begin{itemize}
  \item<1-> Τι είναι γραμμή?
  \item<2-> Γραφικά ή Αλγεβρικά?
 \end{itemize}
\end{frame}

\begin{frame}{Γραφικά}
 Εύκολο!
\end{frame}

\begin{frame}{Αλγεβρικά}
 \begin{block}{Ορισμός γραμμής}
  Μία εξίσωση με τουλάχιστον έναν άγνωστο
 \end{block}
 \begin{block}{Σημείο στη γραμμή}
  Κάθε σημείο που επαληθεύει την εξίσωση
 \end{block}
\end{frame}

\begin{frame}{Ας φτιάξουμε απλές γραμμές}
 \begin{itemize}
  \item<1-> $y=2$
  \item<2-> $x=1$
  \item<3-> $x-y=0$
  \item<4-> $y=2x$
 \end{itemize}
\end{frame}

\begin{frame}{Ορισμοί}
 \begin{block}{Γωνία Ευθείας}
  Ονομάζουμε \emph{γωνία της ευθείας με τον άξονα $x'x$}, την γωνία που σχηματίζει ο $x'x$ όταν στραφεί αντίστροφα με τους δείκτες του ρολογιού έως ότου συμπέσει με την ευθεία
 \end{block}
 \begin{block}{Συντελεστής Διεύθυνσης Ευθείας}
  Ονομάζουμε \emph{συντελεστή διεύθυνσης} (ή κλίση) της ευθείας την εφαπτομένη της γωνίας της ευθείας με τον $x'x$
 \end{block}
\end{frame}

\begin{frame}{Ξεπηδούν οι απορίες}
 \begin{itemize}
  \item<2-> Τι τιμές παίρνει μία γωνία
  \item<3-> Τι τιμές παίρνει η κλίση
  \item<4-> Πότε είναι παράλληλες δύο ευθείες
  \item<5-> Πότε είναι παράλληλη μία ευθεία με ένα διάνυσμα
  \item<6-> Ποιά άλλα διανύσματα είναι παράλληλα με την ευθεία?
  \item<7-> Πότε είναι κάθετες δύο ευθείες? \pause μην βιάζεστε!!!!!
 \end{itemize}
\end{frame}

\begin{frame}{Λίγη ιστορία}
 \begin{block}{Κλίση διανύσματος}
  $λ=\frac{y_2-y_1}{x_2-x_1}$
 \end{block}
\end{frame}

\begin{frame}{Εξισώση ευθείας 1 (από κλίση και σημείο)}
 Ας θεωρήσουμε ότι \emph{υπάρχει συντελεστής διεύθυνσης $λ$} και ας έχουμε γνωστό \emph{ένα σημείο} $(x_0,y_0)$. Κάθε σημείο $(x,y)$ που ανήκει στην ευθεία θα έχει με το γνωστό σημείο κλίση $λ$. Άρα
 \begin{align*}
  \frac{y-y_0}{x-x_0}=λ \\
  \pause y-y_0=λ(x-x_0)
 \end{align*}
\end{frame}

\begin{frame}{Εξισώση ευθείας 2 (από δύο σημεία)}
 Ας είναι δύο σημεία $(x_1,y_1)$ και $(x_2,y_2)$. Αν $x_1\ne x_2$... \pause

 $λ=\frac{y_2-y_1}{x_2-x_1}$... \pause

 και έχουμε κλίση και σημείο (κοίτα προηγούμενη διαφάνεια)
\end{frame}

\begin{frame}{Εξισώση ευθείας 3 (δεν έχει κλίση)}
 Εύκολο?
\end{frame}

\section{Ασκήσεις}

\begin{frame}
 \vfill
 \centering
 \begin{beamercolorbox}[sep=8pt,center,shadow=true,rounded=true]{title}
  \usebeamerfont{title}Ασκήσεις
 \end{beamercolorbox}
 \vfill
\end{frame}

\subsection{Άσκηση 1}
\begin{frame}[label=Άσκηση1]{Εξάσκηση 1}
 Να βρείτε το συντελεστή διαύθυνσης $λ$ μιας ευθείας η οποία:
 \begin{enumerate}
  \item<1-> σχηματίζει με τον άξονα $x'x$ γωνία $ω=\frac{π}{3}$
  \item<2-> είναι παράλληλη στο διάνυσμα $\vec{α}=(2,-4)$
  \item<3-> διέρχεται από τα σημεία $Α(1,3)$ και $Β(3,6)$
 \end{enumerate}

 % \hyperlink{Λύση1}{\beamerbutton{Λύση}}
\end{frame}

\subsection{Άσκηση 2}
\begin{frame}[label=Άσκηση2]{Εξάσκηση 2}
 Να βρείτε τη γωνία που σχηματίζουν με τον άξονα $x'x$ οι ευθείες που διέρχονται από τα σημεία
 \begin{enumerate}
  \item<1-> $Α(1,0)$ και $Β(2,\sqrt{3})$
  \item<2-> $Α(2,3)$ και $Β(1,3)$
 \end{enumerate}

 % \hyperlink{Λύση2}{\beamerbutton{Λύση}}
\end{frame}

\subsection{Άσκηση 3}
\begin{frame}[label=Άσκηση3]{Εξάσκηση 3}
 Να βρείτε τον συντελεστή διεύθυνσης $λ$ μιας ευθείας $ε$, η οποία:
 \begin{enumerate}
  \item<1-> είναι παράλληλη στην ευθεία $ε_1$ που σχηματίζει με τον άξονα $x'x$ γωνία $ω=120^{\circ}$
  \item<2-> είναι κάθετη στην ευθεία $ε_2$ που διέρχεται από τα σημεία $Α(2,3)$ και $Β(3,5)$
 \end{enumerate}

 % \hyperlink{Λύση3}{\beamerbutton{Λύση}}
\end{frame}

\subsection{Άσκηση 4}
\begin{frame}[label=Άσκηση4]{Εξάσκηση 4}
 Έστω η ευθεία $ε$ που σχηματίζει με τον άξονα $x'x$ γωνία $ω=45^{\circ}$ και η ευθεία $ζ$ που διέρχεται από τα σημεία $Α(3,α)$ και $Β(5,3α-2)$. Να βρείτε την τιμή του $α$, ώστε:
 \begin{enumerate}
  \item<1-> Οι ευθείες $ε$ και $ζ$ να είναι παράλληλες
  \item<2-> Οι ευθείες $ε$ και $ζ$ να είναι κάθετες
 \end{enumerate}

 % \hyperlink{Λύση4}{\beamerbutton{Λύση}}
\end{frame}

\subsection{Άσκηση 5}
\begin{frame}[label=Άσκηση5]{Εξάσκηση 5}
 Θεωρούμε την ευθεία $ε$ που διέρχεται από το σημείο $Α(1,2)$ και έχει συντελεστή διεύθυνσης $λ=3$. Να βρείτε:
 \begin{enumerate}
  \item<1-> Την εξίσωση της ευθείας $ε$
  \item<2-> Την τιμή του $λ$, για την οποία το σημείο $Μ(λ-1,2λ)$ ανήκει στην ευθεία $ε$.
 \end{enumerate}

 % \hyperlink{Λύση5}{\beamerbutton{Λύση}}
\end{frame}

\subsection{Άσκηση 6}
\begin{frame}[label=Άσκηση6]{Εξάσκηση 6}
 Να βρείτε την εξίσωση της ευθείας $ε$ που διέρχεται από το σημείο $Α(3,2)$ και:
 \begin{enumerate}
  \item<1-> σχηματίζει με τον άξονα $x'x$ γωνία $ω=45^{\circ}$
  \item<2-> είναι παράλληλη στο διάνυσμα $\vec{α}=(2,-4)$
  \item<3-> είναι κάθετη στην ευθεία $ζ$ με συντελεστή διεύθυνσης $-\frac{1}{2}$
 \end{enumerate}

 % \hyperlink{Λύση6}{\beamerbutton{Λύση}}
\end{frame}

\subsection{Άσκηση 7}
\begin{frame}[label=Άσκηση7]{Εξάσκηση 7}
 Έστω μία ευθεία $ε$ που διέρχεται από το σημείο $Μ(α,2α+1)$ και έχει συντελεστή διεύθυνσης $λ=1$.
 \begin{enumerate}
  \item<1-> Να βρείτε την εξίσωση της ευθείας $ε$
  \item<2-> Αν επιπλέον η ευθεία $ε$ διέρχεται από το σημείο $Ν(1,-2)$, να βρείτε:
   \begin{enumerate}
    \item<2-> την τιμή του $α$
    \item<3-> τα σημεία τομής της ευθείας $ε$ με τους άξονες και στη συνέχεια να τη σχεδιάσετε
    \item<4-> το εμβαδό του τριγώνου που σχηματίζεται από την ευθεία $ε$ και τους άξονες
   \end{enumerate}
 \end{enumerate}

 %\hyperlink{Λύση7}{\beamerbutton{Λύση}}
\end{frame}

\subsection{Άσκηση 8}
\begin{frame}[label=Άσκηση8]{Εξάσκηση 8}
 Δίνεται τρίγωνο $ΑΒΓ$ με $Α(2,-3)$, $Β(1,5)$ και $Γ(2,3)$. Να βρείτε την εξίσωση:
 \begin{enumerate}
  \item<1-> της ευθείας $ε$ που διέρχεται από το σημείο $Α$ και είναι παράλληλη στην ευθεία $ΒΓ$
  \item<2-> του ύψους $ΑΔ$
  \item<3-> της διαμέσου $ΒΜ$
 \end{enumerate}

 %\hyperlink{Λύση8}{\beamerbutton{Λύση}}
\end{frame}

\subsection{Άσκηση 9}
\begin{frame}[label=Άσκηση9]{Εξάσκηση 9}
 Δίνονται τα σημεία $Α(1,4)$ και $Β(3,-6)$. Να βρείτε την εξίσωση της μεσοκαθέτου $ε$ του τμήματος $ΑΒ$

 %\hyperlink{Λύση9}{\beamerbutton{Λύση}}
\end{frame}

\subsection{Άσκηση 10}
\begin{frame}[label=Άσκηση10]{Εξάσκηση 10}
 Να βρείτε την εξίσωση της ευθείας που διέρχεται από τα σημεία:
 \begin{enumerate}
  \item<1-> $Α(3,2)$ και $Β(-1,6)$
  \item<2-> $Γ(5,-3)$ και $Δ(5,-4)$
 \end{enumerate}

 %\hyperlink{Λύση10}{\beamerbutton{Λύση}}
\end{frame}

\subsection{Άσκηση 11}
\begin{frame}[label=Άσκηση11]{Εξάσκηση 11}
 Δίνεται παραλληλόγραμμο $ΑΒΓΔ$ με $Α(2,5)$, $Β(1,7)$ και $Γ(4,1)$. Να βρείτε την εξίσωση της διαγωνίου $ΒΔ$.

 %\hyperlink{Λύση11}{\beamerbutton{Λύση}}
\end{frame}

\subsection{Άσκηση 12}
\begin{frame}[label=Άσκηση12]{Εξάσκηση 12}
 Να βρείτε την εξίσωση της ευθείας $ε$ όταν:
 \begin{enumerate}
  \item<1-> η ευθεία $ε$ τέμνει τον άξονα $y'y$ στο σημείο $Α(0,-3)$ και έχει συντελεστή διεύθυνσης $λ=2$
  \item<2-> η ευθεία $ε$ διέρχεται από την αρχή των αξόνων και έχει συντελεστή διεύθυνσης $λ=-\frac{2}{3}$
  \item<3-> η ευθεία $ε$ διέρχεται από τα σημεία $Α(-1,4)$ και $Β(λ^2,4)$
 \end{enumerate}

 %\hyperlink{Λύση12}{\beamerbutton{Λύση}}
\end{frame}

\subsection{Άσκηση 13}
\begin{frame}[label=Άσκηση13]{Εξάσκηση 13}
 Δίνονται οι ευθείες $ε:y=\frac{x}{2}-1$ και $ζ:(|μ|-2)x-5$. Να βρείτε τις τιμές του $μ$ ώστε η ευθεία να είναι:
 \begin{enumerate}
  \item<1-> παράλληλη στην ευθεία $ζ$
  \item<2-> κάθετη στην ευθεία $ζ$
 \end{enumerate}

 %\hyperlink{Λύση13}{\beamerbutton{Λύση}}
\end{frame}

\subsection{Άσκηση 14}
\begin{frame}[label=Άσκηση14]{Εξάσκηση 14}
 Να βρείτε την εξίσωση της ευθείας $ζ$ που διέρχεται από το σημείο $Α(-1,2)$ και:
 \begin{enumerate}
  \item<1-> είναι παράλληλη στην ευθεία $ε_1:y=3x+1$
  \item<2-> είναι κάθετη στην ευθεία $ε_2:y=-2x+3$
 \end{enumerate}

 %\hyperlink{Λύση14}{\beamerbutton{Λύση}}
\end{frame}

\subsection{Άσκηση 15}
\begin{frame}[label=Άσκηση15]{Εξάσκηση 15}
 Δίνονται οι ευθείες $ε_1:y=2x-1$ και $ε_2:y=x+1$.
 \begin{enumerate}
  \item<1-> Να βρείτε το σημείο τομής $Μ$ των ευθειών $ε_1$ και $ε_2$
  \item<2-> Να βρείτε την εξίσωση της ευθείας $ε$ που διέρχεται από το σημείο τομής των ευθειών $ε_1$ και $ε_2$ και σχηματίζει με τον άξονα $x'x$ γωνία $ω=135^{\circ}$
  \item<3-> Να δείξετε ότι οι ευθείες $ε_1$, $ε_2$ και $ζ:y=5x-7$ συντρέχουν
 \end{enumerate}

 %\hyperlink{Λύση15}{\beamerbutton{Λύση}}
\end{frame}

\subsection{Άσκηση 16}
\begin{frame}[label=Άσκηση16]{Εξάσκηση 16}
 Δίνεται τρίγωνο $ΑΒΓ$ με $Γ(4,3)$. Αν η εξίσωση της ευθείας $ΑΒ$ είναι $y=2x+1$ και του ύψους $ΑΔ$ είναι $y=x-1$, να βρείτε τις συντεταγμένες των σημείων $Α$ και $Β$

 %\hyperlink{Λύση16}{\beamerbutton{Λύση}}
\end{frame}

\subsection{Άσκηση 17}
\begin{frame}[label=Άσκηση17]{Εξάσκηση 17}
 Να βρείτε το πλησιέστερο σημείο της ευθείας $ε:y=-2x+1$ από την αρχή των αξόνων και στη συνέχεια την ελάχιστη απόσταση του σημείο $Ο$ από τα σημεία της ευθείας $ε$

 %\hyperlink{Λύση17}{\beamerbutton{Λύση}}
\end{frame}

\subsection{Άσκηση 18}
\begin{frame}[label=Άσκηση18]{Εξάσκηση 18}
 Να βρείτε το συμμετρικό σημείο του σημείου $Α(5,4)$ ως προς την ευθεία $ε:y=-4x+7$

 %\hyperlink{Λύση18}{\beamerbutton{Λύση}}
\end{frame}

\subsection{Άσκηση 19}
\begin{frame}[label=Άσκηση19]{Εξάσκηση 19}
 Δίνεται τρίγωνο $ΑΒΓ$ με $Β(1,2)$. Το ύψος και η διάμεσος από μία κορυφή του τριγώνου έχουν εξισώσεις $y=\frac{1}{2}x+\frac{1}{2}$ και $y=x$. Να βρείτε τις άλλες κορυφές και το βαρύκεντρο του τριγώνου

 %\hyperlink{Λύση19}{\beamerbutton{Λύση}}
\end{frame}

\subsection{Άσκηση 20}
\begin{frame}[label=Άσκηση20 ]{Εξάσκηση 20}
 Δίνεται τρίγωνο $ΑΒΓ$ με $Α(1,2)$, $ΒΓ:y=2x+5$ και η διάμεσος $ΒΜ$ έχει εξίσωση $y=\frac{1}{2}x-\frac{1}{2}$. Να βρείτε:
 \begin{enumerate}
  \item<1-> τις συντεταγμένες του σημείου $Γ$
  \item<2-> την εξίσωση του ύψους $ΓΔ$
 \end{enumerate}

 %\hyperlink{Λύση20 }{\beamerbutton{Λύση}}
\end{frame}

\subsection{Άσκηση 21}
\begin{frame}[label=Άσκηση21]{Εξάσκηση 21}
 Δίνονται τα σημεία $Α(-2,2)$ και $Β(3,1)$. Να βρείτε το σημείο $Μ$ της ευθείας $ε:y=x+3$, τέτοιο ώστε το τρίγωνο $ΑΜΒ$ να είναι ορθογώνιο στην κορυφή $Μ$

 %\hyperlink{Λύση21}{\beamerbutton{Λύση}}
\end{frame}

\subsection{Άσκηση 22}
\begin{frame}[label=Άσκηση22]{Εξάσκηση 22}
 Δίνεται τρίγωνο $ΑΒΓ$ με $ΑΒ:y=2x$ και $ΑΓ:y=3x-1$. Αν το σημείο $Μ(1,0)$ είναι μέσον της πλευράς $ΒΓ$

 %\hyperlink{Λύση22}{\beamerbutton{Λύση}}
\end{frame}

\subsection{Άσκηση 23}
\begin{frame}[label=Άσκηση23]{Εξάσκηση 23}
 Θεωρούμε το σημείο $Α(2,1)$ και το συμμετρικό του $Α'$ ως προς τον άξονα $x'x$. Να βρείτε το γεωμετρικό τόπο των σημείων $Μ$ για τα οποία ισχύει

 $$\overrightarrow{ΟΜ}\cdot\overrightarrow{ΟΑ}+\overrightarrow{ΟΜ'}\cdot\overrightarrow{ΟΑ'}=2$$

 όπου $Μ'$ το συμμετρικό του $Μ$ ως προς τον άξονα $x'x$

 %\hyperlink{Λύση23}{\beamerbutton{Λύση}}
\end{frame}

\subsection{Άσκηση 24}
\begin{frame}[label=Άσκηση24]{Εξάσκηση 24}
 Να βρείτε το γεωμετρικό τόπο των σημείων $Μ(x,y)$ όταν:
 \begin{enumerate}
  \item<1-> $Μ(λ-1,2λ-3)$, $λ\in\mathbb{R}$
  \item<2-> $Μ(-3,λ+1)$, $λ\in\mathbb{R}$
  \item<3-> $Μ(λ^2+1,2)$, $λ\in\mathbb{R}$
  \item<4-> $Μ(-3,ημλ)$, $λ\in\mathbb{R}$
 \end{enumerate}

 %\hyperlink{Λύση24}{\beamerbutton{Λύση}}
\end{frame}

\subsection{Άσκηση 25}
\begin{frame}[label=Άσκηση25]{Εξάσκηση 25}
 Αν το σημείο $Μ(α,β)$ κινείται στην ευθεία $ε:y=2x-4$, να βρείτε πού κινείται το σημείο $Ν\left(   \frac{α}{2},\frac{β}{2}\right)$

 %\hyperlink{Λύση25}{\beamerbutton{Λύση}}
\end{frame}

\subsection{Άσκηση 26}
\begin{frame}[label=Άσκηση26]{Εξάσκηση 26}
 Να αποδείξετε ότι το σημείο $Μ(3-συν^2θ,1-ημ^2θ)$, $θ\in\mathbb{R}$, κινείται σε σταθερή ευθεία.

 %\hyperlink{Λύση26}{\beamerbutton{Λύση}}
\end{frame}

%
% \appendix
% \section{Λύσεις Ασκήσεων}
% \begin{frame}
%  \tableofcontents
% \end{frame}
%
% \subsection{Άσκηση 1}
% \begin{frame}[label=Λύση1]
%  Με θεώρημα ενδιαμέσων τιμών. Η συνάρτηση είναι συνεχής στο $[10,11]$ με $f(10)=1024$ και $f(11)=2048$. Αφού $2023\in (1024,2048)$ υπάρχει $x_0$...
%
%  \hyperlink{Άσκηση1}{\beamerbutton{Πίσω στην άσκηση}}
% \end{frame}
%
% \subsection{Άσκηση 2}
% \begin{frame}[label=Λύση2]
%  Με Bolzano ή με μέγιστης ελάχιστης τιμής και ΘΕΤ.
%
%  \begin{gather*}
%   f(3)<f(2)<f(1) \\
%   3f(3)<f(1)+f(2)+f(3)<3f(1) \\
%   f(3)<\frac{f(1)+f(2)+f(3)}{3}<f(1)
%  \end{gather*}
%
%  \hyperlink{Άσκηση2}{\beamerbutton{Πίσω στην άσκηση}}
% \end{frame}
%
% \subsection{Άσκηση 3}
% \begin{frame}[label=Λύση3]
%  Προφανές ελάχιστο στα $x_1=1$ και $x_2=3$. Ως συνεχής στο $[1,3]$ έχει σίγουρα ΚΑΙ μέγιστο στο $(1,3)$
%
%  \hyperlink{Άσκηση3}{\beamerbutton{Πίσω στην άσκηση}}
% \end{frame}
%
% \subsection{Άσκηση 4}
% \begin{frame}[label=Λύση4]
%  Η συνάρτηση `απόστασης` $f(x)-x$ είναι ορισμένη στο κλειστό διάστημα και έχει σίγουρα μέγιστο
%
%  \hyperlink{Άσκηση4}{\beamerbutton{Πίσω στην άσκηση}}
% \end{frame}
%
% \subsection{Άσκηση 5}
% \begin{frame}[label=Λύση5]
%  Όμοια με την Άσκηση 2
%
%  \hyperlink{Άσκηση5}{\beamerbutton{Πίσω στην άσκηση}}
% \end{frame}
%
% \subsection{Άσκηση 6}
% \begin{frame}[label=Λύση6]
%  \begin{enumerate}
%   \item Είναι γνησίως αύξουσα άρα $(f(+\infty),f(-\infty))$
%   \item Προφανώς $[f(0),f(1)]$...
%  \end{enumerate}
%
%  \hyperlink{Άσκηση6}{\beamerbutton{Πίσω στην άσκηση}}
% \end{frame}
%
\end{document}
