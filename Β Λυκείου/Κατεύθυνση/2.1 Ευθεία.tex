\documentclass[greek]{beamer}
\usepackage{amsmath,amsthm} % needed for mathematics
\usepackage{unicode-math}
\usepackage{xltxtra}
\usepackage{graphicx}
\usetheme{CambridgeUS}
\usecolortheme{seagull}
\usepackage{hyperref}
\usepackage{ulem} % underline words package
\usepackage{xgreek}

\usepackage{pgfpages} 
\usepackage{tikz} % package for shapes and more
%\setbeameroption{show notes on second screen}
%\setbeameroption{show only notes}

\setsansfont{Calibri} % it is said that Calibri is the proper font for reading difficulties

\usepackage{multicol} % package for two or more columns

\usepackage{appendixnumberbeamer} % remove page numbering in appendix

\usepackage{polynom} % polynomial divisions package

\usepackage{pgffor} % macros

\setbeamercovered{highly dynamic}
\setbeamertemplate{navigation symbols}{}

\newcounter{askisi} % enviroment for exercises
\newenvironment{askisi}
{
 \refstepcounter{askisi}\par
 \subsection{Άσκηση \theaskisi}
 \begin{frame}[label=Άσκηση\theaskisi,t]{Εξάσκηση \theaskisi}
}{
 \end{frame}
}
 
\newcounter{lisi} % enviroment for solutions
\newenvironment{lisi}
{
 \refstepcounter{lisi}\par
 \subsection{Άσκηση \thelisi}
 \begin{frame}[label=Λύση\thelisi,t]{Λύση \thelisi}
}{
 \end{frame} 
}

\title{Ευθεία}
\subtitle{Εξίσωση Ευθείας}
\author[Λόλας]{Κωνσταντίνος Λόλας}
\date{}

\begin{document}

\begin{frame}
  \titlepage
\end{frame}

\section{Θεωρία}

\begin{frame}{Το μεγάλο ταξίδι}
  \begin{itemize}[<+->]
    \item Ορισμός
    \item Εξίσωση
    \item Γενική Εξίσωση, to rule them all!
    \item Ελάχιστος συνδυασμός με διανύσματα, shame!
    \item Εύρεση εξίσωσης από κάθε περίπτωση, brace yourselfs!
    \item 2 νέοι τύποι (απόστασης και εμβαδού)
  \end{itemize}
\end{frame}

\begin{frame}{Γνωστά ή Άγνωστα νερά?}
  Λέξεις κλειδιά
  \begin{itemize}
    \item Κλίση
    \item Συντελεστής διεύθυνσης
    \item $εφθ$
    \item $α$
    \item Σημεία
    \item Παραλληλία
    \item Καθετότητα
    \item Σημεία τομής...
  \end{itemize}
  είναι μερικά που θυμάμαι!
\end{frame}

\begin{frame}{Γραμμές, γραμμές παντού}
  \begin{itemize}[<+->]
    \item Τι είναι γραμμή?
    \item Γραφικά ή Αλγεβρικά?
  \end{itemize}
\end{frame}

\begin{frame}{Γραφικά}
  Εύκολο!
\end{frame}

\begin{frame}{Αλγεβρικά}
  \begin{block}{Ορισμός γραμμής}
    Μία εξίσωση με τουλάχιστον έναν άγνωστο
  \end{block}
  \begin{block}{Σημείο στη γραμμή}
    Κάθε σημείο που επαληθεύει την εξίσωση
  \end{block}
\end{frame}

\begin{frame}{Ας φτιάξουμε απλές γραμμές}
  \begin{itemize}[<+->]
    \item $y=2$
    \item $x=1$
    \item $x-y=0$
    \item $y=2x$
  \end{itemize}
\end{frame}

\begin{frame}{Ορισμοί}
  \begin{block}{Γωνία Ευθείας}
    Ονομάζουμε \emph{γωνία της ευθείας με τον άξονα $x'x$}, την γωνία που σχηματίζει ο $x'x$ όταν στραφεί αντίστροφα με τους δείκτες του ρολογιού έως ότου συμπέσει με την ευθεία
  \end{block}
  \begin{block}{Συντελεστής Διεύθυνσης Ευθείας}
    Ονομάζουμε \emph{συντελεστή διεύθυνσης} (ή κλίση) της ευθείας την εφαπτομένη της γωνίας της ευθείας με τον $x'x$
  \end{block}
\end{frame}

\begin{frame}{Ξεπηδούν οι απορίες}
  \begin{itemize}[<+->]
    \item Τι τιμές παίρνει μία γωνία
    \item Τι τιμές παίρνει η κλίση
    \item Πότε είναι παράλληλες δύο ευθείες
    \item Πότε είναι παράλληλη μία ευθεία με ένα διάνυσμα
    \item Ποιά άλλα διανύσματα είναι παράλληλα με την ευθεία?
    \item Πότε είναι κάθετες δύο ευθείες? μην βιάζεστε!!!!!
  \end{itemize}
\end{frame}

\begin{frame}{Λίγη ιστορία}
  \begin{block}{Κλίση διανύσματος}
    $λ=\frac{y_2-y_1}{x_2-x_1}$
  \end{block}
\end{frame}

\begin{frame}{Εξισώση ευθείας 1 (από κλίση και σημείο)}
  Ας θεωρήσουμε ότι \emph{υπάρχει συντελεστής διεύθυνσης $λ$} και ας έχουμε γνωστό \emph{ένα σημείο} $(x_0,y_0)$. Κάθε σημείο $(x,y)$ που ανήκει στην ευθεία θα έχει με το γνωστό σημείο κλίση $λ$. Άρα
  \begin{align*}
    \frac{y-y_0}{x-x_0} & =λ        \\
    y-y_0               & =λ(x-x_0)
  \end{align*}
\end{frame}

\begin{frame}{Εξισώση ευθείας 2 (από δύο σημεία)}
  Ας είναι δύο σημεία $(x_1,y_1)$ και $(x_2,y_2)$. Αν $x_1\ne x_2$... \pause

  $λ=\frac{y_2-y_1}{x_2-x_1}$... \pause

  και έχουμε κλίση και σημείο (κοίτα προηγούμενη διαφάνεια)
\end{frame}

\begin{frame}{Εξισώση ευθείας 3 (δεν έχει κλίση)}
  Εύκολο?
\end{frame}

\begin{frame}[noframenumbering]
  Στο moodle θα βρείτε τις ασκήσεις που πρέπει να κάνετε, όπως και αυτή τη παρουσίαση
\end{frame}

\section{Ασκήσεις}

\begin{frame}[noframenumbering]
  \vfill
  \centering
  \begin{beamercolorbox}[sep=8pt,center,shadow=true,rounded=true]{title}
    \usebeamerfont{title}Ασκήσεις
  \end{beamercolorbox}
  \vfill
\end{frame}

\begin{askisi}
  Να βρείτε το συντελεστή διεύθυνσης $λ$ μιας ευθείας η οποία:
  \begin{enumerate}[<+->]
    \item σχηματίζει με τον άξονα $x'x$ γωνία $ω=\frac{π}{3}$
    \item είναι παράλληλη στο διάνυσμα $\vec{α}=(2,-4)$
    \item διέρχεται από τα σημεία $Α(1,3)$ και $Β(3,6)$
  \end{enumerate}
\end{askisi}

\begin{askisi}
  Να βρείτε τη γωνία που σχηματίζουν με τον άξονα $x'x$ οι ευθείες που διέρχονται από τα σημεία
  \begin{enumerate}[<+->]
    \item $Α(1,0)$ και $Β(2,\sqrt{3})$
    \item $Α(2,3)$ και $Β(1,3)$
  \end{enumerate}
\end{askisi}

\begin{askisi}
  Να βρείτε τον συντελεστή διεύθυνσης $λ$ μιας ευθείας $ε$, η οποία:
  \begin{enumerate}[<+->]
    \item είναι παράλληλη στην ευθεία $ε_1$ που σχηματίζει με τον άξονα $x'x$ γωνία $ω=120^{\circ}$
    \item είναι κάθετη στην ευθεία $ε_2$ που διέρχεται από τα σημεία $Α(2,3)$ και $Β(3,5)$
  \end{enumerate}
\end{askisi}

\begin{askisi}
  Έστω η ευθεία $ε$ που σχηματίζει με τον άξονα $x'x$ γωνία $ω=45^{\circ}$ και η ευθεία $ζ$ που διέρχεται από τα σημεία $Α(3,α)$ και $Β(5,3α-2)$. Να βρείτε την τιμή του $α$, ώστε:
  \begin{enumerate}[<+->]
    \item Οι ευθείες $ε$ και $ζ$ να είναι παράλληλες
    \item Οι ευθείες $ε$ και $ζ$ να είναι κάθετες
  \end{enumerate}
\end{askisi}

\begin{askisi}
  Θεωρούμε την ευθεία $ε$ που διέρχεται από το σημείο $Α(1,2)$ και έχει συντελεστή διεύθυνσης $λ=3$. Να βρείτε:
  \begin{enumerate}[<+->]
    \item Την εξίσωση της ευθείας $ε$
    \item Την τιμή του $λ$, για την οποία το σημείο $Μ(λ-1,2λ)$ ανήκει στην ευθεία $ε$.
  \end{enumerate}
\end{askisi}

\begin{askisi}
  Να βρείτε την εξίσωση της ευθείας $ε$ που διέρχεται από το σημείο $Α(3,2)$ και:
  \begin{enumerate}[<+->]
    \item σχηματίζει με τον άξονα $x'x$ γωνία $ω=45^{\circ}$
    \item είναι παράλληλη στο διάνυσμα $\vec{α}=(2,-4)$
    \item είναι κάθετη στην ευθεία $ζ$ με συντελεστή διεύθυνσης $-\frac{1}{2}$
  \end{enumerate}
\end{askisi}

\begin{askisi}
  Έστω μία ευθεία $ε$ που διέρχεται από το σημείο $Μ(α,2α+1)$ και έχει συντελεστή διεύθυνσης $λ=1$.
  \begin{enumerate}[<+->]
    \item Να βρείτε την εξίσωση της ευθείας $ε$
    \item Αν επιπλέον η ευθεία $ε$ διέρχεται από το σημείο $Ν(1,-2)$, να βρείτε:
          \begin{enumerate}[<+(-1)->]
            \item την τιμή του $α$
            \item τα σημεία τομής της ευθείας $ε$ με τους άξονες και στη συνέχεια να τη σχεδιάσετε
            \item το εμβαδό του τριγώνου που σχηματίζεται από την ευθεία $ε$ και τους άξονες
          \end{enumerate}
  \end{enumerate}
\end{askisi}

\begin{askisi}
  Δίνεται τρίγωνο $ΑΒΓ$ με $Α(2,-3)$, $Β(1,5)$ και $Γ(2,3)$. Να βρείτε την εξίσωση:
  \begin{enumerate}
    \item της ευθείας $ε$ που διέρχεται από το σημείο $Α$ και είναι παράλληλη στην ευθεία $ΒΓ$
    \item του ύψους $ΑΔ$
    \item της διαμέσου $ΒΜ$
  \end{enumerate}
\end{askisi}

\begin{askisi}
  Δίνονται τα σημεία $Α(1,4)$ και $Β(3,-6)$. Να βρείτε την εξίσωση της μεσοκαθέτου $ε$ του τμήματος $ΑΒ$
\end{askisi}

\begin{askisi}
  Να βρείτε την εξίσωση της ευθείας που διέρχεται από τα σημεία:
  \begin{enumerate}
    \item $Α(3,2)$ και $Β(-1,6)$
    \item $Γ(5,-3)$ και $Δ(5,-4)$
  \end{enumerate}
\end{askisi}

\begin{askisi}
  Δίνεται παραλληλόγραμμο $ΑΒΓΔ$ με $Α(2,5)$, $Β(1,7)$ και $Γ(4,1)$. Να βρείτε την εξίσωση της διαγωνίου $ΒΔ$.
\end{askisi}

\begin{askisi}
  Να βρείτε την εξίσωση της ευθείας $ε$ όταν:
  \begin{enumerate}
    \item η ευθεία $ε$ τέμνει τον άξονα $y'y$ στο σημείο $Α(0,-3)$ και έχει συντελεστή διεύθυνσης $λ=2$
    \item η ευθεία $ε$ διέρχεται από την αρχή των αξόνων και έχει συντελεστή διεύθυνσης $λ=-\frac{2}{3}$
    \item η ευθεία $ε$ διέρχεται από τα σημεία $Α(-1,4)$ και $Β(λ^2,4)$
  \end{enumerate}
\end{askisi}

\begin{askisi}
  Δίνονται οι ευθείες $ε:y=\frac{x}{2}-1$ και $ζ:(|μ|-2)x-5$. Να βρείτε τις τιμές του $μ$ ώστε η ευθεία να είναι:
  \begin{enumerate}
    \item παράλληλη στην ευθεία $ζ$
    \item κάθετη στην ευθεία $ζ$
  \end{enumerate}
\end{askisi}

\begin{askisi}
  Να βρείτε την εξίσωση της ευθείας $ζ$ που διέρχεται από το σημείο $Α(-1,2)$ και:
  \begin{enumerate}
    \item είναι παράλληλη στην ευθεία $ε_1:y=3x+1$
    \item είναι κάθετη στην ευθεία $ε_2:y=-2x+3$
  \end{enumerate}
\end{askisi}

\begin{askisi}
  Δίνονται οι ευθείες $ε_1:y=2x-1$ και $ε_2:y=x+1$.
  \begin{enumerate}
    \item Να βρείτε το σημείο τομής $Μ$ των ευθειών $ε_1$ και $ε_2$
    \item Να βρείτε την εξίσωση της ευθείας $ε$ που διέρχεται από το σημείο τομής των ευθειών $ε_1$ και $ε_2$ και σχηματίζει με τον άξονα $x'x$ γωνία $ω=135^{\circ}$
    \item Να δείξετε ότι οι ευθείες $ε_1$, $ε_2$ και $ζ:y=5x-7$ συντρέχουν
  \end{enumerate}
\end{askisi}

\begin{askisi}
  Δίνεται τρίγωνο $ΑΒΓ$ με $Γ(4,3)$. Αν η εξίσωση της ευθείας $ΑΒ$ είναι $y=2x+1$ και του ύψους $ΑΔ$ είναι $y=x-1$, να βρείτε τις συντεταγμένες των σημείων $Α$ και $Β$
\end{askisi}

\begin{askisi}
  Να βρείτε το πλησιέστερο σημείο της ευθείας $ε:y=-2x+1$ από την αρχή των αξόνων και στη συνέχεια την ελάχιστη απόσταση του σημείο $Ο$ από τα σημεία της ευθείας $ε$
\end{askisi}

\begin{askisi}
  Να βρείτε το συμμετρικό σημείο του σημείου $Α(5,4)$ ως προς την ευθεία $ε:y=-4x+7$
\end{askisi}

\begin{askisi}
  Δίνεται τρίγωνο $ΑΒΓ$ με $Β(1,2)$. Το ύψος και η διάμεσος από μία κορυφή του τριγώνου έχουν εξισώσεις $y=\frac{1}{2}x+\frac{1}{2}$ και $y=x$. Να βρείτε τις άλλες κορυφές και το βαρύκεντρο του τριγώνου
\end{askisi}

\begin{askisi}
  Δίνεται τρίγωνο $ΑΒΓ$ με $Α(1,2)$, $ΒΓ:y=2x+5$ και η διάμεσος $ΒΜ$ έχει εξίσωση $y=\frac{1}{2}x-\frac{1}{2}$. Να βρείτε:
  \begin{enumerate}
    \item τις συντεταγμένες του σημείου $Γ$
    \item την εξίσωση του ύψους $ΓΔ$
  \end{enumerate}
\end{askisi}

\begin{askisi}
  Δίνονται τα σημεία $Α(-2,2)$ και $Β(3,1)$. Να βρείτε το σημείο $Μ$ της ευθείας $ε:y=x+3$, τέτοιο ώστε το τρίγωνο $ΑΜΒ$ να είναι ορθογώνιο στην κορυφή $Μ$
\end{askisi}

\begin{askisi}
  Δίνεται τρίγωνο $ΑΒΓ$ με $ΑΒ:y=2x$ και $ΑΓ:y=3x-1$. Αν το σημείο $Μ(1,0)$ είναι μέσον της πλευράς $ΒΓ$
\end{askisi}

\begin{askisi}
  Θεωρούμε το σημείο $Α(2,1)$ και το συμμετρικό του $Α'$ ως προς τον άξονα $x'x$. Να βρείτε το γεωμετρικό τόπο των σημείων $Μ$ για τα οποία ισχύει

  $$\overrightarrow{ΟΜ}\cdot\overrightarrow{ΟΑ}+\overrightarrow{ΟΜ'}\cdot\overrightarrow{ΟΑ'}=2$$

  όπου $Μ'$ το συμμετρικό του $Μ$ ως προς τον άξονα $x'x$
\end{askisi}

\begin{askisi}
  Να βρείτε το γεωμετρικό τόπο των σημείων $Μ(x,y)$ όταν:
  \begin{enumerate}
    \item $Μ(λ-1,2λ-3)$, $λ\in\mathbb{R}$
    \item $Μ(-3,λ+1)$, $λ\in\mathbb{R}$
    \item $Μ(λ^2+1,2)$, $λ\in\mathbb{R}$
    \item $Μ(-3,ημλ)$, $λ\in\mathbb{R}$
  \end{enumerate}
\end{askisi}

\begin{askisi}
  Αν το σημείο $Μ(α,β)$ κινείται στην ευθεία $ε:y=2x-4$, να βρείτε πού κινείται το σημείο $Ν\left(  \frac{α}{2},\frac{β}{2}\right)$
\end{askisi}

\begin{askisi}
  Να αποδείξετε ότι το σημείο $Μ(3-συν^2θ,1-ημ^2θ)$, $θ\in\mathbb{R}$, κινείται σε σταθερή ευθεία.
\end{askisi}

\end{document}
