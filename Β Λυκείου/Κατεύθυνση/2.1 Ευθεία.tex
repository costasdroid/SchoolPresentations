\documentclass[greek]{beamer}
%\usepackage{fontspec}
\usepackage{amsmath,amsthm}
\usepackage{unicode-math}
\usepackage{xltxtra}
\usepackage{graphicx}
\usetheme{CambridgeUS}
\usecolortheme{seagull}
\usepackage{hyperref}
\usepackage{ulem}
\usepackage{xgreek}

\usepackage{pgfpages}
\usepackage{tikz}
\usepackage{tkz-tab}
%\setbeameroption{show notes on second screen}
%\setbeameroption{show only notes}

\setsansfont{Noto Serif}

\usepackage{multicol}

\usepackage{appendixnumberbeamer}

\setbeamercovered{transparent}
\beamertemplatenavigationsymbolsempty

\title{Ευθεία}
\subtitle{Εξίσωση Ευθείας}
\author[Λόλας]{Κωνσταντίνος Λόλας}
\date{}

\begin{document}

\begin{frame}
 \titlepage
\end{frame}

\section{Θεωρία}
\begin{frame}{Το μεγάλο ταξίδι}
 \begin{itemize}
  \item<1-> Ορισμός
  \item<2-> Εξίσωση
  \item<3-> Γενική Εξίσωση, to rule them all!
  \item<4-> Ελάχιστος συνδυασμός με διανύσματα, shame!
  \item<5-> Εύρεση εξίσωσης από κάθε περίπτωση, brace yourselfs!
  \item<6-> 2 νέοι τύποι (απόστασης και εμβαδού)
 \end{itemize}
\end{frame}

\begin{frame}{Γνωστά ή Άγνωστα νερά?}
 Λέξεις κλειδιά
 \begin{itemize}
  \item Κλίση
  \item Συντελεστής διεύθυνσης
  \item $εφθ$
  \item $α$
  \item Σημεία
  \item Παραλληλία
  \item Καθετότητα
  \item Σημεία τομής...
 \end{itemize}
 είναι μερικά που θυμάμαι!
\end{frame}

\begin{frame}{Γραμμές, γραμμές παντού}
 \begin{itemize}
  \item<1-> Τι είναι γραμμή?
  \item<2-> Γραφικά ή Αλγεβρικά?
 \end{itemize}
\end{frame}

\begin{frame}{Γραφικά}
 Εύκολο!
\end{frame}

\begin{frame}{Αλγεβρικά}
 \begin{block}{Ορισμός γραμμής}
  Μία εξίσωση με τουλάχιστον έναν άγνωστο
 \end{block}
 \begin{block}{Σημείο στη γραμμή}
  Κάθε σημείο που επαληθεύει την εξίσωση
 \end{block}
\end{frame}

\begin{frame}{Ας φτιάξουμε απλές γραμμές}
 \begin{itemize}
  \item<1-> $y=2$
  \item<2-> $x=1$
  \item<3-> $x-y=0$
  \item<4-> $y=2x$
 \end{itemize}
\end{frame}

\begin{frame}{Ορισμοί}
 \begin{block}{Γωνία Ευθείας}
  Ονομάζουμε \emph{γωνία της ευθείας με τον άξονα $x'x$}, την γωνία που σχηματίζει ο $x'x$ όταν στραφεί αντίστροφα με τους δείκτες του ρολογιού έως ότου συμπέσει με την ευθεία
 \end{block}
 \begin{block}{Συντελεστής Διεύθυνσης Ευθείας}
  Ονομάζουμε \emph{συντελεστή διεύθυνσης} (ή κλίση) της ευθείας την εφαπτομένη της γωνίας της ευθείας με τον $x'x$
 \end{block}
\end{frame}

\begin{frame}{Ξεπηδούν οι απορίες}
 \begin{itemize}
  \item<2-> Τι τιμές παίρνει μία γωνία
  \item<3-> Τι τιμές παίρνει η κλίση
  \item<4-> Πότε είναι παράλληλες δύο ευθείες
  \item<5-> Πότε είναι παράλληλη μία ευθεία με ένα διάνυσμα
  \item<6-> Ποιά άλλα διανύσματα είναι παράλληλα με την ευθεία?
  \item<7-> Πότε είναι κάθετες δύο ευθείες? \pause μην βιάζεστε!!!!!
 \end{itemize}
\end{frame}

\begin{frame}{Λίγη ιστορία}
 \begin{block}{Κλίση διανύσματος}
  $λ=\frac{y_2-y_1}{x_2-x_1}$
 \end{block}
\end{frame}

\begin{frame}{Εξισώση ευθείας 1 (από κλίση και σημείο)}
 Ας θεωρήσουμε ότι \emph{υπάρχει συντελεστής διεύθυνσης $λ$} και ας έχουμε γνωστό \emph{ένα σημείο} $(x_0,y_0)$. Κάθε σημείο $(x,y)$ που ανήκει στην ευθεία θα έχει με το γνωστό σημείο κλίση $λ$. Άρα
 \begin{align*}
  \frac{y-y_0}{x-x_0}=λ \\
  \pause y-y_0=λ(x-x_0)
 \end{align*}
\end{frame}

\begin{frame}{Εξισώση ευθείας 2 (από δύο σημεία)}
 Ας είναι δύο σημεία $(x_1,y_1)$ και $(x_2,y_2)$. Αν $x_1\ne x_2$... \pause

 $λ=\frac{y_2-y_1}{x_2-x_1}$... \pause

 και έχουμε κλίση και σημείο (κοίτα προηγούμενη διαφάνεια)
\end{frame}

\begin{frame}{Εξισώση ευθείας 3 (δεν έχει κλίση)}
 Εύκολο?
\end{frame}

\section{Ασκήσεις}
\subsection{Άσκηση 1}
\begin{frame}[label=Άσκηση1]{Εξάσκηση 1}
 Να βρείτε το συντελεστή διαύθυνσης $λ$ μιας ευθείας η οποία:
 \begin{enumerate}
  \item<1-> σχηματίζει με τον άξονα $x'x$ γωνία $ω=\frac{π}{3}$
  \item<2-> είναι παράλληλη στο διάνυσμα $\vec{α}=(2,-4)$
  \item<3-> διέρχεται από τα σημεία $Α(1,3)$ και $Β(3,6)$
 \end{enumerate}

 % \hyperlink{Λύση1}{\beamerbutton{Λύση}}
\end{frame}

\subsection{Άσκηση 2}
\begin{frame}[label=Άσκηση2]{Εξάσκηση 2}
 Να βρείτε τη γωνία που σχηματίζουν με τον άξονα $x'x$ οι ευθείες που διέρχονται από τα σημεία
 \begin{enumerate}
  \item<1-> $Α(1,0)$ και $Β(2,\sqrt{3})$
  \item<2-> $Α(2,3)$ και $Β(1,3)$
 \end{enumerate}

 % \hyperlink{Λύση2}{\beamerbutton{Λύση}}
\end{frame}

\subsection{Άσκηση 3}
\begin{frame}[label=Άσκηση3]{Εξάσκηση 3}
 Να βρείτε τον συντελεστή διεύθυνσης $λ$ μιας ευθείας $ε$, η οποία:
 \begin{enumerate}
  \item<1-> είναι παράλληλη στην ευθεία $ε_1$ που σχηματίζει με τον άξονα $x'x$ γωνία $ω=120^{\circ}$
  \item<2-> είναι κάθετη στην ευθεία $ε_2$ που διέρχεται από τα σημεία $Α(2,3)$ και $Β(3,5)$
 \end{enumerate}

 % \hyperlink{Λύση3}{\beamerbutton{Λύση}}
\end{frame}

\subsection{Άσκηση 4}
\begin{frame}[label=Άσκηση4]{Εξάσκηση 4}
 Έστω η ευθεία $ε$ που σχηματίζει με τον άξονα $x'x$ γωνία $ω=45^{\circ}$ και η ευθεία $ζ$ που διέρχεται από τα σημεία $Α(3,α)$ και $Β(5,3α-2)$. Να βρείτε την τιμή του $α$, ώστε:
 \begin{enumerate}
   \item<1-> Οι ευθείες $ε$ και $ζ$ να είναι παράλληλες
   \item<2-> Οι ευθείες $ε$ και $ζ$ να είναι κάθετες
 \end{enumerate}

 % \hyperlink{Λύση4}{\beamerbutton{Λύση}}
\end{frame}

\subsection{Άσκηση 5}
\begin{frame}[label=Άσκηση5]{Εξάσκηση 5}
 Θεωρούμε την ευθεία $ε$ που διέρχεται από το σημείο $Α(1,2)$ και έχει συντελεστή διεύθυνσης $λ=3$. Να βρείτε:
 \begin{enumerate}
   \item<1-> Την εξίσωση της ευθείας $ε$
   \item<2-> Την τιμή του $λ$, για την οποία το σημείο $Μ(λ-1,2λ)$ ανήκει στην ευθεία $ε$.
 \end{enumerate}

 % \hyperlink{Λύση5}{\beamerbutton{Λύση}}
\end{frame}

\subsection{Άσκηση 6}
\begin{frame}[label=Άσκηση6]{Εξάσκηση 6}
 Να βρείτε την εξίσωση της ευθείας $ε$ που διέρχεται από το σημείο $Α(3,2)$ και:
 \begin{enumerate}
   \item<1-> σχηματίζει με τον άξονα $x'x$ γωνία $ω=45^{\circ}$
   \item<2-> είναι παράλληλη στο διάνυσμα $\vec{α}=(2,-4)$
   \item<3-> είναι κάθετη στην ευθεία $ζ$ με συντελεστή διεύθυνσης $-\frac{1}{2}$
 \end{enumerate}

 % \hyperlink{Λύση6}{\beamerbutton{Λύση}}
\end{frame}
%
% \appendix
% \section{Λύσεις Ασκήσεων}
% \begin{frame}
%  \tableofcontents
% \end{frame}
%
% \subsection{Άσκηση 1}
% \begin{frame}[label=Λύση1]
%  Με θεώρημα ενδιαμέσων τιμών. Η συνάρτηση είναι συνεχής στο $[10,11]$ με $f(10)=1024$ και $f(11)=2048$. Αφού $2023\in (1024,2048)$ υπάρχει $x_0$...
%
%  \hyperlink{Άσκηση1}{\beamerbutton{Πίσω στην άσκηση}}
% \end{frame}
%
% \subsection{Άσκηση 2}
% \begin{frame}[label=Λύση2]
%  Με Bolzano ή με μέγιστης ελάχιστης τιμής και ΘΕΤ.
%
%  \begin{gather*}
%   f(3)<f(2)<f(1) \\
%   3f(3)<f(1)+f(2)+f(3)<3f(1) \\
%   f(3)<\frac{f(1)+f(2)+f(3)}{3}<f(1)
%  \end{gather*}
%
%  \hyperlink{Άσκηση2}{\beamerbutton{Πίσω στην άσκηση}}
% \end{frame}
%
% \subsection{Άσκηση 3}
% \begin{frame}[label=Λύση3]
%  Προφανές ελάχιστο στα $x_1=1$ και $x_2=3$. Ως συνεχής στο $[1,3]$ έχει σίγουρα ΚΑΙ μέγιστο στο $(1,3)$
%
%  \hyperlink{Άσκηση3}{\beamerbutton{Πίσω στην άσκηση}}
% \end{frame}
%
% \subsection{Άσκηση 4}
% \begin{frame}[label=Λύση4]
%  Η συνάρτηση `απόστασης` $f(x)-x$ είναι ορισμένη στο κλειστό διάστημα και έχει σίγουρα μέγιστο
%
%  \hyperlink{Άσκηση4}{\beamerbutton{Πίσω στην άσκηση}}
% \end{frame}
%
% \subsection{Άσκηση 5}
% \begin{frame}[label=Λύση5]
%  Όμοια με την Άσκηση 2
%
%  \hyperlink{Άσκηση5}{\beamerbutton{Πίσω στην άσκηση}}
% \end{frame}
%
% \subsection{Άσκηση 6}
% \begin{frame}[label=Λύση6]
%  \begin{enumerate}
%   \item Είναι γνησίως αύξουσα άρα $(f(+\infty),f(-\infty))$
%   \item Προφανώς $[f(0),f(1)]$...
%  \end{enumerate}
%
%  \hyperlink{Άσκηση6}{\beamerbutton{Πίσω στην άσκηση}}
% \end{frame}
%
\end{document}
