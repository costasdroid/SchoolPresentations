\documentclass[greek]{beamer}
%\usepackage{fontspec}
\usepackage{amsmath,amsthm}
\usepackage{unicode-math}
\usepackage{xltxtra}
\usepackage{graphicx}
\usetheme{CambridgeUS}
\usecolortheme{seagull}
\usepackage{hyperref}
\usepackage{ulem}
\usepackage{xgreek}

\usepackage{pgfpages}
\usepackage{tikz}
%\setbeameroption{show notes on second screen}
%\setbeameroption{show only notes}

\setsansfont{Calibri}

\usepackage{multicol}

\usepackage{appendixnumberbeamer}

\usepackage{polynom}

\usepackage{pgffor}

\setbeamercovered{transparent}
\beamertemplatenavigationsymbolsempty

\title{Εκθετική Λογαριθμική Συνάρτηση}
\subtitle{Τα ωραία των λογαρίθμων}
\author[Λόλας]{Κωνσταντίνος Λόλας}
\date{}

\begin{document}

\begin{frame}
  \titlepage
\end{frame}

\section{Θεωρία}
\begin{frame}{Παρατήρηση 1}
  \begin{block}{Ιδιότητα 1 Λογάριθμου}
    \begin{itemize}
      \item<1-> $\log 54.324=\log 10\cdot 5.4324=\log 10+\log 5.4324$
      \item<2-> $\log 432.53=\log 100\cdot 4.3253 = \log 100 + \log 4.3253$
    \end{itemize}

  \end{block}
  \onslide<3->Άρα αν ξέραμε τους λογάριθμους από $1$ έως $9.999999...$ θα ξέραμε τους λογάριθμους όλων των αριθμών!
\end{frame}

\begin{frame}{Παρατήρηση 2}
  \begin{block}{Ιδιότητα 2 Λογάριθμου}

    $$\log 54.324\cdot 432.53 = \log 54.342 + \log 432.53 = $$

  \end{block}
  \onslide<3->Άρα αν ξέραμε τους λογάριθμους από $0$ έως $0.999999...$ θα ξέραμε τους λογάριθμους όλων των αριθμών!
\end{frame}

\begin{frame}{Σχέση με προηγούμενα?}
  Από τον ορισμό του λογάριθμου έχουμε
  $$\ln x=y\iff e^y=x$$
  \begin{itemize}
    \item Αν λοιπόν $f(x)=\ln x$ τότε τα σημεία της είναι τα $(x,\ln x)$
    \item Αντίστοιχα, αν $g(x)=e^x$ τότε τα σημεία της είναι τα $(x,e^x)$
  \end{itemize}
  Με απλή αντικατάσταση

  $$(x,e^x)=(\ln x, e^{\ln x})=(\ln x,x)$$
  \only<1> {Αυτό σημαίνει ότι...}

  \only<2> {Η γραφική παράσταση της $\ln x$ είναι συμμετρική της $e^x$ ως προς την ευθεία $y=x$}
\end{frame}

\begin{frame}{Ιδιότητες της $C_{\ln}$}
  \begin{itemize}
    \item Μονοτονία \only<2-> {$\rightarrow$ Γνησίως αύξουσα}
    \item Ακρότατα \only<3-> {$\rightarrow$ Δεν έχει}
    \item Ένα προς ένα? \only<4-> {$\rightarrow$ Ως γνησίως αύξουσα είναι!}
  \end{itemize}
\end{frame}

\begin{frame}{Και τελειώνοντας...}

\end{frame}

\section{Ασκήσεις}
\begin{askisi}
  Να βρείτε το πεδίο ορισμού των συναρτήσεων
  \begin{enumerate}
    \item<1-> $f(x)=\ln (e^x-1)$
    \item<2-> $f(x)=\ln \left( x-\dfrac{4}{x} \right) $
  \end{enumerate}

  % \hyperlink{Λύση1}{\beamerbutton{Λύση}}

\end{askisi}

\begin{askisi}
  Να Παραστήσετε γραφικά τις συναρτήσεις
  \begin{enumerate}
    \item<1-> $f(x)=\ln x+1$
    \item<2-> $f(x)=\ln (x-1)$
  \end{enumerate}

  % \hyperlink{Λύση2}{\beamerbutton{Λύση}}

\end{askisi}

\begin{askisi}
  Στο ίδιο σύστημα αξόνων να κάνετε τις γραφικές παραστάσεις των συναρτήσεων:
  \begin{enumerate}
    \item $y=x$
    \item $y=x+1$
    \item $y=x-1$
    \item $y=e^x$
    \item $y=\ln x$
  \end{enumerate}

  % \hyperlink{Λύση3}{\beamerbutton{Λύση}}

\end{askisi}

\begin{askisi}
  Να παραστήσετε γραφικά τις παρακάτω συναρτήσεις:
  \begin{enumerate}
    \item<1-> $g(x)=\ln \dfrac{1}{x}$
    \item<2-> $h(x)=|\ln x|$
    \item<3-> $φ(x)=\ln |x|$
  \end{enumerate}

  % \hyperlink{Λύση4}{\beamerbutton{Λύση}}

\end{askisi}

\begin{askisi}
  Έστω η συνάρτηση $f(x)=\ln \dfrac{5-x}{5+x}$.
  \begin{enumerate}
    \item<1-> Να βρείτε το πεδίο ορισμού της συνάρτησης $f$
    \item<2-> Να δείξετε ότι η συνάρτηση $f$ είναι περιττή
  \end{enumerate}

  %\hyperlink{Λύση5}{\beamerbutton{Λύση}}

\end{askisi}

\begin{askisi}
  Έστω η συνάρτηση $f(x)=\ln \left( \sqrt{x^2+1}-x \right) $
  \begin{enumerate}
    \item<1-> Να βρείτε το πεδίο ορισμού της $f$
    \item<2-> Να δείξετε ότι η συνάρτηση $f$ είναι περιττή
  \end{enumerate}

  %\hyperlink{Λύση6}{\beamerbutton{Λύση}}

\end{askisi}

\begin{askisi}
  Δίνεται η συνάρτηση $f(x)=x+\ln(1+x)$
  \begin{enumerate}
    \item<1-> Να μελετήσετε τη συνάρτηση $f$ ως προς τη μονοτονία
    \item<2-> Να βρείτε τις ρίζες και το πρόσημο της συνάρτησης $f$
    \item<3-> Αν $α$, $β>0$ και $α<β$, να δείξετε ότι $α-β<\ln \dfrac{1+β}{1+α}$
  \end{enumerate}

  %\hyperlink{Λύση7}{\beamerbutton{Λύση}}

\end{askisi}

\begin{askisi}
  Να κάνετε τον πίνακα προσήμων της συνάρτησης $f(x)=x-1+\dfrac{\ln x}{x}$

  %\hyperlink{Λύση8}{\beamerbutton{Λύση}}

\end{askisi}

\begin{askisi}
  Να λύσετε τις εξισώσεις
  \begin{enumerate}
    \item<1-> $\ln x-1=0$
    \item<2-> $\ln (x-1)=0$
    \item<3-> $\ln x^2-1=0$
    \item<4-> $\ln 3x-\ln(4-x)=0$
  \end{enumerate}

  %\hyperlink{Λύση9}{\beamerbutton{Λύση}}

\end{askisi}

\begin{askisi}
  Να λύσετε την εξίσωση $\ln x+\ln (x+1)=\ln (x+3)+\ln 2$

  %\hyperlink{Λύση10}{\beamerbutton{Λύση}}

\end{askisi}

\begin{askisi}
  Να λύσετε την εξίσωση $\log (x^2+7x)=1+\log (x+1)$

  %\hyperlink{Λύση11}{\beamerbutton{Λύση}}

\end{askisi}

\begin{askisi}
  Να λύσετε την εξίσωση $$\ln^2x-3\ln x-4=0$$

  %\hyperlink{Λύση12}{\beamerbutton{Λύση}}

\end{askisi}

\begin{askisi}
  Να λύσετε τις εξισώσεις
  \begin{enumerate}
    \item<1-> $e^x=2$
    \item<2-> $3^x=10$
    \item<3-> $2e^x-1=0$
    \item<4-> $e^{-x}=2^x$
  \end{enumerate}

  %\hyperlink{Λύση13}{\beamerbutton{Λύση}}

\end{askisi}

\begin{askisi}
  Να λύσετε τις εξισώσεις
  \begin{enumerate}
    \item<1-> $\ln x+1\le 0$
    \item<2-> $3\ln x-1<0$
  \end{enumerate}

  %\hyperlink{Λύση14}{\beamerbutton{Λύση}}

\end{askisi}

\begin{askisi}
  Να λύσετε την ανίσωση $$\ln (1-x)>1+\ln x$$

  %\hyperlink{Λύση15}{\beamerbutton{Λύση}}

\end{askisi}

\begin{askisi}
  Να λύσετε τις ανισώσεις
  \begin{enumerate}
    \item<1-> $3e^x-1<0$
    \item<2-> $e^{-x}-2>0$
    \item<3-> $\left( \dfrac{2}{5} \right)^x>3 $
  \end{enumerate}

  %\hyperlink{Λύση16}{\beamerbutton{Λύση}}

\end{askisi}

\begin{askisi}
  Να λύσετε την ανίσωση
  $$\ln^2x-ln x^2-3>0$$

  %\hyperlink{Λύση17}{\beamerbutton{Λύση}}

\end{askisi}

\begin{askisi}
  Να λύσετε την ανίσωση $$\dfrac{e^x-1}{2e^x-1}<\dfrac{1}{3}$$

  %\hyperlink{Λύση18}{\beamerbutton{Λύση}}

\end{askisi}

\begin{askisi}
  Να κάνετε τον πίνακα προσήμων της συνάρτησης $$f(x)=2e^x-1$$

  %\hyperlink{Λύση19}{\beamerbutton{Λύση}}

\end{askisi}

\begin{askisi}
  Να κάνετε τον πίνακα προσήμων της συνάρτησης $$f(x)=2\ln x-1$$

  %\hyperlink{Λύση20}{\beamerbutton{Λύση}}

\end{askisi}

\begin{askisi}
  Έστω η συνάρτηση $f(x)=2\ln (x-1)-1$
  \begin{enumerate}
    \item<1-> Να βρείτε το πεδίο ορισμού της συνάρτησης $f$
    \item<2-> Να βρείτε τις ρίζες της $f$ και να κάνετε τον πίνακα προσήμων της
    \item<3-> Να βρείτε τα διαστήματα του $x$ που η $C_f$ είναι κάτω από τον άξονα $x'x$
  \end{enumerate}

  %\hyperlink{Λύση21}{\beamerbutton{Λύση}}

\end{askisi}

\begin{askisi}
  Να κάνετε τον πίνακα προσήμων των συναρτήσεων
  \begin{enumerate}
    \item<1-> $f(x)=\dfrac{e^x-2}{x-1}$
    \item<2-> $f(x)=\dfrac{1-2\ln x}{x-2}$
    \item<3-> $f(x)=\dfrac{\ln x}{x-1}$
    \item<4-> $f(x)=x^2-1+\ln x$
  \end{enumerate}

  %\hyperlink{Λύση22}{\beamerbutton{Λύση}}

\end{askisi}

\begin{askisi}
  Να λύσετε την εξίσωση
  $$2\ln (2x-1)-\dfrac{1}{2}\ln 9=\ln (x-1)+\ln (x+1)$$

  %\hyperlink{Λύση23}{\beamerbutton{Λύση}}

\end{askisi}

\begin{askisi}
  Να λύσετε το σύστημα

  $$\begin{cases}
      \ln x^3+\ln y^4=11 \\
      \ln (xy)=3
    \end{cases}$$

  %\hyperlink{Λύση24}{\beamerbutton{Λύση}}

\end{askisi}

\begin{askisi}
  Να λύσετε το σύστημα

  $$\begin{cases}
      y=xe^x \\
      \ln y-\ln^2x=x
    \end{cases}$$

  %\hyperlink{Λύση25}{\beamerbutton{Λύση}}

\end{askisi}

\begin{askisi}
  Να λύσετε τις εξισώσεις
  \begin{enumerate}
    \item<1-> $3^{x-1}=e^{2-x}$
    \item<2-> $\ln (e^x+2^x)=x+\ln 3$
  \end{enumerate}

  %\hyperlink{Λύση26}{\beamerbutton{Λύση}}

\end{askisi}

\begin{askisi}
  \begin{enumerate}
    \item<1-> Να δείξετε ότι $5^{\ln x}=x^{ln 5}$
    \item<2-> Να λύσετε την εξίσωση $25^{\ln x}-4x^{\ln 5}+3=0$
  \end{enumerate}

  %\hyperlink{Λύση27}{\beamerbutton{Λύση}}

\end{askisi}

\begin{askisi}
  Να λύσετε την εξίσωση

  $$x^{\ln x^2}=e^2x$$

  %\hyperlink{Λύση28}{\beamerbutton{Λύση}}

\end{askisi}

\begin{askisi}
  Το ενοίκιο $Q(t)$ ενός σπιτιού που πληρώνει ένας ενοικιαστής μετά από $t$ χρόνια από όταν το ενοικίασε, δίνεται από τη συνάρτηση
  $$Q(t)=α+β(1-e^{-γt}) \text{, } β \text{, } γ\ne 0\text{, } t\ge 0$$
  Στην αρχή πλήρωσε ενοίκιο $200$ Ευρώ, μετά από $2$ χρόνια πλήρωσε $232$ Ευρώ και μετά από άλλα $2$ χρόνια πλήρωσε $248$ Ευρώ
  \begin{enumerate}
    \item<1-> Να βρείτε τα $α$, $β$ και $γ$
    \item<2-> Να δείξετε ότι $Q(t)=200+64\left( 1-2^{-\frac{1}{2}} \right) $
    \item<3-> Να βρείτε ποιο χρόνο θα πληρώνει ενοίκιο $262$ Ευρώ
    \item<4-> Να βρείτε για πόσα χρόνια το ενοίκιο δεν θα ξεπεράσει τα $260$ Ευρώ
    \item<5-> Να δείξετε ότι το ποσόν του ενοικίου δεν θα ξεπεράσει τα $264$ Ευρώ
  \end{enumerate}

  %\hyperlink{Λύση29}{\beamerbutton{Λύση}}

\end{askisi}

\appendix

\section{}
\begin{frame}
  \centering
  \includegraphics[width=0.8\textwidth]{"./images/thatsall.png"}

\end{frame}

\begin{frame}
  Στο moodle θα βρείτε τις ασκήσεις που πρέπει να κάνετε, όπως και αυτή τη παρουσίαση
\end{frame}

% \section{Αποδείξεις}
% \begin{frame}[label=Απόδειξη1,t]{Απόδειξη Ιδιοτήτων λογαρίθμων}
%
%  \begin{itemize}
%   \item<1-> $\log_α\left( θ_1\cdot θ_2 \right)=\log_αθ_1+\log_αθ_2 $
%   \item<3-> $\log_α\dfrac{θ_1}{θ_2}=\log_αθ_1-\log_αθ_2 $
%   \item<4-> $\log_αθ^κ=κ\log_αθ$
%  \end{itemize}
%  \only<1-3>{Θέτουμε $\log_αθ_1=x_1\implies α^{x_1}=θ_1$ και $\log_αθ_2=x_2\implies α^{x_2}=θ_2$ και έχουμε}
%  \only<2>{$$\log_α\left( θ_1\cdot θ_2 \right)=\log_α\left( α^{x_1}α^{x_2} \right)=\log_α α^{x_1+x_2} =x_1+x_2=\log_αθ_1+\log_αθ_2$$}
%
%  \only<3>{$$\log_α\dfrac{θ_1}{θ_2}=\log_α \dfrac{α^{x_1}}{α^{x_2}}=\log_α α^{x_1-x_2} =x_1-x_2=\log_αθ_1-\log_αθ_2$$}
%
%  \only<4-5>{Θέτουμε $\log_αθ=x\implies α^{x}=θ$ και έχουμε}
%  \only<5>{$$\log_αθ^κ=\log_α{(α^x)}^κ=\log_αα^{κx} =κx=κ\log_αθ$$}
%
%  \hyperlink{Ιδιότητες}{\beamerbutton{Πίσω στη θεωρία}}
% \end{frame}


% \section{Λύσεις Ασκήσεων}
% \begin{frame}
%  \tableofcontents
% \end{frame}
%
% \subsection{Άσκηση 1}
% \begin{frame}[label=Λύση1]
%  Με θεώρημα ενδιαμέσων τιμών. Η συνάρτηση είναι συνεχής στο $[10,11]$ με $f(10)=1024$ και $f(11)=2048$. Αφού $2023\in (1024,2048)$ υπάρχει $x_0$...
%
%  \hyperlink{Άσκηση1}{\beamerbutton{Πίσω στην άσκηση}}
% \end{frame}
%
% \subsection{Άσκηση 2}
% \begin{frame}[label=Λύση2]
%  Με Bolzano ή με μέγιστης ελάχιστης τιμής και ΘΕΤ.
%
%  \begin{gather*}
%   f(3)<f(2)<f(1) \\
%   3f(3)<f(1)+f(2)+f(3)<3f(1) \\
%   f(3)<\dfrac{f(1)+f(2)+f(3)}{3}<f(1)
%  \end{gather*}
%
%  \hyperlink{Άσκηση2}{\beamerbutton{Πίσω στην άσκηση}}
% \end{frame}
%
% \subsection{Άσκηση 3}
% \begin{frame}[label=Λύση3]
%  Προφανές ελάχιστο στα $x_1=1$ και $x_2=3$. Ως συνεχής στο $[1,3]$ έχει σίγουρα ΚΑΙ μέγιστο στο $(1,3)$
%
%  \hyperlink{Άσκηση3}{\beamerbutton{Πίσω στην άσκηση}}
% \end{frame}
%
% \subsection{Άσκηση 4}
% \begin{frame}[label=Λύση4]
%  Η συνάρτηση `απόστασης` $f(x)-x$ είναι ορισμένη στο κλειστό διάστημα και έχει σίγουρα μέγιστο
%
%  \hyperlink{Άσκηση4}{\beamerbutton{Πίσω στην άσκηση}}
% \end{frame}
%
% \subsection{Άσκηση 5}
% \begin{frame}[label=Λύση5]
%  Όμοια με την Άσκηση 2
%
%  \hyperlink{Άσκηση5}{\beamerbutton{Πίσω στην άσκηση}}
% \end{frame}
%
% \subsection{Άσκηση 6}
% \begin{frame}[label=Λύση6]
%  \begin{enumerate}
%   \item Είναι γνησίως αύξουσα άρα $(f(+\infty),f(-\infty))$
%   \item Προφανώς $[f(0),f(1)]$...
%  \end{enumerate}
%
%  \hyperlink{Άσκηση6}{\beamerbutton{Πίσω στην άσκηση}}
% \end{frame}

\end{document}
