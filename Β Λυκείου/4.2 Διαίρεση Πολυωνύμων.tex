\documentclass[greek]{beamer}
%\usepackage{fontspec}
\usepackage{amsmath,amsthm}
\usepackage{unicode-math}
\usepackage{xltxtra}
\usepackage{graphicx}
\usetheme{CambridgeUS}
\usecolortheme{seagull}
\usepackage{hyperref}
\usepackage{ulem}
\usepackage{xgreek}

\usepackage{pgfpages}
\usepackage{tikz}
\usepackage{tkz-tab}
%\setbeameroption{show notes on second screen}
%\setbeameroption{show only notes}

\setsansfont{Noto Serif}

\usepackage{multicol}

\usepackage{appendixnumberbeamer}

\usepackage{polynom}

\usepackage{pgffor}

\setbeamercovered{transparent}
\beamertemplatenavigationsymbolsempty

\title{Πολυώνυμα}
\subtitle{Διαίρεση}
\author[Λόλας]{Κωνσταντίνος Λόλας}
\date{}

\begin{document}

\begin{frame}
  \titlepage
\end{frame}

\section{Θεωρία}
\begin{frame}{Ποια πράξη σιχαίνεστε περισσότερο}
  Εννοείται τη διαίρεση! \only<2>{και ειδικά την κάθετη!}

  \only<3>{Να σας δω: Να γίνει η διαίρεση $$51.32:2.7$$}

  \only<4>{Το τέλειο φέτος είναι ότι θα ασχολούμαστε \emph{ΜΟΝΟ} με ακέραιους συντελεστές}
\end{frame}

\begin{frame}{Δημοτικό ολέ!}
  \begin{block}{Ευκελείδεια Διαίρεση}
    Για κάθε ζευγάρι φυσικών $Δ$ και $δ>0$, υπάρχουν μοναδικοί $π\in\mathbb{N}$ και $υ\in\mathbb{N}$ ώστε
    $$Δ=δ\cdot π+υ \text{, με } 0\le υ <δ$$
    Με ονομασίες: $Δ$ διαιρετέος, $δ$ διαιρέτης, $π$ πηλίκο και $υ$ υπόλοιπο
  \end{block}
  \begin{block}{Άλλη μορφή της διαίρεσης}<2->
    Αν $Δ=δ\cdot π+υ$ τότε
    $$\dfrac{Δ}{δ}=π + \dfrac{υ}{δ}$$
  \end{block}
\end{frame}

\begin{frame}{Περνάμε στα καινούρια}
  \begin{block}{Διαίρεση Πολυωνύμων}
    Για κάθε ζευγάρι πολυωνύμων $Δ(x)$ και $δ(x)\ne 0$, υπάρχουν μοναδικά πολυώνυμα $π(x)$ και $υ(x)$ ώστε
    $$Δ(x)=δ(x)\cdot π(x)+υ(x)$$
    Με $υ(x)$ να είναι μηδενικό πολυώνυμο ή βαθμού μικρότερου ή ίσου του $δ(x)$
  \end{block}
  \begin{block}{Άλλη μορφή της διαίρεσης}<2->
    Αν $Δ(x)=δ(x)\cdot π(x)+υ(x)$ τότε
    $$\dfrac{Δ(x)}{δ(x)}=π(x) + \dfrac{υ(x)}{δ(x)}$$
  \end{block}
\end{frame}

\begin{frame}{Πάμε για γνωστά!}
  \begin{enumerate}
    \item<1-> $(x^2-1):(x-1)$
    \item<2-> $(x^2-1):(x)$
    \item<3-> $(x^2+2x+3):(x+1)$
  \end{enumerate}
\end{frame}

\begin{frame}[t]{Πάμε για άγνωστα!}
  Από γνωστά
  \only<2>{\polylongdiv[stage=1,style=D]{x^3-5x^2+2x-1}{x-3}}
  \only<3>{\polylongdiv[stage=2,style=D]{x^3-5x^2+2x-1}{x-3}}
  \only<4>{\polylongdiv[stage=3,style=D]{x^3-5x^2+2x-1}{x-3}}
  \only<5>{\polylongdiv[stage=4,style=D]{x^3-5x^2+2x-1}{x-3}}
  \only<6>{\polylongdiv[stage=5,style=D]{x^3-5x^2+2x-1}{x-3}}
  \only<7>{\polylongdiv[stage=6,style=D]{x^3-5x^2+2x-1}{x-3}}
  \only<8>{\polylongdiv[stage=7,style=D]{x^3-5x^2+2x-1}{x-3}}
  \only<9>{\polylongdiv[stage=8,style=D]{x^3-5x^2+2x-1}{x-3}}
  \only<10>{\polylongdiv[stage=9,style=D]{x^3-5x^2+2x-1}{x-3}}
  \only<11->{\polylongdiv[stage=10,style=D]{x^3-5x^2+2x-1}{x-3}}

  \only<12->{Άρα $$\dfrac{x^3-5x^2+2x-1}{x-3}=x^2-2x-4+\dfrac{-13}{x-3}$$}
\end{frame}

\begin{frame}{Ουφ! Επιτέλους, στα 17 μου κατάλαβα διαίρεση}
  Διαίρεση με
  \begin{itemize}
    \item<1-> $x-5$ θα δώσει υπόλοιπο πολυώνυμο μορφής...
    \item<2-> $x^2+2x-1$ θα δώσει υπόλοιπο πολυώνυμο μορφής...
    \item<3-> $x^5+3x+2$ θα δώσει υπόλοιπο πολυώνυμο μορφής...
  \end{itemize}
  \onslide<4-> Άρα $P(x)=(x-ρ)π(x)+υ$
\end{frame}

\begin{frame}{1 Θεωρηματάκι}
  $$P(x)=(x-ρ)π(x)+υ$$
  Βρείτε τρόπο να υπολογίσετε το $υ$
  \begin{block}{Θεώρημα Υπολοίπου}<2->
    Το υπόλοιπο της διαίρεσης του $P(x)$ με το $x-ρ$ δίνει υπόλοιπο $P(ρ)$
  \end{block}
\end{frame}

\begin{frame}{Και το πόρισμά του!}
  $$P(x)=(x-ρ)π(x)+P(ρ)$$
  Τι γίνεται αν $P(ρ)=0$?
  \begin{block}{Πόρισμα Υπολοίπου}<2->
    Αν $P(ρ)=0$ τότε $P(x)=(x-ρ)π(x)$ δηλαδή το $x-ρ$ είναι παράγοντας
  \end{block}

  \begin{block}{Γενικά}<3->
    Ένα πολυώνυμο $P(x)$ έχει παράγοντα το $x-ρ$ αν και μόνο αν το $ρ$ είναι ρίζα του
  \end{block}

\end{frame}

\begin{frame}{Σχήμα Horner}
  Αν έχουμε να διαιρέσουμε ένα πολυώνυμο με $x-α$ τότε \emph{ίσως} είναι καλύτερα να χρησιμοποιήσουμε Horner
  \begin{enumerate}
    \item γράφουμε \emph{όλους} τους συντελεστές του διαιρεταίου
    \item γράφουμε το $α$ δεξιά
    \item ράβουμε!!!!!
  \end{enumerate}
  \centering
  \foreach \xxslide in {1,2,3,4,5,6,7,8}{
      \only<\xxslide>{\polyhornerscheme[tutor=true, x=1, stage=\xxslide]{x^3-5x^2+2x-1}}
    }
\end{frame}


\section{Ασκήσεις}
\begin{askisi}
  Δίνεται το πολυώνυμο $P(x)=x^3+3x+1$
  \begin{enumerate}
    \item<1-> Να κάνετε τη διαίρεση του πολυωνύμου $P(x)$ δια το $x^2-1$
    \item<2-> Να βρείτε το πηλίκο $π(x)$ και το υπόλοιπο $υ(x)$ της παραπάνω διαίρεσης
    \item<3-> Να γράψετε την ταυτότητα της παραπάνω διαίρεσης
    \item<4-> Να εξετάσετε αν η παραπάνω διαίρεση είναι τέλεια
    \item<5-> Να δείξετε ότι το πολυώνυμο $Q(x)=P(x)-(4x+1)$ διαιρείται από το $x^2-1$

  \end{enumerate}

  % \hyperlink{Λύση1}{\beamerbutton{Λύση}}

\end{askisi}

\begin{askisi}
  Έστω το πολυώνυμο $P(x)=(x^2-1)(3x-2)+5x-1$
  \begin{enumerate}
    \item<1-> Να βρείτε το πηλίκο και το υπόλοιπο της διαίρεσης $P(x):(x^2-1)$
    \item<2-> Η παραπάνω ισότητα είναι η ταυτότητα διαίρεσης του $P(x)$ με το $3x-2$;
  \end{enumerate}

  % \hyperlink{Λύση2}{\beamerbutton{Λύση}}

\end{askisi}

\begin{askisi}
  \begin{enumerate}
    \item<1-> Να βρείτε το υπόλοιπο της διαίρεσης του πολυωνύμου $P(x)=x^3-5x+1$ με το $x+1$
    \item<2-> Να βρείτε τις τιμές του $α$, για τις οποίες το υπόλοιπο της διαίρεσης του $P(x)=2x^2-αx+α$ με το $x+2α$ είναι $11$
  \end{enumerate}

  % \hyperlink{Λύση3}{\beamerbutton{Λύση}}

\end{askisi}

\begin{askisi}
  Αν το υπόλοιπο της διαίρεσης του πολυωνύμου $P(x)$ με το $x-1$ είναι $2$, να δείξετε ότι το πολυώνυμο $Q(x)=P(3x+7)-x^2+2$ έχει ρίζα το $-2$

  % \hyperlink{Λύση4}{\beamerbutton{Λύση}}

\end{askisi}

\begin{askisi}
  Αν το υπόλοιπο της διαίρεσης ενός πολυωνύμου $P(x)$ με το $3x^2-x-4$ είναι $2x+5$, να βρείτε το υπόλοιπο της διαίρεσης του $P(x)$ με το $x+1$

  % \hyperlink{Λύση5}{\beamerbutton{Λύση}}

\end{askisi}

\begin{askisi}
  \begin{enumerate}
    \item<1-> Να εξετάσετε, αν τα πολυώνυμα $x+1$ και $x-1$ είναι παράγοντας του πολυωνύμου
    \item<2-> Να βρείτε τις τιμές του $α$, για τις οποίες το πολυώνυμο $P(x)=α^3x^3-5x+2$ έχει παράγοντα το $x-2$
  \end{enumerate}

  % \hyperlink{Λύση6}{\beamerbutton{Λύση}}

\end{askisi}

\begin{askisi}
  Να δείξετε ότι τα παρακάτω πολυώνυμο δεν έχουν παράγοντα της μορφής $x-ρ$
  \begin{enumerate}
    \item<1-> $P(x)=3x^6+5x^2+3$
    \item<2-> $Q(x)=-3x^4-5x^2-3$
  \end{enumerate}

  %\hyperlink{Λύση7}{\beamerbutton{Λύση}}

\end{askisi}

\begin{askisi}
  Να βρείτε τις τιμές των $α$ και $β$ για τις οποίες το πολυώνυμο $P(x)=-x^3+2αx-β+α$, έχει παράγοντα το $x+2$ και το υπόλοιπο της διαίρεσης του $P(x)$ με το $x+1$ είναι 3.

  %\hyperlink{Λύση8}{\beamerbutton{Λύση}}

\end{askisi}

\begin{askisi}
  Αν το πολυώνυμο $P(x)=x^3+αx-α+β$ έχει παράγοντες όλους τους παράγοντες του πολυωνύμου $x^2-2x$, να βρείτε τις τιμές των $α$ και $β$

  %\hyperlink{Λύση9}{\beamerbutton{Λύση}}

\end{askisi}

\begin{askisi}
  Με τη βοήθεια του σχήματος Horner, να βρείτε το πηλίκο $π(x)$ και το υπόλοιπο των παρακάτω διαιραίσεων και να γράψετε τις ταυτότητες των διαιρέσεων
  \begin{enumerate}
    \item<1-> $(2x^3-3x^2+5):(x-1)$
    \item<2-> $-2x^3:(x+1)$
  \end{enumerate}

  %\hyperlink{Λύση10}{\beamerbutton{Λύση}}

\end{askisi}

\begin{askisi}
  Με τη βοήθεια του σχήματος Horner μόνο, να αποδείξετε ότι το πολυώνυμο $P(x)=x^3-7x+6$ διαιρείται με το πολυώνυμο $Q(x)=(x-1)(x+3)$ και στη συνέχεια, να βρείτε το πηλίκο της διαίρεσης $P(x):Q(x)$

  %\hyperlink{Λύση11}{\beamerbutton{Λύση}}

\end{askisi}

\begin{askisi}
  Δίνεται η συνάρτηση $f(x)=x^5$, όπου $ν$ θετικός αριθμός. Να απλοποιήσετε την παράσταση $\dfrac{f(x)-f(1)}{x-1}$

  %\hyperlink{Λύση12}{\beamerbutton{Λύση}}

\end{askisi}

\begin{askisi}
  Να βρείτε το υπόλοιπο της διαίρεσης του $P(x)$ με το $x^2+x$, όταν ισχύουν $P(0)=2$ και το υπόλοιπο της διαίρεσής του $P(x)$ με το $x+1$ είναι $3$

  %\hyperlink{Λύση13}{\beamerbutton{Λύση}}

\end{askisi}

\begin{askisi}
  Να γράψετε τις παρακάτω συναρτήσεις στη μορφή $f(x)=αx+β+\dfrac{γ}{x-1}$
  \begin{enumerate}
    \item<1-> $f(x)=\dfrac{2x-3}{x-1}$
    \item<2-> $f(x)=\dfrac{x^2}{x-1}$
  \end{enumerate}
  %\hyperlink{Λύση14}{\beamerbutton{Λύση}}

\end{askisi}

\begin{askisi}
  Έστω $P(x)$ πολυώνυμο 3ου βαθμού, το οποίο διαιρείται με το $x^2+3$
  \begin{enumerate}
    \item<1-> Να βρείτε τη μορφή του πηλίκου $π(x)$
    \item<2-> Αν επιπλέον έχει ρίζα το $-1$ και ισχύει $P(-2)=14$, να βρείτε το $P(x)$
  \end{enumerate}

  %\hyperlink{Λύση15}{\beamerbutton{Λύση}}

\end{askisi}


% \appendix
%
% \section{Αποδείξεις}
% \begin{frame}[label=Απόδειξη1]{Απόδειξη μονοτονίας συνάρτησης}
%  Θα δείξουμε ότι για κάθε $x_1<x_2\in Δ \implies f(x_1)<f(x_2)$.
%
%  \onslide<1-> Στο $[x_1,x_2]$ είναι παραγωγίσιμη, άρα θα ισχύει το ΘΜΤ
%
%  \onslide<2-> Υπάρχει $ξ\in Δ$ ώστε $f'(ξ)=\dfrac{f(x_2)-f(x_1)}{x_2-x_1}$.
%
%  \onslide<3-> Αλλά $f'(x)>0$ για κάθε $x\in Δ$
%
%  \onslide<4-> Άρα $f'(ξ)=\dfrac{f(x_2)-f(x_1)}{x_2-x_1}>0$
%
%  \onslide<5-> $f(x_2)>f(x_1)$
%
%  \hyperlink{Θεώρημα1}{\beamerbutton{Πίσω στη θεωρία}}
% \end{frame}


% \section{Λύσεις Ασκήσεων}
% \begin{frame}
%  \tableofcontents
% \end{frame}
%
% \subsection{Άσκηση 1}
% \begin{frame}[label=Λύση1]
%  Με θεώρημα ενδιαμέσων τιμών. Η συνάρτηση είναι συνεχής στο $[10,11]$ με $f(10)=1024$ και $f(11)=2048$. Αφού $2023\in (1024,2048)$ υπάρχει $x_0$...
%
%  \hyperlink{Άσκηση1}{\beamerbutton{Πίσω στην άσκηση}}
% \end{frame}
%
% \subsection{Άσκηση 2}
% \begin{frame}[label=Λύση2]
%  Με Bolzano ή με μέγιστης ελάχιστης τιμής και ΘΕΤ.
%
%  \begin{gather*}
%   f(3)<f(2)<f(1) \\
%   3f(3)<f(1)+f(2)+f(3)<3f(1) \\
%   f(3)<\dfrac{f(1)+f(2)+f(3)}{3}<f(1)
%  \end{gather*}
%
%  \hyperlink{Άσκηση2}{\beamerbutton{Πίσω στην άσκηση}}
% \end{frame}
%
% \subsection{Άσκηση 3}
% \begin{frame}[label=Λύση3]
%  Προφανές ελάχιστο στα $x_1=1$ και $x_2=3$. Ως συνεχής στο $[1,3]$ έχει σίγουρα ΚΑΙ μέγιστο στο $(1,3)$
%
%  \hyperlink{Άσκηση3}{\beamerbutton{Πίσω στην άσκηση}}
% \end{frame}
%
% \subsection{Άσκηση 4}
% \begin{frame}[label=Λύση4]
%  Η συνάρτηση `απόστασης` $f(x)-x$ είναι ορισμένη στο κλειστό διάστημα και έχει σίγουρα μέγιστο
%
%  \hyperlink{Άσκηση4}{\beamerbutton{Πίσω στην άσκηση}}
% \end{frame}
%
% \subsection{Άσκηση 5}
% \begin{frame}[label=Λύση5]
%  Όμοια με την Άσκηση 2
%
%  \hyperlink{Άσκηση5}{\beamerbutton{Πίσω στην άσκηση}}
% \end{frame}
%
% \subsection{Άσκηση 6}
% \begin{frame}[label=Λύση6]
%  \begin{enumerate}
%   \item Είναι γνησίως αύξουσα άρα $(f(+\infty),f(-\infty))$
%   \item Προφανώς $[f(0),f(1)]$...
%  \end{enumerate}
%
%  \hyperlink{Άσκηση6}{\beamerbutton{Πίσω στην άσκηση}}
% \end{frame}

\end{document}
