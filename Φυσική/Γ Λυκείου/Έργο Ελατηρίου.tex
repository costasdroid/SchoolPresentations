\documentclass{../../presentation}

\usepackage{tikz}
\usetikzlibrary{decorations.pathmorphing,calc}

% Define custom TikZ styles
\tikzset{
  mydashed/.style={dashed, gray},
  axis/.style={thick, ->, >=stealth},
  ell/.style={thick, <->, >=stealth},
  dx/.style={thick, <->, >=stealth},
  spring/.style={decorate, decoration={aspect=0.3, segment length=3mm, amplitude=2mm, coil}},
  ground/.style={fill=gray!30},
  mass/.style={draw, fill=blue!20},
  force/.style={thick, ->, >=stealth, red}
}

% Define custom commands
\newcommand{\tick}[1]{\draw[thick] (#1) --++ (0,-0.1);}
\newcommand{\vb}[1]{\mathbf{#1}}

% Define a custom TikZ command for the spring system
\newcommand{\springdiagram}[2]{
  \begin{tikzpicture}
    \def\H{1.1}  % wall height
    \def\T{0.3}  % wall thickness
    \def\W{7}  % ground length
    \def\D{0.25} % ground depth
    \def\h{0.7}  % mass height
    \def\w{0.8}  % mass width
    \def\x{4.0}  % mass x position
    \def\dx{#1}  % extension
    \def\y{1.22*\H} % x axis y position
    \def\F{0.8}  % force

    % AXIS
    \draw[mydashed] (\x,0) --++ (0,\y) (\x+\dx,0) --++ (0,1.1*\y);
    \draw[ell] (0,1.3*\h) --++ (\x,0) node[midway,fill=white,inner sep=0] {$\ell_0$};
    \draw[dx] (\x,1.6*\h) --++ (\dx,0) node[pos=0.45,fill=white,inner sep=0] {#2};

    % SPRING & MASS
    \draw[spring,segment length=7.5] (0,\h/2) --++ (\x+\dx,0);
    \draw[ground] (0,0) |-++ (-\T,\H) |-++ (\T+\W,-\H-\D) -- (\W,0) -- cycle;
    \draw (0,\H) -- (0,0) -- (\W,0);
  \end{tikzpicture}
}

\title{Ελατήριο}
\subtitle{Έργο Ελατηρίου}
\author[Λόλας]{Κωνσταντίνος Λόλας}
\institute[$10^ο$ ΓΕΛ]{$10^ο$ ΓΕΛ Θεσσαλονίκης}
\date{}

\begin{document}

\begin{frame}
  \titlepage
\end{frame}

\section{Θεωρία}

\begin{frame}
  Σε κάθε περίπτωση το ελατήριο από θέση $x_1$ μεταβαίνει σε θέση $x_2$ και η ενέργεια του ελατηρίου μεταβάλλεται από $U(x_1)$ σε $U(x_2)$. Η μεταβολή της ενέργειας του ελατηρίου είναι:
  \begin{equation*}
    \Delta U = U(x_2) - U(x_1) = \frac{1}{2}k x_2^2 - \frac{1}{2}k x_1^2 = \frac{1}{2}k (x_2^2 - x_1^2)
  \end{equation*}

\end{frame}

\begin{frame}
  Αν $|x_2| > |x_1|$ τότε το ελατήριο αύξησε την ενέργεια του ($ΔU>0$), συνεπώς καταναλώθηκε έργο από εξωτερική δύναμη και άρα το έργο της δύναμης αυτής είναι αρνητικό.
  \begin{equation*}
    W = -\Delta U = -\frac{1}{2}k (x_2^2 - x_1^2) = \frac{1}{2}k (x_1^2 - x_2^2)
  \end{equation*}
  \springdiagram{1}{$x_1$}
  \springdiagram{2}{$x_2$}
  \springdiagram{-2}{$x_2$}
\end{frame}

\begin{frame}
  Αν $|x_2| < |x_1|$ τότε το ελατήριο έχασε ενέργεια ($ΔU<0$) και άρα προσφέρει έργο στο περιβάλλον, το οποίο π.χ. για εξωτερικό σώμα είναι θετικό.
  \begin{equation*}
    W = -\Delta U = -\frac{1}{2}k (x_1^2 - x_2^2) = \frac{1}{2}k (x_2^2 - x_1^2)
  \end{equation*}
  \springdiagram{2}{$x_1$}
  \springdiagram{1}{$x_2$}
  \springdiagram{-1}{$x_2$}
\end{frame}

\begin{frame}
  Από Αρχή Διατήρησης Ενέργειας:
  \begin{align*}
    Ε     & =Κ+U \\
    ΔΕ    & =0   \\
    ΔΚ+ΔU & =0   \\
    ΔΚ    & =-ΔU \\
    W     & =-ΔU
  \end{align*}
\end{frame}


\end{document}