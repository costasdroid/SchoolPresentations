\documentclass{presentation}

\title{Τίτλος}
\subtitle{Υπότιτλος}
\author[Λόλας]{Κωνσταντίνος Λόλας}
\institute[$10^ο$ ΓΕΛ]{$10^ο$ ΓΕΛ Θεσσαλονίκης}
\date{}

\begin{document}

\begin{frame}
  \titlepage
\end{frame}

\section{Θεωρία}
\begin{frame}{Το μεγάλο ταξίδι}
  \begin{itemize}[<+-|alert@+>]
    \item Ορισμός
    \item Εξίσωση
    \item Γενική Εξίσωση, to rule them all!
    \item Ελάχιστος συνδυασμός με διανύσματα, shame!
    \item Εύρεση εξίσωσης από κάθε περίπτωση, brace yourselfs!
    \item νέοι τύποι (απόστασης και εμβαδού)
  \end{itemize}
\end{frame}

\begin{frame}[noframenumbering]
  Στο moodle θα βρείτε τις ασκήσεις που πρέπει να κάνετε, όπως και αυτή τη παρουσίαση
\end{frame}

\section{Ασκήσεις}

\begin{frame}[noframenumbering]
  \vfill
  \centering
  \begin{beamercolorbox}[sep=8pt,center,shadow=true,rounded=true]{title}
    \usebeamerfont{title}Ασκήσεις
  \end{beamercolorbox}
  \vfill
\end{frame}

\begin{askisi}

\end{askisi}

\appendix
\section*{Λύσεις Ασκήσεων}

\begin{frame}[noframenumbering]
  \vfill
  \centering
  \begin{beamercolorbox}[sep=8pt,center,shadow=true,rounded=true]{title}
    \usebeamerfont{title}Λύσεις
  \end{beamercolorbox}
  \vfill
\end{frame}

\begin{frame}{Περιεχόμενα }
  \tableofcontents
\end{frame}

\subsection{Λύση ηβη}
\begin{lisi}
  Με θεώρημα ενδιαμέσων τιμών. Η συνάρτηση είναι συνεχής στο $[10,11]$ με $f(10)=1024$ και $f(11)=2048$. Αφού $2023\in (1024,2048)$ υπάρχει $x_0$...

  \hyperlink{Άσκηση1}{\beamerbutton{Πίσω στην άσκηση}}
\end{lisi}

% regex
% (\\subsection\{Άσκηση)().*\n.*
% \begin{askisi}
%   \end{frame}
% \end{askisi}

\end{document}
