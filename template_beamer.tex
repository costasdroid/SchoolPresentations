\documentclass[greek]{beamer}
\usepackage{amsmath,amsthm} % needed for mathematics
\usepackage{unicode-math}
\usepackage{xltxtra}
\usepackage{graphicx}
\usetheme{CambridgeUS}
\usecolortheme{seagull}
\usepackage{hyperref}
\usepackage{ulem} % underline words package
\usepackage{xgreek}

\usepackage{pgfpages} 
\usepackage{tikz} % package for shapes and more
%\setbeameroption{show notes on second screen}
%\setbeameroption{show only notes}

\setsansfont{Calibri} % it is said that Calibri is the proper font for reading difficulties

\usepackage{multicol} % package for two or more columns

\usepackage{appendixnumberbeamer} % remove page numbering in appendix

\usepackage{polynom} % polynomial divisions package

\usepackage{pgffor} % macros

\setbeamercovered{highly dynamic}
\setbeamertemplate{navigation symbols}{}

\newcounter{askisi} % enviroment for exercises
\newenvironment{askisi}
{
  \refstepcounter{askisi}\par
  \subsection{Άσκηση \theaskisi}
  \begin{frame}[label=Άσκηση\theaskisi,t]{Εξάσκηση \theaskisi}
}{
  \end{frame}
}

\newcounter{lisi} % enviroment for solutions
\newenvironment{lisi}
{
  \refstepcounter{lisi}\par
  \subsection{Άσκηση \thelisi}
  \begin{frame}[label=Λύση\thelisi,t]{Λύση \thelisi}
}{
  \end{frame}
}

\title{Τίτλος}
\subtitle{Υπότιτλος}
\author[Λόλας]{Κωνσταντίνος Λόλας}
\date{}

\begin{document}

\begin{frame}
  \titlepage
\end{frame}

\section{Θεωρία}
\begin{frame}{Το μεγάλο ταξίδι}
  \begin{itemize}
    \item<1-> Ορισμός
    \item<2-> Εξίσωση
    \item<3-> Γενική Εξίσωση, to rule them all!
    \item<4-> Ελάχιστος συνδυασμός με διανύσματα, shame!
    \item<5-> Εύρεση εξίσωσης από κάθε περίπτωση, brace yourselfs!
    \item<6-> 2 νέοι τύποι (απόστασης και εμβαδού)
  \end{itemize}
\end{frame}

\begin{frame}[noframenumbering]
  Στο moodle θα βρείτε τις ασκήσεις που πρέπει να κάνετε, όπως και αυτή τη παρουσίαση
\end{frame}

\section{Ασκήσεις}

\begin{frame}[noframenumbering]
  \vfill
  \centering
  \begin{beamercolorbox}[sep=8pt,center,shadow=true,rounded=true]{title}
    \usebeamerfont{title}Ασκήσεις
  \end{beamercolorbox}
  \vfill
\end{frame}

\begin{askisi}

\end{askisi}

% \appendix
% \section{Λύσεις Ασκήσεων}
% \begin{frame}
%   \tableofcontents
% \end{frame}

% \begin{lisi} Με θεώρημα ενδιαμέσων τιμών. Η συνάρτηση είναι συνεχής στο $[10,11]$ με $f(10)=1024$ και $f(11)=2048$. Αφού $2023\in (1024,2048)$ υπάρχει $x_0$...

%   \hyperlink{Άσκηση1}{\beamerbutton{Πίσω στην άσκηση}}
% \end{lisi}


% regex
% (\\subsection\{Άσκηση)().*\n.*\n
% \begin{askisi}
%   \end{frame}
% \end{askisi}

\end{document}
