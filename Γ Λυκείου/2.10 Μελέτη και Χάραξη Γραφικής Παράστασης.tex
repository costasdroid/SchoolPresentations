\documentclass{../presentation}

\title{Συναρτήσεις}
\subtitle{Μελέτη και χάραξη γραφικής παράστασης συνάρτησης}
\author[Λόλας]{Κωνσταντίνος Λόλας}
\institute[$10^ο$ ΓΕΛ]{$10^ο$ ΓΕΛ Θεσσαλονίκης}
\date{}

\begin{document}

\begin{frame}
  \titlepage
\end{frame}

\section{Θεωρία}

\begin{frame}{Τέλος Διαφορικού Λογισμού ΟΛΕΕΕΕΕΕΕΕ}
  Κλείνουμε το μαγαζί! Σκουπίζουμε και πάμε για άλλα!
\end{frame}

\begin{frame}{Όλα μαζί!}
  \begin{itemize}
    \item Πεδίο ορισμού
    \item Άρτια - Περιττή
    \item Σημεία τομής με άξονες
    \item Συνέχεια
    \item Παραγωγισιμότητα
    \item Μονοτονία - Ακρότατα
    \item Κυρτότητα - Σημεία καμπής
    \item Ασύμπτωτες
  \end{itemize}
  Και τα βάζουμε όλα μαζί σε άξονες!
\end{frame}

\section{Ασκήσεις}

\begin{askisi}
  Δίνεται η συνάρτηση $f(x)=2e^{x-1}-x^2$
  \begin{enumerate}[<+->]
    \item Να μελετήσετε τη συνάρτηση $f$ ως προς τη μονοτονία, τα ακρότατα, τη κυρτότητα και τα σημεία καμπής
    \item Να βρείτε τις οριακές τιμές της $f$ στα άκρα του διαστήματος του πεδίου ορισμού της
    \item Με βάση τις απαντήσεις σας στα προηγούμενα ερωτήματα, να κάνετε τον πίνακα μεταβολών της $f$ και να σχεδιάσετε τη $C_f$
  \end{enumerate}
\end{askisi}

\end{document}
