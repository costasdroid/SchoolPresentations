\documentclass{../presentation}

\title{Συναρτήσεις}
\subtitle{Μελέτη και χάραξη γραφικής παράστασης συνάρτησης}
\author[Λόλας]{Κωνσταντίνος Λόλας}
\institute[$10^ο$ ΓΕΛ]{$10^ο$ ΓΕΛ Θεσσαλονίκης}
\date{}

\begin{document}

\begin{frame}
  \titlepage
\end{frame}

\section{Θεωρία}

\begin{frame}{Τέλος Διαφορικού Λογισμού ΟΛΕΕΕΕΕΕΕΕ}
  Κλείνουμε το μαγαζί! Σκουπίζουμε και πάμε για άλλα!
\end{frame}

\begin{frame}{Όλα μαζί!}
  \begin{itemize}
    \item Πεδίο ορισμού
    \item Άρτια - Περιττή
    \item Σημεία τομής με άξονες
    \item Συνέχεια
    \item Παραγωγισιμότητα
    \item Μονοτονία - Ακρότατα
    \item Κυρτότητα - Σημεία καμπής
    \item Ασύμπτωτες
  \end{itemize}
  Και τα βάζουμε όλα μαζί σε άξονες!
\end{frame}

\section{Ασκήσεις}

\begin{askisi}
  Δίνεται η συνάρτηση $f(x)=2e^{x-1}-x^2$
  \begin{enumerate}[<+->]
    \item Να μελετήσετε τη συνάρτηση $f$ ως προς τη μονοτονία, τα ακρότατα, τη κυρτότητα και τα σημεία καμπής
    \item Να βρείτε τις οριακές τιμές της $f$ στα άκρα του διαστήματος του πεδίου ορισμού της
    \item Με βάση τις απαντήσεις σας στα προηγούμενα ερωτήματα, να κάνετε τον πίνακα μεταβολών της $f$ και να σχεδιάσετε τη $C_f$
  \end{enumerate}
\end{askisi}

\begin{askisi}
  Δίνεται η συνάρτηση $f(x)=εφx-x$, $x\in \left( -\dfrac{π}{2},\dfrac{π}{2} \right).
  \begin{enumerate}[<+->]
    \item Να μελετήσετε τη συνάρτηση $f$ ως προς τη μονοτονία, τα ακρότατα, τη κυρτότητα και τα σημεία καμπής
    \item Να βρείτε τις ασύμπτωτές της $C_f$
    \item Με βάση τις απαντήσεις σας στα προηγούμενα ερωτήματα, να κάνετε τον πίνακα μεταβολών της $f$ και να σχεδιάσετε τη $C_f$
  \end{enumerate}
\end{askisi}

\begin{askisi}
  Δίνεται η συνάρτηση $f(x)=\dfrac{\ln x}{x}$
  \begin{enumerate}[<+->]
    \item Να μελετήσετε τη συνάρτηση $f$ ως προς τη μονοτονία, τα ακρότατα, τη κυρτότητα και τα σημεία καμπής
    \item Να βρείτε τις οριακές τιμές της $f$ στα άκρα του διαστήματος του πεδίου ορισμού της και τις ασύμπτωτές της $C_f$
    \item Να κάνετε τον πίνακα μεταβολών της $f$ και να σχεδιάσετε τη $C_f$
  \end{enumerate}
\end{askisi}

\begin{askisi}
  Δίνεται η συνάρτηση $f(x)=x^x$, $x>0$.
  \begin{enumerate}
    \item Να μελετήσετε τη συνάρτηση $f$ ως προς τη μονοτονία, τα ακρότατα, τη κυρτότητα και τα σημεία καμπής
    \item Να βρείτε τις οριακές τιμές της $f$ στα άκρα του διαστήματος του πεδίου ορισμού της
    \item Με βάση τις απαντήσεις σας στα προηγούμενα ερωτήματα, να κάνετε τον πίνακα μεταβολών της $f$ και να σχεδιάσετε τη $C_f$
  \end{enumerate}
\end{askisi}

\begin{askisi}
  Δίνεται η συνάρτηση $f(x)=x+\dfrac{1}{x-1}$
  \begin{enumerate}[<+->]
    \item Να μελετήσετε τη συνάρτηση $f$ ως προς τη μονοτονία, τα ακρότατα, τη κυρτότητα και τα σημεία καμπής
    \item Να βρείτε τις οριακές τιμές της $f$ στα άκρα του διαστήματος του πεδίου ορισμού της και τις ασύμπτωτές της $C_f$
    \item Να κάνετε τον πίνακα μεταβολών της $f$ και να σχεδιάσετε τη $C_f$
  \end{enumerate}
\end{askisi}

\begin{askisi}
  Δίνεται η συνάρτηση $f(x)=\dfrac{x}{x^2+1}$
  \begin{enumerate}[<+->]
    \item Να μελετήσετε τη συνάρτηση $f$ ως προς τη μονοτονία, τα ακρότατα, τη κυρτότητα και τα σημεία καμπής
    \item Να βρείτε τις οριακές τιμές της $f$ στα άκρα του διαστήματος του πεδίου ορισμού της και τις ασύμπτωτές της $C_f$
    \item Να κάνετε τον πίνακα μεταβολών της $f$ και να σχεδιάσετε τη $C_f$
  \end{enumerate}
\end{askisi}

\begin{askisi}
  Δίνεται η συνάρτηση $f(x)=x^2e^{1-x}$
  \begin{enumerate}[<+->]
    \item Να μελετήσετε τη συνάρτηση $f$ ως προς τη μονοτονία, τα ακρότατα, τη κυρτότητα και τα σημεία καμπής
    \item Να βρείτε τις οριακές τιμές της $f$ στα άκρα του διαστήματος του πεδίου ορισμού της και τις ασύμπτωτές της $C_f$
    \item Με βάση τις απαντήσεις σας στα προηγούμενα ερωτήματα, να κάνετε τον πίνακα μεταβολών της $f$ και να σχεδιάσετε τη $C_f$
    \item Με τη βοήθεια της $C_f$, να βρείτε το πλήθος των λύσεων της εξίσωσης $f(x)=a$, $a\in\mathbb{R}$
  \end{enumerate}
\end{askisi}

\begin{askisi}
  Έστω $f:[1,7]\to\mathbb{R}$ για την οποία ισχύουν:
  \begin{itemize}
    \item $f(6)=2f(4)=-4f(1)=4$, $f(2)=0$
    \item Η $f$ είναι ικανοποιεί τις υποθέσεις του Θ. Rolle στο $[1,7]$.
  \end{itemize}
  Η γραφική παράσταση της $f'$, φαίνεται στο σχήμα.
  \begin{enumerate}[<+->]
    \item Να μελετήσετε τη συνάρτηση $f$ ως προς τη μονοτονία και τα ακρότατα
    \item Να μελετήσετε τη συνάρτηση $f$ ως προς την κυρτότητα και τα σημεία καμπής
    \item Με βάση τις απαντήσεις σας στα προηγούμενα ερωτήματα, να σχεδιάσετε τη $C_f$
  \end{enumerate}
\end{askisi}

\end{document}
