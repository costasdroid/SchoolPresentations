\documentclass{presentation}

\title{Συναρτήσεις}
\subtitle{Εφαπτομένη}
\author[Λόλας]{Κωνσταντίνος Λόλας}
\institute[$10^ο$ ΓΕΛ]{$10^ο$ ΓΕΛ Θεσσαλονίκης}
\date{}

\begin{document}

\begin{frame}
    \titlepage
\end{frame}

\section{Θεωρία}
\begin{frame}{Τι μάθαμε?}
    Ξέρουμε την κλίση ΚΑΘΕ συνάρτησης σε ΚΑΘΕ σημείο. \pause Σημείο - Κλίση...
    \begin{block}{Εφαπτομένη}
        Η εφαπτομένη της γραφικής παράστασης της $f$ στο σημείο $x_0$ είναι η
        $$y-f(x_0)=f'(x_0)(x-x_0)$$
    \end{block}
\end{frame}

\begin{frame}{Μελέτη τύπου}
    \begin{enumerate}
        \item<1-> Άν έχουμε μία συνάρτηση και ένα σημείο είναι ΟΚ
        \item<2-> Ο μόνος άγνωστος είναι το $x_0$
            \begin{enumerate}
                \item<3-> είτε θα δίνεται
                \item<4-> είτε δίνεται το $f(x_0)$ και θα βρεθεί
                \item<5-> είτε δίνεται το $f'(x_0)$ και θα βρεθεί
                \item<6-> είτε η εφαπτόμενη είναι παράλληλη σε ευθεία
                \item<7-> είτε η εφαπτόμενη είναι κάθετη σε ευθεία
                \item<8-> είτε δίνεται η εφαπτόμενη
                \item<9-> είτε διέρχεται από ένα σημείο
                \item<10-> είτε κοινή εφαπτόμενη σε κοινό σημείο
                \item<11-> είτε κοινή εφαπτόμενη...

            \end{enumerate}
    \end{enumerate}
\end{frame}

\begin{frame}[noframenumbering]
    Στο moodle θα βρείτε τις ασκήσεις που πρέπει να κάνετε, όπως και αυτή τη παρουσίαση
\end{frame}

\section{Ασκήσεις}

\begin{frame}[noframenumbering]
    \vfill
    \centering
    \begin{beamercolorbox}[sep=8pt,center,shadow=true,rounded=true]{title}
        \usebeamerfont{title}Ασκήσεις
    \end{beamercolorbox}
    \vfill
\end{frame}

\begin{askisi}
    Να βρείτε την εξίσωση της εφαπτομένης $ε$ της γραφικής παράστασης της συνάρτησης $f(x)=x^2-x+2$ στο σημείο της με τετμημένη $x_0=1$.

    % \hyperlink{Λύση1}{\beamerbutton{Λύση}}
\end{askisi}

\begin{askisi}
    Δίνεται η συνάρτηση $f(x)=x\ln x$. Να βρείτε την εφαπτομένη $ε$ της $C_f$ που σχηματίζει με τον άξονα $x'x$ γωνία $ω=45^{\circ}$

    % \hyperlink{Λύση2}{\beamerbutton{Λύση}}
\end{askisi}

\begin{askisi}
    Δίνεται η συνάρτηση $f(x)=-x^2+2x$.
    \begin{enumerate}
        \item<1-> Να βρείτε τις εξισώσεις των εφαπτομένων στη γραφική παράσταση της συνάρτησης, που διέρχονται από το σημείο $Μ(1,2)$
        \item<2-> Να σχεδιάσετε τη $C_f$ και να βρείτε το εμβαδόν $Ε$ του τριγώνουν που σχηματίζουν οι εφαπτόμενες του ερωτήματος $1$, με τον άξονα $x'x$
    \end{enumerate}

    % \hyperlink{Λύση3}{\beamerbutton{Λύση}}
\end{askisi}

\begin{askisi}
    Δίνεται η συνάρτηση $f(x)=x^2-2x+3$. Να βρείτε τις τετμημένες $x$ των σημείων της γραφικής παράστασης της $f$ που οι εφαπτόμενες σε αυτά:
    \begin{enumerate}
        \item<1-> Έχουν κλίση $2$
        \item<2-> Σχηματίζουν με τον άξονα $x'x$ γωνία $ω$ ώστε:
            \begin{enumerate}
                \item<2-> $ω=\dfrac{3π}{4}$
                \item<3-> $εφω>1$
                \item<4-> $ω$: αμβλεία
            \end{enumerate}
    \end{enumerate}

    % \hyperlink{Λύση4}{\beamerbutton{Λύση}}
\end{askisi}

\begin{askisi}
    Δίνεται η συνάρτηση $f(x)=\dfrac{x^3}{3}-\dfrac{x^2}{2}+1$. Να βρείτε τα σημεία της $C_f$, που οι εφαπτόμενες σ' αυτά είναι:
    \begin{enumerate}
        \item<1-> Κάθετες στην ευθεία $ε:x+2y-1=0$
        \item<2-> Παράλληλες στον άξονα $x'x$
    \end{enumerate}

    % \hyperlink{Λύση5}{\beamerbutton{Λύση}}
\end{askisi}

\begin{askisi}
    Δίνεται η συνάρτηση $f(x)=αx^3+β\ln x-\ln β$. Να βρείτε τις τιμές των $α$ και $β$ για τις οποίες η εφαπτομένη της $C_f$ στο σημείο $Α(1,1)$ έχει κλίση $4$.

    % \hyperlink{Λύση6}{\beamerbutton{Λύση}}
\end{askisi}

\begin{askisi}
    Δίνεται η συνάρτηση $f(x)=x^2+x-1$. Να δείξετε ότι η ευθεία $ε:y=3x-2$ εφάπτεται της $C_f$ και να βρείτε το σημείο επαφής.

    %\hyperlink{Λύση7}{\beamerbutton{Λύση}}
\end{askisi}

\begin{askisi}
    Δίνεται η συνάρτηση $f(x)=x^2+λx+2$ και η ευθεία $ε:y=-x+λ$. Να βρείτε τις τιμές του $λ\in\mathbb{R}$, για τις οποίες η ευθεία $ε$ εφάπτεται της $C_f$

    %\hyperlink{Λύση8}{\beamerbutton{Λύση}}
\end{askisi}

\begin{askisi}
    Έστω οι συναρτήσεις $f(x)=αx^2+βx+3$ και $g(x)=x^2-αx-β$. Να βρείτε τις τιμές των $α$ και $β$ για τις οποίες οι $C_f$ και $C_g$ να έχουν κοινή εφαπτόμενη στο σημείο τους με τετμημένη $x_0=-2$

    %\hyperlink{Λύση9}{\beamerbutton{Λύση}}
\end{askisi}

\begin{askisi}
    Δίνονται οι συναρτήσεις $f(x)=x^2+3x+3$ και $g(x)=-\dfrac{1}{x}$. Να αποδείξετε ότι οι $C_f$ και $C_g$ έχουν κοινές εφαπτόμενες στα κοινά τους σημεία

    %\hyperlink{Λύση10}{\beamerbutton{Λύση}}
\end{askisi}

\begin{askisi}
    Έστω οι συναρτήσεις $f(x)=x^2+1$ και $g(x)=2x^2+2x$. Να βρείτε τις κοινές εφαπτόμενες των $C_f$ και $C_g$

    %\hyperlink{Λύση11}{\beamerbutton{Λύση}}
\end{askisi}

\begin{askisi}
    Να δείξετε ότι υπάρχει ακριβώς ένα $x_0\in (0,1)$, ώστε η εφαπτομένη στη γραφική παράσταση της συνάρτησης $f(x)=-2x^2+\ln x$ στο σημείο της με τετμημένη $x_0$, να διέρχεται από την αρχή των αξόνων

    %\hyperlink{Λύση12}{\beamerbutton{Λύση}}
\end{askisi}

\begin{askisi}
    Δίνεται η συνάρτηση $f(x)=\dfrac{x^3}{3}+x-1$.
    \begin{enumerate}
        \item<1-> Να βρείτε το σύνολο τιμών της $f'$
        \item<2-> Να βρείτε τις δυνατές τιμές της γωνίας $ω$ που σχηματίζει η εφαπτόμενη της $C_f$ στο σημείο $Μ(x,f(x))$ με τον άξονα $x'x$
    \end{enumerate}

    %\hyperlink{Λύση13}{\beamerbutton{Λύση}}
\end{askisi}

\begin{askisi}
    Δίνεται η συνάρτηση $f(x)=x^3+x+1$.
    \begin{enumerate}
        \item<1-> Να δείξετε ότι η $f$ αντιστρέφεται και να βρείτε το $D_{f^{-1}}$
        \item<2-> Αν θεωρήσουμε γνωστό ότι η συνάρτηση $f^{-1}$ είναι παραγωγίσιμη, να βρείτε την εφαπτόμενη της $C_{f^{-1}}$ στο σημείο με τετμημένη $x_0=3$
    \end{enumerate}


    %\hyperlink{Λύση14}{\beamerbutton{Λύση}}
\end{askisi}

\end{document}
