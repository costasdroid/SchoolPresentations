\documentclass[greek]{beamer}
%\usepackage{fontspec}
\usepackage{amsmath,amsthm}
\usepackage{unicode-math}
\usepackage{xltxtra}
\usepackage{graphicx}
\usetheme{CambridgeUS}
\usecolortheme{seagull}
\usepackage{hyperref}
\usepackage{ulem}
\usepackage{xgreek}

\usepackage{pgfpages}
\usepackage{tikz}
\usepackage{tkz-tab}
%\setbeameroption{show notes on second screen}
%\setbeameroption{show only notes}

\setsansfont{Noto Serif}

\usepackage{multicol}

\usepackage{appendixnumberbeamer}

\setbeamercovered{transparent}
\beamertemplatenavigationsymbolsempty

\title{Συναρτήσεις}
\subtitle{Εφαπτομένη}
\author[Λόλας]{Κωνσταντίνος Λόλας}
\date{}

\begin{document}

\begin{frame}
 \titlepage
\end{frame}

\section{Θεωρία}
\begin{frame}{Τι μάθαμε?}
 Ξέρουμε την κλίση ΚΑΘΕ συνάρτησης σε ΚΑΘΕ σημείο. \pause Σημείο - Κλίση...
 \begin{block}{Εφαπτομένη}
  Η εφαπτομένη της γραφικής παράστασης της $f$ στο σημείο $x_0$ είναι η
  $$y-f(x_0)=f'(x_0)(x-x_0)$$
 \end{block}
\end{frame}

\begin{frame}{Μελέτη τύπου}
 \begin{enumerate}
  \item<1-> Άν έχουμε μία συνάρτηση και ένα σημείο είναι ΟΚ
  \item<2-> Ο μόνος άγνωστος είναι το $x_0$
   \begin{enumerate}
    \item<3-> είτε θα δίνεται
    \item<4-> είτε δίνεται το $f(x_0)$ και θα βρεθεί
    \item<5-> είτε δίνεται το $f'(x_0)$ και θα βρεθεί
    \item<6-> είτε η εφαπτόμενη είναι παράλληλη σε ευθεία
    \item<7-> είτε η εφαπτόμενη είναι κάθετη σε ευθεία
    \item<8-> είτε δίνεται η εφαπτόμενη
    \item<9-> είτε διέρχεται από ένα σημείο...

   \end{enumerate}
 \end{enumerate}
\end{frame}

\section{Ασκήσεις}
\subsection{Άσκηση 1}
\begin{frame}[label=Άσκηση1]{Εξάσκηση 1}
 Να βρείτε την εξίσωση της εφαπτομένης $ε$ της γραφικής παράστασης της συνάρτησης $f(x)=x^2-x+2$ στο σημείο της με τετμημένη $x_0=1$.

 % \hyperlink{Λύση1}{\beamerbutton{Λύση}}
\end{frame}

\subsection{Άσκηση 2}
\begin{frame}[label=Άσκηση2]{Εξάσκηση 2}
 Δίνεται η συνάρτηση $f(x)=x\ln x$. Να βρείτε την εφαπτομένη $ε$ της $C_f$ που σχηματίζει με τον άξονα $x'x$ γωνία $ω=45^{\circ}$

 % \hyperlink{Λύση2}{\beamerbutton{Λύση}}
\end{frame}

\subsection{Άσκηση 3}
\begin{frame}[label=Άσκηση3]{Εξάσκηση 3}
 Δίνεται η συνάρτηση $f(x)=-x^2+2x$.
 \begin{enumerate}
  \item<1-> Να βρείτε τις εξισώσεις των εφαπτομένων στη γραφική παράσταση της συνάρτησης, που διέρχονται από το σημείο $Μ(1,2)$
  \item<2-> Να σχεδιάσετε τη $C_f$ και να βρείτε το εμβαδόν $Ε$ του τριγώνουν που σχηματίζουν οι εφαπτόμενες του ερωτήματος $1$, με τον άξονα $x'x$
 \end{enumerate}

 % \hyperlink{Λύση3}{\beamerbutton{Λύση}}
\end{frame}

\subsection{Άσκηση 4}
\begin{frame}[label=Άσκηση4]{Εξάσκηση 4}
 Δίνεται η συνάρτηση $f(x)=x^2-2x+3$. Να βρείτε τις τετμημένες $x$ των σημείων της γραφικής παράστασης της $f$ που οι εφαπτόμενες σε αυτά:
 \begin{enumerate}
  \item<1-> Έχουν κλίση $2$
  \item<2-> Σχηματίζουν με τον άξονα $x'x$ γωνία $ω$ ώστε:
   \begin{enumerate}
    \item<2-> $ω=\frac{3π}{4}$
    \item<3-> $εφω>1$
    \item<4-> $ω$:αμβλεία
   \end{enumerate}
 \end{enumerate}

 % \hyperlink{Λύση4}{\beamerbutton{Λύση}}
\end{frame}

\subsection{Άσκηση 5}
\begin{frame}[label=Άσκηση5]{Εξάσκηση 5}
 Δίνεται η συνάρτηση $f(x)=\frac{x^3}{3}-\frac{x^2}{2}+1$. Να βρείτε τα σημεία της $C_f$, που οι εφαπτόμενες σ' αυτά είναι:
 \begin{enumerate}
  \item<1-> Κάθετες στην ευθεία $ε:x+2y-1=0$
  \item<2-> Παράλληλες στον άξονα $x'x$
 \end{enumerate}

 % \hyperlink{Λύση5}{\beamerbutton{Λύση}}
\end{frame}

\subsection{Άσκηση 6}
\begin{frame}[label=Άσκηση6]{Εξάσκηση 6}
 Δίνεται η συνάρτηση $f(x)=αx^3+β\ln x-\ln β$. Να βρείτε τις τιμές των $α$ και $β$ για τις οποίες η εφαπτομένη της $C_f$ στο σημείο $Α(1,1)$ έχει κλίση $4$.

 % \hyperlink{Λύση6}{\beamerbutton{Λύση}}
\end{frame}

\subsection{Άσκηση 7}
\begin{frame}[label=Άσκηση7]{Εξάσκηση 7}
 Δίνεται η συνάρτηση $f(x)=x^2+x-1$. Να δείξετε ότι η ευθεία $ε:y=3x-2$ εφάπτεται της $C_f$ και να βρείτε το σημείο επαφής.

 %\hyperlink{Λύση7}{\beamerbutton{Λύση}}
\end{frame}

\subsection{Άσκηση 8}
\begin{frame}[label=Άσκηση8]{Εξάσκηση 8}
 Δίνεται η συνάρτηση $f(x)=x^2+λx+2$ και η ευθεία $ε:y=-x+λ$. Να βρείτε τις τιμές του $λ\in\mathbb{R}$, για τις οποίες η ευθεία $ε$ εφάπτεται της $C_f$

 %\hyperlink{Λύση8}{\beamerbutton{Λύση}}
\end{frame}

\subsection{Άσκηση 9}
\begin{frame}[label=Άσκηση9]{Εξάσκηση 9}
 Έστω οι συναρτήσεις $f(x)=αx^2+βx+3$ και $g(x)=x^2-αx-β$. Να βρείτε τις τιμές των $α$ και $β$ για τις οποίες οι $C_f$ και $C_g$ να έχουν κοινή εφαπτόμενη στο σημείο τους με τετμημένη $x_0=-2$

 %\hyperlink{Λύση9}{\beamerbutton{Λύση}}
\end{frame}

\subsection{Άσκηση 10}
\begin{frame}[label=Άσκηση10]{Εξάσκηση 10}
 Δίνονται οι συναρτήσεις $f(x)=x^2+3x+3$ και $g(x)=-\frac{1}{x}$. Να αποδείξετε ότι οι $C_f$ και $C_g$ έχουν κοινές εφαπτόμενες στα κοινά τους σημεία

 %\hyperlink{Λύση10}{\beamerbutton{Λύση}}
\end{frame}

\subsection{Άσκηση 11}
\begin{frame}[label=Άσκηση11]{Εξάσκηση 11}
 Έστω οι συναρτήσεις $f(x)=x^2+1$ και $g(x)=2x^2+2x$. Να βρείτε τις κοινές εφαπτόμενες των $C_f$ και $C_g$

 %\hyperlink{Λύση11}{\beamerbutton{Λύση}}
\end{frame}

\subsection{Άσκηση 12}
\begin{frame}[label=Άσκηση12]{Εξάσκηση 12}
 Να δείξετε ότι υπάρχει ακριβώς ένα $x_0\in (0,1)$, ώστε η εφαπτομένη στη γραφική παράσταση της συνάρτησης $f(x)=-2x^2+\ln x$ στο σημείο της με τετμημένη $x_0$, να διέρχεται από την αρχή των αξόνων

 %\hyperlink{Λύση12}{\beamerbutton{Λύση}}
\end{frame}

\subsection{Άσκηση 13}
\begin{frame}[label=Άσκηση13]{Εξάσκηση 13}
 Δίνεται η συνάρτηση $f(x)=\frac{x^3}{3}+x-1$.
 \begin{enumerate}
  \item<1-> Να βρείτε το σύνολο τιμών της $f'$
  \item<2-> Να βρείτε τις δυνατές τιμές της γωνίας $ω$ που σχηματίζει η εφαπτόμενη της $C_f$ στο σημείο $Μ(x,f(x))$ με τον άξονα $x'x$
 \end{enumerate}

 %\hyperlink{Λύση13}{\beamerbutton{Λύση}}
\end{frame}

\subsection{Άσκηση 14}
\begin{frame}[label=Άσκηση14]{Εξάσκηση 14}
 Δίνεται η συνάρτηση $f(x)=x^3+x+1$.
 \begin{enumerate}
  \item<1-> Να δείξετε ότι η $f$ αντιστρέφεται και να βρείτε το $D_{f^{-1}}$
  \item<2-> Αν θεωρήσουμε γνωστό ότι η συνάρτηση $f^{-1}$ είναι παραγωγίσιμη, να βρείτε την εφαπτόμενη της $C_{f^{-1}}$ στο σημείο με τετμημένη $x_0=3$
 \end{enumerate}


  %\hyperlink{Λύση14}{\beamerbutton{Λύση}}
\end{frame}


%
% \appendix
% \section{Λύσεις Ασκήσεων}
% \begin{frame}
%  \tableofcontents
% \end{frame}
%
% \subsection{Άσκηση 1}
% \begin{frame}[label=Λύση1]
%  Με θεώρημα ενδιαμέσων τιμών. Η συνάρτηση είναι συνεχής στο $[10,11]$ με $f(10)=1024$ και $f(11)=2048$. Αφού $2023\in (1024,2048)$ υπάρχει $x_0$...
%
%  \hyperlink{Άσκηση1}{\beamerbutton{Πίσω στην άσκηση}}
% \end{frame}
%
% \subsection{Άσκηση 2}
% \begin{frame}[label=Λύση2]
%  Με Bolzano ή με μέγιστης ελάχιστης τιμής και ΘΕΤ.
%
%  \begin{gather*}
%   f(3)<f(2)<f(1) \\
%   3f(3)<f(1)+f(2)+f(3)<3f(1) \\
%   f(3)<\frac{f(1)+f(2)+f(3)}{3}<f(1)
%  \end{gather*}
%
%  \hyperlink{Άσκηση2}{\beamerbutton{Πίσω στην άσκηση}}
% \end{frame}
%
% \subsection{Άσκηση 3}
% \begin{frame}[label=Λύση3]
%  Προφανές ελάχιστο στα $x_1=1$ και $x_2=3$. Ως συνεχής στο $[1,3]$ έχει σίγουρα ΚΑΙ μέγιστο στο $(1,3)$
%
%  \hyperlink{Άσκηση3}{\beamerbutton{Πίσω στην άσκηση}}
% \end{frame}
%
% \subsection{Άσκηση 4}
% \begin{frame}[label=Λύση4]
%  Η συνάρτηση `απόστασης` $f(x)-x$ είναι ορισμένη στο κλειστό διάστημα και έχει σίγουρα μέγιστο
%
%  \hyperlink{Άσκηση4}{\beamerbutton{Πίσω στην άσκηση}}
% \end{frame}
%
% \subsection{Άσκηση 5}
% \begin{frame}[label=Λύση5]
%  Όμοια με την Άσκηση 2
%
%  \hyperlink{Άσκηση5}{\beamerbutton{Πίσω στην άσκηση}}
% \end{frame}
%
% \subsection{Άσκηση 6}
% \begin{frame}[label=Λύση6]
%  \begin{enumerate}
%   \item Είναι γνησίως αύξουσα άρα $(f(+\infty),f(-\infty))$
%   \item Προφανώς $[f(0),f(1)]$...
%  \end{enumerate}
%
%  \hyperlink{Άσκηση6}{\beamerbutton{Πίσω στην άσκηση}}
% \end{frame}

\end{document}
