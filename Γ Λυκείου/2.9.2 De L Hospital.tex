\documentclass[greek]{beamer}
%\usepackage{fontspec}
\usepackage{amsmath,amsthm}
\usepackage{unicode-math}
\usepackage{xltxtra}
\usepackage{graphicx}
\usetheme{CambridgeUS}
\usecolortheme{seagull}
\usepackage{hyperref}
\usepackage{ulem}
\usepackage{xgreek}

\usepackage{pgfpages}
\usepackage{tikz}
\usepackage{tkz-tab}
%\setbeameroption{show notes on second screen}
%\setbeameroption{show only notes}

\setsansfont{Noto Serif}

\usepackage{multicol}

\usepackage{appendixnumberbeamer}

\setbeamercovered{transparent}
\beamertemplatenavigationsymbolsempty

\title{Συναρτήσεις}
\subtitle{Κανόνας De L' Hospital}
\author[Λόλας]{Κωνσταντίνος Λόλας}
\date{}

\begin{document}

\begin{frame}
    \titlepage
\end{frame}

\section{Θεωρία}
\begin{frame}{Ας τελειώσουμε με τα όρια ΕΠΙΤΕΛΟΥΣ}
    Αφήσαμε κάποια όρια
    \begin{itemize}
        \item $\dfrac{0}{0}$
        \item $\dfrac{\pm\infty}{\pm\infty}$
        \item $0\cdot \infty$
        \item $1^{+\infty}$
        \item $(+\infty)^0$
        \item $0^0$
    \end{itemize}
\end{frame}

\begin{frame}{Ένας για όλους και όλοι για έναν}
    Θα ήταν τέλεια αν κατέληγαν όλες οι περιπτώσεις σε μία!!!!
    \begin{itemize}
        \item<1-> $0\cdot \infty$ \onslide<2->$\implies \dfrac{0}{0}$ ή $\dfrac{\pm\infty}{\pm\infty}$
        \item<3-> $1^{+\infty}$ \onslide<4-> $\implies e^{+\infty\ln 1}=e^{+\infty\cdot 0}$
        \item<5-> $(+\infty)^0$ \onslide<6-> $\implies e^{0\ln\infty}=e^{+\infty\cdot 0}$
        \item<7-> $0^0$ \onslide<8-> $\implies e^{0\ln 0}=e^{+\infty\cdot 0}$
    \end{itemize}
\end{frame}

\begin{frame}{Ορισμός}
    \begin{block}{Κανόνας De L' Hospital}
        Αν $\lim\limits_{x \to x_0}{ \dfrac{f(x)}{g(x)} }=\dfrac{\pm\infty}{\pm\infty}$ ή $\dfrac{0}{0}$, με $x_0\in\bar{\mathbb{R}}$ τότε αν
        $$\lim\limits_{x \to x_0}{ \dfrac{f'(x)}{g'(x)} }=k\in\bar{\mathbb{R}}$$
        όπου $\bar{\mathbb{R}}=\mathbb{R}\cup \{-\infty,+\infty\}$ τότε
        $$\lim\limits_{x \to x_0}{ \dfrac{f(x)}{g(x)} }=k$$
    \end{block}
\end{frame}

\begin{frame}{Παρατηρήσεις}

    Ο κανόνας De L' Hospital δεν είναι απλό παιχνίδι!

    \begin{itemize}
        \item<1-> Έχει προϋποθέσεις που απλά
        \item<2-> Τι πρέπει να ισχύει για τις ρητές συναρτήσεις ώστε να έχουν πλάγιες ασύμπτωτες?
        \item<3-> Ποιές συναρτήσεις έχουν κατακόρυφες ασύμπτωτες?
        \item<4-> Πού ψάχνουμε κατακόρυφες ασύμπτωτες?
    \end{itemize}

\end{frame}

\section{Ασκήσεις}
\subsection{Άσκηση 1}
\begin{frame}[label=Άσκηση1,t]{Εξάσκηση 1}
    Να υπολογιστούν τα παρακάτω όρια:
    \begin{enumerate}
        \item<1-> $\lim\limits_{x \to 1}{ \dfrac{x^7+x-2}{x^2-x} }$
        \item<2-> $\lim\limits_{x \to 0}{ \dfrac{e^x-1}{x} }$
        \item<3-> $\lim\limits_{x \to 1}{ \dfrac{\ln x}{x-1} }$
        \item<4-> $\lim\limits_{x \to 0}{ \dfrac{x-ημx}{1-συνx} }$
    \end{enumerate}

    % \hyperlink{Λύση1}{\beamerbutton{Λύση}}
\end{frame}

\subsection{Άσκηση 2}
\begin{frame}[label=Άσκηση2,t]{Εξάσκηση 2}
    Να υπολογίσετε τα παρακάτω όρια:
    \begin{enumerate}
        \item<1-> $\lim\limits_{x \to +\infty}{ \dfrac{\ln x}{x} }$
        \item<2-> $\lim\limits_{x \to +\infty}{ \dfrac{e^x}{x^2} }$
    \end{enumerate}

    % \hyperlink{Λύση2}{\beamerbutton{Λύση}}
\end{frame}

\subsection{Άσκηση 3}
\begin{frame}[label=Άσκηση3,t]{Εξάσκηση 3}
    Να υπολογίσετε τα παρακάτω όρια:
    \begin{enumerate}
        \item<1-> $\lim\limits_{x \to +\infty}{ (x-\ln x-e^x) }$
        \item<2-> $\lim\limits_{x \to +\infty}{ (x-\ln x-1) }$
        \item<3-> $\lim\limits_{x \to +\infty}{ \dfrac{e^x-3^x}{x-\ln x} }$
    \end{enumerate}

    % \hyperlink{Λύση3}{\beamerbutton{Λύση}}
\end{frame}

\subsection{Άσκηση 4}
\begin{frame}[label=Άσκηση4,t]{Εξάσκηση 4}
    Να βρείτε τα όρια:
    \begin{enumerate}
        \item<1-> $\lim\limits_{x \to +\infty}{ (x-\ln (1+e^x)) }$
        \item<2-> $\lim\limits_{x \to 1}{ \left( \dfrac{1}{\ln x} - \dfrac{1}{x-1} \right)  }$
    \end{enumerate}

    % \hyperlink{Λύση4}{\beamerbutton{Λύση}}
\end{frame}

\subsection{Άσκηση 5}
\begin{frame}[label=Άσκηση5,t]{Εξάσκηση 5}
    Να υπολογίσετε τα παρακάτω όρια:
    \begin{enumerate}
        \item<1-> $\lim\limits_{x \to -\infty}{ (xe^x) }$
        \item<2-> $\lim\limits_{x \to -0}{ (x\ln x) }$
        \item<3-> $\lim\limits_{x \to 0^+}{ \left( xe^{\frac{1}{x}} \right)  }$
        \item<4-> $\lim\limits_{x \to -\infty}{ \dfrac{1}{xe^{x^3}} }$
    \end{enumerate}

    % \hyperlink{Λύση5}{\beamerbutton{Λύση}}
\end{frame}

\subsection{Άσκηση 6}
\begin{frame}[label=Άσκηση6,t]{Εξάσκηση 6}
    Να υπολογίσετε τα παρακάτω όρια:
    \begin{enumerate}
        \item<1-> $\lim\limits_{x \to 0^-}{ x^x }$
        \item<2-> $\lim\limits_{x \to +\infty}{ \left( 1+2x \right)^{\frac{1}{x}}  }$
    \end{enumerate}

    % \hyperlink{Λύση6}{\beamerbutton{Λύση}}
\end{frame}

\subsection{Άσκηση 7}
\begin{frame}[label=Άσκηση7,t]{Εξάσκηση 7}
    Έστω η συνάρτηση $f(x)=\dfrac{x^2+x+2a}{x-a^2}$. Να βρείτε τις τιμές του $α\in\mathbb{R}$, για τις οποίες η ευθεία $ε:x=1$ είναι ασύμπτωτη της $C_f$

    %\hyperlink{Λύση7}{\beamerbutton{Λύση}}
\end{frame}

\subsection{Άσκηση 8}
\begin{frame}[label=Άσκηση8,t]{Εξάσκηση 8}
    Δίνεται η συνάρτηση $f(x)=\dfrac{a^2x^n+5x+1}{x^2+1}$. Να βρείτε τις τιμές των $a\in\mathbb{R}^*$ και $n\in\mathbb{N}-{0,1}$ για τις οποίες η ευθεία $ε:y=1$ είναι οριζόντια ασύμπτωτη της $C_f$ στο $+\infty$

    %\hyperlink{Λύση8}{\beamerbutton{Λύση}}
\end{frame}

\subsection{Άσκηση 9}
\begin{frame}[label=Άσκηση9,t]{Εξάσκηση 9}
    Να βρείτε τις τιμές των $α$ και $β\in\mathbb{R}$, ώστε
    $$\lim\limits_{x \to +\infty}{ \left(   \dfrac{αx^2+βx+3}{x-1}-x \right)}=2$$

    %\hyperlink{Λύση9}{\beamerbutton{Λύση}}
\end{frame}

\begin{frame}
    Στο moodle θα βρείτε τις ασκήσεις που πρέπει να κάνετε, όπως και αυτή τη παρουσίαση
\end{frame}


\appendix

\section{Αποδείξεις}
\begin{frame}[label=Απόδειξη1]{Απόδειξη σημείο καμπής}
    \onslide<1-> Έστω ότι η $f$ έχει σημείο καμπής στο $x_0$ με κυρτή αριστερά και κοίλη δεξιά του σημείου.

    Άρα $f'(x)< f'(x_0)$ για κάθε $x<x_0$ και $f'(x)<f'(x_0)$ για κάθε $x>x_0$

    \onslide<2-> Αφού $f'$ παραγωγίσιμη, θα υπάρχει το όριο

    $$f''(x_0)=\lim\limits_{x \to x_0^-}{ \dfrac{f'(x)-f'(x_0)}{x-x_0} }\ge 0$$

    \onslide<3-> όμοια
    $$f''(x_0)=\lim\limits_{x \to x_0^+}{ \dfrac{f'(x)-f'(x_0)}{x-x_0} } \le 0$$

    \onslide<4-> Άρα $f''(x_0)=0$ \hyperlink{Θεώρημα1}{\beamerbutton{Πίσω στη θεωρία}}
\end{frame}


% \section{Λύσεις Ασκήσεων}
% \begin{frame}
%  \tableofcontents
% \end{frame}
%
% \subsection{Άσκηση 1}
% \begin{frame}[label=Λύση1]
%  Με θεώρημα ενδιαμέσων τιμών. Η συνάρτηση είναι συνεχής στο $[10,11]$ με $f(10)=1024$ και $f(11)=2048$. Αφού $2023\in (1024,2048)$ υπάρχει $x_0$...
%
%  \hyperlink{Άσκηση1}{\beamerbutton{Πίσω στην άσκηση}}
% \end{frame}
%
% \subsection{Άσκηση 2}
% \begin{frame}[label=Λύση2]
%  Με Bolzano ή με μέγιστης ελάχιστης τιμής και ΘΕΤ.
%
%  \begin{gather*}
%   f(3)<f(2)<f(1) \\
%   3f(3)<f(1)+f(2)+f(3)<3f(1) \\
%   f(3)<\dfrac{f(1)+f(2)+f(3)}{3}<f(1)
%  \end{gather*}
%
%  \hyperlink{Άσκηση2}{\beamerbutton{Πίσω στην άσκηση}}
% \end{frame}
%
% \subsection{Άσκηση 3}
% \begin{frame}[label=Λύση3]
%  Προφανές ελάχιστο στα $x_1=1$ και $x_2=3$. Ως συνεχής στο $[1,3]$ έχει σίγουρα ΚΑΙ μέγιστο στο $(1,3)$
%
%  \hyperlink{Άσκηση3}{\beamerbutton{Πίσω στην άσκηση}}
% \end{frame}
%
% \subsection{Άσκηση 4}
% \begin{frame}[label=Λύση4]
%  Η συνάρτηση `απόστασης` $f(x)-x$ είναι ορισμένη στο κλειστό διάστημα και έχει σίγουρα μέγιστο
%
%  \hyperlink{Άσκηση4}{\beamerbutton{Πίσω στην άσκηση}}
% \end{frame}
%
% \subsection{Άσκηση 5}
% \begin{frame}[label=Λύση5]
%  Όμοια με την Άσκηση 2
%
%  \hyperlink{Άσκηση5}{\beamerbutton{Πίσω στην άσκηση}}
% \end{frame}
%
% \subsection{Άσκηση 6}
% \begin{frame}[label=Λύση6]
%  \begin{enumerate}
%   \item Είναι γνησίως αύξουσα άρα $(f(+\infty),f(-\infty))$
%   \item Προφανώς $[f(0),f(1)]$...
%  \end{enumerate}
%
%  \hyperlink{Άσκηση6}{\beamerbutton{Πίσω στην άσκηση}}
% \end{frame}

\end{document}
