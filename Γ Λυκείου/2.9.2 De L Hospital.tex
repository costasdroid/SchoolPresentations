\documentclass{presentation}

\title{Συναρτήσεις}
\subtitle{Κανόνας De L' Hospital}
\author[Λόλας]{Κωνσταντίνος Λόλας}
\institute[$10^ο$ ΓΕΛ]{$10^ο$ ΓΕΛ Θεσσαλονίκης}


\begin{document}

\begin{frame}
  \titlepage
\end{frame}

\section{Θεωρία}
\begin{frame}{Ας τελειώσουμε με τα όρια ΕΠΙΤΕΛΟΥΣ}
  Αφήσαμε κάποια όρια
  \begin{itemize}
    \item $\dfrac{0}{0}$
    \item $\dfrac{\pm\infty}{\pm\infty}$
    \item $0\cdot \infty$
    \item $1^{+\infty}$
    \item $(+\infty)^0$
    \item $0^0$
  \end{itemize}
\end{frame}

\begin{frame}{Ένας για όλους και όλοι για έναν}
  Θα ήταν τέλεια αν κατέληγαν όλες οι περιπτώσεις σε μία!!!!
  \begin{itemize}
    \item<1-> $0\cdot \infty$ \onslide<2->$\implies \dfrac{0}{0}$ ή $\dfrac{\pm\infty}{\pm\infty}$
    \item<3-> $1^{+\infty}$ \onslide<4-> $\implies e^{+\infty\ln 1}=e^{+\infty\cdot 0}$
    \item<5-> $(+\infty)^0$ \onslide<6-> $\implies e^{0\ln\infty}=e^{+\infty\cdot 0}$
    \item<7-> $0^0$ \onslide<8-> $\implies e^{0\ln 0}=e^{+\infty\cdot 0}$
  \end{itemize}
\end{frame}

\begin{frame}{Ορισμός}
  \begin{block}{Κανόνας De L' Hospital}
    Αν $\lim\limits_{x \to x_0}{ \dfrac{f(x)}{g(x)} }=\dfrac{\pm\infty}{\pm\infty}$ ή $\dfrac{0}{0}$, με $x_0\in\bar{\mathbb{R}}$ τότε αν
    $$\lim\limits_{x \to x_0}{ \dfrac{f'(x)}{g'(x)} }=k\in\bar{\mathbb{R}}$$
    όπου $\bar{\mathbb{R}}=\mathbb{R}\cup \{-\infty,+\infty\}$ τότε
    $$\lim\limits_{x \to x_0}{ \dfrac{f(x)}{g(x)} }=k$$
  \end{block}
\end{frame}

\begin{frame}{Παρατηρήσεις}

  Ο κανόνας De L' Hospital δεν είναι απλό παιχνίδι!

  \begin{itemize}[<+->]
    \item Έχει προϋποθέσεις!
    \item Αν προκύπτει πάλι απροσδιοριστία ίσως ΞΑΝΑ DLH
    \item Ισχύουν και για πλευρικά
    \item Δεν ισχύει το αντίστροφο!
  \end{itemize}

\end{frame}

\begin{frame}{Τα γνωστά!}

  Ας αποδείξουμε τα γνωστά όρια:

  \begin{itemize}[<+->]
    \item $\lim\limits_{x \to 0}{ \dfrac{ημx}{x} }$
    \item $\lim\limits_{x \to 0}{ \dfrac{1-συνx}{x} }$
  \end{itemize}

\end{frame}

\begin{frame}{Πώς θα το γράφουμε!}
  Κανονικά θα έπρεπε... αλλά! π.χ.
  \begin{align*}
    \lim\limits_{x \to 0}{ \frac{ημx}{x} }\underset{\text{DLH}}{\overset{\left( \frac{0}{0} \right) }{=}} \lim\limits_{x \to 0}{ \frac{(ημx)'}{(x)'} }=\lim\limits_{x \to 0}{ \frac{συνx}{1} }=1
  \end{align*}
\end{frame}

\section{Ασκήσεις}

\begin{frame}[noframenumbering]
  \vfill
  \centering
  \begin{beamercolorbox}[sep=8pt,center,shadow=true,rounded=true]{title}
    \usebeamerfont{title}Ασκήσεις
  \end{beamercolorbox}
  \vfill
\end{frame}

\begin{askisi}
  Να υπολογιστούν τα παρακάτω όρια:
  \begin{enumerate}
    \item<1-> $\lim\limits_{x \to 1}{ \dfrac{x^7+x-2}{x^2-x} }$
    \item<2-> $\lim\limits_{x \to 0}{ \dfrac{e^x-1}{x} }$
    \item<3-> $\lim\limits_{x \to 1}{ \dfrac{\ln x}{x-1} }$
    \item<4-> $\lim\limits_{x \to 0}{ \dfrac{x-ημx}{1-συνx} }$
  \end{enumerate}

  % \hyperlink{Λύση1}{\beamerbutton{Λύση}}
\end{askisi}

\begin{askisi}
  Να υπολογίσετε τα παρακάτω όρια:
  \begin{enumerate}
    \item<1-> $\lim\limits_{x \to +\infty}{ \dfrac{\ln x}{x} }$
    \item<2-> $\lim\limits_{x \to +\infty}{ \dfrac{e^x}{x^2} }$
  \end{enumerate}

  % \hyperlink{Λύση2}{\beamerbutton{Λύση}}
\end{askisi}

\begin{askisi}
  Να υπολογίσετε τα παρακάτω όρια:
  \begin{enumerate}
    \item<1-> $\lim\limits_{x \to +\infty}{ (x-\ln x-e^x) }$
    \item<2-> $\lim\limits_{x \to +\infty}{ (x-\ln x-1) }$
    \item<3-> $\lim\limits_{x \to +\infty}{ \dfrac{e^x-3^x}{x-\ln x} }$
  \end{enumerate}

  % \hyperlink{Λύση3}{\beamerbutton{Λύση}}
\end{askisi}

\begin{askisi}
  Να βρείτε τα όρια:
  \begin{enumerate}
    \item<1-> $\lim\limits_{x \to +\infty}{ (x-\ln (1+e^x)) }$
    \item<2-> $\lim\limits_{x \to 1}{ \left( \dfrac{1}{\ln x} - \dfrac{1}{x-1} \right)  }$
  \end{enumerate}

  % \hyperlink{Λύση4}{\beamerbutton{Λύση}}
\end{askisi}

\begin{askisi}
  Να υπολογίσετε τα παρακάτω όρια:
  \begin{enumerate}
    \item<1-> $\lim\limits_{x \to -\infty}{ (xe^x) }$
    \item<2-> $\lim\limits_{x \to -0}{ (x\ln x) }$
    \item<3-> $\lim\limits_{x \to 0^+}{ \left( xe^{\frac{1}{x}} \right)  }$
    \item<4-> $\lim\limits_{x \to -\infty}{ \dfrac{1}{xe^{x^3}} }$
  \end{enumerate}

  % \hyperlink{Λύση5}{\beamerbutton{Λύση}}
\end{askisi}

\begin{askisi}
  Να υπολογίσετε τα παρακάτω όρια:
  \begin{enumerate}[<+->]
    \item $\lim\limits_{x \to 0^-}{ x^x }$
    \item $\lim\limits_{x \to +\infty}{ \left( 1+2x \right)^{\frac{1}{x}}  }$
    \item $\lim\limits_{x \to 0^-}{ \left( 1+x \right)^{σφx}  }$
  \end{enumerate}
\end{askisi}

\begin{askisi}
  Να υπολογίσετε τα παρακάτω όρια:
  \begin{enumerate}[<+->]
    \item $\lim\limits_{x \to 0}{ \left( ημx\cdot \ln x \right)  }$
    \item $\lim\limits_{x \to 0}{ \left[ (e^x-1)ημ\frac{1}{x} \right]  }$
  \end{enumerate}
\end{askisi}

\begin{askisi}
  Να υπολογίσετε το

  $$\lim\limits_{x \to 0}{ \frac{(e^x-1)^2(συνx-1)^3}{ημ^4x\cdot \ln (1+x)}  }$$
\end{askisi}

\begin{askisi}
  Έστω $f:\mathbb{R}\to\mathbb{R}$ μία συνάρτηση με $f(1)=f'(1)=0$. Να υπολογίσετε το $\lim\limits_{x \to 1}{ \frac{f(x+\ln x)}{x-1}}$
\end{askisi}

\begin{askisi}
  Έστω $f:\mathbb{R}\to\mathbb{R}$ μία συνάρτηση παραγωγίσιμη με $f(0)=f'(0)=0$ και f''(0)=2. Έστω $g(x)=\begin{cases}\frac{f(x)}{x} & ,x\ne 0 \\
             0              & ,x=0\end{cases}$.

  \begin{enumerate}[<+->]
    \item Να βρείτε την $g'(0)$
    \item Να δείξετε ότι η $g'$ είναι συνεχής στο $x_0=0$
  \end{enumerate}
\end{askisi}

\begin{askisi}
  Έστω $f:\mathbb{R}\to\mathbb{R}$ μία συνάρτηση η οποία είναι δύο φορές παραγωγίσιμη. Να δείξετε ότι:

  $$\lim\limits_{h \to 0}{ \frac{f(x+2h)-3f(x)+2f(x-h)}{h^2} }=3f''(x)$$
\end{askisi}

\begin{frame}
  Στο moodle θα βρείτε τις ασκήσεις που πρέπει να κάνετε, όπως και αυτή τη παρουσίαση
\end{frame}

\end{document}
