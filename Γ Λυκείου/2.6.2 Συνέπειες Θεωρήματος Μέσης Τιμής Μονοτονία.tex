\documentclass[greek]{beamer}
%\usepackage{fontspec}
\usepackage{amsmath,amsthm}
\usepackage{unicode-math}
\usepackage{xltxtra}
\usepackage{graphicx}
\usetheme{CambridgeUS}
\usecolortheme{seagull}
\usepackage{hyperref}
\usepackage{ulem}
\usepackage{xgreek}

\usepackage{pgfpages}
\usepackage{tikz}
\usepackage{tkz-tab}
%\setbeameroption{show notes on second screen}
%\setbeameroption{show only notes}

\setsansfont{Noto Serif}

\usepackage{multicol}

\usepackage{appendixnumberbeamer}

\setbeamercovered{transparent}
\beamertemplatenavigationsymbolsempty

\title{Συναρτήσεις}
\subtitle{Μονοτονία}
\author[Λόλας]{Κωνσταντίνος Λόλας}
\date{}

\begin{document}

\begin{frame}
 \titlepage
\end{frame}

\section{Θεωρία}
\begin{frame}{Κάτι που αφήσαμε πιο πριν...}
 Δεν μπορούσαμε να υπολογίσουμε όλων των συναρτήσεων την μονοτονία π.χ.
 \begin{enumerate}
  \item $e^x-x$
  \item $x^3-2x+1$
  \item $\ln x - x^2$
 \end{enumerate}
\end{frame}

\begin{frame}{Spoiler}
 Γιατί υπολογίζαμε κλίση (εκτός διαφορικών εξισώσεων??)
 \onslide<2-> θετική κλίση? Αρνητική?
\end{frame}

\begin{frame}{Μονοτονία}
 \begin{block}{Μονοτονία}
  Έστω μία συνάρτηση $f$ συνεχής στο $Δ$. Αν $f'(x)>0$ σε κάθε εσωτερικό σημείο $x$ του $Δ$, τότε η $f$ είναι γνησίως αύξουσα στο $Δ$
 \end{block}
 \onslide<2-> Όμοια για $f'(x)<0$
\end{frame}

\begin{frame}{Από εδώ και εμπρός...}
 \begin{enumerate}
  \item<1-> Ξαναλύνουμε όλες τις ασκήσεις με μονοτονία, τώρα όμως ΟΛΕΣ
  \item<2-> Όσοι δεν τα μάθανε καλά, είναι ευκαιρία τώρα να επιστρέψουν
  \item<3-> Όσοι τα είχατε καταλάβει ευκαιρία για επανάληψη
  \item<4-> Θα ασχολούμαστε με πρόσημα!!!!
 \end{enumerate}
\end{frame}

\begin{frame}{Και λίγο κατανόηση δεν βλάπτει}
 \begin{enumerate}
   \item<1-> Αν $f'>0$ τότε $f\uparrow$ \only<2> {\emph{ΛΑΘΟΣ!!!!!!!!!!}}
   \item<3-> Αν $f\uparrow$ τότε $f>0$ \only<4> {\emph{ΛΑΘΟΣ!!!!!!!!!!}}
   \item<5-> Αν $f\uparrow$ τότε $f\ge 0$ \only<6> {\emph{ΛΑΘΟΣ!!!!!!!!!!}}
   \item<7-> Αν $f'\ne 0$ τότε $f$ γνησίως μονότονη \only<8> {\emph{ΛΑΘΟΣ!!!!!!!!!!}}
 \end{enumerate}
\end{frame}

\section{Ασκήσεις}
\subsection{Άσκηση 1}
\begin{frame}[label=Άσκηση1]{Εξάσκηση 1}
 Έστω $f:(0,+\infty)\to\mathbb{R}$ μία συνάρτηση, η οποία είναι παραγωγίσιμη και ισχύει $xf'(x)-2f(x)=0$ για κάθε $x>0$
 \begin{enumerate}
  \item<1-> Να δείξετε ότι η συνάρτηση $g(x)=\frac{f(x)}{x^2}$. $x>0$ είναι σταθερή
  \item<2-> Αν επιπλέον $f(1)=2$ να βρείτε τον τύπο υης $f$
 \end{enumerate}

 % \hyperlink{Λύση1}{\beamerbutton{Λύση}}
\end{frame}

\subsection{Άσκηση 2}
\begin{frame}[label=Άσκηση2]{Εξάσκηση 2}
 Έστω $f:[0,π]\to\mathbb{R}$ μία συνάρτηση με $f(0)=1$, η οποία είναι συνεχής και ισχύει $f'(x)=xσυνx$ για κάθε $x\in (0,π)$

 Να δείξετε ότι $f(x)=xημx+συνx$, $x\in [0,π]$

 % \hyperlink{Λύση2}{\beamerbutton{Λύση}}
\end{frame}

\subsection{Άσκηση 3}
\begin{frame}[label=Άσκηση3]{Εξάσκηση 3}
 Έστω $f:(0,+\infty)\to\mathbb{R}$ μία συνάρτηση με $f(1)=0$, η οποία είναι παραγωγίσιμη και ισχύει $f'(x)=\frac{1-xf(x)}{x^2}$ για κάθε $x>0$

 Να δείξετε ότι $f(x)=\frac{\ln x}{x}$, $x>0$

 % \hyperlink{Λύση3}{\beamerbutton{Λύση}}
\end{frame}

\subsection{Άσκηση 4}
\begin{frame}[label=Άσκηση4]{Εξάσκηση 4}
 Έστω $f:(0,π)\to\mathbb{R}$ μία συνάρτηση με $f(\frac{π}{2})=1$, η οποία είναι παραγωγίσιμη και ισχύει $f'(x)-σφx\cdot f(x)=0$ για κάθε $x\in (0,π)$

 Να δείξετε ότι $f(x)=ημx$, $x\in (0,π)$

 % \hyperlink{Λύση4}{\beamerbutton{Λύση}}
\end{frame}

\subsection{Άσκηση 5}
\begin{frame}[label=Άσκηση5]{Εξάσκηση 5}
 Έστω $f:\mathbb{R}\to\mathbb{R}$ μία συνάρτηση με $f(0)=1$, η οποία είναι παραγωγίσιμη και ισχύει $f'(x)=f(x)-e^xημx$ για κάθε $x\in\mathbb{R}$

 Να δείξετε ότι $f(x)=e^xσυνx$, $x\in\mathbb{R}$

 % \hyperlink{Λύση5}{\beamerbutton{Λύση}}
\end{frame}

\subsection{Άσκηση 6}
\begin{frame}[label=Άσκηση6]{Εξάσκηση 6}
 Έστω $f:\mathbb{R}\to\mathbb{R}$ μία συνάρτηση με $f(0)=0$, η οποία είναι παραγωγίσιμη και ισχύει $f'(x)=2xe^{-f(x)}$ για κάθε $x\in\mathbb{R}$

 Να δείξετε ότι $f(x)=\ln (x^2+1)$, $x\in\mathbb{R}$

 % \hyperlink{Λύση6}{\beamerbutton{Λύση}}
\end{frame}

\subsection{Άσκηση 7}
\begin{frame}[label=Άσκηση7]{Εξάσκηση 7}
 Έστω $f$, $g:\mathbb{R}\to\mathbb{R}$ δύο συναρτήσεις οι οποίες είναι δύο φορές παραγωγίσιμες και ισχύουν
 \begin{itemize}
  \item $f''(x)=g''(x)$ για κάθε $x\in\mathbb{R}$
  \item $f(0)=g(0)+1$
 \end{itemize}

 Να δείξετε ότι $f(0)=g(0)+1$

 %\hyperlink{Λύση7}{\beamerbutton{Λύση}}
\end{frame}

\subsection{Άσκηση 8}
\begin{frame}[label=Άσκηση8]{Εξάσκηση 8}
 Έστω $f:(0,+\infty)\to\mathbb{R}$ μία συνάρτηση με $f(0)=0$, η οποία είναι συνεχής και ισχύει $f(x)=x(f(x)-f'(x))$ για κάθε $x>0$
 \begin{enumerate}
  \item<1-> Αν $g(x)=xf(x)$, $x>0$ να δείξετε ότι $g(x)=c\cdot e^x$, $x>0$
  \item<2-> Αν επιπλέον $f(1)=e$ να βρείτε τον τύπο της $f$
 \end{enumerate}

 %\hyperlink{Λύση8}{\beamerbutton{Λύση}}
\end{frame}

\subsection{Άσκηση 9}
\begin{frame}[label=Άσκηση9]{Εξάσκηση 9}
 Έστω $f:\mathbb{R}\to\mathbb{R}$ μία συνάρτηση η οποία είναι δύο φορές παραγωγίσιμη και ισχύουν
 \begin{itemize}
  \item $2f(x)+4xf'(x)+(x^2+9)f''(x)=0$ για κάθε $x\in\mathbb{R}$
  \item $f(0)=0$ και $f'(0)=\frac{1}{9}$
 \end{itemize}

 Να δείξετε ότι $f(x)=\frac{x}{x^2+9}$

 %\hyperlink{Λύση9}{\beamerbutton{Λύση}}
\end{frame}

\subsection{Άσκηση 10}
\begin{frame}[label=Άσκηση10]{Εξάσκηση 10}
 Έστω $f:\mathbb{R}\to\mathbb{R}$ μία συνάρτηση για την οποία ισχύει
 $$|f(x)-f(y)|\le (x-y)^2 \text{ για κάθε } x,y\in\mathbb{R}$$

 Να δείξετε ότι η $f$ είναι σταθερή

 %\hyperlink{Λύση10}{\beamerbutton{Λύση}}
\end{frame}

\subsection{Άσκηση 11}
\begin{frame}[label=Άσκηση11]{Εξάσκηση 11}
 Αν για τις συναρτήσεις $f$, $g$ ισχύουν $f(0)=0$, $g(0)=1$, $f'(x)=g(x)$ και $g'(x)=-f(x)$ για κάθε $x\in\mathbb{R}$ να αποδείξετε ότι:
 \begin{enumerate}
  \item<1-> $f^2(x)+g^2(x)=1$, $x\in\mathbb{R}$
  \item<2-> $f(x)=ημx$, $x\in\mathbb{R}$ και $g(x)=συνx$, $x\in\mathbb{R}$
 \end{enumerate}

 %\hyperlink{Λύση11}{\beamerbutton{Λύση}}
\end{frame}


\appendix

\section{Αποδείξεις}
\begin{frame}{Απόδειξη σταθερής συνάρτησης}
 Θα δείξουμε ότι για κάθε $x_1<x_2\in Δ \implies f(x_1)<f(x_2)$.

 \onslide<1-> Στο $(x_1,x_2)$ είναι παραγωγίσιμη, άρα θα ισχύει το ΘΜΤ

 \onslide<2-> Υπάρχει $ξ\in Δ$ ώστε $f'(ξ)=\frac{f(x_2)-f(x_1)}{x_2-x_1}$.

 \onslide<3-> Αλλά $f'(x)>0$ για κάθε $x\in Δ$

 \onslide<4-> Άρα $f'(ξ)=\frac{f(x_2)-f(x_1)}{x_2-x_1}>0$

 \onslide<5-> $f(x_2)>f(x_1)$
\end{frame}


% \section{Λύσεις Ασκήσεων}
% \begin{frame}
%  \tableofcontents
% \end{frame}
%
% \subsection{Άσκηση 1}
% \begin{frame}[label=Λύση1]
%  Με θεώρημα ενδιαμέσων τιμών. Η συνάρτηση είναι συνεχής στο $[10,11]$ με $f(10)=1024$ και $f(11)=2048$. Αφού $2023\in (1024,2048)$ υπάρχει $x_0$...
%
%  \hyperlink{Άσκηση1}{\beamerbutton{Πίσω στην άσκηση}}
% \end{frame}
%
% \subsection{Άσκηση 2}
% \begin{frame}[label=Λύση2]
%  Με Bolzano ή με μέγιστης ελάχιστης τιμής και ΘΕΤ.
%
%  \begin{gather*}
%   f(3)<f(2)<f(1) \\
%   3f(3)<f(1)+f(2)+f(3)<3f(1) \\
%   f(3)<\frac{f(1)+f(2)+f(3)}{3}<f(1)
%  \end{gather*}
%
%  \hyperlink{Άσκηση2}{\beamerbutton{Πίσω στην άσκηση}}
% \end{frame}
%
% \subsection{Άσκηση 3}
% \begin{frame}[label=Λύση3]
%  Προφανές ελάχιστο στα $x_1=1$ και $x_2=3$. Ως συνεχής στο $[1,3]$ έχει σίγουρα ΚΑΙ μέγιστο στο $(1,3)$
%
%  \hyperlink{Άσκηση3}{\beamerbutton{Πίσω στην άσκηση}}
% \end{frame}
%
% \subsection{Άσκηση 4}
% \begin{frame}[label=Λύση4]
%  Η συνάρτηση `απόστασης` $f(x)-x$ είναι ορισμένη στο κλειστό διάστημα και έχει σίγουρα μέγιστο
%
%  \hyperlink{Άσκηση4}{\beamerbutton{Πίσω στην άσκηση}}
% \end{frame}
%
% \subsection{Άσκηση 5}
% \begin{frame}[label=Λύση5]
%  Όμοια με την Άσκηση 2
%
%  \hyperlink{Άσκηση5}{\beamerbutton{Πίσω στην άσκηση}}
% \end{frame}
%
% \subsection{Άσκηση 6}
% \begin{frame}[label=Λύση6]
%  \begin{enumerate}
%   \item Είναι γνησίως αύξουσα άρα $(f(+\infty),f(-\infty))$
%   \item Προφανώς $[f(0),f(1)]$...
%  \end{enumerate}
%
%  \hyperlink{Άσκηση6}{\beamerbutton{Πίσω στην άσκηση}}
% \end{frame}

\end{document}
