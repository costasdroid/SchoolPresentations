\documentclass[greek]{beamer}
%\usepackage{fontspec}
\usepackage{amsmath,amsthm}
\usepackage{unicode-math}
\usepackage{xltxtra}
\usepackage{graphicx}
\usetheme{CambridgeUS}
\usecolortheme{seagull}
\usepackage{hyperref}
\usepackage{ulem}
\usepackage{xgreek}

\usepackage{pgfpages}
\usepackage{tikz}
\usepackage{tkz-tab}
%\setbeameroption{show notes on second screen}
%\setbeameroption{show only notes}

\setsansfont{Noto Serif}

\usepackage{multicol}

\usepackage{appendixnumberbeamer}

\setbeamercovered{transparent}
\beamertemplatenavigationsymbolsempty

\title{Συναρτήσεις}
\subtitle{Θεώρημα Rolle}
\author[Λόλας]{Κωνσταντίνος Λόλας}
\date{}

\begin{document}

\begin{frame}
  \titlepage
\end{frame}

\section{Θεωρία}
\begin{frame}{Ήρθε η ώρα για τα δύσκολα}
  Το πιό εύκολο κεφάλαιο, με τις πιο δύσκολες ασκήσεις!
  \begin{enumerate}
    \item<1-> το θεώρημα... γελοίο
    \item<2-> οι εφαρμογές... πφφφφ
    \item<3-> οι αρχικές ασκήσεις... παιχνιδάκι
    \item<4-> όταν είναι όμως στις πανελλαδικές... ΠΑΤΟΣ
  \end{enumerate}
\end{frame}

\begin{frame}{But Whyyyyyyyyyy!}
  Xωρίς τον Rolle ξεχάστε
  \begin{enumerate}
    \item<1-> μονοτονία
    \item<2-> ακρότατα
    \item<3-> αντιπαράγωγο, διαφορικές κτλ
  \end{enumerate}
\end{frame}

\begin{frame}{Ώρα για Ζωγραφιές}
  \begin{enumerate}
    \item<1-> Φτιάξτε άξονες
    \item<2-> Διαλέξτε δύο διαφορετικές τιμές στον άξονα των $x$ τις $α$ και $β$
    \item<3-> θεωρήστε δύο σημεία μιας συνάρτησης με $f(α)=f(β)$
    \item<4-> Φτιάξτε παραγωγίσιμη συνάρτηση στο $[α,β]$ και μελετήστε την εφαπτόμενή της
    \item<5-> Εντοπίστε σημεία στα οποία η εφαπτόμενη είναι παράλληλη με τον άξονα $x'x$
    \item<6-> Επαναλάβετε όλη τη διαδικασία, δημιουργώντας συνάρτηση που δεν έχει οριζόντια εφαπτόμενη
  \end{enumerate}
  \onslide<7-> Συμπέρασμα

\end{frame}

\begin{frame}{Θεώρημα Rolle}
  \begin{block}{Θεώρημα Rolle}
    Έστω μία συνάρτηση $f$:
    \begin{itemize}
      \item συνεχής στο $[α,β]$
      \item παραγωγίσιμη στο $(α,β)$
      \item $f(α)=f(β)$
    \end{itemize}
    τότε υπάρχει $ξ\in (α,β)$ με $f'(ξ)=0$
  \end{block}
\end{frame}

\begin{frame}{Συμπεράσματα}
  \begin{enumerate}
    \item<1-> ο Rolle όπως και ο Bolzano δεν βρίσκει ρίζες, αλλά βεβαιώνει την ύπαρξη
    \item<2-> ο Rolle από την συνάρτηση βγάζει συμπέρασμα για την παράγωγο
    \item<3-> έχει περίεργες προϋποθέσεις
  \end{enumerate}
\end{frame}

\begin{frame}{Πώς θα τον χρησιμοποιούμε?}
  \begin{enumerate}
    \item<1-> βεβαιώνουμε ύπαρξη, αν ο Bolzano δεν μας κάνει
    \item<2-> βρίσκουμε πλήθος ριζών
    \item<3-> βεβαιώνουμε ότι η συνάρτηση είναι 1-1 ?????????
  \end{enumerate}
\end{frame}

\begin{frame}{Δεν σας πείθω για την δυσκολία έ?}
  \onslide<1->
  \begin{block}{Άσκηση 22}
    Αν για τους αριθμούς $α$ και $β$ με $α<β$ ισχύει $\frac{συνα-συνβ}{α-β}=\frac{α+β}{2}$ να δείξετε ότι οι αριθμοί $α$ και $β$ είναι ετερόσημοι
  \end{block}
  \onslide<2->
  \begin{block}{Άσκηση 24}
    Έστω συνάρτηση παραγωγίσιμη στο $[2,3]$ με $2f(3)=3f(2)$. Να δείξετε ότι υπάρχει $x_0\in (2,3)$ ώστε $f'(x_0)=\frac{f(x_0)}{x_0}$
  \end{block}
  \onslide<3-> Και να φανταστείτε ΣΑΣ ΛΕΩ ότι λύνονται με Rolle
\end{frame}

\section{Ασκήσεις}
\subsection{Άσκηση 1}
\begin{frame}[label=Άσκηση1]{Εξάσκηση 1}
  Δίνεται η συνάρτηση $f(x)=x^3-6x^2+9x$
  \begin{enumerate}
    \item<1-> Να δείξετε ότι η $f$ ικανοποιεί τις υποθέσεις του θ. Rolle στο $Δ=[0,3]$
    \item<2-> Να βρείτε τα $ξ\in(0,3)$ για τα οποία ισχύει $f'(ξ)=0$
  \end{enumerate}
  % \hyperlink{Λύση1}{\beamerbutton{Λύση}}
\end{frame}

\subsection{Άσκηση 2}
\begin{frame}[label=Άσκηση2]{Εξάσκηση 2}
  Δίνεται η συνάρτηση $f(x)=(x-2)ημx$. Να αποδείξετε ότι:
  \begin{enumerate}
    \item<1-> Η εξίσωση $f'(x)=0$ έχει μία τουλάχιστον ρίζα στο διάστημα $(2,π)$
    \item<2-> Η εξίσωση $εφx=2-x$ έχει μία τουλάχιστον ρίζα στο $(2,π)$
  \end{enumerate}

  % \hyperlink{Λύση2}{\beamerbutton{Λύση}}
\end{frame}

\subsection{Άσκηση 3}
\begin{frame}[label=Άσκηση3]{Εξάσκηση 3}
  Δίνεται η συνάρτηση $f(x)=x^4-20x^3-25x^2-x+1$
  \begin{enumerate}
    \item<1-> Να αποδείξετε ότι η εξίσωση $f(x)=0$ έχει μία τουλάχιστον ρίζα στο διάστημα $(-1,0)$ και μία τουλάχιστον στο διάστημα $(0,1)$
    \item<2-> Να αποδείξετε ότι η εξίσωση $4x^3-60x^2-50x-1=0$ έχει μία τουλάχιστον ρίζα στο διάστημα $(-1,1)$
  \end{enumerate}
  % \hyperlink{Λύση3}{\beamerbutton{Λύση}}
\end{frame}

\subsection{Άσκηση 4}
\begin{frame}[label=Άσκηση4]{Εξάσκηση 4}
  Έστω $f:\mathbb{R}\to\mathbb{R}$ μία συνάρτηση, η οποία είναι παραγωγίσιμη και ισχύει
  $$f'(x)\ne 1, \text{ για κάθε } x\in\mathbb{R}$$
  Να δείξετε ότι η εξίσωση $f(x)=x$ έχει μία το πολύ ρίζα
  % \hyperlink{Λύση4}{\beamerbutton{Λύση}}
\end{frame}

\subsection{Άσκηση 5}
\begin{frame}[label=Άσκηση5]{Εξάσκηση 5}
  Δίνεται η συνάρτηση $f(x)=2^x+x^2-2x-1$
  \begin{enumerate}
    \item<1-> Να αποδείξετε ότι για την $f$ ισχύουν οι υποθέσεις του Rolle στο $[0,1]$
    \item<2-> Να αποδείξετε ότι η $f$ έχει δύο το πολύ ρίζες
    \item<3-> Να βρείτε τα κοινά σημεία των γραφικών παραστάσεων των συναρτήσεων:
      $$g(x)=2^x \text{ και } h(x)=2x-x^2+1$$
  \end{enumerate}
  % \hyperlink{Λύση5}{\beamerbutton{Λύση}}
\end{frame}

\subsection{Άσκηση 6}
\begin{frame}[label=Άσκηση6]{Εξάσκηση 6}
  Δίνεται η συνάρτηση $f(x)=ημ(2x)$. Να δείξετε ότι η $f$ ικανοποιεί τις υποθέσεις του Rolle στο διάστημα $[0,π]$ και στη συνέχεια, να βρείτε όλα τα $ξ\in (0,π)$ για τα οποία ισχύει $f'(ξ)=0$
  % \hyperlink{Λύση6}{\beamerbutton{Λύση}}
\end{frame}

\subsection{Άσκηση 7}
\begin{frame}[label=Άσκηση7]{Εξάσκηση 7}
  Αν $0<α<β$ και $α^β=β^α$, να δείξετε ότι:
  \begin{enumerate}
    \item<1-> Για τη συνάρτηση $f(x)=\frac{\ln x}{x}$ ισχύουν οι υποθέσεις Rolle στο $[α,β]$
    \item<2-> $1<α<e<β$
  \end{enumerate}
  %\hyperlink{Λύση7}{\beamerbutton{Λύση}}
\end{frame}

\subsection{Άσκηση 8}
\begin{frame}[label=Άσκηση8]{Εξάσκηση 8}
  Έστω $f:\mathbb{R}\to\mathbb{R}$ μία συνάρτηση, η οποία είναι παραγωγίσιμη και ισχύει
  $$f'(x)\ne 0, \text{ για κάθε } x\in\mathbb{R}$$
  Να δείξετε ότι η $f$ είναι συνάρτηση $1-1$
  %\hyperlink{Λύση8}{\beamerbutton{Λύση}}
\end{frame}

\subsection{Άσκηση 9}
\begin{frame}[label=Άσκηση9]{Εξάσκηση 9}
  Έστω $f:\mathbb{R}\to\mathbb{R}$ μία συνάρτηση, η οποία είναι παραγωγίσιμη και ισχύουν
  \begin{itemize}
    \item $f'(x)\ne 2x$ για κάθε $x\in\mathbb{R}$
    \item $1<f(x)<2$ για κάθε $x\in\mathbb{R}$
  \end{itemize}
  Να δείξετε ότι υπάρχει μοναδικό $x_0\in (0,1)$ ώστε $f(x_0)=x_0^2+1$
  %\hyperlink{Λύση9}{\beamerbutton{Λύση}}
\end{frame}

\subsection{Άσκηση 10}
\begin{frame}[label=Άσκηση10]{Εξάσκηση 10}
  Έστω $f:\mathbb{R}\to\mathbb{R}$ μία συνάρτηση, η οποία είναι παραγωγίσιμη και η γραφική της παράσταση τέμνει τον άξονα $x'x$ στα σημεία $x_1=1$ και $x_2=2$. Να αποδείξετε ότι:
  \begin{itemize}
    \item Για την συνάρτηση $G(x)=\frac{f(x)}{x-3}$ εφαρμόζεται το θ. Rolle στο $[1,2]$
    \item Υπάρχει $ξ\in (1,2)$ τέτοιο ώστε η εφαπτομένη της $C_f$ στο σημείο $Μ(ξ,f(ξ))$ να διέρχεται από το σημείο $Α(3,0)$
  \end{itemize}
  %\hyperlink{Λύση10}{\beamerbutton{Λύση}}
\end{frame}


%
% \appendix
% \section{Λύσεις Ασκήσεων}
% \begin{frame}
%  \tableofcontents
% \end{frame}
%
% \subsection{Άσκηση 1}
% \begin{frame}[label=Λύση1]
%  Με θεώρημα ενδιαμέσων τιμών. Η συνάρτηση είναι συνεχής στο $[10,11]$ με $f(10)=1024$ και $f(11)=2048$. Αφού $2023\in (1024,2048)$ υπάρχει $x_0$...
%
%  \hyperlink{Άσκηση1}{\beamerbutton{Πίσω στην άσκηση}}
% \end{frame}
%
% \subsection{Άσκηση 2}
% \begin{frame}[label=Λύση2]
%  Με Bolzano ή με μέγιστης ελάχιστης τιμής και ΘΕΤ.
%
%  \begin{gather*}
%   f(3)<f(2)<f(1) \\
%   3f(3)<f(1)+f(2)+f(3)<3f(1) \\
%   f(3)<\frac{f(1)+f(2)+f(3)}{3}<f(1)
%  \end{gather*}
%
%  \hyperlink{Άσκηση2}{\beamerbutton{Πίσω στην άσκηση}}
% \end{frame}
%
% \subsection{Άσκηση 3}
% \begin{frame}[label=Λύση3]
%  Προφανές ελάχιστο στα $x_1=1$ και $x_2=3$. Ως συνεχής στο $[1,3]$ έχει σίγουρα ΚΑΙ μέγιστο στο $(1,3)$
%
%  \hyperlink{Άσκηση3}{\beamerbutton{Πίσω στην άσκηση}}
% \end{frame}
%
% \subsection{Άσκηση 4}
% \begin{frame}[label=Λύση4]
%  Η συνάρτηση `απόστασης` $f(x)-x$ είναι ορισμένη στο κλειστό διάστημα και έχει σίγουρα μέγιστο
%
%  \hyperlink{Άσκηση4}{\beamerbutton{Πίσω στην άσκηση}}
% \end{frame}
%
% \subsection{Άσκηση 5}
% \begin{frame}[label=Λύση5]
%  Όμοια με την Άσκηση 2
%
%  \hyperlink{Άσκηση5}{\beamerbutton{Πίσω στην άσκηση}}
% \end{frame}
%
% \subsection{Άσκηση 6}
% \begin{frame}[label=Λύση6]
%  \begin{enumerate}
%   \item Είναι γνησίως αύξουσα άρα $(f(+\infty),f(-\infty))$
%   \item Προφανώς $[f(0),f(1)]$...
%  \end{enumerate}
%
%  \hyperlink{Άσκηση6}{\beamerbutton{Πίσω στην άσκηση}}
% \end{frame}

\end{document}
