\documentclass{presentation}

\title{Συνδυαστική}
\subtitle{Πόσοι τρόποι υπάρχουν να...}
\author[Λόλας]{Κωνσταντίνος Λόλας}

\begin{document}

\begin{frame}
  \titlepage
\end{frame}

% --- Θεωρία ---

\section{Θεωρία}

\begin{frame}{Η ζωή είναι... συνδυαστική!}
  \begin{itemize}[<+->]
    \item Πόσες διαφορετικές ομάδες φίλων μπορώ να καλέσω για βραδιά Monopoly;
    \item Πόσες τριάδες μαθητών μπορώ να διαλέξω για διαγωνισμό;
    \item Πόσες σαλάτες μπορώ να φτιάξω με 5 υλικά από τα 10 του ψυγείου;
    \item (Και όχι, δεν θα μιλήσουμε για πίτσες...)
  \end{itemize}
\end{frame}

\begin{frame}{Διατάξεις: Το προηγούμενο επεισόδιο}
  \begin{itemize}[<+->]
    \item Θυμάσαι που μετρούσαμε τρόπους με σειρά; Αυτές ήταν οι διατάξεις!
    \item Εδώ όμως... η σειρά ΔΕΝ μετράει!
    \item Πάμε να δούμε πώς μετράμε ομάδες, όχι σειρές!
  \end{itemize}
\end{frame}

\begin{frame}{Ο θρυλικός τύπος των συνδυασμών}
  \begin{block}{Συνδυασμοί}
    Ο αριθμός των τρόπων να επιλέξουμε $k$ αντικείμενα από $n$ χωρίς να μας νοιάζει η σειρά:
    \[
      \binom{n}{k} = \frac{n!}{k!(n-k)!}
    \]
  \end{block}
\end{frame}

\begin{frame}{Από πού ξεφύτρωσε ο τύπος;}
  \begin{block}{Απόδειξη}
    \begin{itemize}[<+->]
      \item Θέλουμε να διαλέξουμε $k$ από $n$ αντικείμενα, χωρίς σειρά.
      \item Αν μας ενδιέφερε η σειρά, θα είχαμε $n \cdot (n-1) \cdots (n-k+1) = \frac{n!}{(n-k)!}$ τρόπους (διατάξεις).
      \item Όμως κάθε ομάδα $k$ αντικειμένων μετρήθηκε $k!$ φορές (όσες οι δυνατές σειρές τους).
      \item Άρα, για να βρούμε τους συνδυασμούς, διαιρούμε με $k!$:
            \[
              \binom{n}{k} = \frac{n!}{k!(n-k)!}
            \]
    \end{itemize}
  \end{block}
\end{frame}

\begin{frame}{Γιατί να μην τα μετρήσω με το χέρι;}
  \begin{itemize}[<+->]
    \item Γιατί αν οι τρόποι είναι 1.000.000, θα χρειαστώ διακοπές πριν τελειώσω!
    \item Γιατί οι συνδυασμοί είναι το "fast forward" της μέτρησης.
    \item Γιατί το να μετράς με το χέρι είναι βαρετό (και επικίνδυνο για τα νεύρα σου).
  \end{itemize}
\end{frame}

\begin{frame}{Συνδυασμοί: Η τέχνη του να διαλέγεις χωρίς να τσακώνεσαι}
  \begin{itemize}[<+->]
    \item Διαλέγουμε 3 από 5 μπάλες:
    \item Πόσοι τρόποι; $\binom{5}{3}=10$
    \item Αν άλλαζε η σειρά, θα μιλούσαμε για διατάξεις (αλλά σήμερα όχι!)
  \end{itemize}
\end{frame}

\moodle

% --- Ασκήσεις ---
\section{Ασκήσεις}


\exercises

\begin{askisi}
  Από μια παρέα 12 φίλων, πόσες διαφορετικές 5μελείς ομάδες μπορούν να φτιαχτούν για να παίξουν escape room;
\end{askisi}

\begin{askisi}
  Μια πίτσα έχει 8 διαφορετικά υλικά. Πόσες διαφορετικές πίτσες μπορείς να φτιάξεις αν κάθε πίτσα έχει ακριβώς 3 υλικά;
\end{askisi}

\begin{askisi}
  Στο σχολείο γίνεται κλήρωση για 4 θέσεις σε ταξίδι στη NASA ανάμεσα σε 30 μαθητές. Πόσοι διαφορετικοί συνδυασμοί μαθητών μπορούν να ταξιδέψουν;
\end{askisi}

\begin{askisi}
  Από τράπουλα 52 φύλλων, πόσες διαφορετικές πεντάδες (χέρι πόκερ) μπορούν να μοιραστούν;
\end{askisi}

\end{document}
