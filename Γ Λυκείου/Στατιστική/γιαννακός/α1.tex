\documentclass[a4paper, 12pt]{article}

\usepackage{amsmath, amsthm, amssymb}

\usepackage{unicode-math}
\usepackage{xltxtra}
\usepackage{xgreek}
\setmainfont{Calibri}

\usepackage{fancyhdr}
\renewcommand{\footrulewidth}{0.4pt}% default is 0pt
\usepackage{lastpage}

\usepackage[a4paper, top=2.5cm, bottom=2.5cm, left=2.5cm, right=2.5cm, headheight=1cm, footskip=1.5cm]{geometry}
\usepackage{enumitem} % Για προσαρμοσμένες λίστες

\usepackage{fancyhdr}
\usepackage{tabularx}
\usepackage{mhchem} % Για χημικούς τύπους

% Ρυθμίσεις σελίδας
\geometry{top=2cm, bottom=2cm, left=2.5cm, right=2.5cm}

% Ρυθμίσεις Επικεφαλίδων
\pagestyle{fancy}
\fancyhf{}
\rhead{Χημεία Γ' Λυκείου}
\lhead{Ογκομέτρηση Ασθενούς Οξέος}
\cfoot{\thepage}

\begin{document}

% ----------------------------------------------------------------------
% 1. ΣΧΕΔΙΟ ΜΑΘΗΜΑΤΟΣ ΓΙΑ ΤΟΝ ΑΞΙΟΛΟΓΗΤΗ
% ----------------------------------------------------------------------
\begin{center}
  \LARGE \textbf{Σχέδιο Μαθήματος (Lesson Plan)} \\
  \large \textbf{Θέμα: Ογκομέτρηση ασθενούς οξέος από ισχυρή βάση} \\
  \vspace{0.2cm}
  \small \textit{Πλαίσιο Αξιολόγησης Α1 - Προσανατολισμός Θετικών Σπουδών}
\end{center}
\section*{Γενικά Στοιχεία}
\begin{itemize}[label=--]
  \item \textbf{Διάρκεια:} 45 λεπτά.
  \item \textbf{Προαπαιτούμενη γνώση:} Ιοντική ισορροπία, pH ασθενών οξέων, υδρόλυση αλάτων, ρυθμιστικά διαλύματα.
  \item \textbf{Μέσα Διδασκαλίας:} Διαδραστικός πίνακας, λογισμικό προσομοίωσης (π.χ. PhET ή Pclabs), ψηφιακό pH-μετρο.
\end{itemize}

\section*{Διδακτικοί Στόχοι}
\begin{enumerate}
  \item Κατανόηση της μεταβολής του pH κατά τη διάρκεια της ογκομέτρησης.
  \item Προσδιορισμός του pH στο ισοδύναμο σημείο (γιατί είναι $pH > 7$;).
  \item Επιλογή κατάλληλου πρωτολυτικού δείκτη.
  \item Εξοικείωση με την καμπύλη ογκομέτρησης και το ημι-ισοδύναμο σημείο ($pH = pK_a$).
\end{enumerate}

\section*{Ροή Μαθήματος (Χρονοδιάγραμμα)}
\begin{itemize}
  \item \textbf{00'--05' (Εισαγωγή):} Υπενθύμιση ογκομέτρησης $HCl/NaOH$. Θέτουμε το ερώτημα: «Τι αλλάζει αν το οξύ μας είναι το ξύδι ($CH_3COOH$);»
  \item \textbf{05'--15' (Θεωρητική Προσέγγιση):} Ανάλυση των 4 σταδίων της ογκομέτρησης:
        \begin{enumerate}
          \item Αρχικό διάλυμα (ασθενές οξύ).
          \item Πριν το ισοδύναμο σημείο (Ρυθμιστικό διάλυμα).
          \item Ισοδύναμο σημείο (Υδρόλυση ανιόντος).
          \item Μετά το ισοδύναμο σημείο (Περίσσεια ισχυρής βάσης).
        \end{enumerate}
  \item \textbf{15'--25' (Προσομοίωση):} Προβολή καμπύλης ογκομέτρησης στον διαδραστικό πίνακα. Παρατήρηση του «άλματος» pH και της περιοχής ρυθμιστικής ικανότητας.
  \item \textbf{25'--40' (Ομαδική Εργασία):} Επίλυση προβλήματος ογκομέτρησης $CH_3COOH$ με $NaOH$ και επιλογή δείκτη.
  \item \textbf{40'--45' (Αξιολόγηση):} Συμπλήρωση Exit Ticket.
\end{itemize}

\newpage

% ----------------------------------------------------------------------
% 2. ΦΥΛΛΟ ΕΡΓΑΣΙΑΣ ΜΑΘΗΤΩΝ (ΟΜΑΔΙΚΟ)
% ----------------------------------------------------------------------
\begin{center}
  \Large \textbf{ΦΥΛΛΟ ΕΡΓΑΣΙΑΣ: «Μελετώντας την καμπύλη ογκομέτρησης»} \\
  \small Τμήμα: \dotfill \quad Ομάδα: \dotfill
\end{center}

\vspace{0.5cm}

\textbf{Δραστηριότητα 1: Η Χημεία του Ισοδύναμου Σημείου} \\
Ογκομετρούμε 20 mL διαλύματος \ce{CH3COOH} 0,1 M με πρότυπο διάλυμα \ce{NaOH} 0,1 M.
\begin{enumerate}
  \item Γράψτε τη χημική εξίσωση της αντίδρασης: \dotfill
  \item Στο ισοδύναμο σημείο, ποιο χημικό είδος καθορίζει το pH; Γιατί το διάλυμα είναι βασικό;
\end{enumerate}
\textit{Απάντηση:} \dotfill

\vspace{0.5cm}

\vspace{0.5cm}

\textbf{Δραστηριότητα 2: Ανάλυση Καμπύλης} \\
Στο παρακάτω διάγραμμα καμπύλης ογκομέτρησης, σημειώστε:
\begin{enumerate}
  \item Το Ισοδύναμο Σημείο (Ι.Σ.).
  \item Την περιοχή όπου το διάλυμα δρα ως Ρυθμιστικό.
  \item Το Ημι-ισοδύναμο σημείο. Ποια σχέση ισχύει τότε μεταξύ $pH$ και $pK_a$;
\end{enumerate}

\vspace{4cm} % Χώρος για σχεδιασμό ή παρατήρηση

\textbf{Δραστηριότητα 3: Επιλογή Δείκτη} \\
Διαθέτετε τους εξής δείκτες:
\begin{itemize}
  \item Ηλιανθίνη (περιοχή αλλαγής χρώματος: 3,1 - 4,4)
  \item Φαινολοφθαλεΐνη (περιοχή αλλαγής χρώματος: 8,0 - 10,0)
\end{itemize}
Ποιος είναι ο κατάλληλος για την ογκομέτρηση του \ce{CH3COOH} και γιατί;
\\ \textit{Απάντηση:} \dotfill

\newpage

% ----------------------------------------------------------------------
% 3 & 4. ΑΥΤΟΑΞΙΟΛΟΓΗΣΗ & ΑΞΙΟΛΟΓΗΣΗ ΜΑΘΗΜΑΤΟΣ
% ----------------------------------------------------------------------
\begin{center}
  \Large \textbf{Exit Ticket: Ογκομέτρηση Ασθενούς - Ισχυρού}
\end{center}

\vspace{0.8cm}

\section*{Μέρος Α: Έλεγχος Κατανόησης}
\textit{Απαντήστε σύντομα.}

\vspace{0.4cm}

\textbf{1. Στην ογκομέτρηση ασθενούς οξέος με ισχυρή βάση, στο ισοδύναμο σημείο ισχύει:} \\
$\square$ $[H_3O^+] = [OH^-]$ \quad $\square$ $[H_3O^+] < [OH^-]$ \quad $\square$ $[H_3O^+] > [OH^-]$

\vspace{0.4cm}

\textbf{2. Τι ονομάζουμε «περιοχή άλματος» του pH στην καμπύλη ογκομέτρησης;} \\
\dotfill

\vspace{0.4cm}

\textbf{3. Αν το $pK_a$ ενός ασθενούς οξέος είναι 4,75, ποιο θα είναι το pH στο σημείο όπου έχει εξουδετερωθεί το μισό οξύ;} \\
\dotfill

\vspace{1cm}
\hrule
\vspace{1cm}

\section*{Μέρος Β: Ανατροφοδότηση Μαθήματος}
\textit{Η γνώμη σου μετράει για να βελτιώσουμε το επόμενο εργαστήριο.}

\vspace{0.5cm}

\begin{tabularx}{\textwidth}{X c c c c c}
  \textbf{Κριτήριο}                                       & \textbf{1} & \textbf{2} & \textbf{3} & \textbf{4} & \textbf{5} \\
  \hline
  Κατάλαβα τη διαφορά Ισοδύναμου - Ημι-ισοδύναμου σημείου & $\square$  & $\square$  & $\square$  & $\square$  & $\square$  \\
  Η χρήση της προσομοίωσης με βοήθησε να «δώ» τις αλλαγές & $\square$  & $\square$  & $\square$  & $\square$  & $\square$  \\
  Η ομαδική συνεργασία λειτούργησε αποτελεσματικά         & $\square$  & $\square$  & $\square$  & $\square$  & $\square$  \\
\end{tabularx}

\vspace{0.8cm}

\textbf{Ποιο σημείο του μαθήματος σου φάνηκε πιο δύσκολο;} \\
\dotfill

\end{document}