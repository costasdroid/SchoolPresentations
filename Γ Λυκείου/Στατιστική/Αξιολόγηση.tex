\documentclass[a4paper, 12pt]{article}

\usepackage{amsmath, amsthm, amssymb}

\usepackage{unicode-math}
\usepackage{xltxtra}
\usepackage{xgreek}
\setmainfont{Calibri}

\usepackage{fancyhdr}
\renewcommand{\footrulewidth}{0.4pt}% default is 0pt
\usepackage{lastpage}

\usepackage[a4paper, top=2.5cm, bottom=2.5cm, left=2.5cm, right=2.5cm, headheight=1cm, footskip=1.5cm]{geometry}

\usepackage{fancyhdr}
\usepackage{tabularx}

% Ρυθμίσεις σελίδας
\geometry{top=2cm, bottom=2cm, left=2.5cm, right=2.5cm}

% Ρυθμίσεις Επικεφαλίδων
\pagestyle{fancy}
\fancyhf{}
\rhead{Στατιστική Γ' Λυκείου}
\lhead{Συνδυασμοί $n$ ανά $\kappa$}
\cfoot{\thepage}

\begin{document}

% ----------------------------------------------------------------------
% 1. ΣΧΕΔΙΟ ΜΑΘΗΜΑΤΟΣ ΓΙΑ ΤΟΝ ΑΞΙΟΛΟΓΗΤΗ
% ----------------------------------------------------------------------
\begin{center}
  \LARGE \textbf{Σχέδιο Μαθήματος (Lesson Plan)} \\
  \large \textbf{Θέμα: Συνδυασμοί $n$ ανά $\kappa$ (Επιλογή χωρίς σειρά)} \\
  \vspace{0.2cm}
  \small \textit{Πλαίσιο Αξιολόγησης Α1 - Προσανατολισμός Ανθρωπιστικών Σπουδών}
\end{center}

\section*{Γενικά Στοιχεία}
\begin{itemize}
  \item \textbf{Κοινό:} Μαθητές Ανθρωπιστικών Σπουδών.
  \item \textbf{Διάρκεια:} 45 λεπτά.
  \item \textbf{Βασική Σύνδεση:} Μετάβαση από τις Διατάξεις (σημασία στη σειρά) στους Συνδυασμούς (σημασία στη σύνθεση της ομάδας).
  \item \textbf{Πλαίσιο Εφαρμογών:} Στατιστική έρευνα, επιλογή δειγμάτων και ανάλυση τυχερών παιχνιδιών.
\end{itemize}

\section*{Διδακτικοί Στόχοι}
\begin{enumerate}
  \item Διάκριση μεταξύ Διάταξης και Συνδυασμού με βάση το κριτήριο της σειράς.
  \item Κατανόηση του συμβόλου $\binom{n}{\kappa}$ και του τρόπου υπολογισμού του.
  \item Αντίληψη της πρακτικής αξίας των συνδυασμών στην επιλογή αντιπροσωπευτικών δειγμάτων για έρευνες πεδίου.
\end{enumerate}

\section*{Ροή Μαθήματος (Χρονοδιάγραμμα)}
\begin{itemize}
  \item \textbf{00'--08' (Εισαγωγή):} Βιωματικό παράδειγμα: Επιλογή μαθητών για ρόλους με ιεραρχία vs επιλογή για μια ενιαία ομάδα εργασίας.
  \item \textbf{08'--15' (Θεωρία):} Ορισμός του Συνδυασμού. Παρουσίαση του τύπου και επεξήγηση της διαίρεσης με το $\kappa!$ (απαλοιφή της σειράς).
  \item \textbf{15'--25' (Εφαρμογή):} Παράδειγμα έρευνας αγοράς/κοινής γνώμης. Επιλογή 3 ατόμων από 10 για συμμετοχή σε ομάδα συζήτησης (Focus Group).
  \item \textbf{25'--35' (Πραγματικά Δεδομένα):} Ανάλυση του Τζόκερ. Υπολογισμός πιθανοτήτων και συζήτηση για το μέγεθος του δειγματικού χώρου.
  \item \textbf{35'--40' (Ομαδική Εργασία):} Επίλυση προβλήματος δειγματοληψίας από δύο διαφορετικές πληθυσμιακές ομάδες.
  \item \textbf{40'--45' (Ανατροφοδότηση):} Συμπλήρωση των φύλλων αξιολόγησης.
\end{itemize}

\newpage

% ----------------------------------------------------------------------
% 2. ΦΥΛΛΟ ΕΡΓΑΣΙΑΣ ΜΑΘΗΤΩΝ (ΟΜΑΔΙΚΟ)
% ----------------------------------------------------------------------
\begin{center}
  \Large \textbf{ΦΥΛΛΟ ΕΡΓΑΣΙΑΣ: «Εφαρμογές της Στατιστικής»} \\
  \small Τμήμα: \dots\dots \quad Ομάδα: \dots\dots\dots\dots\dots\dots
\end{center}

\vspace{0.5cm}

\textbf{Δραστηριότητα 1: Το κριτήριο της σειράς} \\
Σε μια ομάδα 5 εθελοντών, θέλουμε να επιλέξουμε 2 άτομα για μια δράση.
\begin{enumerate}
  \item Αν ο ένας θα είναι «Υπεύθυνος» και ο άλλος «Βοηθός», πόσες διαφορετικές επιλογές έχουμε; (Διατάξεις $P_{5,2}$).
  \item Αν και οι δύο έχουν ακριβώς τις ίδιες αρμοδιότητες ως μέλη της ίδιας ομάδας, πόσες επιλογές έχουμε; (Συνδυασμοί $\binom{5}{2}$).
\end{enumerate}

\vspace{3.5cm}
\textit{Απάντηση:} \dots\dots\dots\dots\dots\dots\dots\dots\dots\dots\dots\dots\dots\dots\dots

\vspace{1cm}

\textbf{Δραστηριότητα 2: Σχεδιασμός Έρευνας} \\
Ένας ερευνητής θέλει να μελετήσει τις συνήθειες μιας κοινότητας που αποτελείται από 6 εργαζόμενους στον ιδιωτικό τομέα και 6 στον δημόσιο. Πρέπει να επιλέξει ένα δείγμα που να αποτελείται από 2 άτομα από κάθε τομέα.
\begin{enumerate}
  \item Με πόσους τρόπους μπορεί να επιλέξει τους εργαζόμενους από τον ιδιωτικό τομέα; $\binom{6}{2} = $ \dots\dots\dots
  \item Με πόσους τρόπους μπορεί να επιλέξει τους εργαζόμενους από τον δημόσιο τομέα; $\binom{6}{2} = $ \dots\dots\dots
  \item Πόσα διαφορετικά δείγματα (συνδυασμοί ατόμων) μπορούν να σχηματιστούν συνολικά;
\end{enumerate}

\vspace{3.5cm}
\textit{Λύση:} \dots\dots\dots\dots\dots\dots\dots\dots\dots\dots\dots\dots\dots\dots\dots

\vspace{1cm}

\textbf{Δραστηριότητα 3: Πιθανότητες και Τζόκερ} \\
Στην κλήρωση του Τζόκερ επιλέγουμε 5 αριθμούς από 45.

\begin{enumerate}
  \item Υπολογίστε το πλήθος των πιθανών πεντάδων: $\binom{45}{5} = \frac{45 \cdot 44 \cdot 43 \cdot 42 \cdot 41}{5 \cdot 4 \cdot 3 \cdot 2 \cdot 1}$.
  \item Στο Τζόκερ, εκτός από τους 5 αριθμούς, επιλέγεται και \textbf{ένας αριθμός Τζόκερ από τα 20}. Πόσες συνολικά διαφορετικές στήλες μπορούν να συμπληρωθούν;
  \item Γράψτε τον τύπο που δίνει το συνολικό πλήθος στηλών:
        \[
          \text{Σύνολο στηλών} = \binom{45}{5} \times 20
        \]
  \item Με βάση το αποτέλεσμα (περίπου 24 εκατομμύρια), πώς θα σχολιάζατε τη φράση «έχω ένα σύστημα που προβλέπει τους αριθμούς»;
\end{enumerate}

\vspace{2.5cm}
\textit{Σχόλιο:} \dots\dots\dots\dots\dots\dots\dots\dots\dots\dots\dots\dots\dots\dots\dots

\newpage

% ----------------------------------------------------------------------
% 3 & 4. ΑΥΤΟΑΞΙΟΛΟΓΗΣΗ & ΑΞΙΟΛΟΓΗΣΗ ΜΑΘΗΜΑΤΟΣ
% ----------------------------------------------------------------------
\begin{center}
  \Large \textbf{Σύνοψη Μαθήματος (Exit Ticket)}
\end{center}

\section*{Μέρος Α: Έλεγχος Κατανόησης}
\textit{Επιλέξτε τη σωστή απάντηση.}

\vspace{0.4cm}

\textbf{1. Πότε χρησιμοποιούμε τον τύπο των Συνδυασμών;} \\
$\square$ Όταν η σειρά εμφάνισης των στοιχείων μεταβάλλει το αποτέλεσμα. \\
$\square$ Όταν μας ενδιαφέρει μόνο ποια στοιχεία αποτελούν την ομάδα, ανεξάρτητα από τη σειρά τους.

\vspace{0.4cm}

\textbf{2. Πόσοι είναι οι συνδυασμοί 10 ατόμων ανά 1;} \\
α) 1 \hspace{2cm} β) 10 \hspace{2cm} γ) 0

\vspace{0.4cm}

\textbf{3. Ποια θεωρείτε τη σημαντικότερη εφαρμογή των συνδυασμών στην έρευνα (π.χ. δημοσκοπήσεις);} \\
\dots\dots\dots\dots\dots\dots\dots\dots\dots\dots\dots\dots\dots\dots\dots\dots\dots\dots\dots\dots\dots\dots\dots\dots\dots\dots

\vspace{1cm}
\hrule
\vspace{1cm}

\section*{Μέρος Β: Αξιολόγηση Διδακτικής Εμπειρίας}
\textit{Η γνώμη σας με βοηθά να βελτιώσω τον τρόπο διδασκαλίας μου.}

\vspace{0.5cm}

\begin{tabularx}{\textwidth}{X c c c c c}
  \textbf{Κριτήριο}                             & \textbf{1} & \textbf{2} & \textbf{3} & \textbf{4} & \textbf{5} \\
  \hline
  Σαφήνεια στη διαφορά Διάταξης - Συνδυασμού    & $\square$  & $\square$  & $\square$  & $\square$  & $\square$  \\
  Σύνδεση των μαθηματικών με την καθημερινότητα & $\square$  & $\square$  & $\square$  & $\square$  & $\square$  \\
  Οργάνωση της ομαδικής εργασίας                & $\square$  & $\square$  & $\square$  & $\square$  & $\square$  \\
  Χρήση παραδειγμάτων (Τζόκερ, δειγματοληψία)   & $\square$  & $\square$  & $\square$  & $\square$  & $\square$  \\
\end{tabularx}

\vspace{0.8cm}

\textbf{Τι σας φάνηκε πιο ενδιαφέρον σήμερα;} \\

\vspace{2.5cm}
\dots\dots\dots\dots\dots\dots\dots\dots\dots\dots\dots\dots\dots\dots\dots\dots\dots\dots\dots\dots\dots\dots\dots\dots\dots\dots

\vspace{0.5cm}

\textbf{Υπάρχει κάτι που θα θέλατε να εξηγηθεί με διαφορετικό τρόπο;} \\

\vspace{2.5cm}
\dots\dots\dots\dots\dots\dots\dots\dots\dots\dots\dots\dots\dots\dots\dots\dots\dots\dots\dots\dots\dots\dots\dots\dots\dots\dots

\end{document}