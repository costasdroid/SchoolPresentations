\documentclass[greek]{beamer}
%\usepackage{fontspec}
\usepackage{amsmath,amsthm}
\usepackage{unicode-math}
\usepackage{xltxtra}
\usepackage{graphicx}
\usetheme{CambridgeUS}
\usecolortheme{seagull}
\usepackage{hyperref}
\usepackage{ulem}
\usepackage{xgreek}
\usepackage{pgfpages}
\usepackage{tikz}
%\setbeameroption{show notes on second screen}
%\setbeameroption{show only notes}

\setsansfont{Noto Serif}

\usepackage{multicol}

% \newtheorem{definition}{Ορισμός}

\title{Συναρτήσεις}
\subtitle{Συνέχεια Συνάρτησης}
\author[Λόλας]{Κωνσταντίνος Λόλας}
\date{}

\begin{document}

\begin{frame}
 \titlepage
\end{frame}

\section{Θεωρία}
\begin{frame}{Όταν εμείς το υπολογίζαμε...}
 Μέχρι στιγμής πλησιάζαμε. Ήρθε ο καιρός να φτάσουμε!
\end{frame}

\begin{frame}{Συνέχεια 1}
 \begin{block}{Συνέχεια σε σημείο}
  Μία συνάρτηση είναι συνεχής στο $x_0$ αν $\lim\limits_{x \to x_0}{ f(x) }=f(x_0)$
 \end{block}
\end{frame}

\begin{frame}{Συνέχεια 2}
 \begin{block}{Συνέχεια σε διάστημα}
  Μία συνάρτηση είναι συνεχής στο $[α,β]$ όταν:
  \begin{itemize}
   \item $\lim\limits_{x \to x_0}{ f(x) }=f(x_0)$ για κάθε $x\in (α,β)$
   \item $\lim\limits_{x \to α^+}{ f(x) }=f(α)$
   \item $\lim\limits_{x \to β^-}{ f(x) }=f(β)$
  \end{itemize}
 \end{block}
\end{frame}

\begin{frame}{Συνέχεια 3}
 \begin{block}{Συνεχής συνάρτηση}
  Μία συνάρτηση είναι συνεχής όταν είναι συνεχής σε κάθε σημείο του πεδίου ορισμού της.
 \end{block}
\end{frame}

\begin{frame}{Ας γνωριστούμε}
 Γνωστές συνεχείς συναρτήσεις:
 \begin{itemize}
  \item Πολυωνυμικές \pause
  \item Εκθετικές \pause
  \item Λογαριθμικές \pause
  \item Τριγωνομετρικές
 \end{itemize}
\end{frame}

\begin{frame}{Και πράξεις αυτών}
 Αν $f$ και $g$ συνεχείς τότε συνεχής
 \begin{itemize}
  \item $f+g$ \pause
  \item $f-g$ \pause
  \item $f\cdot g$ \pause
  \item $\frac{f}{g}$ \pause
  \item $f\circ g$ \pause
  \item ΟΛΕΣ ΟΙ ΓΝΩΣΤΕΣ
 \end{itemize}
\end{frame}

\begin{frame}{Το μέλλον...}
 \begin{itemize}
  \item Αντί να υπολογίζουμε όρια, θα υπολογίζουμε τιμές \pause
  \item Αν δεν μπορούμε να υπολογίζουμε τιμές, θα υπολογίζουμε όρια \pause
  \item Αφού η συνάρτηση δεν "διακόπτεται" βγάζουμε ωραία θεωρήματα
 \end{itemize}
\end{frame}

\begin{frame}{Πιο άπειρο είναι μεγαλύτερο κάνει κουμάντο}
 \begin{itemize}
  \item Υπάρχει μεγαλύτερο? το βγάζω κοινό παράγοντα
  \item Είναι ίσα? κάνω πράξεις και τα διώχνω
 \end{itemize}
\end{frame}

\section{Ασκήσεις}
\begin{frame}{Εξάσκηση}
 Να εξετάσετε, αν καθεμιά από τις παρακάτω συναρτήσεις είναι συνεχής στο $x_0$:
 \begin{enumerate}
  \item $f(x)=\begin{cases}
          \frac{x^2-1}{x-1}, & x\ne 1 \\
          2,                 & x=1
         \end{cases}$, $x_0=1$ \pause
  \item $f(x)=\begin{cases}
          \frac{ημx}{x}, & x<0    \\
          2x+1,          & x\ge 0
         \end{cases}$, $x_0=0$
 \end{enumerate}
\end{frame}

\begin{frame}{Εξάσκηση}
 Να μελετήσετε τη συνάρτηση $f(x)=e^x+\ln (x+1)$ ως προς τη συνέχεια και να βρείτε το $\lim\limits_{x \to 0}{ f(x) }$.
\end{frame}

\begin{frame}{Εξάσκηση}
 Δίνεται η συνάρτηση $f(x)=\begin{cases}
   e^x+ημx,             & x<0  \\
   1,                   & x=0 \\
   συνx\cdot \ln (x+1), & x>0
  \end{cases}$
  \begin{enumerate}
    \item Να μελετήσετε τη συνάρτηση $f$ ως προς τη συνέχεια.
    \item Να αποδείξετε ότι η $f$ είναι συνεχής στο διάστημα $[-π,0]$.
  \end{enumerate}
\end{frame}

\begin{frame}{Εξάσκηση}
 Δίνεται η συνάρτηση $f(x)=\begin{cases}
   4αe^x+βσυνx,             & x<0  \\
   x+2,                   & 0\ge x\ge 1 \\
   \ln x+αx-β, & x>1
  \end{cases}$

  Να βρείτε τις τιμές των $α$ και $β$ για τις οποίες η $f$ είναι συνεχής.
\end{frame}

\begin{frame}{Εξάσκηση}
 Να βρείτε το όριο $\lim\limits_{x \to +\infty}{ \left( 2x-|x^3-x-1| \right)  }$
\end{frame}

\begin{frame}{Εξάσκηση}
 Να βρείτε τα όρια:
 \begin{itemize}
  \item $\lim\limits_{x \to +\infty}{ \sqrt{4x^2-2x+1} }$ \pause
  \item $\lim\limits_{x \to -\infty}{ \left( \sqrt{x^2+5} -x \right)  }$
 \end{itemize}
\end{frame}

\begin{frame}{Εξάσκηση}
 Να βρείτε τα όρια:
 \begin{itemize}
  \item $\lim\limits_{x \to +\infty}{ \left( \sqrt{4x^2+2x+1}-2x \right)  }$ \pause
  \item $\lim\limits_{x \to -\infty}{ \left( x+ \sqrt{x^2+1} \right)  }$
 \end{itemize}
\end{frame}

\begin{frame}{Εξάσκηση}
 Να βρείτε το όριο $\lim\limits_{x \to +\infty}{ \left( (a-1)x^3-2x+1 \right)  }$, για τις διάφορες τιμές του $a\in\mathbb{R}$
\end{frame}

\begin{frame}{Εξάσκηση}
 Να βρείτε τις τιμές του $μ\in\mathbb{R}$, για τις οποίες το $\lim\limits_{x \to +\infty}{ \frac{(μ-1)x^3+μx^2-2}{(μ-2)x^2+3x+1}  }$, είναι πραγματικός αριθμός
\end{frame}

\begin{frame}{Εξάσκηση}
 Για τις διάφορες πραγματικές τιμές του $μ$, να υπολογίσετε το $\lim\limits_{x \to -\infty}{ \left( \sqrt{4x^2+1}+μx \right)  }$
\end{frame}

\begin{frame}{Εξάσκηση}
 Δίνεται η συνάρτηση $f(x)=\frac{x^n+x-1}{x^2+1}$, $n\in\mathbb{N}^*$. Να βρείτε το $\lim\limits_{x \to +\infty}{ f(x) } $ για τις διάφορες τιμές του $n\in\mathbb{N}^*$.
\end{frame}

\begin{frame}{Εξάσκηση}
 Έστω $f:\mathbb{R}\to\mathbb{R}$ μια συνάρτηση, για την οποία ισχύει $\lim\limits_{x \to +\infty}{ \left( xf\left( \frac{x-1}{x} \right)  \right)  }=2$, να υπολογίσετε το $\lim\limits_{x \to 1}{ \frac{f(x)}{x-1} }$.
\end{frame}

\begin{frame}{Εξάσκηση}
 Έστω $f:\mathbb{R}\to\mathbb{R}$ μια συνάρτηση, για την οποία ισχύει $\lim\limits_{x \to 1}{ f(x)  }=-\infty$, να υπολογίσετε τα όρια
 \begin{enumerate}
  \item $\lim\limits_{x \to 1}{ \frac{2f^2(x)+f(x)-1}{f^3(x)-f(x)-2} }$ \pause
  \item $\lim\limits_{x \to 1}{ \left( \sqrt{f^2(x)+1}-f(x) \right)  }$
 \end{enumerate}
\end{frame}

\begin{frame}{Εξάσκηση}
 Έστω $f:(-\infty,0)\to\mathbb{R}$ μια συνάρτηση, για την οποία ισχύει $\lim\limits_{x \to -\infty}{ \frac{xf(x)-2x+3}{x+2}  }=1$
 \begin{enumerate}
  \item να βρείτε τα όρια:
        \begin{enumerate}
         \item $\lim\limits_{x \to -\infty}{ f(x) }$ \pause
         \item $\lim\limits_{x \to -\infty}{ \frac{3x^2f(x)-x^2+1}{xf(x)+3}  }$
        \end{enumerate}
  \item Αν επιπλέον ισχύει $f\left( (-\infty,0) \right)=(3,+\infty) $, να βρείτε το $\lim\limits_{x \to -\infty}{ \frac{x}{f(x)-3} }$
 \end{enumerate}
\end{frame}

\begin{frame}{Εξάσκηση}
 Έστω $f:(0,+\infty)\to\mathbb{R}$ μια συνάρτηση, για την οποία ισχύουν
 $$\lim\limits_{x \to +\infty}{ \frac{f(x)}{x} }=5 \text{ και } \lim\limits_{x \to +\infty}{ (f(x)-5x) }=2$$
 Να βρείτε το $λ\in\mathbb{R}$, ώστε
 $$\lim\limits_{x \to +\infty}{ \frac{3f(x)+λx-2}{xf(x)-5x^2+1} }=3$$
\end{frame}

\begin{frame}{Εξάσκηση}
 Να βρείτε τα όρια:
 \begin{enumerate}
  \item $\lim\limits_{x \to +\infty}{ ημ\frac{2x-1}{x^2+1} }$ \pause
  \item $\lim\limits_{x \to +\infty}{ \frac{x}{x^2+1}συνx  }$
 \end{enumerate}
\end{frame}

\begin{frame}{Εξάσκηση}
 Να αποδείξετε ότι:
 \begin{enumerate}
  \item $\lim\limits_{x \to -\infty}{ xημ\frac{1}{x} }=1$ \pause
  \item $\lim\limits_{x \to +\infty}{ \frac{ημx}{x}  }=0$  \pause
  \item $\lim\limits_{x \to +\infty}{ ημx\cdot ημ\frac{1}{x}  }=0$  \pause
  \item $\lim\limits_{x \to +\infty}{ \frac{x-ημx}{x-1}  }=1$
 \end{enumerate}
\end{frame}

\begin{frame}{Εξάσκηση}
 Να βρείτε τα όρια:
 \begin{enumerate}
  \item $\lim\limits_{x \to +\infty}{ (x+ημx) }$ \pause
  \item $\lim\limits_{x \to +\infty}{ \frac{x}{2-ημx}  }$
 \end{enumerate}
\end{frame}

\begin{frame}{Εξάσκηση}
 Να βρείτε τα όρια:
 \begin{enumerate}
  \item $\lim\limits_{x \to +\infty}{ \frac{e^x-2^x+1}{3^x-5^x-2} }$ \pause
  \item $\lim\limits_{x \to -\infty}{ \frac{3^x-5^x}{3^x-2^x}  }$
 \end{enumerate}
\end{frame}

\begin{frame}{Εξάσκηση}
 Να βρείτε το $\lim\limits_{x \to +\infty}{ \frac{2^x-a^x}{2^x+3a^x}  }$, $a>0$
\end{frame}

\begin{frame}{Εξάσκηση}
 Να βρείτε τα όρια:
 \begin{enumerate}
  \item $\lim\limits_{x \to +\infty}{ e^{-x^2+1} }$ \pause
  \item $\lim\limits_{x \to 0^-}{ e^{-\frac{1}{x}} }$\pause
  \item $\lim\limits_{x \to 0}{ \frac{1}{e^{x^2}-1}}$
 \end{enumerate}
\end{frame}

\begin{frame}{Εξάσκηση}
 Να βρείτε τα όρια:
 \begin{enumerate}
  \item $\lim\limits_{x \to 0}{ \frac{1}{x}-\ln x }$ \pause
  \item $\lim\limits_{x \to 0}{ \frac{x}{\ln x} }$\pause
  \item $\lim\limits_{x \to 1}{ \frac{1+\sqrt{x-1}}{\ln x}}$\pause
  \item $\lim\limits_{x \to 0}{ \frac{\ln x}{ημx}}$
 \end{enumerate}
\end{frame}

\begin{frame}{Εξάσκηση}
 Να βρείτε τα όρια:
 \begin{enumerate}
  \item $\lim\limits_{x \to +\infty}{\left(    \ln x + e^{-\frac{1}{x}} \right)}$ \pause
  \item $\lim\limits_{x \to 1}{ \ln\frac{x}{x-1} }$\pause
  \item $\lim\limits_{x \to +\infty}{\left( \ln (1+e^{2x})-x \right)}$
 \end{enumerate}
\end{frame}

\begin{frame}{Εξάσκηση}
 Να βρείτε τα όρια:
 \begin{enumerate}
  \item $\lim\limits_{x \to +\infty}{\left( \ln x+συνx \right)}$ \pause
  \item $\lim\limits_{x \to +\infty}{ \frac{συνx}{\ln x} }$
 \end{enumerate}
\end{frame}

\begin{frame}{Εξάσκηση}
 Δίνεται η συνάρτηση $f(x)=\ln x + \sqrt{x-1}$ με σύνολο τιμών το $[0,+\infty)$
 \begin{enumerate}
  \item Να δείξετε ότι η $f$ αντιστρέφεται \pause
  \item Να βρείτε το $\lim\limits_{x \to +\infty}{ \left( x^2f^{-1}(x) \right)  }$
 \end{enumerate}
\end{frame}

\begin{frame}{Εξάσκηση}
 Δίνεται η συνάρτηση $f(x)=x^x$, $x>0$. Να βρείτε το $\lim\limits_{x \to +\infty}{ f(x)  }$
\end{frame}

\section{}
\begin{frame}
 Στο moodle θα βρείτε τις ασκήσεις που πρέπει να κάνετε, όπως και αυτή τη παρουσίαση
\end{frame}

\end{document}
