\documentclass{presentation}

\title{Συναρτήσεις} 
\subtitle{Έννοια Πραγματικής Συνάρτησης}
\author[Λόλας]{Κωνσταντίνος Λόλας }
\institute[$10^ο$ ΓΕΛ]{$10^ο$ ΓΕΛ Θεσσαλονίκης}
\date{}

\begin{document}

\begin{frame}
      \titlepage
\end{frame}
\begin{frame}{Ορισμοί}
      \begin{block}{Ορισμός Συνάρτησης}
            Έστω $Α$ ένα υποσύνολο του $\mathbb{R}$. Ονομάζουμε \emph{πραγματική συνάρτηση} με πεδίο ορισμού το $Α$ μια διαδικασία (κανόνα) $f$, με την οποία κάθε στοιχείο $x\in A$ αντιστοιχίζεται σε ένα μόνο πραγματικό αριθμό $y$. Το $y$ ονομάζεται τιμή της $f$ στο $x$ και συμβολίζεται με $f(x)$.
      \end{block}
\end{frame}

\begin{frame}{Τρόπος ορισμού}
      \begin{itemize}
            \item<1-> Απλός τύπος
                  $$f(x)=x^2+2,x\in\mathbb{R}\quad g(a)=\frac{2}{ημa}, a\le1$$
            \item<2-> Ορισμένη κατά "κλάδους"
                  $$
                        f(x)=
                        \begin{cases}
                              x^2+2,         & x<2 \\
                              \frac{2}{ημx}, & x>5 \\
                              -\sqrt{2},     & x=3
                        \end{cases}
                  $$
            \item<3-> Περιγραφικά
      \end{itemize}
\end{frame}

\begin{frame}{Συμβολισμοί}
      \begin{itemize}
            \item $f$, $g$, $h$
            \item $f:A\to\mathbb{R}$
            \item $x$, $y$, $t\ldots$
            \item $D_f$, $A_f$
            \item $x\to f(x)$
            \item $f(A)=\{y| y=f(x)\text{ για κάποιο }x\in A\}$
      \end{itemize}
\end{frame}

\begin{frame}{Πεδίο Ορισμού}
      \begin{itemize}
            \item<1-> Διαίρεση
                  $$\frac{a}{b},b\ne 0$$
            \item<2-> Ρίζες
                  $$\sqrt[n]{a},a\ge 0$$
            \item<3-> Λογάριθμοι
                  $$\ln a,a > 0$$
            \item<4-> Κρυφά
                  $$εφx,\quad x^x\ldots$$
      \end{itemize}
\end{frame}

\begin{frame}{Προαπαιτούμενα}
      Πρέπει να γνωρίζετε πολύ καλά
      \begin{itemize}
            \item Εξισώσεις
            \item Ανισώσεις
      \end{itemize}
\end{frame}

\begin{frame}[noframenumbering]
      Στο moodle θα βρείτε τις ασκήσεις που πρέπει να κάνετε, όπως και αυτή τη παρουσίαση
\end{frame}

\section{Ασκήσεις}

\begin{frame}[noframenumbering]
      \vfill
      \centering
      \begin{beamercolorbox}[sep=8pt,center,shadow=true,rounded=true]{title}
            \usebeamerfont{title}Ασκήσεις
      \end{beamercolorbox}
      \vfill
\end{frame}

\begin{askisi}
      Δίνεται η συνάρτηση $f(x)=x^3-x+a$ με $f(-1)=1$
      \begin{enumerate}
            \item<1-> Να βρείτε την τιμή του $a$.
            \item<2-> Να λύσετε την εξίσωση $f(x)=1$.
      \end{enumerate}
      %\hyperlink{Λύση}{\beamerbutton{Λύση}}
\end{askisi}

\begin{askisi}
      Για τη συνάρτηση $f(x)=\begin{cases}
                  2x+a^2,  & x<3   \\
                  x-3+b^2, & x\ge3
            \end{cases}$, ισχύει $f(0)+f(3)=0$.
      \begin{enumerate}
            \item<1-> Να βρείτε το πεδίο ορισμού της.
            \item<2-> Να υπολογίσετε τα $a$ και $b$.
            \item<3-> Να βρείτε τις τιμές $f(\pi)$ και $f(e)$.
      \end{enumerate}
      %\hyperlink{Λύση2}{\beamerbutton{Λύση}}
\end{askisi}

\begin{askisi}
      Να βρείτε το πεδίο ορισμού των συναρτήσεων
      \begin{enumerate}
            \item<1-> $\frac{1}{x-1}$
            \item<2-> $\frac{2x}{x^2-3x+2}$
            \item<3-> $\sqrt{x-1}$
            \item<4-> $\ln (x-1)$
            \item<5-> $\sqrt{x-1}-\sqrt[3]{2-x}$
            \item<6-> $\frac{\sqrt{1-x^2}}{x}$
            \item<7-> $\frac{\ln x}{x-1}$
            \item<8-> $\frac{x-1}{\ln x}$
            \item<9-> $\sqrt{\ln x}$
            \item<10-> $\frac{1}{\sqrt{x}-1}$
            \item<11-> $\ln \left(\sqrt{x-1}-x+3\right)$
      \end{enumerate}
      %\hyperlink{Λύση3}{\beamerbutton{Λύση}}
\end{askisi}

\begin{askisi}
      Δίνεται η συνάρτηση $f(x)=\begin{cases}
                  1-x^2,     & x\le 1 \\
                  \ln (x-1), & x>1
            \end{cases}$.
      \begin{enumerate}
            \item<1-> Να βρείτε τις τιμές
                  \begin{enumerate}
                        \item<1-> $f(ημ a)$, $a\in\mathbb{R}$
                        \item<2-> $f(x^2+1)$, $x\ne 0$
                  \end{enumerate}
            \item<2-> Να λύσετε την εξίσωση $f(x)=0$
      \end{enumerate}
      %\hyperlink{Λύση4}{\beamerbutton{Λύση}}
\end{askisi}

\end{document}
