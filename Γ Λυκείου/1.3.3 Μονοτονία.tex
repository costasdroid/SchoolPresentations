\documentclass{presentation}

\title{Συναρτήσεις}
\subtitle{Ένα προς ένα (1-1)}
\author[Λόλας]{Κωνσταντίνος Λόλας }
\institute[$10^ο$ ΓΕΛ]{$10^ο$ ΓΕΛ Θεσσαλονίκης}
\date{}

\begin{document}

\begin{frame}
  \titlepage
\end{frame}

\section{Θεωρία}
\begin{frame}{1-1}
  \begin{block}{Ορισμός}
    Μία συνάρτηση $f:Α\to\mathbb{R}$ λέγεται συνάρτηση \emph{συνάστηση 1-1} (ένα προς ένα), όταν για οποιαδήποτε $x_1$, $x_2\in Α$ ισχύει η συνεπαγωγή
    $$x_1\ne x_2 \iff f(x_1)\ne f(x_2)$$
  \end{block}
\end{frame}

\begin{frame}{Αντέχετε?}
  \begin{exampleblock}{Ενδιαφέροντα}
    \begin{itemize}
      \item $x_1\ne x_2 \iff f(x_1)\ne f(x_2)$
      \item $x_1= x_2 \iff f(x_1)= f(x_2)$
      \item $f(x_1)\ne f(x_2) \iff x_1\ne x_2$
      \item $f(x_1)= f(x_2) \iff x_1= x_2$
    \end{itemize}
  \end{exampleblock}
\end{frame}

\begin{frame}{Φαντασία θέλει}
  Μια συνάρτηση:
  \begin{itemize}
    \item Κάθε $y$ το πολύ μία φορά \pause
    \item Κάθε $y$ του συνόλου τιμών ΑΚΡΙΒΩΣ μία φορά \pause
    \item Κάθε οριζόντια γραμμή...
  \end{itemize}
  Άρα... φαίνονται οι διαφορετικοί!
\end{frame}

\begin{frame}{ΤΟΟΟ μπέρδεμα}
  \begin{alertblock}{Προσοχή}
    \begin{itemize}
      \item Γνησίως μονότονη $\implies$ \pause είναι 1-1 \pause
      \item  1-1 $\implies$ \pause ΜΠΟΡΕΙ!
    \end{itemize}
  \end{alertblock}
  Βρείτε την!
\end{frame}

\begin{frame}{Ανακεφαλαίωση}
  Θα δείχνουμε ότι η συνάρτηση είναι 1-1, αλλά πώς? \pause
  Κυρίως
  \begin{itemize}
    \item Κατασκευή \pause
    \item Μονοτονία σε διάστημα
  \end{itemize} \pause
  \begin{alertblock}{Γιατί να το κάνουμε?}
    Λύνουμε σύνθετες εξισώσεις διώχνοντας $f$
  \end{alertblock}
\end{frame}

\section{Ασκήσεις}
\begin{frame}{Εξάσκηση}
  Να βρείτε, ποιες από τις  παρακάτω συναρτήσεις είναι 1-1.
  \begin{enumerate}
    \item $f(x)=\frac{x-1}{x-2}$\pause
    \item $f(x)=2x+e^x-1$\pause
    \item $f(x)=x^2-1$
  \end{enumerate}
\end{frame}

\begin{frame}{Εξάσκηση}
  Δίνεται η συνάρτηση $f(x)=e^{x-1}+x^3-2$.
  \begin{enumerate}
    \item Να δείξετε ότι η $f$ είναι συνάρτηση 1-1 \pause
    \item Να λύσετε τις εξισώσεις:
          \begin{enumerate}
            \item $f(x)=0$ \pause
            \item $f(\ln x)=0$ \pause
            \item $f(x^2-2x)=f(x-2)$ \pause
            \item $f\left(f(x)+1\right)=0$ \pause
          \end{enumerate}
    \item Να λύσετε το σύστημα $\begin{cases}
              α^3-β=2 \\
              e^{α-1}+β=0
            \end{cases}$
  \end{enumerate}
\end{frame}

\begin{frame}{Εξάσκηση}
  Δίνεται η συνάρτηση $f(x)=\frac{e^x}{e^x-1}$.
  \begin{enumerate}
    \item Να δείξετε ότι η $f$ είναι συνάρτηση 1-1 \pause
    \item Να λύσετε την εξίσωση $(1-e^{-x})f(x^2+2x)=1$
  \end{enumerate}
\end{frame}

\begin{frame}{Εξάσκηση}
  Στο διπλανό σχήμα φαίνεται η γραφική παράσταση μιας συνάρτησης $f$ που είναι ορισμένη στο $\mathbb{R}$. Να λύσετε τις εξισώσεις:
  \begin{enumerate}
    \item $f(x^4+1)=f(x^2+1)$ \pause
    \item $f(ημ x)=f(συν x)$ \pause
    \item $f\left( f(x) \right)=1$
  \end{enumerate}
\end{frame}

\begin{frame}{Εξάσκηση}
  Έστω συνάρτηση $f:\mathbb{R}\to \mathbb{R}$ η οποία είναι γνησίως αύξουσα. Να λύσετε:
  \begin{enumerate}
    \item Την ανίσωση $f(x)-x>f(2x)$ \pause
    \item Την εξίσωση $f(x)-\ln x=f(x^2)$
  \end{enumerate}
\end{frame}

\begin{frame}{Εξάσκηση}
  Δίνεται η συνάρτηση $f(x)=e^x+x-1$. Να λύσετε το σύστημα
  $$\begin{cases}
      y=f(x) \\
      x=f(y)
    \end{cases}$$
\end{frame}

\begin{frame}{Εξάσκηση}
  Έστω $f,g:\mathbb{R}\to\mathbb{R}$ δύο συναρτήσεις, όπου η συνάρτηση $g\circ f$ είναι 1-1. Να δείξετε ότι η $f$ είναι 1-1.
\end{frame}

\begin{frame}{Εξάσκηση}
  Έστω $f:\mathbb{R}\to\mathbb{R}$ μία συνάρτηση, για την οποία ισχύει
  $$f\left(f(x)\right)+f^3(x)-x=0\text{, για κάθε }x\in\mathbb{R}$$
  \begin{enumerate}
    \item Να δείξετε ότι η $f$ είναι συνάρτηση 1-1 \pause
    \item Να λύσετε την εξίσωση $f\left(f(x)+x^2-x\right)=f\left(f(x)+2x-2\right)$ \pause
    \item Να λύσετε την εξίσωση $f\left(f(2x+1)\right)-f\left(f(x)\right)=x+1$
  \end{enumerate}
\end{frame}

\begin{frame}
  Στο moodle θα βρείτε τις ασκήσεις που πρέπει να κάνετε, όπως και αυτή τη παρουσίαση
\end{frame}

\end{document}
