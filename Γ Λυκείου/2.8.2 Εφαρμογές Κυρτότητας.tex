\documentclass{../presentation}

\title{Συναρτήσεις}
\subtitle{Εφαρμογές Κυρτότητας}
\author[Λόλας]{Κωνσταντίνος Λόλας}
\institute[$10^ο$ ΓΕΛ]{$10^ο$ ΓΕΛ Θεσσαλονίκης}
\date{}

\begin{document}

\begin{frame}
  \titlepage
\end{frame}

\section{Ασκήσεις}

\begin{askisi}
  Δίνεται η συνάρτηση $f(x)=\ln x-\dfrac{1}{x}$
  \begin{enumerate}[<+->]
    \item Να δείξετε ότι η $f$ είναι κοίλη.
    \item Να λύσετε την ανίσωση $f'(x^2+1)<2$
    \item Να αποδείξετε ότι για κάθε $x_0>0$, η $C_f$ και η εφαπτομένη της στο $Μ(x_0,f(x_0))$ έχουν ένα μόνο κοινό σημείο
    \item Να βρείτε την εφαπτόμενη στη γραφική παράσταση της $f$ στο $x_0=1$
    \item
          \begin{enumerate}[<+->]
            \item Να δείξετε ότι $f(x)-2x\le -3$, για κάθε $x>0$
            \item Να λύσετε την εξίσωση $\dfrac{3+f(x)}{x}=2$
          \end{enumerate}
    \item
          \begin{enumerate}[<+->]
            \item Να δείξετε ότι $f(x^2+1)+1 \le 2x^2$, για κάθε $x\in\mathbb{R}$
            \item Να λύσετε την εξίσωση $f(e^x)-2e^x=-3$
          \end{enumerate}
  \end{enumerate}
\end{askisi}

\begin{askisi}
  Δίνεται η συνάρτηση $f(x)=e^x+x^2$.
  \begin{enumerate}[<+->]
    \item Να δείξετε ότι η $f$ είναι κυρτή.
    \item Να λύσετε τις εξισώσεις:
          \begin{enumerate}[<+->]
            \item $f'(f(x)-x)=2+e$
            \item $e^{ημx}=ημx+συν^2x$
          \end{enumerate}
    \item Να λύσετε τις ανισώσεις:
          \begin{enumerate}[<+->]
            \item $e^x(x^2+x-1)>-1$
            \item $\dfrac{e^{x-1}-3x}{x^2+1}+1>0$
          \end{enumerate}
  \end{enumerate}
\end{askisi}

\begin{askisi}
  Έστω $f:\mathbb{R}\to\mathbb{R}$ μια συνάρτηση με $f(\mathbb{R})=\mathbb{R}$, η οποία είναι παραγωίσιμη, γνησίως αύξουσα και κυρτή. Επιπλέον είναι $f(0)=2$ και $f'(0)=1$.
  \begin{enumerate}[<+->]
    \item Να λύσετε την εξίσωση $f(x)=2+x-x^2$
    \item Να υπολογίσετε το $\lim_{x\to 0}\dfrac{1}{f(x)-(x+2)}$
    \item Να λύσετε την εξίσωση $f\left(f(x-2)-x\right)=2$
    \item Να δείξετε ότι η $f$ αντιστρέφεται και $f^{-1}(x)\le x-2$, για κάθε $x\in \mathbb{R}$
  \end{enumerate}
\end{askisi}

\begin{askisi}
  Δίνεται η συνάρτηση $f(x)=3x^5-5x^4$.
  \begin{enumerate}[<+->]
    \item Να δείξετε ότι η $C_f$ έχει ακριβώς ένα σημείο καμπής.
    \item Να λύσετε την ανίσωση $f'(x^2+2)>f'(2x^2+1)$
    \item Να δείξετε ότι $f(x)+5x\le 3$ για κάθε $x\le 1$
    \item Να λύσετε:
          \begin{enumerate}[<+->]
            \item την εξίσωση $f(x)=3-5x$
            \item την ανίσωση $3-f(x)< 5x$
          \end{enumerate}
    \item Να δείξετε ότι η $f(e^x-x)+5e^x \ge 5x+3$
    \item Για κάθε $x\ge 1$, να δείξετε ότι $xf \left( \dfrac{1}{x} \right) \le 3x-5$
  \end{enumerate}
\end{askisi}

\begin{askisi}
  Δίνεται η συνάρτηση $f(x)=εφx$, $x\in A=\left( -\dfrac{π}{2},\dfrac{π}{2} \right)$
  \begin{enumerate}[<+->]
    \item Να μελετήσετε τη συνάρτηση $f$ ως προς την κυρτότητα και τα σημεία καμπής
    \item Να δείξετε ότι $ημx<x<εφx$, για κάθε $x\in \left( 0 , \dfrac{π}{2} \right)$
    \item Να βρείτε το $\lim\limits_{x\to 0}\dfrac{\ln x}{εφx-x}$
  \end{enumerate}
\end{askisi}

\begin{askisi}
  Έστω $f:\mathbb{R}\to\mathbb{R}$ μια συνάρτηση δύο φορές παραγωγίσιμη με $f''(1)=1$, συνεχή δεύτερη παράγωγο και ισχύει $f''(x)\ne 0$, για κάθε $x\in\mathbb{R}$.
  \begin{enumerate}[<+->]
    \item Να αποδείξετε ότι η $f$ δεν έχει σημεία καμπής και είναι κυρτή

          Αν $f(1)=1$ και $f'(1)=1$, τότε:

    \item Να υπολογίσετε τα όρια
          \begin{enumerate}[<+->]
            \item $\lim\limits_{x\to 1}\dfrac{\ln(x-1)}{f'(x)-f'(x^2)}$
            \item $\lim\limits_{x\to +\infty}f(x)$
          \end{enumerate}

    \item Να λύσετε την εξίσωση $f(x)+f'(x-1)=f'(\ln x)+x$
  \end{enumerate}
\end{askisi}

\begin{askisi}
  Έστω $f:\mathbb{R}\to\mathbb{R}$ μια συνάρτηση με $f(0)=1$, η οποία είναι παραγωγίσιμη, κυρτή και ισχύει $f(x)\ge 1$, για κάθε $x\in\mathbb{R}$.
  \begin{enumerate}[<+->]
    \item Να μελετήσετε την $f$ ως προς την μονοτονία

          Αν επιπλέον $f(1)=f'(1)=2$, τότε:

    \item Να λύσετε την εξίσωση $f(x)+f(x+1)=2x+3$
    \item Να βρείτε το $\lim\limits_{x\to 1}\dfrac{1}{f\left( f(x)\right)-f(2x)}$
  \end{enumerate}
\end{askisi}

\begin{askisi}
  Έστω $f:\mathbb{R}\to\mathbb{R}$ μια συνάρτηση η οποία είναι κυρτή. Να δείξετε ότι

  $$f(e^x)-f(x)>(e^x-x)f'(x) \text{, για κάθε } x\in\mathbb{R}$$
\end{askisi}

\begin{askisi}
  Έστω $f:[0,1]\to\mathbb{R}$ μια παραγωγίσιμη συνάρτηση με $f(0)=f(1)=0$ η οποία είναι κυρτή. Να δείξετε ότι:
  \begin{enumerate}[<+->]
    \item Υπάρχει μοναδικό $ξ\in(0,1)$ τέτοιο ώστε $f'(ξ)=0$
    \item $f(x)<0$, για κάθε $x\in(0,1)$
  \end{enumerate}
\end{askisi}

\begin{askisi}
  Έστω $f:\mathbb{R}\to\mathbb{R}$ μια συνάρτηση με $f(0)=0$ η οποία είναι παραγωγίσιμη με $f'(0)=1$ και κυρτή. Αν $α>1$, να δείξετε ότι η εξίσωση
  $$\frac{f'(α)-1}{x}+\frac{f(2α)-α}{x-1}+\frac{f(α^2)-α}{x-2}=0$$
  έχει ακριβώς δύο ρίζες στο διάστημα $(0,2)$.
\end{askisi}

\begin{askisi}
  Έστω $f:(0,+\infty)\to\mathbb{R}$ μια συνάρτηση η οποία είναι παραγωγίσιμη, γνησίως αύξουσα και κοίλη
  \begin{enumerate}[<+->]
    \item Αν $g(x)=\dfrac{f(x+1)-f(x)}{x}$, $x>0$, να δείξετε ότι $g\downarrow (0,+\infty)$
    \item Αν $f(2)=1$ και $f'(1)=2$, να δείξετε ότι
          $$-1<f(1)<1+συν1-συν2$$
  \end{enumerate}
\end{askisi}

\begin{askisi}
  Έστω $f:[0,+\infty]\to\mathbb{R}$ μια συνάρτηση με $f(0)=0$ η οποία είναι κυρτή. Να δείξετε ότι:
  \begin{enumerate}[<+->]
    \item $f(x+1)-f(x)>f'(x)$, για κάθε $x>0$
    \item Η συνάρτηση $g(x)=(x+1)f(x)-xf(x+1)-x+2$, $x\ge 0$ είναι γνησίως φθίνουσα
    \item $f(x)<\dfrac{2}{3}f\left(\dfrac{3x}{2}\right)$, για κάθε $x>0$
    \item Υπάρχει μοναδικό $α\in (0,2)$ τέτοιο ώστε
          $$\frac{f(α)}{α}-\frac{f(α+1)}{α+1}=\frac{α-2}{α^2+α}$$
  \end{enumerate}
\end{askisi}

\end{document}
