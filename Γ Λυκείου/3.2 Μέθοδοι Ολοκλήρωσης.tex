\documentclass[greek]{beamer}
%\usepackage{fontspec}
\usepackage{amsmath,amsthm}
\usepackage{unicode-math}
\usepackage{xltxtra}
\usepackage{graphicx}
\usetheme{CambridgeUS}
\usecolortheme{seagull}
\usepackage{hyperref}
\usepackage{ulem}
\usepackage{xgreek}

\usepackage{pgfpages}
\usepackage{tikz}
\usepackage{tkz-tab}
%\setbeameroption{show notes on second screen}
%\setbeameroption{show only notes}

\setsansfont{Noto Serif}

\usepackage{multicol}

\usepackage{appendixnumberbeamer}

\setbeamercovered{transparent}
\beamertemplatenavigationsymbolsempty

\title{Συναρτήσεις}
\subtitle{Μέθοδοι Ολοκλήρωσης}
\author[Λόλας]{Κωνσταντίνος Λόλας}
\date{}

\begin{document}

\begin{frame}
 \titlepage
\end{frame}

\section{Θεωρία}
\begin{frame}{Σιγά τα ολοκληρώματα!}
 Τι μπορούμε να ολοκληρώσουμε
 \begin{enumerate}
  \item Πολυώνυμα
  \item Εκθετικές
  \item Τριγωνομετρικές
  \item Ρητές με πρωτοβάθμιο διαιρέτη
  \item Πρωτοβάθμιες άρρητες
  \item Έτοιμες από σύνθεση και φυσικά
  \item κάθε πρόσθεση ή αφαίρεση αυτών ΜΟΝΟ
 \end{enumerate}
 Τι γίνεται με τον πολλαπλασιασμό? Διαίρεση? Ακόμα και την απλή $\ln x$?
\end{frame}

\begin{frame}{Ιστορία}
 Ξέρουμε να παραγωγίζουμε γινόμενο
 \begin{align*}
  (f\cdot g)' & =f'g+fg'         \\
  f'g         & =(f\cdot g)'-fg'
 \end{align*}
 Άρα
 $$\int f'g\, dx=\int (f\cdot g)'\,dx - \int fg'\, dx$$
 $$\int f'g\, dx=f\cdot g - \int fg'\, dx$$
\end{frame}

\begin{frame}{Έ και?}
 $$\int f'g\, dx=f\cdot g - \int fg'\, dx$$
 Βρείτε λόγους για να περάσουμε την παράγωγο από την μία συνάρτηση στην άλλη
 \begin{itemize}
  \item<2-> Γιατί τελικά... εξαφανίζεται
  \item<3-> Γιατί δεν ξέρουμε να την ολοκληρώνουμε
  \item<4-> Γιατί μπορούμε να ξαναφτάσουμε στον ίδιο τύπο!!!!!!
 \end{itemize}
\end{frame}

\begin{frame}{Παραδείγματα}
 \begin{enumerate}
  \item<1-> $\int xe^x\,dx$
  \item<2-> $\int x^3e^x\,dx$
  \item<3-> $\int x\ln x\,dx$
  \item<4-> $\int e^{2x}ημ(3x+1)\,dx$
 \end{enumerate}
\end{frame}

\begin{frame}{Και στα εντός ύλης!}
 \begin{block}{Κατά παράγοντες}
  $$\int_a^b f'(x)g(x)\,dx=\left[ f(x)g(x) \right]_a^b -\int_a^b f(x)g'(x)\,dx$$
 \end{block}
\end{frame}

\begin{frame}{Hold your horses}

 Δεν θα μάθουμε \emph{ΠΟΤΕ} να ολοκληρώνουμε όλες τις συναρτήσεις! Μαθηματικό...!

 Μεθόδους για "όμορφες"

 \begin{itemize}
  \item<2-> ρητές
  \item<3-> άρρητες
  \item<4-> τριγωνομετρικές
  \item<5-> από σύνθεση?????
 \end{itemize}

\end{frame}

\begin{frame}{Δοκιμές σύνθεσης}
 \begin{enumerate}
  \item<1-> $\int \dfrac{x}{x^2+1}\,dx$
  \item<2-> $\int \dfrac{\ln x}{x}\, dx$
  \item<3-> $\int 4xεφ(x^2)\ln (ημ(x^2))\, dx$
 \end{enumerate}

\end{frame}

\begin{frame}{Ναι, αλλά... τύπο έχουμε?}
 \begin{block}{Μέθοδος Αντικατάστασης}
   $$\int_a^b f(g(x))\, dx$$
   Θέτω $u=g(x)$, άρα
   \begin{itemize}
     \item για $x=a\implies u=g(a)$
     \item για $x=b\implies u=g(b)$
     \item $du=g'(x)dx$
   \end{itemize}
   $$\int_a^b f(x)\, dx=\int_k^l f(g(u))g'(u)\,du$$
 \end{block}

\end{frame}

\section{Ασκήσεις}
\subsection{Άσκηση 1}
\begin{frame}[label=Άσκηση1,t]{Εξάσκηση 1}
 Να μελετήσετε τις παρακάτω συναρτήσεις ως προς την κυρτότητα
 \begin{enumerate}
  \item<1-> $f(x)=x^2-\ln x$
  \item<2-> $f(x)=\sqrt{x}-e^x$
  \item<3-> $f(x)=x^4-2x+1$
  \item<4-> $f(x)=x\ln x-e^{-x}$
 \end{enumerate}

 % \hyperlink{Λύση1}{\beamerbutton{Λύση}}
\end{frame}

\subsection{Άσκηση 2}
\begin{frame}[label=Άσκηση2,t]{Εξάσκηση 2}
 Δίνεται η συνάρτηση $f(x)=e^x-x$.
 \begin{enumerate}
  \item<1-> Να μελετήσετε τη συνάρτηση $f$ ως προς τη μονοτονία, τα ακρότατα και την κυρτότητα
  \item<2-> Να βρείτε τις οριακές τιμές της $f$ στα άκρα του πεδίου ορισμού της, να κάνετε τον πίνακα μεταβολών της $f$ και να σχεδιάσετε τη $C_f$
  \item<3-> Να λύσετε την εξίσωση $f(x)=συνx$
 \end{enumerate}

 % \hyperlink{Λύση2}{\beamerbutton{Λύση}}
\end{frame}

\subsection{Άσκηση 3}
\begin{frame}[label=Άσκηση3,t]{Εξάσκηση 3}
 Να βρείτε τα διαστήματα στα οποία οι παρακάτω συναρτήσεις είναι κυρτές ή κοίλες και να προσδιορίσετε (αν υπάρχουν) τα σημεία καμπής
 \begin{enumerate}
  \item<1-> $f(x)=x^3-3x^2+5$
  \item<2-> $f(x)=3x^5-5x^4$
 \end{enumerate}

 % \hyperlink{Λύση3}{\beamerbutton{Λύση}}
\end{frame}

\subsection{Άσκηση 4}
\begin{frame}[label=Άσκηση4,t]{Εξάσκηση 4}
 Να βρείτε τα διαστήματα στα οποία οι παρακάτω συναρτήσεις είναι κυρτές ή κοίλες και να προσδιορίσετε (αν υπάρχουν) τα σημεία καμπής
 \begin{enumerate}
  \item<1-> $f(x)=\dfrac{1}{x^2+1}$
  \item<2-> $f(x)=x+\dfrac{1}{x}$
 \end{enumerate}

 % \hyperlink{Λύση4}{\beamerbutton{Λύση}}
\end{frame}

\subsection{Άσκηση 5}
\begin{frame}[label=Άσκηση5,t]{Εξάσκηση 5}
 Να βρείτε τις θέσεις των σημείων καμπής των συναρτήσεων:
 \begin{enumerate}
  \item<1-> $f(x)=συνx-\dfrac{x^2}{3}+\dfrac{x^2}{2}-1$
  \item<2-> $f(x)=2x(\ln x-1)-\ln^2x$
 \end{enumerate}

 % \hyperlink{Λύση5}{\beamerbutton{Λύση}}
\end{frame}

\subsection{Άσκηση 6}
\begin{frame}[label=Άσκηση6,t]{Εξάσκηση 6}
 Να βρείτε τα διαστήματα στα οποία οι παρακάτω συναρτήσεις είναι κυρτές ή κοίλες και να προσδιορίσετε (αν υπάρχουν) τα σημεία καμπής
 \begin{enumerate}
  \item<1-> $f(x)=σφx$, $x\in (0,\pi)$
  \item<2-> $f(x)=εφx-x+2\ln (συνx)$, $x\in (-\dfrac{\pi}{2},\dfrac{\pi}{2})$
 \end{enumerate}

 % \hyperlink{Λύση6}{\beamerbutton{Λύση}}
\end{frame}

\subsection{Άσκηση 7}
\begin{frame}[label=Άσκηση7,t]{Εξάσκηση 7}
 Δίνεται η συνάρτηση $f(x)=x^3-3x$
 \begin{enumerate}
  \item<1-> Να μελετήσετε τη συνάρτηση $f$ ως προς τη μονοτονία, τα ακρότατα, την κυρτότητα και τα σημεία καμπής
  \item<2-> Να βρείτε τις οριακές τιμές της $f$ στα άκρα του διαστήματος του πεδίου ορισμού της, να κάνετε τον πίνακα μεταβολών της $f$ και με βάση τις απαντήσεις σας στα προηγούμενα ερωτήματα, να σχεδιάσετε τη γραφική παράσταση της $f$
 \end{enumerate}

 %\hyperlink{Λύση7}{\beamerbutton{Λύση}}
\end{frame}

\subsection{Άσκηση 8}
\begin{frame}[label=Άσκηση8,t]{Εξάσκηση 8}
 Δίνεται η συνάρτηση $f(x)=e^x+\ln x$. Να δείξετε ότι:
 \begin{enumerate}
  \item<1-> Η $C_f$ έχει μοναδικό σημείο καμπής το $Α(x_0,f(x_0))$
  \item<2-> $x_0<\dfrac{4}{5}$
 \end{enumerate}

 %\hyperlink{Λύση8}{\beamerbutton{Λύση}}
\end{frame}

\subsection{Άσκηση 9}
\begin{frame}[label=Άσκηση9,t]{Εξάσκηση 9}
 Δίνεται η συνάρτηση $f(x)=6e^x-x^3-4x^2$. Να δείξετε ότι η $f$ έχει ακριβώς δύο σημεία καμπής

 %\hyperlink{Λύση9}{\beamerbutton{Λύση}}
\end{frame}

\subsection{Άσκηση 10}
\begin{frame}[label=Άσκηση10,t]{Εξάσκηση 10}
 Δίνεται η συνάρτηση $$f(x)=\dfrac{x^4}{12}-\dfrac{α^2x^3}{3}+\dfrac{αx^2}{2}-3x+1$$
 Να βρείτε τις τιμές του $α\in\mathbb{R}$ για τις οποίες:
 \begin{enumerate}
  \item<1-> Η $f$ παρουσιάζει καμπή στο $x_0=1$
  \item<2-> Η $C_f$ έχει ακριβώς δύο σημεία καμπής
 \end{enumerate}

 %\hyperlink{Λύση10}{\beamerbutton{Λύση}}
\end{frame}

\subsection{Άσκηση 11}
\begin{frame}[label=Άσκηση11,,t]{Εξάσκηση 11}
 Να αποδείξετε ότι για κάθε $α\in (-2,2)$ η συνάρτηση $f(x)=x^4-2αx^3+6x^2-1$ είναι κυρτή σε όλο το $\mathbb{R}$

 %\hyperlink{Λύση11}{\beamerbutton{Λύση}}
\end{frame}

\subsection{Άσκηση 12}
\begin{frame}[label=Άσκηση12,t]{Εξάσκηση 12}
 Έστω $f:\mathbb{R}\to\mathbb{R}$ μία συνάρτηση για την οποία ισχύει
 $$f''(x)+f(x)\ne 2f'(x)\text{, για κάθε } x\in\mathbb{R}$$
 Να δείξετε ότι η συνάρτηση $g(x)=e^{-x}f(x)$, $x\in\mathbb{R}$ δεν έχει σημεία καμπής.

 %\hyperlink{Λύση12}{\beamerbutton{Λύση}}
\end{frame}

\subsection{Άσκηση 13}
\begin{frame}[label=Άσκηση13,t]{Εξάσκηση 13}

 Έστω $f:(1,3)\to\mathbb{R}$ μία συνάρτηση, η οποία είναι δύο φορές παραγωγίσιμη και ισχύει:
 $$f^2(x)+xf(x)+x^2-3x+1=0\text{, για κάθε } x\in (1,3)$$
 Να δείξετε ότι η συνάρτηση $f$, δεν παρουσιάζει καμπή

 %\hyperlink{Λύση13}{\beamerbutton{Λύση}}
\end{frame}


\appendix

\section{Αποδείξεις}
\begin{frame}[label=Απόδειξη1]{Απόδειξη σημείο καμπής}
 \onslide<1-> Έστω ότι η $f$ έχει σημείο καμπής στο $x_0$ με κυρτή αριστερά και κοίλη δεξιά του σημείου.

 Άρα $f'(x)< f'(x_0)$ για κάθε $x<x_0$ και $f'(x)<f'(x_0)$ για κάθε $x>x_0$

 \onslide<2-> Αφού $f'$ παραγωγίσιμη, θα υπάρχει το όριο

 $$f''(x_0)=\lim\limits_{x \to x_0^-}{ \dfrac{f'(x)-f'(x_0)}{x-x_0} }\ge 0$$

 \onslide<3-> όμοια
 $$f''(x_0)=\lim\limits_{x \to x_0^+}{ \dfrac{f'(x)-f'(x_0)}{x-x_0} } \le 0$$

 \onslide<4-> Άρα $f''(x_0)=0$ \hyperlink{Θεώρημα1}{\beamerbutton{Πίσω στη θεωρία}}
\end{frame}


% \section{Λύσεις Ασκήσεων}
% \begin{frame}
%  \tableofcontents
% \end{frame}
%
% \subsection{Άσκηση 1}
% \begin{frame}[label=Λύση1]
%  Με θεώρημα ενδιαμέσων τιμών. Η συνάρτηση είναι συνεχής στο $[10,11]$ με $f(10)=1024$ και $f(11)=2048$. Αφού $2023\in (1024,2048)$ υπάρχει $x_0$...
%
%  \hyperlink{Άσκηση1}{\beamerbutton{Πίσω στην άσκηση}}
% \end{frame}
%
% \subsection{Άσκηση 2}
% \begin{frame}[label=Λύση2]
%  Με Bolzano ή με μέγιστης ελάχιστης τιμής και ΘΕΤ.
%
%  \begin{gather*}
%   f(3)<f(2)<f(1) \\
%   3f(3)<f(1)+f(2)+f(3)<3f(1) \\
%   f(3)<\dfrac{f(1)+f(2)+f(3)}{3}<f(1)
%  \end{gather*}
%
%  \hyperlink{Άσκηση2}{\beamerbutton{Πίσω στην άσκηση}}
% \end{frame}
%
% \subsection{Άσκηση 3}
% \begin{frame}[label=Λύση3]
%  Προφανές ελάχιστο στα $x_1=1$ και $x_2=3$. Ως συνεχής στο $[1,3]$ έχει σίγουρα ΚΑΙ μέγιστο στο $(1,3)$
%
%  \hyperlink{Άσκηση3}{\beamerbutton{Πίσω στην άσκηση}}
% \end{frame}
%
% \subsection{Άσκηση 4}
% \begin{frame}[label=Λύση4]
%  Η συνάρτηση `απόστασης` $f(x)-x$ είναι ορισμένη στο κλειστό διάστημα και έχει σίγουρα μέγιστο
%
%  \hyperlink{Άσκηση4}{\beamerbutton{Πίσω στην άσκηση}}
% \end{frame}
%
% \subsection{Άσκηση 5}
% \begin{frame}[label=Λύση5]
%  Όμοια με την Άσκηση 2
%
%  \hyperlink{Άσκηση5}{\beamerbutton{Πίσω στην άσκηση}}
% \end{frame}
%
% \subsection{Άσκηση 6}
% \begin{frame}[label=Λύση6]
%  \begin{enumerate}
%   \item Είναι γνησίως αύξουσα άρα $(f(+\infty),f(-\infty))$
%   \item Προφανώς $[f(0),f(1)]$...
%  \end{enumerate}
%
%  \hyperlink{Άσκηση6}{\beamerbutton{Πίσω στην άσκηση}}
% \end{frame}

\end{document}
