\documentclass{presentation}

\title{Συναρτήσεις}
\subtitle{Μέθοδοι Ολοκλήρωσης}
\author[Λόλας]{Κωνσταντίνος Λόλας}
\institute[$10^ο$ ΓΕΛ]{$10^ο$ ΓΕΛ Θεσσαλονίκης}
\date{}

\begin{document}

\begin{frame}
    \titlepage
\end{frame}

\section{Θεωρία}
\begin{frame}{Σιγά τα ολοκληρώματα!}
    Τι μπορούμε να ολοκληρώσουμε
    \begin{enumerate}
        \item Πολυώνυμα
        \item Εκθετικές
        \item Τριγωνομετρικές
        \item Ρητές με πρωτοβάθμιο διαιρέτη
        \item Πρωτοβάθμιες άρρητες
        \item Έτοιμες από σύνθεση και φυσικά
        \item κάθε πρόσθεση ή αφαίρεση αυτών ΜΟΝΟ
    \end{enumerate}
    Τι γίνεται με τον πολλαπλασιασμό? Διαίρεση? Ακόμα και την απλή $\ln x$?
\end{frame}

\begin{frame}{Ιστορία}
    Ξέρουμε να παραγωγίζουμε γινόμενο
    \begin{align*}
        (f\cdot g)' & =f'g+fg'         \\
        f'g         & =(f\cdot g)'-fg'
    \end{align*}
    \onslide<2-> {
        Άρα
        $$\int f'g\, dx=\int (f\cdot g)'\,dx - \int fg'\, dx$$
        $$\int f'g\, dx=f\cdot g - \int fg'\, dx$$}
\end{frame}

\begin{frame}{Έ και?}
    $$\int f'g\, dx=f\cdot g - \int fg'\, dx$$
    Βρείτε λόγους για να περάσουμε την παράγωγο από την μία συνάρτηση στην άλλη
    \begin{itemize}
        \item<2-> Γιατί τελικά... εξαφανίζεται
        \item<3-> Γιατί δεν ξέρουμε να την ολοκληρώνουμε
        \item<4-> Γιατί μπορούμε να ξαναφτάσουμε στον ίδιο τύπο!!!!!!
    \end{itemize}
\end{frame}

\begin{frame}{Παραδείγματα}
    \begin{enumerate}
        \item<1-> $\int xe^x\,dx$
        \item<2-> $\int x^3e^x\,dx$
        \item<3-> $\int x\ln x\,dx$
        \item<4-> $\int e^{2x}ημ(3x+1)\,dx$
    \end{enumerate}
\end{frame}

\begin{frame}{Και στα εντός ύλης!}
    \begin{block}{Κατά παράγοντες}
        $$\int_a^b f'(x)g(x)\,dx=\left[ f(x)g(x) \right]_a^b -\int_a^b f(x)g'(x)\,dx$$
    \end{block}
\end{frame}

\begin{frame}{Hold your horses}

    Δεν θα μάθουμε \emph{ΠΟΤΕ} να ολοκληρώνουμε όλες τις συναρτήσεις! Μαθηματικό...!

    Μεθόδους για "όμορφες"

    \begin{itemize}
        \item<2-> ρητές
        \item<3-> άρρητες
        \item<4-> τριγωνομετρικές
        \item<5-> από σύνθεση?????
    \end{itemize}

\end{frame}

\begin{frame}{Δοκιμές σύνθεσης}
    \begin{enumerate}
        \item<1-> $\int \dfrac{x}{x^2+1}\,dx$
        \item<2-> $\int \dfrac{\ln x}{x}\, dx$
        \item<3-> $\int 4xεφ(x^2)\ln (ημ(x^2))\, dx$
    \end{enumerate}

\end{frame}

\begin{frame}{Ναι, αλλά... τύπο έχουμε?}
    \begin{block}{Μέθοδος Αντικατάστασης}
        $$\int_a^b f(x)\, dx$$
        Θέτω $x=g(u)$, άρα
        \begin{itemize}
            \item για $x=a\implies u=k$
            \item για $x=b\implies u=l$
            \item $dx=g'(u)du$
        \end{itemize}
        $$\int_a^b f(x)\, dx=\int_{k}^{l} f(g(u))g'(u)\,du$$
    \end{block}

\end{frame}

\section{Ασκήσεις}
\begin{askisi}
    Να υπολογίσετε τα ολοκληρώματα
    \begin{enumerate}
        \item<1-> $\int_{0}^{1}x\,dx$
        \item<2-> $\int_{0}^{\frac{π}{2}}xημx\,dx$
        \item<3-> $\int_{0}^{π}x^2συνx\,dx$
    \end{enumerate}

    % \hyperlink{Λύση1}{\beamerbutton{Λύση}}
\end{askisi}

\begin{askisi}
    Να υπολογίσετε τα ολοκληρώματα
    \begin{enumerate}
        \item<1-> $\int_{1}^{2} 2x\ln x \,dx$
        \item<2-> $\int_{1}^{2} \ln x \,dx$
        \item<3-> $\int_{0}^{1} \ln (x+1) \,dx$
        \item<4-> $\int_{1}^{2} \dfrac{\ln x}{x^2} \,dx$
        \item<5-> $\int_{1}^{9} \dfrac{\ln x}{\sqrt{x}} \,dx$
    \end{enumerate}

    % \hyperlink{Λύση2}{\beamerbutton{Λύση}}
\end{askisi}

\begin{askisi}
    Να υπολογίσετε τα ολοκληρώματα
    \begin{enumerate}
        \item<1-> $\int_{0}^{π} e^xσυνx \,dx$
        \item<2-> $\int_{0}^{π} \dfrac{ημ2x}{e^x} \,dx$
    \end{enumerate}

    % \hyperlink{Λύση3}{\beamerbutton{Λύση}}
\end{askisi}

\begin{askisi}
    Να υπολογίσετε το ολοκλήρωμα $\int_{0}^{\frac{0}{4}} \dfrac{x}{συν^2x} \,dx$

    % \hyperlink{Λύση4}{\beamerbutton{Λύση}}
\end{askisi}

\begin{askisi}
    Έστω $F$ μία παράγουσα στο $\mathbb{R}$ της συνάρτησης $f(x)=e^{x^2}$, με $F(1)=0$. Να υπολογίσετε το ολοκλήρωμα $\int_{0}^{1} F(x) \,dx$

    % \hyperlink{Λύση5}{\beamerbutton{Λύση}}
\end{askisi}

\begin{askisi}
    Έστω $f:\mathbb{R}\to\mathbb{R}$ μία συνάρτηση με $f(0)=0$ και συνεχή δεύτερη παράγωγο για την οποία ισχύει $\int_{0}^{π} \left( f(x)+f''(x) \right)ημx  \,dx=π$. Να δείξετε ότι $f(π)=π$

    % \hyperlink{Λύση6}{\beamerbutton{Λύση}}
\end{askisi}

\begin{askisi}
    Έστω $f:\mathbb{R}\to\mathbb{R}$ μία συνάρτηση, η οποία παρουσιάζει τοπικό ακρότατο στο $x_0=2$, έχει συνεχή $f''$ και ισχύει
    $$\int_{0}^{2} \left( xf''(x)+2f'(x) \right) \,dx=0$$
    \begin{enumerate}
        \item<1-> Να δείξετε ότι $f(0)=f(2)$
        \item<2-> Να δείξετε ότι υπάρχει $ξ\in (0,2)$, τέτοιο ώστε $f'(ξ)=0$
    \end{enumerate}

    %\hyperlink{Λύση7}{\beamerbutton{Λύση}}
\end{askisi}

\begin{askisi}
    Δίνεται η συνάρτηση $f(x)=4x-2x+1$. Να υπολογίσετε τα ολοκληρώματα:
    \begin{enumerate}
        \item<1-> $\int_{-1}^{0} f(x+1) \,dx$
        \item<2-> $\int_{1}^{e} \dfrac{f(\ln x)}{x} \,dx$
    \end{enumerate}

    %\hyperlink{Λύση8}{\beamerbutton{Λύση}}
\end{askisi}

\begin{askisi}
    Έστω $f:\mathbb{R}\to\mathbb{R}$ μία συνάρτηση η οποία είναι συνεχής. Να δείξετε ότι
    $$\int_{2}^{4} f\left( \dfrac{2}{x} \right)  \,dx=2\int_{\frac{1}{2}}^{1} \dfrac{f(x)}{x^2} \,dx$$

    %\hyperlink{Λύση9}{\beamerbutton{Λύση}}
\end{askisi}

\begin{askisi}
    ΔΝα υπολογίσετε τα ολοκληρώματα
    \begin{enumerate}
        \item<1-> $\int_{0}^{\frac{π}{2}} συν(x-\frac{π}{3}) \,dx$
        \item<2-> $\int_{0}^{1} \dfrac{1}{2x+1} \,dx$
        \item<3-> $\int_{0}^{\sqrt{3}} \dfrac{x}{\sqrt{x^2+1}} \,dx$
        \item<4-> $\int_{1}^{e} \dfrac{\sqrt{\ln x}}{x} \,dx$
    \end{enumerate}

    %\hyperlink{Λύση10}{\beamerbutton{Λύση}}
\end{askisi}

\begin{askisi}
    Δίνεται η συνάρτηση $f(x)=e^x+x-1$
    \begin{enumerate}
        \item<1-> Να δείξετε ότι ορίζεται η αντίστροφη συνάρτηση $f^{-1}$ και να βρείτε το πεδίο ορισμού της
        \item<2-> Να υπολογίσετε το $\int_{0}^{e} f^{-1}(x) \,dx$
    \end{enumerate}

    %\hyperlink{Λύση11}{\beamerbutton{Λύση}}
\end{askisi}

\begin{askisi}
    Έστω $f:\mathbb{R}\to\mathbb{R}$ μία συνάρτηση με $f(\mathbb{R})=\mathbb{R}$, η οποία είναι παραγωγίσιμη και ισχύει
    $$f^3(x)+f(x)=x \text{, για κάθε } x\in\mathbb{R}$$
    \begin{enumerate}
        \item<1-> Να δείξετε ότι η συνάρτηση $f$ αντιστρέφεται και να βρείτε την $f^{-1}$
        \item<2-> Να υπολογίσετε το $\int_{0}^{2} f(x) \,dx$
    \end{enumerate}

    %\hyperlink{Λύση12}{\beamerbutton{Λύση}}
\end{askisi}

\begin{askisi}
    Έστω $f:[-a,a]\to\mathbb{R}$ μία συνάρτηση, η οποία είναι συνεχής. Να δείξετε ότι:
    \begin{enumerate}
        \item
              \begin{enumerate}
                  \item<1-> Αν η $f$ είναι περιττή, τότε $\int_{-a}^{a} f(x) \,dx=0$
                  \item<2-> Να υπολογίσετε το ολοκλήρωμα $J=\int_{-1}^{1} \dfrac{x}{2+συνx} \,dx$
              \end{enumerate}
        \item<3-> Αν η $f$ είναι άρτια, τότε $\int_{-a}^{a} f(x) \,dx=2\int_{0}^{a} f(x) \,dx$
    \end{enumerate}

    %\hyperlink{Λύση13}{\beamerbutton{Λύση}}
\end{askisi}

\begin{askisi}
    Έστω μία συνεχής συνάρτηση $f:[0,2]\to\mathbb{R}$ για την οποία ισχύει
    $$f(1-x)+f(1+x)=2\text{ για κάθε }x\in [-1,1]$$
    Να υπολογίσετε το ολοκλήρωμα $\int_{0}^{2} f(x) \,dx$

    %\hyperlink{Λύση14}{\beamerbutton{Λύση}}
\end{askisi}

\begin{askisi}
    Να υπολογίσετε το ολοκλήρωμα $\int_{1}^{e^2} |\ln x-1| \,dx$

    %\hyperlink{Λύση15}{\beamerbutton{Λύση}}
\end{askisi}

\begin{askisi}
    Να υπολογίσετ το ολοκλήρωμα $\int_{1}^{e} ημ(\ln x) \,dx$

    %\hyperlink{Λύση16}{\beamerbutton{Λύση}}
\end{askisi}

\begin{askisi}
    Έστω $f:\mathbb{R}\to\mathbb{R}$ μία συνάρτηση η οποία είναι συνεχής και ισχύει:
    $$f(x)=e^x+\int_{0}^{1} xf(x) \,dx\, , x\in\mathbb{R}$$
    Να βρείτε την $f$

    %\hyperlink{Λύση17}{\beamerbutton{Λύση}}
\end{askisi}

\begin{askisi}
    Αν η συνάρτηση $f$ είναι συνεχής στο $[α,β]$ και ισχύει $f(x)=f(α+β-x)$, για κάθε $x\in [α,β]$, να δείξετε ότι:
    $$\int_{α}^{β} xf(x) \,dx=\dfrac{α+β}{2}\int_{α}^{β} f(x) \,dx$$

    %\hyperlink{Λύση18}{\beamerbutton{Λύση}}
\end{askisi}


% \appendix
%
% \section{Αποδείξεις}
% \begin{frame}[label=Απόδειξη1]{Απόδειξη σημείο καμπής}
%  \onslide<1-> Έστω ότι η $f$ έχει σημείο καμπής στο $x_0$ με κυρτή αριστερά και κοίλη δεξιά του σημείου.
%
%  Άρα $f'(x)< f'(x_0)$ για κάθε $x<x_0$ και $f'(x)<f'(x_0)$ για κάθε $x>x_0$
%
%  \onslide<2-> Αφού $f'$ παραγωγίσιμη, θα υπάρχει το όριο
%
%  $$f''(x_0)=\lim\limits_{x \to x_0^-}{ \dfrac{f'(x)-f'(x_0)}{x-x_0} }\ge 0$$
%
%  \onslide<3-> όμοια
%  $$f''(x_0)=\lim\limits_{x \to x_0^+}{ \dfrac{f'(x)-f'(x_0)}{x-x_0} } \le 0$$
%
%  \onslide<4-> Άρα $f''(x_0)=0$ \hyperlink{Θεώρημα1}{\beamerbutton{Πίσω στη θεωρία}}
% \end{frame}


% \section{Λύσεις Ασκήσεων}
% \begin{frame}
%  \tableofcontents
% \end{frame}
%
% \begin{askisi}
%  Με θεώρημα ενδιαμέσων τιμών. Η συνάρτηση είναι συνεχής στο $[10,11]$ με $f(10)=1024$ και $f(11)=2048$. Αφού $2023\in (1024,2048)$ υπάρχει $x_0$...
%
%  \hyperlink{Άσκηση1}{\beamerbutton{Πίσω στην άσκηση}}
% \end{frame}
%
% \begin{askisi}
%  Με Bolzano ή με μέγιστης ελάχιστης τιμής και ΘΕΤ.
%
%  \begin{gather*}
%   f(3)<f(2)<f(1) \\
%   3f(3)<f(1)+f(2)+f(3)<3f(1) \\
%   f(3)<\dfrac{f(1)+f(2)+f(3)}{3}<f(1)
%  \end{gather*}
%
%  \hyperlink{Άσκηση2}{\beamerbutton{Πίσω στην άσκηση}}
% \end{frame}
%
% \begin{askisi}
%  Προφανές ελάχιστο στα $x_1=1$ και $x_2=3$. Ως συνεχής στο $[1,3]$ έχει σίγουρα ΚΑΙ μέγιστο στο $(1,3)$
%
%  \hyperlink{Άσκηση3}{\beamerbutton{Πίσω στην άσκηση}}
% \end{frame}
%
% \begin{askisi}
%  Η συνάρτηση `απόστασης` $f(x)-x$ είναι ορισμένη στο κλειστό διάστημα και έχει σίγουρα μέγιστο
%
%  \hyperlink{Άσκηση4}{\beamerbutton{Πίσω στην άσκηση}}
% \end{frame}
%
% \begin{askisi}
%  Όμοια με την Άσκηση 2
%
%  \hyperlink{Άσκηση5}{\beamerbutton{Πίσω στην άσκηση}}
% \end{frame}
%
% \begin{askisi}
%  \begin{enumerate}
%   \item Είναι γνησίως αύξουσα άρα $(f(+\infty),f(-\infty))$
%   \item Προφανώς $[f(0),f(1)]$...
%  \end{enumerate}
%
%  \hyperlink{Άσκηση6}{\beamerbutton{Πίσω στην άσκηση}}
% \end{frame}

\end{document}
