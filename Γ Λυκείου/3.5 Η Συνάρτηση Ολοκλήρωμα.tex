\documentclass{presentation}

\title{Συναρτήσεις}
\subtitle{Συνάρτηση Ολοκλήρωμα $\int_α^x f(t)\, dt$}
\author[Λόλας]{Κωνσταντίνος Λόλας}
\institute[$10^ο$ ΓΕΛ]{$10^ο$ ΓΕΛ Θεσσαλονίκης}
\date{}

\begin{document}

\begin{frame}
    \titlepage
\end{frame}

\section{Θεωρία}

\begin{frame}{Στο σημερινό επεισόδιο}
    Θα μάθουμε:
    \begin{enumerate}
        \item Πώς υπολογίζουμε τα $\int_α^β f(x)\, dx$ και
        \item Τι σχέση έχουμε με τις αρχικές
    \end{enumerate}
\end{frame}

\begin{frame}{Ήρθε η ώρα της σύνδεσης!}
    \begin{block}{Θεώρημα Παράγουσας - Ορισμένου}
        Αν $f$ είναι μια συνεχής συνάρτηση σε ένα διάστημα $Δ$ και $α$ είναι ένα σημείο του Δ, τότε η συνάρτηση
        $$F(x)=\int_α^x f(t) \, dt$$

        είναι μια παράγουσα της $f$ στο $Δ$. Δηλαδή ισχύει:
        $$\left( \int_α^x f(t) \, dt \right)'=f(x)$$
    \end{block}
\end{frame}

\begin{frame}{Ήρθε η ώρα της σύνδεσης!}
    \begin{block}{Θεμελιώδες Θεώρημα Ολοκληρωτικού Λογισμού}
        Αν $f$ είναι μια συνεχής συνάρτηση σε ένα διάστημα $[α,β]$ και $G$ είναι μία παράγουσα της $f$ στο $[α,β]$, τότε
        $$\int_α^β f(x) \, dx=G(β)-G(α)=\left[ G(x) \right]_α^β $$
    \end{block}
\end{frame}

\begin{frame}{Ρεζουμέ!}
    Για να υπολογίσουμε το $\int_α^β f(x)\, dx$
    \begin{enumerate}[<+->]
        \item Δεν το κάνουμε με τα άπειρα αθροίσματα!
        \item Υπολογίζουμε το $G(x)=\int f(x)\, dx$
        \item Υπολογίζουμε το $\left[ G(x) \right]_α^β=G(β)-G(α) $
    \end{enumerate}
\end{frame}



\appendix
\section{Αποδείξεις}
\begin{frame}{Μέσης τιμής}
    \begin{block}{Θεώρημα Μέσης Τιμής}
        Αν $f(x)$ συνεχής στο $[α,β]$, τότε υπάρχει $ξ\in [α,β]$ ώστε
        $$\int_α^βf(x)\, dx=f(ξ)(β-α)$$
    \end{block}
\end{frame}

\begin{frame}{Απόδειξη Μέσης τιμής}
    Στο $[α,β]$ η συνάρτηση $f(x)$ είναι συνεχής άρα έχει μέγιστη $Μ$ και ελάχιστη $μ$ τιμή, συνεπώς
    $$μ\le f(x) \le Μ$$
    Με τον ορισμό του ορισμοένου ολοκληρώματος η σχέση γίνεται

    $$μ(β-α)\le \int_α^βf(x)\, dx \le Μ(β-α)$$
    $$μ\le \frac{\int_α^βf(x)\, dx}{β-α} \le Μ$$

    που σύμφωνα με το ΘΕΤ θα ισχύει

    $$\frac{\int_α^βf(x)\, dx}{β-α}=f(ξ)\Rightarrow \int_α^βf(x)\, dx=f(ξ)(β-α)$$

\end{frame}

\begin{frame}{Αρχικής}

    \begin{align*}
        F'(x) & =\lim_{h\to 0}\frac{F(x+h)-F(x)}{h}=\lim_{h\to 0}\frac{\int_α^{x+h} f(t) \, dt-\int_α^x f(t) \, dt}{h} \\
              & =\lim_{h\to 0}\frac{\int_x^{x+h} f(t) \, dt}{h}                                                        \\
              & = \lim_{h\to 0}\frac{f(ξ)h}{h},ξ\in[x,x+h]                                                             \\
              & = \lim_{h\to 0}f(ξ)                                                                                    \\
              & =f(x)
    \end{align*}

\end{frame}

\begin{frame}{Ολοκληρωτικού}
    Η $F(x)=\int_α^x f(t) \, dt$ είναι μία αρχική της $f$ άρα θα ισχύει
    $$G(x)=F(x)+c$$
    Αλλά $$G(α)=F(α)+c=0+c=c$$
    Έτσι $$G(x)=F(x)+G(α)$$
    Τότε $$G(β)=F(β)+G(α)\Rightarrow$$
    $$ F(β)=\int_α^β f(x) \, dx=G(β)-G(α)$$
\end{frame}

\end{document}