\documentclass[a4paper, 12pt]{article}

\usepackage{amsmath, amsthm}

\usepackage{unicode-math}
\usepackage{xltxtra}
\usepackage{xgreek}
\setmainfont{Calibri}

\usepackage[margin=1.2cm]{geometry}

\usepackage[most]{tcolorbox}

\tcbuselibrary{listingsutf8}
\tcbset{
  theoremstyle/.style={
    colback=blue!5!white,
    colframe=blue!75!black,
    fonttitle=\bfseries,
    coltitle=black,
    colbacktitle=blue!10!white,
    enhanced,
    attach boxed title to top left={xshift=2mm,yshift=-2mm}
  },
}

\newtcolorbox[auto counter]{theorem}[2][]{%
  theoremstyle,
  title={Απόδειξη~\thetcbcounter: #2}, #1
}


\title{Θεωρήματα και Αποδείξεις Γ' Λυκείου}
\author{Κωνσταντίνος Λόλας}
\date{2025}

\begin{document}

\maketitle

\begin{theorem}{}
  Έστω $P(x)=α_νx^n+α_{n-1}x^{n-1}+...+α_1x+α_0$. Τότε
  $$\lim_{x\to x_0}P(x)=P(x_0)$$
\end{theorem}
\begin{proof}
  \begin{align*}
    \lim_{x\to x_0}P(x) & =\lim_{x\to x_0}α_νx^n+\lim_{x\to x_0}α_{n-1}x^{n-1}+...+\lim_{x\to x_0}α_1x+\lim_{x\to x_0}α_0 \\
                        & =α_ν\lim_{x\to x_0}x^n+α_{n-1}\lim_{x\to x_0}x^{n-1}+...+α_1\lim_{x\to x_0}x+\lim_{x\to x_0}α_0 \\
                        & =α_νx_0^n+α_{n-1}x_0^{n-1}+...+α_1x_0+\lim_{x\to x_0}α_0                                        \\
                        & =α_νx_0^n+α_{n-1}x_0^{n-1}+...+α_1x_0+α_0                                                       \\
                        & =P(x_0)
  \end{align*}
\end{proof}

\begin{theorem}{}
  Έστω $f(x)=\dfrac{P(x)}{Q(x)}$ με $P,Q$ πολυώνυμα. Τότε
  $$\lim_{x\to x_0}f(x)=\frac{P(x_0)}{Q(x_0)}$$
  αν $Q(x_0)\neq 0$.
\end{theorem}
\begin{proof}
  \begin{align*}
    \lim_{x\to x_0}f(x) & =\lim_{x\to x_0}\frac{P(x)}{Q(x)}                \\
                        & =\frac{\lim_{x\to x_0}P(x)}{\lim_{x\to x_0}Q(x)} \\
                        & =\frac{P(x_0)}{Q(x_0)}
  \end{align*}
\end{proof}

\begin{theorem}{}{[Θεώρημα Ενδιάμεσων Τιμών]}
  Έστω μία συνάρτηση $f$ η οποία είναι ορισμένη στο $[a,b]$. αν
  \begin{itemize}
    \item $f$ είναι συνεχής στο $[a,b]$,
    \item $f(a)\ne f(b)$
  \end{itemize}
  τότε για κάθε αριθμό $η$ μεταξύ των $f(a)$ και $f(b)$ υπάρχει τουλάχιστον ένα $x_0\in (a,b)$ τέτοιο ώστε $f(x_0)=η$.
\end{theorem}
\begin{proof}
  Ας υποθέσουμε ότι $f(a)<f(b)$. Τότε θα ισχύει ότι $f(a)<η<f(b)$. Αν θεωρήσουμε συνάρτηση $g(x)=f(x)-η$, παρατηρούμε ότι
  \begin{itemize}
    \item η συνάρτηση $g$ είναι συνεχής στο $[a,b]$,
    \item $g(a)g(b)<0$ (γιατί $g(a)=f(a)-η<0$ και $g(b)=f(b)-η>0$).
  \end{itemize}
  Επομένως, από το θεώρημα Bolzano, υπάρχει τουλάχιστον ένα $x_0\in (a,b)$ τέτοιο ώστε $g(x_0)=0$. Άρα $f(x_0)=η$.
\end{proof}

\begin{theorem}{}
  Αν μία συνάρτηση είναι παραγωγίσιμη σε ένα σημείο $x_0$ τότε είναι και συνεχής στο σημείο αυτό.
\end{theorem}
\begin{proof}
  Για $x\ne x_0$ έχουμε
  \begin{align*}
    \lim_{x\to x_0}f(x) -f(x_0) & =\lim_{x\to x_0}\frac{f(x) -f(x_0)}{x-x_0}\cdot(x-x_0)                \\
                                & =\lim_{x\to x_0}\frac{f(x) -f(x_0)}{x-x_0}\cdot\lim_{x\to x_0}(x-x_0) \\
                                & =\lim_{x\to x_0}\frac{f(x) -f(x_0)}{x-x_0}\cdot 0                     \\
                                & =f'(x_0) \cdot 0 = 0
  \end{align*}
  Άρα $\lim_{x\to x_0}f(x)=f(x_0)$.
\end{proof}

\begin{theorem}{}[]
  Έστω η συνάρτηση $f(x)=c$, $c\in\mathbb{R}$. Η συνάρτηση $f$ είναι παραγωγίσιμη στο $\mathbb{R}$ και $f'(x)=0$ για κάθε $x\in\mathbb{R}$. Δηλαδή
  $$(c)'=0$$
\end{theorem}
\begin{proof}
  Πράγματι, αν $x_0\in\mathbb{R}$, τότε για κάθε $x\ne x_0$ έχουμε
  $$ \frac{f(x)-f(x_0)}{x-x_0}=\frac{c-c}{x-x_0}=0$$
  Επομένως
  $$\lim_{x\to x_0}\frac{f(x)-f(x_0)}{x-x_0}=\lim_{x\to x_0}0=0$$
  Άρα $f'(x_0)=0$.
\end{proof}

\begin{theorem}{}
  ́Εστω η συνάρτηση $f(x) = x$. Η συνάρτηση $f$ είναι παραγωγίσιμη
  στο $\mathbb{R}$ και ισχύει $f'(x)=1$, Δηλαδή
  $$(x)'=1$$
\end{theorem}
\begin{proof}
  Πράγματι, αν $x_0\in\mathbb{R}$, τότε για κάθε $x\ne x_0$ έχουμε
  $$ \frac{f(x)-f(x_0)}{x-x_0}=\frac{x-x_0}{x-x_0}=1$$
  Επομένως
  $$\lim_{x\to x_0}\frac{f(x)-f(x_0)}{x-x_0}=\lim_{x\to x_0}1=1$$
  Άρα $f'(x_0)=1$.
\end{proof}

\begin{theorem}{}
  Έστω η  συνάρτηση $f(x)=x^n$, $n\in\mathbb{N}-\{0,1\}$. Η συνάρτηση $f$ είναι παραγωγίσιμη στο $\mathbb{R}$ και ισχύει $f'(x)=nx^{n-1}$, δηλαδή
  $$(x^n)'=nx^{n-1}$$
\end{theorem}
\begin{proof}
  Πράγματι, αν $x_0\in\mathbb{R}$, τότε για κάθε $x\ne x_0$ έχουμε
  \begin{align*}
    \frac{f(x)-f(x_0)}{x-x_0} & =\frac{x^n-(x_0)^n}{x-x_0}                                          \\
                              & =\frac{(x-x_0)(x^{n-1}+x^{n-2}x_0+...+xx_0^{n-2}+x_0^{n-1})}{x-x_0} \\
                              & =x^{n-1}+x^{n-2}x_0+...+xx_0^{n-2}+x_0^{n-1}
  \end{align*}
  Επομένως
  $$\lim_{x\to x_0}\frac{f(x)-f(x_0)}{x-x_0}=nx_0^{n-1}$$
  Άρα $f'(x_0)=nx_0^{n-1}$.
\end{proof}

\begin{theorem}{}
  Έστω η συνάρτηση $f(x)=\sqrt{x}$, $x\geq 0$. Η συνάρτηση $f$ είναι παραγωγίσιμη στο $\mathbb{R}^+$ και ισχύει $f'(x)=\dfrac{1}{2\sqrt{x}}$, δηλαδή
  $$(\sqrt{x})'=\frac{1}{2\sqrt{x}}$$
\end{theorem}
\begin{proof}
  Πράγματι, αν $x_0\in\mathbb{R}^+$, τότε για κάθε $x\ne x_0$ έχουμε
  \begin{align*}
    \frac{f(x)-f(x_0)}{x-x_0} & =\frac{\sqrt{x}-\sqrt{x_0}}{x-x_0}                                               \\
                              & =\frac{(\sqrt{x}-\sqrt{x_0})(\sqrt{x}+\sqrt{x_0})}{(x-x_0)(\sqrt{x}+\sqrt{x_0})} \\
                              & =\frac{x-x_0}{(x-x_0)(\sqrt{x}+\sqrt{x_0})}                                      \\
                              & =\frac{1}{(\sqrt{x}+\sqrt{x_0})}
  \end{align*}
  Επομένως
  $$\lim_{x\to x_0}\frac{f(x)-f(x_0)}{x-x_0}=\frac{1}{2\sqrt{x_0}}$$
  Άρα $f'(x_0)=\dfrac{1}{2\sqrt{x_0}}$.
\end{proof}

\begin{theorem}{}
  Αν οι συναρτήσεις $f,g$ είναι παραγωγίσιμες στο $x_0$, τότε και η συνάρτηση $f+g$ είναι παραγωγίσιμη στο $x_0$ και ισχύει
  $$(f+g)'(x_0)=f'(x_0)+g'(x_0)$$
\end{theorem}
\begin{proof}
  Πράγματι, αν $x_0\in\mathbb{R}$, τότε για κάθε $x\ne x_0$ έχουμε
  \begin{align*}
    \frac{(f+g)(x)-(f+g)(x_0)}{x-x_0} & =\frac{f(x)+g(x)-f(x_0)-g(x_0)}{x-x_0}               \\
                                      & =\frac{(f(x)-f(x_0))+(g(x)-g(x_0))}{x-x_0}           \\
                                      & =\frac{f(x)-f(x_0)}{x-x_0}+\frac{g(x)-g(x_0)}{x-x_0}
  \end{align*}
  Επομένως, αφού οι συναρτήσεις $f,g$ είναι παραγωγίσιμες στο $x_0$, ισχύει
  $$\lim_{x\to x_0}\frac{(f+g)(x)-(f+g)(x_0)}{x-x_0}=\lim_{x\to x_0}\frac{f(x)-f(x_0)}{x-x_0}+\lim_{x\to x_0}\frac{g(x)-g(x_0)}{x-x_0}$$
  Άρα $(f+g)'(x_0)=f'(x_0)+g'(x_0)$.
\end{proof}

\begin{theorem}{}
  Έστω η συνάρτηση $f(x)=x^{-ν}$, $ν\in\mathbb{R}^*$. Η συνάρτηση $f$ είναι παραγωγίσιμη στο $\mathbb{R}^*$ και ισχύει $f'(x)=-νx^{-ν-1}$, δηλαδή
  $$(x^{-ν})'=-νx^{-ν-1}$$
\end{theorem}
\begin{proof}
  Πράγματι, για κάθε $x\ne x_0$ έχουμε
  \begin{align*}
    (x^{-ν})'=\left( \frac{1}{x^ν} \right)'=\frac{1'\cdot x^{ν}-ν\cdot 1\cdot x^{ν-1}}{(x^ν)^2} & =\frac{0-ν \cdot x^{-ν-1}}{x^{2ν}}=-ν\cdot x^{-ν-1}
  \end{align*}
\end{proof}

\begin{theorem}{}
  Έστω η συνάρτηση $f(x)=εφx$. Η συνάρτηση είναι παραγωγίσιμη στο $\mathbb{R_1}=\mathbb{R}-\{x:συνx\ne 0\}$ και ισχύει $f'(x)=\dfrac{1}{συν^2x}$, δηλαδή
  $$(εφx)'=\frac{1}{συν^2x}$$
\end{theorem}
\begin{proof}
  Πράγματι, για κάθε $x\ne x_0$ έχουμε
  \begin{align*}
    (εφx)' & =\left( \frac{ημx}{συνx}\right)'                \\
           & =\frac{ημ'x\cdot συνx-ημx\cdot συν'x}{(συνx)^2} \\
           & =\frac{συνx\cdot συνx+ημx\cdot ημx}{(συνx)^2}   \\
           & =\frac{συν^2x+ημ^2x}{(συνx)^2}                  \\
           & =\frac{1}{συν^2x}
  \end{align*}
\end{proof}

\begin{theorem}{}
  Έστω η συνάρτηση $f(x)=x^a$, $a\in\mathbb{R}$, $a\in\mathbb{R}-\mathbb{Z}$. Η συνάρτηση $f$ είναι παραγωγίσιμη στο $(0,+\infty)$ και ισχύει $f'(x)=ax^{a-1}$, δηλαδή
\end{theorem}
\begin{proof}
  Πράγματι, αν $y=x^a$, τότε $y=e^{a\ln x}$. Άρα
  \begin{align*}
    (x^a)' & =\left( e^{a\ln x} \right)'         \\
           & =e^{a\ln x}\cdot a\cdot \frac{1}{x} \\
           & =x^a\cdot a\cdot \frac{1}{x}        \\
           & =ax^{a-1}
  \end{align*}
\end{proof}

\begin{theorem}{}
  Έστω η συνάρτηση $f(x)=a^x$, $a>0$. Η συνάρτηση $f$ είναι παραγωγίσιμη στο $\mathbb{R}$ και ισχύει $f'(x)=a^x\cdot \ln a$, δηλαδή
  $$(a^x)'=a^x\cdot \ln a$$
\end{theorem}
\begin{proof}
  Πράγματι, αν $y=a^x$, τότε $y=e^{x\ln a}$. Άρα
  \begin{align*}
    (a^x)' & =\left( e^{x\ln a} \right)' \\
           & =e^{x\ln a}\cdot \ln a      \\
           & =a^x\cdot \ln a
  \end{align*}
\end{proof}

\begin{theorem}{}
  Έστω η συνάρτηση $f(x)=\ln |x|$, $x\in\mathbb{R}^*$. Η συνάρτηση $f$ είναι παραγωγίσιμη στο $\mathbb{R}^*$ και ισχύει $f'(x)=\dfrac{1}{x}$, δηλαδή
  $$(\ln |x|)'=\frac{1}{x}$$
\end{theorem}
\begin{proof}
  Πράγματι,
  \begin{itemize}
    \item Για $x>0$ έχουμε $(\ln x)' =\dfrac{1}{x}$
    \item Για $x<0$ έχουμε $(\ln (-x))' =\dfrac{1}{-x}(-x)'=\dfrac{1}{x}$
  \end{itemize}
\end{proof}

\begin{theorem}{Συνέπειες του Θεωρήματος Μέσης Τιμής}
  ́Εστω μία συνάρτηση $f$ ορισμένη σε ένα διάστημα $∆$. Αν
  \begin{itemize}
    \item η f είναι συνεχής στο $∆$,
    \item $f'(x)=0$ για κάθε εσωτερικό σημείο $x$ του $∆$,
  \end{itemize}
  τότε η $f$ είναι σταθερή σε όλο το διάστημα $∆$.
\end{theorem}
\begin{proof}
  Αρκεί να αποδείξουμε ότι για οποιαδήποτε $x_1,x_2\in ∆$ ισχύει $f(x_1)=f(x_2)$. Πράγματι:
  \begin{itemize}
    \item Αν $x_1=x_2$, τότε $f(x_1)=f(x_2)$.
    \item Αν $x_1\ne x_2$, τότε το διάστημα $[x_1,x_2]$ είναι εσωτερικό του $∆$. Επομένως, από το θεώρημα Μέσης Τιμής, υπάρχει τουλάχιστον ένα σημείο $ξ\in (x_1,x_2)$ τέτοιο ώστε
          $$f'(ξ)=\frac{f(x_2)-f(x_1)}{x_2-x_1}$$
          Επειδή $f'(ξ)=0$, προκύπτει ότι
          $$\frac{f(x_2)-f(x_1)}{x_2-x_1}=0$$
          Άρα $f(x_2)-f(x_1)=0$ και επομένως $f(x_2)=f(x_1)$.
  \end{itemize}
\end{proof}

\begin{theorem}{}
  Έστω δύο συναρτήσεις $f,g$ ορισμένες στο $Δ$. Αν
  \begin{itemize}
    \item οι $f,g$ είναι συνεχείς στο $Δ$,
    \item $f'(x)=g'(x)$ για κάθε εσωτερικό σημείο $x$ του $Δ$,
  \end{itemize}
  τότε υπάρχει σταθερός αριθμός $c$ ώστε για κάθε $x\in Δ$ ισχύει
  $$f(x)=g(x)+c$$
\end{theorem}
\begin{proof}
  Η συνάρτηση $f(x)-g(x)$ είναι συνεχής στο $Δ$ και για κάθε εσωτερικό σημείο $x$ του $Δ$ ισχύει
  $$\left( f-g \right)'(x)=f'(x)-g'(x)=0$$
  Επομένως, από το θεώρημα συνέπειας του Θεωρήματος Μέσης Τιμής, προκύπτει ότι η συνάρτηση $f-g$ είναι σταθερή στο $Δ$. Άρα υπάρχει σταθερός αριθμός $c$ ώστε για κάθε $x\in Δ$ ισχύει $f(x)-g(x)=c$. Δηλαδή $f(x)=g(x)+c$.
\end{proof}

\begin{theorem}{}
  Έστω η συνάρτηση $f$ η οποία είναι συνεχής στο $Δ$.
  \begin{itemize}
    \item Αν $f'(x)>0$ για κάθε εσωτερικό σημείο $x$ του $Δ$, τότε η $f$ είναι αύξουσα στο $Δ$.
    \item Αν $f'(x)<0$ για κάθε εσωτερικό σημείο $x$ του $Δ$, τότε η $f$ είναι φθίνουσα στο $Δ$.
  \end{itemize}
\end{theorem}
\begin{proof}
  Αποδεικνύουμε την περίπτωση που είναι $f'(x)>0$.

  Έστω $x_1,x_2\in Δ$ με $x_1<x_2$. Επειδή η $f$ είναι συνεχής στο $Δ$, το διάστημα $[x_1,x_2]$ είναι εσωτερικό του $Δ$. Επομένως, από το θεώρημα Μέσης Τιμής, υπάρχει τουλάχιστον ένα σημείο $ξ\in (x_1,x_2)$ τέτοιο ώστε
  $$f'(ξ)=\frac{f(x_2)-f(x_1)}{x_2-x_1}$$
  Απειδή $f'(ξ)>0$, τότε $\frac{f(x_2)-f(x_1)}{x_2-x_1}>0$. Άρα $f(x_2)>f(x_1)$ και επομένως η $f$ είναι αύξουσα στο $Δ$.

  Στην περίπτωση που είναι  $f'(x)<0$, η απόδειξη γίνεται με παρόμοιο τρόπο.
\end{proof}

\begin{theorem}{}
  Έστω η συνάρτηση $f$ ορισμένη σε ένα διάστημα $Δ$ και $x_0$ ένα εσωτερικό σημείο του $Δ$. Αν η συνάρτηση $f$ παρουσιάζει τοπικό ακρότατο στο $x_0$ και είναι παραγωγίσιμη στο $x_0$, τότε
  $$f'(x_0)=0$$
\end{theorem}
\begin{proof}
  Ας υποθέσουμε ότι η $f$ παρουσιάζει στο $x0$ τοπικό μέγιστο.
  Επειδή το $x_0$ είναι εσωτερικό σημείο του $Δ$ και η $f$ παρουσιάζει τοπικό μέγιστο στο $x_0$, τότε υπάρχει $δ>0$ τέτοιο ώστε
  $$f(x_0)\geq f(x) \text{ για κάθε } x\in (x_0-δ,x_0+δ)$$
  Επειδή η $f$ είναι παραγωγίσιμη στο $x_0$
  $$f'(x_0)=\lim_{x\to x_0^-} \dfrac{f(x)-f(x_0)}{x-x_0}=\lim_{x\to x_0^+} \dfrac{f(x)-f(x_0)}{x-x_0}$$
  Επομένως
  \begin{itemize}
    \item Αν $x\in (x_0-δ,x_0)$, τότε $\frac{f(x)-f(x_0)}{x-x_0}\geq 0$, οπότε θα έχουμε
          $$f'(x_0)=\lim_{x\to x_0^-} \frac{f(x)-f(x_0)}{x-x_0}\geq 0$$
    \item Αν $x\in (x_0,x_0+δ)$, τότε $\frac{f(x)-f(x_0)}{x-x_0}\leq 0$, οπότε θα έχουμε
          $$f'(x_0)=\lim_{x\to x_0^+} \frac{f(x)-f(x_0)}{x-x_0}\leq 0$$
  \end{itemize}
  Συνεπώς έχουμε
  $$f'(x_0)\geq 0 \text{ και } f'(x_0)\leq 0$$
  Άρα $f'(x_0)=0$.

  Η απόδειξη για τοπικό ελάχιστο είναι ανάλογη.
\end{proof}

\begin{theorem}{}

  Έστω μία συνάρτηση $f$ ορισμένη σε ένα διάστημα $(α,β)$, με εξαίρεση ίσως ένα σημείο $x_0\in (α,β)$ στο οποίο όμως η $f$ είναι συνεχής.
  \begin{enumerate}
    \item Αν $f'(x)>0$ στο $(α,x_0)$ και $f'(x)<0$ στο $(x_0,β)$, τότε η $f$ παρουσιάζει τοπικό μέγιστο στο $x_0$.
    \item Αν $f'(x)<0$ στο $(α,x_0)$ και $f'(x)>0$ στο $(x_0,β)$, τότε η $f$ παρουσιάζει τοπικό ελάχιστο στο $x_0$.
    \item Αν $f'(x)>0$ διατηρεί πρόσημο στο $(α,x_0)\cup (x_0,β)$, τότε το $f(x_0)$ δεν είναι τοπικό ακρότατο και η $f$ είναι γνησίως μονότονη στο $(α,β)$.
  \end{enumerate}
\end{theorem}
\begin{proof}
  \begin{enumerate}
    \item   Επειδή $f'(x)>0$ στο $(α,x_0)$, και η $f$ είναι συνεχής στο $(α,x_0)$, τότε η $f$ είναι αύξουσα στο $(α,x_0]$. Έτσι έχουμε $f(x)\le f(x_0)$ για κάθε $x\in (α,x_0]$.
          Επειδή $f'(x)<0$ στο $(x_0,β)$, και η $f$ είναι συνεχής στο $(x_0,β)$, τότε η $f$ είναι φθίνουσα στο $[x_0,β)$. Έτσι έχουμε $f(x)\ge f(x_0)$ για κάθε $x\in [x_0,β)$.
          Επομένως έχουμε
          $$f(x)\le f(x_0) \text{ για κάθε } x\in (α,β)$$
          Άρα η $f$ παρουσιάζει τοπικό μέγιστο στο $x_0$.
          Για το
    \item Εργαζόμαστε αναλόγως
    \item Έστω ότι $f'(x)>0$ για κάθε $x\in (α,x_0)\cup (x_0,β)$.

          Επειδή η $f$ είναι συνεχής στο $x_0$ θα είναι γνησίως αύξουσα σε καθένα από τα διαστήματα $(α,x_0]$ και $[x_0,β)$. Επομένως για $x_1<x_0<x_2$ έχουμε $f(x_1)<f(x_0)<f(x_2)$. Άρα το $f(x_0)$ δεν είναι τοπικό ακρότατο.

          Θα δείξουμε τώρα ότι η $f$ είναι γνησίως αύξουσα στο $(α,β)$.

          Πράγματι, έστω $x_1,x_2\in (α,β)$ με $x_1<x_2$.

          \begin{itemize}
            \item Αν $x_1,x_2\in (α,x_0]$, επειδή η $f$ είναι γνησίως αύξουσα στο $(α,x_0]$, τότε $f(x_1)<f(x_2)$.
            \item Αν $x_1,x_2\in [x_0,β)$, επειδή η $f$ είναι γνησίως αύξουσα στο $[x_0,β)$, τότε $f(x_1)<f(x_2)$.
            \item Τέλος αν $x_1<x_0<x_2$ τότε όπως είδαμε παραπάνω έχουμε $f(x_1)<f(x_0)<f(x_2)$.
          \end{itemize}

          Επομένως σε όλες τις περιπτώσεις ισχύει $f(x_1)<f(x_2)$ και άρα η $f$ είναι γνησίως αύξουσα στο $(α,β)$.
          Ομοίως μπορούμε να δείξουμε ότι αν $f'(x)<0$ για κάθε $x\in (α,x_0)\cup (x_0,β)$, τότε η $f$ είναι γνησίως φθίνουσα στο $(α,β)$.
  \end{enumerate}
\end{proof}

\begin{theorem}{}
  Έστω $f$ μία συνάρτηση ορισμένη σε ένα διάστημα $Δ$. Αν $F$ είναι μία παράγουσα της $f$ στο $Δ$, τότε
  \begin{itemize}
    \item όλες οι συναρτήσεις της μορφής
          $$G(x)=F(x)+c \text{, } c\in\mathbb{R}$$
          είναι παράγουσες της $f$ στο $Δ$.
    \item κάθε άλλη παράγουσα της $f$ στο $Δ$ παίρνει τη μορφή
          $$G(x)=F(x)+c \text{, } c\in\mathbb{R}$$
  \end{itemize}
\end{theorem}
\begin{proof}
  \begin{itemize}
    \item Κάθε συνάρτηση της μορφής $G(x)=F(x)+c$ είναι μία παράγουσα της $f$ στο $Δ$.
          Πράγματι, αν $x_0\in Δ$, τότε για κάθε $x\ne x_0$ έχουμε
          \begin{align*}
            G'(x) & =F'(x)+c'=f(x)+0 =f(x)
          \end{align*}

    \item Αν $G$ είναι μία άλλη παράγουσα της $f$ στο $Δ$, τότε για κάθε $x\in Δ$ έχουμε $F'(x)=f(x)$ και $G'(x)=f(x)$, οπότε
          $$F'(x)=G'(x) \text{, } x\in Δ$$
          Άρα $F(x)=G(x)+c$ για κάποιο σταθερό $c\in\mathbb{R}$.
  \end{itemize}
\end{proof}

\begin{theorem}{Θεμελιώδες Θεώρημα του Ολοκληρωτικού Λογισμού}
  ́Εστω $f$ μία συνεχής συνάρτηση σε ένα διάστημα $[α, β]$. Αν $G$ είναι
  μία παράγουσα της $f$ στο $[α, β]$, τότε
  $$\int_{α}^{β} f(x)dx=G(β)-G(α)$$
\end{theorem}
\begin{proof}
  Η συνάρτηση $F(x)=\int_{a}^{x} f(t)dt$ είναι μια παράγουσα της $f$ στο $[α, β]$. Επειδή η $G$ είναι παράγουσα της $f$ στο $[α, β]$, θα υπάρχει σταθερός αριθμός $c$ τέτοιος ώστε
  $$G(x)=F(x)+c \text{ για κάθε } x\in [α, β]$$
  Ειδικότερα, για $x=α$ έχουμε
  $$G(α)=F(α)+c=\int_{a}^{a}f(t)\,dt+c=c$$
  Επομένως, $c=G(α)$ και άρα
  $$G(x)=F(x)+G(α) \text{ για κάθε } x\in [α, β]$$
  Άρα
  $$G(β)=F(β)+G(α)$$
  Επομένως
  $$\int_{α}^{β} f(t)\,dt=G(β)-G(α)$$
\end{proof}

\end{document}