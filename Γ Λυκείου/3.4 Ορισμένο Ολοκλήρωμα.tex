\documentclass{../presentation}

\title{Συναρτήσεις}
\subtitle{Ορισμένο Ολοκλήρωμα}
\author[Λόλας]{Κωνσταντίνος Λόλας}
\institute[$10^ο$ ΓΕΛ]{$10^ο$ ΓΕΛ Θεσσαλονίκης}
\date{}

\begin{document}

\begin{frame}
  \titlepage
\end{frame}

\section{Θεωρία}
\begin{frame}{Κάτι πολύ θεωρητικό}
  Αλλά πρώτα ας παίξουμε με \href{https://www.geogebra.org/classic/asyjagxx}{Geogebra}
\end{frame}

\begin{frame}{Σε ακόμα πιο θεωρητικό}
  Συμβολίζουμε με
  $$\lim_{n\to +\infty}\left( {\sum_{i=0}^{n}f(ξ_i)Δx_i} \right)=\int_α^β f(x)\,dx $$
\end{frame}

\begin{frame}{Που οδηγεί σε ιδιότητες}
  Άμεσες συνέπειες του ορισμού:
  \begin{enumerate}[<+->]
    \item $\int_α^α f(x)\,dx =0$
    \item $\int_α^β f(x)\,dx + \int_β^γ f(x)\,dx =\int_α^γ f(x)\,dx $
    \item $\int_α^β f(x)\,dx = -\int_β^α f(x)\,dx$
    \item $\int_α^β f(x)\,dx + \int_α^β g(x)\,dx =\int_α^β f(x)+g(x)\,dx $
    \item $\int_α^β λf(x)\,dx =λ \int_α^β f(x)\,dx$
    \item Αν $f(x)\ge 0$ τότε $\int_α^β f(x)\,dx \ge 0$
    \item Αν $f(x)\ge 0$ και υπάρχει $ξ$ με $f(ξ)\ne 0$ τότε $\int_α^β f(x)\,dx > 0$
  \end{enumerate}
\end{frame}

\begin{frame}[noframenumbering]
  Στο moodle θα βρείτε τις ασκήσεις που πρέπει να κάνετε, όπως και αυτή τη παρουσίαση
\end{frame}

\section{Ασκήσεις}

\begin{frame}[noframenumbering]
  \vfill
  \centering
  \begin{beamercolorbox}[sep=8pt,center,shadow=true,rounded=true]{title}
    \usebeamerfont{title}Ασκήσεις
  \end{beamercolorbox}
  \vfill
\end{frame}

\begin{askisi}
  Αν $\int_{1}^{3}f(x)\,dx=1$, $\int_{2}^{5}f(x)\,dx=2$, $\int_{2}^{3}f(x)\,dx=3$ και $\int_{1}^{3}g(x)\,dx=2$, να υπολογίσετε τα ολοκληρώματα:
  \begin{enumerate}[<+->]
    \item $\int_{3}^{2}f(x)\,dx$
    \item $\int_{1}^{5}f(x)\,dx$
    \item $\int_{1}^{3}2f(x)-3g(x)\,dx$
  \end{enumerate}
\end{askisi}

\begin{askisi}
  Να εξετάσετε αν είναι καλά ορισμένο το ολοκλήρωμα $\int_{0}^{2}\dfrac{1}{x-1}\, dx$
\end{askisi}

\begin{askisi}
  Αν η γραφική παράσταση της συνάρτησης $f$ διέρχεται από την αρχή των αξόνων και το σημείο $Α(1,2)$, να βρείτε την τιμή του ολοκληρώματος $\int_{0}^{1}f'(x)\,dx$, εφόσον η $f'$ είναι συνεχής
\end{askisi}

\begin{askisi}
  Να υπολογίσετε το $κ$ ώστε
  $$\int_{2}^{κ}\frac{x^2-1}{x^2+1}\, dx-2\int_{k}^{2}\frac{1}{x^2+1}\, dx=1$$
\end{askisi}

\begin{askisi}
  Να υπολογίσετε τα ολοκληρώματα:
  \begin{enumerate}[<+->]
    \item $\int_{0}^{\frac{π}{2}}ημx\,dx$
    \item $\int_{1}^{e^2}\frac{1}{x}\,dx$
    \item $\int_{0}^{1}x^2\,dx$
    \item $\int_{2}^{1}\frac{1}{x^5}\,dx$
    \item $\int_{0}^{4}x\sqrt{x}\,dx$
    \item $\int_{1}^{2}1\,dx$
  \end{enumerate}
\end{askisi}

\begin{askisi}
  Να υπολογίσετε τα ολοκληρώματα:
  \begin{enumerate}[<+->]
    \item $\int_{\frac{π}{2}}^{π}\left( 2συνx+\frac{3}{x} \right)\,dx$
    \item $\int_{1}^{2}\left( 2^x+\frac{1}{x^2} \right)\,dx$
    \item $\int_{0}^{1}x(3x-1)\,dx$
    \item $\int_{0}^{π}(3x^2-ημx)\,dx$
    \item $\int_{0}^{1}(4x^3-x^2-3x-1)\,dx$
  \end{enumerate}
\end{askisi}

\begin{askisi}
  Να υπολογίσετε τα ολοκληρώματα:
  \begin{enumerate}[<+->]
    \item $\int_{0}^{π}συν(2x)\,dx$
    \item $\int_{0}^{1}e^{2x}\,dx$
    \item $\int_{0}^{1}e^{-x}\,dx$
    \item $\int_{0}^{1}\frac{dt}{e^{3t}}$
    \item $\int_{0}^{1}e^{2x-1}\,dx$
    \item $\int_{0}^{1}\frac{du}{u+3}$
    \item $\int_{0}^{1}\frac{1}{3x+2}\,dx$
    \item $\int_{0}^{1}(x-2)^4\,dx$
    \item $\int_{2}^{5}\frac{1}{\sqrt{x-1}}\,dx$
  \end{enumerate}
\end{askisi}

\begin{askisi}
  Να υπολογίσετε τα ολοκληρώματα:
  \begin{enumerate}[<+->]
    \item $\int_{0}^{1}\frac{x}{x^2+1}\,dx$
    \item $\int_{0}^{1}\frac{x}{\sqrt{x^2+1}}\,dx$
    \item $\int_{0}^{1}xe^{x^2}\,dx$
    \item $\int_{\frac{π}{6}}^{\frac{π}{2}}\frac{συνx}{ημ^2x}\, dx$
    \item $\int_{0}^{\frac{π}{3}}εφx\,dx$
    \item $\int_{1}^{2}\frac{\ln x}{x}\, dx$
    \item $\int_{α}^{α^2}\frac{dx}{x\ln x}$, $α>1$
  \end{enumerate}
\end{askisi}

\begin{askisi}
  Να υπολογίσετε τα ολοκληρώματα:
  \begin{enumerate}[<+->]
    \item $\int_{1}^{2}\frac{(x-1)^2}{x}\, dx$
    \item $\int_{0}^{1}\frac{1}{1+e^x}\, dx$
    \item $\int_{\frac{π}{6}}^{\frac{π}{4}}\frac{1}{ημ^2x\cdot συν^2x}\, dx$
  \end{enumerate}
\end{askisi}

\begin{askisi}
  \begin{enumerate}[<+->]
    \item Δίνεται η συνάρτηση $f(x)=\begin{cases}
              2x    & ,  x<0   \\
              e^x-1 & ,  x\ge0
            \end{cases}$
          \begin{enumerate}[<+->]
            \item Να δείξετε ότι η $f$ είναι συνεχής
            \item Να βρείτε το $\int_{-1}^{1}f(x)\, dx$
          \end{enumerate}
    \item Να υπολογίσετε το $\int_{0}^{1}|2x-1|\, dx$
  \end{enumerate}
\end{askisi}

\begin{askisi}
  \begin{enumerate}[<+->]
    \item Αν $f(0)=1$, $f(1)=3$, να υπολογίσετε το $\int_{0}^{1}f(x)f'(x)\, dx$
    \item Αν $f(1)=2f(0)$ και ισχύουν $f^2(x)-f'(x)=f(x)$, $x\in [0,1]$, $f(x)>0$,$x\in [0,1]$, να υπολογίσετε το $\int_{0}^{1}f(x)\, dx$
  \end{enumerate}
\end{askisi}

\begin{askisi}
  Να βρείτε τον τύπο της συνάρτησης $f(x)$ όταν ισχύει:
  \begin{enumerate}[<+->]
    \item $f(x)=x\int_{0}^{1}e^{xt}\, dt$, $x\in\mathbb{R}$
    \item $f(x)=x\int_{0}^{1}\frac{1}{|x-t|+1}\, dt$, $x\ge 1$
  \end{enumerate}
\end{askisi}

\begin{askisi}
  Να δείξετε ότι $$\int_{α}^{β}\left( \int_{α}^{β}f(x)g(t) \, dt \right) \, dx=\int_{α}^{β}f(x) \, dx\int_{α}^{β}g(x) \, dx$$
\end{askisi}

\begin{askisi}
  Έστω μια συνάρτηση $f$ συνεχής στο $\mathbb{R}$, για την οποία ισχύει
  $$f(x)=2x+\int_{0}^{2}f(x) \, dx \text{, } x\in\mathbb{R}$$
  Να αποδείξετε ότι $f(x)=2x-4$, $x\in\mathbb{R}$
\end{askisi}

\appendix

\section{Αποδείξεις Θεωρήματα}

\begin{apodiksi}[Μέσης Τιμής Ολοκληρωτικού]
  $$\int_a^b f(x)\,dx = f(ξ)(b-a)$$
  Έστω $f$ συνεχής στο $[a,b]$. Τότε η $f$ παρουσιάζει ελάχιστο $μ$ και μέγιστο $Μ$ στο $[a,b]$. \pause
  \begin{gather*}
    μ\le f(x) \le Μ                        \\
    μ(b-a)\le \int_a^b f(x)\,dx \le Μ(b-a) \\
    μ \le \frac{1}{b-a}\int_a^b f(x)\,dx \le Μ
  \end{gather*}
  Άρα από ΘΕΤ υπάρχει $ξ\in [a,b]$ τέτοιο ώστε $\frac{\int_a^b f(x)\,dx}{b-a} = f(ξ)$.
\end{apodiksi}

\begin{apodiksi}[Αρχική = Ορισμένο Ολοκλήρωμα]
  $$\text{Αν } F(x)=\int_c^x f(t)\,dt \implies \int_a^b f(x)\,dx = F(b)-F(a)$$
  \begin{align*}
    F'(x) & =\lim_{h\to 0}\frac{F(x+h)-F(x)}{h}                                                                         \\
          & =\lim_{h\to 0}\frac{\int_c^{x+h}f(t)\,dt-\int_c^x f(t)\,dt}{h} =\lim_{h\to 0}\frac{\int_x^{x+h}f(t)\,dt}{h} \\
          & =\lim_{h\to 0}\frac{f(ξ)h}{h} \text{, } ξ\in [x,x+h]                                                        \\
          & =f(ξ) \text{, } ξ\in [x,x+h]                                                                                \\
          & =f(x) \text{, } \text{αφού } h\to 0
  \end{align*}
\end{apodiksi}

\begin{apodiksi}[Θεμελιώδες θεώρημα του ολοκληρωτικού λογισμού]
  Αν $G$ είναι μια αρχική συνάρτηση της $f$ στο $[a,b]$, τότε
  $$\int_a^b f(x)\,dx = G(b)-G(a)$$
  Αφού $G$ αρχική της $f$ ισχύει $G(x) =F(x)+c$

  Έχουμε $G(a)=F(a)+c\implies G(a)=c \implies G(x)=F(x)+G(a)$

  Άρα    $G(b) =F(b)+G(a) \implies F(b)=G(b)-G(a)$

\end{apodiksi}

\end{document}