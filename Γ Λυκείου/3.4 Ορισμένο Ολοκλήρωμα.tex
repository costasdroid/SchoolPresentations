\documentclass{presentation}

\title{Συναρτήσεις}
\subtitle{Ορισμένο Ολοκλήρωμα}
\author[Λόλας]{Κωνσταντίνος Λόλας}
\institute[$10^ο$ ΓΕΛ]{$10^ο$ ΓΕΛ Θεσσαλονίκης}
\date{}

\begin{document}

\begin{frame}
    \titlepage
\end{frame}

\section{Θεωρία}
\begin{frame}{Κάτι πολύ θεωρητικό}
    Αλλά πρώτα ας παίξουμε με \href{https://www.geogebra.org/classic/asyjagxx}{Geogebra}
\end{frame}

\begin{frame}{Σε ακόμα πιο θεωρητικό}
    Συμβολίζουμε με
    $$\lim_{n\to +\infty}\left( {\sum_{i=0}^{n}f(ξ_i)Δx_i} \right)=\int_α^β f(x)\,dx $$
\end{frame}

\begin{frame}{Που οδηγεί σε ιδιότητες}
    Άμεσες συνέπειες του ορισμού:
    \begin{enumerate}[<+->]
        \item $\int_α^α f(x)\,dx =0$
        \item $\int_α^β f(x)\,dx + \int_β^γ f(x)\,dx =\int_α^γ f(x)\,dx $
        \item $\int_α^β f(x)\,dx = -\int_β^α f(x)\,dx$
        \item $\int_α^β f(x)\,dx + \int_α^β g(x)\,dx =\int_α^β f(x)+g(x)\,dx $
        \item $\int_α^β λf(x)\,dx =λ \int_α^β f(x)\,dx$
        \item Αν $f(x)\ge 0$ τότε $\int_α^β f(x)\,dx \ge 0$
        \item Αν $f(x)\ge 0$ και υπάρχει $ξ$ με $f(ξ)\ne 0$ τότε $\int_α^β f(x)\,dx > 0$
    \end{enumerate}
\end{frame}

\end{document}