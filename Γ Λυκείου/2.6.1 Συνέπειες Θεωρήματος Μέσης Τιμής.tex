\documentclass{presentation}

\title{Συναρτήσεις}
\subtitle{Συνέπειες Θεωρήματος Μέσης Τιμής}
\author[Λόλας]{Κωνσταντίνος Λόλας}
\institute[$10^ο$ ΓΕΛ]{$10^ο$ ΓΕΛ Θεσσαλονίκης}
\date{}

\begin{document}

\begin{frame}
    \titlepage
\end{frame}

\section{Θεωρία}
\begin{frame}{Ανάσες, τελείωσε η ανηφόρα}
    Από εδώ και πέρα θα εφαρμόζουμε ΜΟΝΟ (υπολογίζουμε)
    \begin{enumerate}
        \item<1-> Θα βρίσκουμε συνάρτηση από την παράγωγο
        \item<2-> Θα βρίσκουμε μονοτονία
        \item<3-> Θα βρίσκουμε ακρότατα
    \end{enumerate}
    \onslide<4-> και όλα αυτά χάρις το ΘΜΤ (για όσους έλεγαν ΠΟΥ θα χρειαστούν αυτά που κάνουμε!!!!)
\end{frame}

\begin{frame}{Από $f'\to f$}
    Μέχρι τώρα, ξέραμε την συνάρτηση και υπολογίζαμε την παράγωγο. Πάμε για το αντίστροφο!
    \begin{enumerate}
        \item<1-> Βρείτε μία συνάρτηση που $f'(x)=0$
        \item<2-> Βρείτε μία συνάρτηση που $f'(x)=x$
        \item<3-> Βρείτε μία συνάρτηση που $f'(x)=ημx+\frac{1}{x}$
        \item<4-> Βρείτε μία συνάρτηση που $f'(x)=\frac{1}{f(x)}$ ΟΥΠΣ!
    \end{enumerate}
\end{frame}

\begin{frame}{Θεώρημα σταθερής συνάρτησης}
    \begin{block}{Θεώρημα σταθερής συνάρτησης}
        Έστω μία συνάρτηση $f$ ορισμένη στο $Δ$ όπου
        \begin{itemize}
            \item $f$ συνεχής στο $Δ$
            \item $f'(x)=0$ για κάθε \emph{εσωτερικό} σημείο του $Δ$
        \end{itemize}
        Τότε
        $$f(x)=c \text{, για κάθε } x\in Δ$$
    \end{block}
\end{frame}

\begin{frame}{Μάλλον μπλέκουμε σε μπελάδες!}
    \begin{enumerate}
        \item<1-> Για αρχή θα δίνεται η συνάρτηση που ζητάται!
        \item<2-> Αν δεν δίνεται... καλώς ήρθατε στις διαφορικές εξισώσεις!
        \item<3-> Αν τα μάθετε καλά, τα ολοκληρώματα θα είναι γελοία
        \item<4-> Στο μαθηματικό είναι 2-3 μαθήματα (Λογισμός 2, Διαφορικές εξισώσεις...)
    \end{enumerate}
\end{frame}

\begin{frame}{Θεώρημα ίσων παραγώγων}
    \begin{block}{Θεώρημα ίσων παραγώγων}
        Έστω δύο συναρτήσεος $f$ και $g$ ορισμένες στο $Δ$ όπου
        \begin{itemize}
            \item $f$ και $g$ συνεχείς στο $Δ$
            \item $f'(x)=g'(x)$ για κάθε \emph{εσωτερικό} σημείο του $Δ$
        \end{itemize}
        Τότε
        $$f(x)=g(x)+c \text{, για κάθε } x\in Δ$$
    \end{block}
\end{frame}

\begin{frame}
    Στο moodle θα βρείτε τις ασκήσεις που πρέπει να κάνετε, όπως και αυτή τη παρουσίαση
\end{frame}

\section{Ασκήσεις}

\begin{frame}[noframenumbering]
    \vfill
    \centering
    \begin{beamercolorbox}[sep=8pt,center,shadow=true,rounded=true]{title}
        \usebeamerfont{title}Ασκήσεις
    \end{beamercolorbox}
    \vfill
\end{frame}

\begin{askisi}
    Έστω $f:(0,+\infty)\to\mathbb{R}$ μία συνάρτηση, η οποία είναι παραγωγίσιμη και ισχύει $xf'(x)-2f(x)=0$ για κάθε $x>0$
    \begin{enumerate}
        \item<1-> Να δείξετε ότι η συνάρτηση $g(x)=\frac{f(x)}{x^2}$. $x>0$ είναι σταθερή
        \item<2-> Αν επιπλέον $f(1)=2$ να βρείτε τον τύπο υης $f$
    \end{enumerate}

    % \hyperlink{Λύση1}{\beamerbutton{Λύση}}
\end{askisi}

\begin{askisi}
    Έστω $f:[0,π]\to\mathbb{R}$ μία συνάρτηση με $f(0)=1$, η οποία είναι συνεχής και ισχύει $f'(x)=xσυνx$ για κάθε $x\in (0,π)$

    Να δείξετε ότι $f(x)=xημx+συνx$, $x\in [0,π]$

    % \hyperlink{Λύση2}{\beamerbutton{Λύση}}
\end{askisi}

\begin{askisi}
    Έστω $f:(0,+\infty)\to\mathbb{R}$ μία συνάρτηση με $f(1)=0$, η οποία είναι παραγωγίσιμη και ισχύει $f'(x)=\dfrac{1-xf(x)}{x^2}$ για κάθε $x>0$

    Να δείξετε ότι $f(x)=\dfrac{\ln x}{x}$, $x>0$

    % \hyperlink{Λύση3}{\beamerbutton{Λύση}}
\end{askisi}

\begin{askisi}
    Έστω $f:(0,π)\to\mathbb{R}$ μία συνάρτηση με $f(\frac{π}{2})=1$, η οποία είναι παραγωγίσιμη και ισχύει $f'(x)-σφx\cdot f(x)=0$ για κάθε $x\in (0,π)$

    Να δείξετε ότι $f(x)=ημx$, $x\in (0,π)$

    % \hyperlink{Λύση4}{\beamerbutton{Λύση}}
\end{askisi}

\begin{askisi}
    Έστω $f:\mathbb{R}\to\mathbb{R}$ μία συνάρτηση με $f(0)=1$, η οποία είναι παραγωγίσιμη και ισχύει $f'(x)=f(x)-e^xημx$ για κάθε $x\in\mathbb{R}$

    Να δείξετε ότι $f(x)=e^xσυνx$, $x\in\mathbb{R}$

    % \hyperlink{Λύση5}{\beamerbutton{Λύση}}
\end{askisi}

\begin{askisi}
    Έστω $f:\mathbb{R}\to\mathbb{R}$ μία συνάρτηση με $f(0)=0$, η οποία είναι παραγωγίσιμη και ισχύει $f'(x)=2xe^{-f(x)}$ για κάθε $x\in\mathbb{R}$

    Να δείξετε ότι $f(x)=\ln (x^2+1)$, $x\in\mathbb{R}$

    % \hyperlink{Λύση6}{\beamerbutton{Λύση}}
\end{askisi}

\begin{askisi}
    Έστω $f$, $g:\mathbb{R}\to\mathbb{R}$ δύο συναρτήσεις οι οποίες είναι δύο φορές παραγωγίσιμες και ισχύουν
    \begin{itemize}
        \item $f''(x)=g''(x)$ για κάθε $x\in\mathbb{R}$
        \item $f(0)=g(0)+1$
    \end{itemize}

    Να δείξετε ότι $f(0)=g(0)+1$

    %\hyperlink{Λύση7}{\beamerbutton{Λύση}}
\end{askisi}

\begin{askisi}
    Έστω $f:(0,+\infty)\to\mathbb{R}$ μία συνάρτηση με $f(0)=0$, η οποία είναι συνεχής και ισχύει $f(x)=x(f(x)-f'(x))$ για κάθε $x>0$
    \begin{enumerate}
        \item<1-> Αν $g(x)=xf(x)$, $x>0$ να δείξετε ότι $g(x)=c\cdot e^x$, $x>0$
        \item<2-> Αν επιπλέον $f(1)=e$ να βρείτε τον τύπο της $f$
    \end{enumerate}

    %\hyperlink{Λύση8}{\beamerbutton{Λύση}}
\end{askisi}

\begin{askisi}
    Έστω $f:\mathbb{R}\to\mathbb{R}$ μία συνάρτηση η οποία είναι δύο φορές παραγωγίσιμη και ισχύουν
    \begin{itemize}
        \item $2f(x)+4xf'(x)+(x^2+9)f''(x)=0$ για κάθε $x\in\mathbb{R}$
        \item $f(0)=0$ και $f'(0)=\frac{1}{9}$
    \end{itemize}

    Να δείξετε ότι $f(x)=\frac{x}{x^2+9}$

    %\hyperlink{Λύση9}{\beamerbutton{Λύση}}
\end{askisi}

\begin{askisi}
    Έστω $f:\mathbb{R}\to\mathbb{R}$ μία συνάρτηση για την οποία ισχύει
    $$|f(x)-f(y)|\le (x-y)^2 \text{ για κάθε } x,y\in\mathbb{R}$$

    Να δείξετε ότι η $f$ είναι σταθερή

    %\hyperlink{Λύση10}{\beamerbutton{Λύση}}
\end{askisi}

\begin{askisi}
    Αν για τις συναρτήσεις $f$, $g$ ισχύουν $f(0)=0$, $g(0)=1$, $f'(x)=g(x)$ και $g'(x)=-f(x)$ για κάθε $x\in\mathbb{R}$ να αποδείξετε ότι:
    \begin{enumerate}
        \item<1-> $f^2(x)+g^2(x)=1$, $x\in\mathbb{R}$
        \item<2-> $f(x)=ημx$, $x\in\mathbb{R}$ και $g(x)=συνx$, $x\in\mathbb{R}$
    \end{enumerate}

    %\hyperlink{Λύση11}{\beamerbutton{Λύση}}
\end{askisi}


\appendix

\section{Αποδείξεις}
\begin{frame}{Απόδειξη σταθερής συνάρτησης}
    Θα δείξουμε ότι $f(x_1)=f(x_2)$ για κάθε $x_1$, $x_2\in Δ$

    \onslide<1-> Αν $x_1=x_2$...

    \onslide<2-> Αν $x_1\ne x_2$ τότε αφού είναι παραγωγίσιμη στο $Δ$ θα ισχύει το ΘΜΤ.

    \onslide<3-> Υπάρχει $ξ\in Δ$ ώστε $f'(ξ)=\frac{f(x_2)-f(x_1)}{x_2-x_1}$.

    \onslide<4-> Αλλά $f'(x)=0$ για κάθε $x\in Δ$

    \onslide<5-> Άρα $f'(ξ)=\frac{f(x_2)-f(x_1)}{x_2-x_1}=0$

    \onslide<6-> $f(x_2)=f(x_1)$
\end{frame}


% \section{Λύσεις Ασκήσεων}
% \begin{frame}
%  \tableofcontents
% \end{frame}
%
% \begin{askisi}
%  Με θεώρημα ενδιαμέσων τιμών. Η συνάρτηση είναι συνεχής στο $[10,11]$ με $f(10)=1024$ και $f(11)=2048$. Αφού $2023\in (1024,2048)$ υπάρχει $x_0$...
%
%  \hyperlink{Άσκηση1}{\beamerbutton{Πίσω στην άσκηση}}
% \end{frame}
%
% \begin{askisi}
%  Με Bolzano ή με μέγιστης ελάχιστης τιμής και ΘΕΤ.
%
%  \begin{gather*}
%   f(3)<f(2)<f(1) \\
%   3f(3)<f(1)+f(2)+f(3)<3f(1) \\
%   f(3)<\frac{f(1)+f(2)+f(3)}{3}<f(1)
%  \end{gather*}
%
%  \hyperlink{Άσκηση2}{\beamerbutton{Πίσω στην άσκηση}}
% \end{frame}
%
% \begin{askisi}
%  Προφανές ελάχιστο στα $x_1=1$ και $x_2=3$. Ως συνεχής στο $[1,3]$ έχει σίγουρα ΚΑΙ μέγιστο στο $(1,3)$
%
%  \hyperlink{Άσκηση3}{\beamerbutton{Πίσω στην άσκηση}}
% \end{frame}
%
% \begin{askisi}
%  Η συνάρτηση `απόστασης` $f(x)-x$ είναι ορισμένη στο κλειστό διάστημα και έχει σίγουρα μέγιστο
%
%  \hyperlink{Άσκηση4}{\beamerbutton{Πίσω στην άσκηση}}
% \end{frame}
%
% \begin{askisi}
%  Όμοια με την Άσκηση 2
%
%  \hyperlink{Άσκηση5}{\beamerbutton{Πίσω στην άσκηση}}
% \end{frame}
%
% \begin{askisi}
%  \begin{enumerate}
%   \item Είναι γνησίως αύξουσα άρα $(f(+\infty),f(-\infty))$
%   \item Προφανώς $[f(0),f(1)]$...
%  \end{enumerate}
%
%  \hyperlink{Άσκηση6}{\beamerbutton{Πίσω στην άσκηση}}
% \end{frame}

\end{document}
