\documentclass{../presentation}

\title{Συναρτήσεις}
\subtitle{Κυρτότητα, Σημεία Καμπής}
\author[Λόλας]{Κωνσταντίνος Λόλας}
\institute[$10^ο$ ΓΕΛ]{$10^ο$ ΓΕΛ Θεσσαλονίκης}
\date{}

\begin{document}

\begin{frame}
  \titlepage
\end{frame}

\section{Θεωρία}
\begin{frame}{Γραφικές παραστάσεις λοιπόν!}
  Τι μπορούμε να "χαράξουμε"
  \begin{enumerate}
    \item Το πεδίο ορισμού
    \item Τα σημεία τομής με άξονες
    \item Τη συμμετρία ως προς $x'x$ ή $y'y$
    \item Τη συνέχεια
    \item Την παραγωγισιμότητα
    \item Τη μονοτονία
    \item Τα ακρότατα
  \end{enumerate}
  \onslide<2-> Έμειναν
  \begin{enumerate}
    \item<3-> Το πώς "ανέρχεται" ή "κατέρχεται" (κυρτότητα)
    \item<4-> Αν πλησιάζει προς ευθείες (ασύμπτωτες)
  \end{enumerate}
\end{frame}

\begin{frame}{Κυρτότητα}
  \begin{block}{Ορισμός}
    Ένα σχήμα λέγεται κυρτό, αν-ν κάθε ευθύγραμμο τμήμα με άκρα εσωτερικά σημεία του σχήματος, βρίσκεται εξολοκλήρου στο σχήμα
  \end{block}
  \onslide<2-> π.χ.
  \begin{itemize}
    \item<3-> Κυρτή γωνία \onslide<4->{$0<θ<180$}
    \item<5-> Κυρτή παραβολή \onslide<6->{$αx^2+βx+γ$ με $α>0$}
    \item<7-> Κυρτή συνάρτηση?
  \end{itemize}
\end{frame}

\begin{frame}{Κυρτότητα}
  \begin{block}{Συναρτήσεων}
    Έστω μια συνάρτηση $f$ συνεχής σε ένα διάστημα $Δ$ και παραγωγίσιμη στο εσωτερικό του $Δ$. Θα λέμε ότι:
    \begin{itemize}
      \item η συνάρτηση στρέφει τα \emph{κοίλα προς τα άνω} ή είναι \emph{κυρτή} στο $Δ$, αν η $f'$ είναι γνησίως αύξουσα στο εσωτερικό του $Δ$
      \item η συνάρτηση στρέφει τα \emph{κοίλα προς τα κάτω} ή είναι \emph{κοίλη} στο $Δ$, αν η $f'$ είναι γνησίως φθίνουσα στο εσωτερικό του $Δ$
    \end{itemize}
  \end{block}
\end{frame}

\begin{frame}{Και άρα στο εύκολο}
  \begin{block}{Κυρτότητα από $f''(x)$}
    Έστω μια συνάρτηση $f$ συνεχής σ’ ένα διάστημα $Δ$ και δυο φορές παραγωγίσιμη στο εσωτερικό του $Δ$
    \begin{itemize}
      \item Αν $f''(x)>0$ για κάθε εσωτερικό σημείο του $Δ$ τότε είναι \emph{κυρτή} στο $Δ$
      \item Αν $f''(x)<0$ για κάθε εσωτερικό σημείο του $Δ$ τότε είναι \emph{κοίλη} στο $Δ$
    \end{itemize}
  \end{block}
  \onslide<2-> Προσοχή, δεν είναι αυτός ο ορισμός. Ούτε ισχύει το αντίστροφο
\end{frame}

\begin{frame}{Και ένα σχόλιο}
  Φτιάξτε μία κυρτή συνάρτηση και μελετήστε ΚΑΘΕ εφαπτομένη της!
  \onslide<2->
  \begin{block}{Ανίσωση SOS}
    Αν μία συνάρτηση είναι κυρτή σ' ένα διάστημα $Δ$ τότε η εφαπτομένη της γραφικής παράστασης της $f$ σε κάθε σημείο του $Δ$ βρίσκεται "κάτω" από τη γραφική της παράσταση. Αντίστοιχα από "πάνω" όταν είναι κοίλη.
  \end{block}
  \onslide<3-> Η απόδειξη είναι πολύ εύκολη και γίνεται με ΘΜΤ ή με ακρότατα.
\end{frame}

\begin{frame}{Σημείο Καμπής}
  \begin{block}{Ορισμός}
    Έστω μία συνάρτηση $f$ παραγωγίσιμη σ' ένα διάστημα $(α,β)$ με εξαίρεση ίσως ένα σημείο του $x_0$. Αν
    \begin{itemize}
      \item η $f$ είναι κυρτή στο $(α,x_0)$ και κοίλη στο $(x_0,β)$ ή αντιστρόφως, και
      \item η $C_f$ έχει εφαπτόμενη στο σημείο $Α(x_0,f(x_0))$
    \end{itemize}
    τότε το σημείο $Α(x_0,f(x_0))$ ονομάζεται \emph{σημείο καμπής} της γραφικής παράστασης της $f$
  \end{block}
\end{frame}

\begin{frame}[label=Θεώρημα1]{Θεώρημα Σημείου Καμπής}
  \begin{block}{Ιδιότητα}
    Αν το $Α(x_0,f(x_0))$ είναι σημείο καμπής της $C_f$ και η $f$ είναι δύο φορές παραγωγίσιμη, τότε $f''(x_0)=0$.
  \end{block}

  \hyperlink{Απόδειξη1}{\beamerbutton{Απόδειξη}}
\end{frame}

\begin{frame}{Πού κρύβονται?}
  \begin{enumerate}
    \item<1-> $f''(x_0)=0$
    \item<2-> $f''(x_0)$ δεν ορίζεται
  \end{enumerate}
  \only<3-> {Πείτε το, αφού δεν κρατιέστε! Τα σημεία καμπής είναι τα ακρότατα της $f'$, χωρίς τα άκρα!}
\end{frame}

\section{Ασκήσεις}
\begin{askisi}
  Να μελετήσετε τις παρακάτω συναρτήσεις ως προς την κυρτότητα
  \begin{enumerate}
    \item<1-> $f(x)=x^2-\ln x$
    \item<2-> $f(x)=\sqrt{x}-e^x$
    \item<3-> $f(x)=x^4-2x+1$
    \item<4-> $f(x)=x\ln x-e^{-x}$
  \end{enumerate}

  % \hyperlink{Λύση1}{\beamerbutton{Λύση}}
\end{askisi}

\begin{askisi}
  Δίνεται η συνάρτηση $f(x)=e^x-x$.
  \begin{enumerate}
    \item<1-> Να μελετήσετε τη συνάρτηση $f$ ως προς τη μονοτονία, τα ακρότατα και την κυρτότητα
    \item<2-> Να βρείτε τις οριακές τιμές της $f$ στα άκρα του πεδίου ορισμού της, να κάνετε τον πίνακα μεταβολών της $f$ και να σχεδιάσετε τη $C_f$
    \item<3-> Να λύσετε την εξίσωση $f(x)=συνx$
  \end{enumerate}

  % \hyperlink{Λύση2}{\beamerbutton{Λύση}}
\end{askisi}

\begin{askisi}
  Να βρείτε τα διαστήματα στα οποία οι παρακάτω συναρτήσεις είναι κυρτές ή κοίλες και να προσδιορίσετε (αν υπάρχουν) τα σημεία καμπής
  \begin{enumerate}
    \item<1-> $f(x)=x^3-3x^2+5$
    \item<2-> $f(x)=3x^5-5x^4$
  \end{enumerate}

  % \hyperlink{Λύση3}{\beamerbutton{Λύση}}
\end{askisi}

\begin{askisi}
  Να βρείτε τα διαστήματα στα οποία οι παρακάτω συναρτήσεις είναι κυρτές ή κοίλες και να προσδιορίσετε (αν υπάρχουν) τα σημεία καμπής
  \begin{enumerate}
    \item<1-> $f(x)=\dfrac{1}{x^2+1}$
    \item<2-> $f(x)=x+\dfrac{1}{x}$
  \end{enumerate}

  % \hyperlink{Λύση4}{\beamerbutton{Λύση}}
\end{askisi}

\begin{askisi}
  Να βρείτε τις θέσεις των σημείων καμπής των συναρτήσεων:
  \begin{enumerate}
    \item<1-> $f(x)=συνx-\dfrac{x^3}{3}+\dfrac{x^2}{2}-1$
    \item<2-> $f(x)=2x(\ln x-1)-\ln^2x$
  \end{enumerate}

  % \hyperlink{Λύση5}{\beamerbutton{Λύση}}
\end{askisi}

\begin{askisi}
  Να βρείτε τα διαστήματα στα οποία οι παρακάτω συναρτήσεις είναι κυρτές ή κοίλες και να προσδιορίσετε (αν υπάρχουν) τα σημεία καμπής
  \begin{enumerate}
    \item<1-> $f(x)=σφx$, $x\in (0,\pi)$
    \item<2-> $f(x)=εφx-x+2\ln (συνx)$, $x\in (-\dfrac{\pi}{2},\dfrac{\pi}{2})$
  \end{enumerate}

  % \hyperlink{Λύση6}{\beamerbutton{Λύση}}
\end{askisi}

\begin{askisi}
  Δίνεται η συνάρτηση $f(x)=x^3-3x$
  \begin{enumerate}
    \item<1-> Να μελετήσετε τη συνάρτηση $f$ ως προς τη μονοτονία, τα ακρότατα, την κυρτότητα και τα σημεία καμπής
    \item<2-> Να βρείτε τις οριακές τιμές της $f$ στα άκρα του διαστήματος του πεδίου ορισμού της, να κάνετε τον πίνακα μεταβολών της $f$ και με βάση τις απαντήσεις σας στα προηγούμενα ερωτήματα, να σχεδιάσετε τη γραφική παράσταση της $f$
  \end{enumerate}

  %\hyperlink{Λύση7}{\beamerbutton{Λύση}}
\end{askisi}

\begin{askisi}
  Δίνεται η συνάρτηση $f(x)=e^x+\ln x$. Να δείξετε ότι:
  \begin{enumerate}
    \item<1-> Η $C_f$ έχει μοναδικό σημείο καμπής το $Α(x_0,f(x_0))$
    \item<2-> $x_0<\dfrac{4}{5}$
  \end{enumerate}

  %\hyperlink{Λύση8}{\beamerbutton{Λύση}}
\end{askisi}

\begin{askisi}
  Δίνεται η συνάρτηση $f(x)=6e^x-x^3-4x^2$. Να δείξετε ότι η $f$ έχει ακριβώς δύο σημεία καμπής

  %\hyperlink{Λύση9}{\beamerbutton{Λύση}}
\end{askisi}

\begin{askisi}
  Δίνεται η συνάρτηση $$f(x)=\dfrac{x^4}{12}-\dfrac{α^2x^3}{3}+\dfrac{αx^2}{2}-3x+1$$
  Να βρείτε τις τιμές του $α\in\mathbb{R}$ για τις οποίες:
  \begin{enumerate}
    \item<1-> Η $f$ παρουσιάζει καμπή στο $x_0=1$
    \item<2-> Η $C_f$ έχει ακριβώς δύο σημεία καμπής
  \end{enumerate}

  %\hyperlink{Λύση10}{\beamerbutton{Λύση}}
\end{askisi}

\begin{askisi}
  Να αποδείξετε ότι για κάθε $α\in (-2,2)$ η συνάρτηση $f(x)=x^4-2αx^3+6x^2-1$ είναι κυρτή σε όλο το $\mathbb{R}$

  %\hyperlink{Λύση11}{\beamerbutton{Λύση}}
\end{askisi}

\begin{askisi}
  Έστω $f:\mathbb{R}\to\mathbb{R}$ μία συνάρτηση για την οποία ισχύει
  $$f''(x)+f(x)\ne 2f'(x)\text{, για κάθε } x\in\mathbb{R}$$
  Να δείξετε ότι η συνάρτηση $g(x)=e^{-x}f(x)$, $x\in\mathbb{R}$ δεν έχει σημεία καμπής.

  %\hyperlink{Λύση12}{\beamerbutton{Λύση}}
\end{askisi}

\begin{askisi}

  Έστω $f:(1,3)\to\mathbb{R}$ μία συνάρτηση, η οποία είναι δύο φορές παραγωγίσιμη και ισχύει:
  $$f^2(x)+xf(x)+x^2-3x+1=0\text{, για κάθε } x\in (1,3)$$
  Να δείξετε ότι η συνάρτηση $f$, δεν παρουσιάζει καμπή

  %\hyperlink{Λύση13}{\beamerbutton{Λύση}}
\end{askisi}


\appendix

\section{Αποδείξεις}
\begin{frame}[label=Απόδειξη1]{Απόδειξη σημείο καμπής}
  \onslide<1-> Έστω ότι η $f$ έχει σημείο καμπής στο $x_0$ με κυρτή αριστερά και κοίλη δεξιά του σημείου.

  Άρα $f'(x)< f'(x_0)$ για κάθε $x<x_0$ και $f'(x)<f'(x_0)$ για κάθε $x>x_0$

  \onslide<2-> Αφού $f'$ παραγωγίσιμη, θα υπάρχει το όριο

  $$f''(x_0)=\lim\limits_{x \to x_0^-}{ \dfrac{f'(x)-f'(x_0)}{x-x_0} }\ge 0$$

  \onslide<3-> όμοια
  $$f''(x_0)=\lim\limits_{x \to x_0^+}{ \dfrac{f'(x)-f'(x_0)}{x-x_0} } \le 0$$

  \onslide<4-> Άρα $f''(x_0)=0$ \hyperlink{Θεώρημα1}{\beamerbutton{Πίσω στη θεωρία}}
\end{frame}

\end{document}
