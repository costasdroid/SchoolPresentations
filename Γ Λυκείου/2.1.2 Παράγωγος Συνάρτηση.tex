\documentclass[greek]{beamer}
%\usepackage{fontspec}
\usepackage{amsmath,amsthm}
\usepackage{unicode-math}
\usepackage{xltxtra}
\usepackage{graphicx}
\usetheme{CambridgeUS}
\usecolortheme{seagull}
\usepackage{hyperref}
\usepackage{ulem}
\usepackage{xgreek}

\usepackage{pgfpages}
\usepackage{tikz}
\usepackage{tkz-tab}
%\setbeameroption{show notes on second screen}
%\setbeameroption{show only notes}

\setsansfont{Noto Serif}

\usepackage{multicol}

\usepackage{appendixnumberbeamer}

\setbeamercovered{transparent}
\beamertemplatenavigationsymbolsempty

\title{Συναρτήσεις}
\subtitle{Παράγωγος}
\author[Λόλας]{Κωνσταντίνος Λόλας}
\date{}

\begin{document}

\begin{frame}
 \titlepage
\end{frame}

\section{Θεωρία}
\begin{frame}{Μαγεία}
 Ξέρετε τι είναι η κλίση...
 \begin{itemize}
  \item<1-> ευθείας
  \item<2-> καμπύλης?
 \end{itemize}
\end{frame}

\begin{frame}{Τι θα μάθουμε}
 \begin{itemize}
  \item<1-> Τι είναι η κλίση μιας οποιαδήποτε συνάρτηση στο $x_0$ της (παράγωγος)
  \item<2-> Από κλίση στο $x_0$ θα πάμε στο $x\in D_f$ (παράγωγος συνάρτηση)
  \item<3-> Από παράγωγο συνάρτησης, μονοτονία και τα συναφή (ακρότατα, Σ.Τ.)
  \item<4-> Από παράγωγο παραγώγου, κυρτότητα
  \item<5-> Νέα θεωρήματα (Rolle, ΘΜΤ)
  \item<6-> Υπολογισμός ορίων που αφήσαμε πιο πίσω ($\infty \over \infty$, $0 \over 0$)
  \item<7-> Μελέτη συνάρτησης (Γραφικά)
  \item<8-> Το αγαπημένο μου \onslide<9> διαφορικές εξισώσεις
 \end{itemize}
\end{frame}

\begin{frame}{Κλίση σε σημείο = Παράγωγος}
 Ας παίξουμε
 \href{https://www.geogebra.org/m/v6g6qvfw}{Geogebra}
\end{frame}

\begin{frame}{Ορισμός}
 \begin{block}{Παράγωγος}
  Έστω μια συνάρτηση $f$. Λέμε ότι η $f$ είναι παραγωγίσιμη στο $x_0\in D_f$ και γράφουμε $f'(x_0)$ αν υπάρχει το όριο:
  $$f'(x_0)=\lim\limits_{x \to x_0}{ \frac{f(x)-f(x_0)}{x-x_0} }$$
 \end{block}

 \onslide<2>Με αντικατάσταση $x=x_0+h$
 \begin{block}{Άλλος τύπος}
  $$f'(x_0)=\lim\limits_{h \to 0}{ \frac{f(x_0+h)-f(x_0)}{h} }$$
 \end{block}
\end{frame}

\begin{frame}{Άμεσες απορίες - παρατηρήσεις}
 \begin{itemize}
  \item<1-> Τι σημαίνει λοιπόν υπάρχει $f'(x_0)$
  \item<2-> Πότε δεν θα υπάρχει?
  \item<3-> Γραφικά πώς θα είναι η συνάρτηση που είναι (δεν είναι) παραγωγίσιμη
  \item<4-> Πάλι όρια!
  \item<5-> Με την συνέχεια τι έγινε?
 \end{itemize}
\end{frame}

\begin{frame}{Θεώρημα}
 \begin{block}{Παράγωγος $\to$ Συνέχεια}
  Αν μία συνάρτηση είναι παραγωγίσιμη σε ένα σημείο, τότε είναι και συνεχής στο σημείο αυτό
 \end{block}
 \onslide<2-> Φτιάξτε συνάρτηση (γραφικά) που ενώ είναι συνεχής σε ένα σημείο, δεν είναι παραγωγίσιμη στο σημείο αυτό.

 \onslide<3-> Άρα Συνέχεια $\nrightarrow$ Παράγωγος
\end{frame}

\begin{frame}{Συμβολισμοί}
 \begin{itemize}
  \item<1-> Lagrange $f'(x)$
  \item<2-> Leibniz $\frac{df}{dx}$
  \item<3-> Euler $f_x(x)$
 \end{itemize}
\end{frame}

\section{Ασκήσεις}
\subsection{Άσκηση 1}
\begin{frame}[label=Άσκηση1]{Εξάσκηση 1}
 Να βρείτε την παράγωγο των παρακάτω συναρτήσεων στο $x_0$ εφόσον υπάρχει
 \begin{enumerate}
  \item<1-> $f(x)=1+ημx$, $x_0=0$
  \item<2-> $f(x)=\sqrt{x-1}$, $x_0=1$
 \end{enumerate}

 % \hyperlink{Λύση1}{\beamerbutton{Λύση}}
\end{frame}

\subsection{Άσκηση 2}
\begin{frame}[label=Άσκηση2]{Εξάσκηση 2}
 Να βρείτε την παράγωγο της συνάρτησης $f(x)=x+1-xημ|x|$, στο σημείο $x_0=0$.

 % \hyperlink{Λύση2}{\beamerbutton{Λύση}}
\end{frame}

\subsection{Άσκηση 3}
\begin{frame}[label=Άσκηση3]{Εξάσκηση 3}
 Να βρείτε την παράγωγο της συνάρτησης $f$ στο σημείο $x_0=0$, όταν
 \begin{enumerate}
  \item<1-> $\begin{cases}
     x^2,    & x<0    \\
     συνx-1, & x\ge 0
    \end{cases}$
  \item<2-> $\begin{cases}
     x^2ημ\frac{1}{x}, & x\ne 0 \\
     0,                & x= 0
    \end{cases}$
 \end{enumerate}

 % \hyperlink{Λύση3}{\beamerbutton{Λύση}}
\end{frame}

\subsection{Άσκηση 4}
\begin{frame}[label=Άσκηση4]{Εξάσκηση 4}
 Αν $x+1\le f(x) \le x^2+x+1$ για κάθε $x\in\mathbb{R}$, να βρείτε την
 $$\frac{df(0)}{dx}$$

 % \hyperlink{Λύση4}{\beamerbutton{Λύση}}
\end{frame}

\subsection{Άσκηση 5}
\begin{frame}[label=Άσκηση5]{Εξάσκηση 5}
 Αν για μια συνάρτηση $f:\mathbb{R}\to\mathbb{R}$ ισχύει
 $$f(3+h)=2+h^2+ημh \text{, για κάθε } h\in\mathbb{R}$$
 Να αποδείξετε ότι $f(3)=2$ και να βρείτε την $f'(3)$.

 % \hyperlink{Λύση5}{\beamerbutton{Λύση}}
\end{frame}

\subsection{Άσκηση 6}
\begin{frame}[label=Άσκηση6]{Εξάσκηση 6}
 Αν η συνάρτηση $f$ είναι συνεχής στο $0$, να αποδείξετε ότι η συνάρτηση $g(x)=f(x)ημ^2x$ είναι παραγωγίσιμη στο $0$.

 % \hyperlink{Λύση6}{\beamerbutton{Λύση}}
\end{frame}

\subsection{Άσκηση 7}
\begin{frame}{Εξάσκηση 7}
 Αφού μελετήσετε ως προς τη συνέχεις στο $x_0$ τις παρακάτω συναρτήσεις, να εξετάσετε αν είναι παραγωγίσιμες στο σημείο αυτό.
 \begin{enumerate}
  \item<1-> $f(x)=\begin{cases}
     e^x, & x<0    \\
     x^2, & x\ge 0
    \end{cases}$, αν $x_0=0$
  \item<2-> $f(x)=|x-1|+3x-2$, αν $x_0=1$
 \end{enumerate}
\end{frame}

\subsection{Άσκηση 8}
\begin{frame}{Εξάσκηση 8}
 Να βρείτε τις τιμές των $α$ και $β$, για τις οποίες η συνάρτηση
 $f(x)=\begin{cases}
   αx^3+1, & x\le 1 \\
   βx+3,   & x >1
  \end{cases}$, είναι παραγωγίσιμη στο $x_0=1$
\end{frame}

\subsection{Άσκηση 9}
\begin{frame}[label=Άσκηση9]{Εξάσκηση 9}
 Έστω η συνάρτηση $f$ με $f(1)=2$ και $f'(1)=-1$. Να βρείτε τα όρια:
 \begin{enumerate}
  \item<1-> $\lim\limits_{x \to 1}{ \frac{f(x)-2x}{x^2-x}}$
  \item<2-> $\lim\limits_{x \to 1}{ \frac{f^2(x)-2f(x)}{x^2-1}}$
  \item<3-> $\lim\limits_{x \to 1}{ \frac{xf(x)-2}{x-1}}$
 \end{enumerate}

 %\hyperlink{Λύση9}{\beamerbutton{Λύση}}
\end{frame}

\subsection{Άσκηση 10}
\begin{frame}[label=Άσκηση10]{Εξάσκηση 10}
 Έστω $f:\mathbb{R}\to\mathbb{R}$ μία συνάρτηση με $f(3)=0$ και $f'(3)=5$. Να βρείτε το $\lim\limits_{x \to 2}{ \frac{f(2x-1)}{x-2} }$

 %\hyperlink{Λύση10}{\beamerbutton{Λύση}}
\end{frame}

\subsection{Άσκηση 11}
\begin{frame}[label=Άσκηση11]{Εξάσκηση 11}
 Έστω μία συνάρτηση $f$ η οποία είναι παραγωγίσιμη στο $x_0=1$. Να αποδείξετε ότι:
 \begin{enumerate}
  \item<1-> $\lim\limits_{h \to 0}{ \frac{f(1+2h)-f(1)}{h} }=2f'(1)$
  \item<2-> $\lim\limits_{h \to 0}{ \frac{f(1+h)-f(1-h)}{h} }=2f'(1)$
  \item<3-> $\lim\limits_{x \to +\infty}{ xf\left( x+\frac{1}{X} \right) }=f'(1)$, αν $f(1)=0$
 \end{enumerate}

 %\hyperlink{Λύση11}{\beamerbutton{Λύση}}
\end{frame}

\subsection{Άσκηση 12}
\begin{frame}[label=Άσκηση12]{Εξάσκηση 12}
 Έστω $f:\mathbb{R}\to\mathbb{R}$ μία συνάρτηση η οποία είναι συνεχής στο $1$. Να βρείτε τις τιμές $f(1)$ και $f'(1)$, όταν:
 \begin{enumerate}
  \item<1-> $\lim\limits_{x \to 1}{ \frac{f(x)-2}{x-1} }=4$
  \item<2-> $\lim\limits_{h \to 0}{ \frac{f(1+2h)-2}{h} }=8$
  \item<3-> $\lim\limits_{x \to +\infty}{ \left[ xf\left( \frac{x+1}{x} \right)-2x   \right]  }$
 \end{enumerate}

 %\hyperlink{Λύση12}{\beamerbutton{Λύση}}
\end{frame}

\subsection{Άσκηση 13}
\begin{frame}[label=Άσκηση13]{Εξάσκηση 13}
 Έστω $f:\mathbb{R}\to\mathbb{R}$ μία συνάρτηση με
 $$f^3(x)+f(x)+1=x^3 \text{, } x\in\mathbb{R}$$
 Να δείξετε ότι:
 \begin{enumerate}
  \item<1-> Η $f$ είναι συνεχής στο $x_0=1$
  \item<2-> $f'(1)=3$
  \item<3-> $\lim\limits_{x \to 1}{ \frac{f(2x^2-x)}{x-1}  }$
 \end{enumerate}

 %\hyperlink{Λύση13}{\beamerbutton{Λύση}}
\end{frame}

\subsection{Άσκηση 14}
\begin{frame}[label=Άσκηση14]{Εξάσκηση 14}
 Έστω $f:\mathbb{R}\to\mathbb{R}$ μία συνάρτηση η οποία είναι παραγωγίσιμη στο $0$ με $f'(0)=1$ και ισχύει:
 $$f(x+y)=f(x)+f(y)+xy \text{, για κάθε } x,y\in\mathbb{R}$$
 Να αποδείξετε ότι η $f$ είναι παραγωγίσιμη σε κάθε $x_0\in\mathbb{R}$

 %\hyperlink{Λύση14}{\beamerbutton{Λύση}}
\end{frame}

\appendix
\section{Λύσεις Ασκήσεων}
\begin{frame}
 \tableofcontents
\end{frame}

\subsection{Άσκηση 1}
\begin{frame}[label=Λύση1]
 Με θεώρημα ενδιαμέσων τιμών. Η συνάρτηση είναι συνεχής στο $[10,11]$ με $f(10)=1024$ και $f(11)=2048$. Αφού $2023\in (1024,2048)$ υπάρχει $x_0$...

 \hyperlink{Άσκηση1}{\beamerbutton{Πίσω στην άσκηση}}
\end{frame}

\subsection{Άσκηση 2}
\begin{frame}[label=Λύση2]
 Με Bolzano ή με μέγιστης ελάχιστης τιμής και ΘΕΤ.

 \begin{gather*}
  f(3)<f(2)<f(1) \\
  3f(3)<f(1)+f(2)+f(3)<3f(1) \\
  f(3)<\frac{f(1)+f(2)+f(3)}{3}<f(1)
 \end{gather*}

 \hyperlink{Άσκηση2}{\beamerbutton{Πίσω στην άσκηση}}
\end{frame}

\subsection{Άσκηση 3}
\begin{frame}[label=Λύση3]
 Προφανές ελάχιστο στα $x_1=1$ και $x_2=3$. Ως συνεχής στο $[1,3]$ έχει σίγουρα ΚΑΙ μέγιστο στο $(1,3)$

 \hyperlink{Άσκηση3}{\beamerbutton{Πίσω στην άσκηση}}
\end{frame}

\subsection{Άσκηση 4}
\begin{frame}[label=Λύση4]
 Η συνάρτηση `απόστασης` $f(x)-x$ είναι ορισμένη στο κλειστό διάστημα και έχει σίγουρα μέγιστο

 \hyperlink{Άσκηση4}{\beamerbutton{Πίσω στην άσκηση}}
\end{frame}

\subsection{Άσκηση 5}
\begin{frame}[label=Λύση5]
 Όμοια με την Άσκηση 2

 \hyperlink{Άσκηση5}{\beamerbutton{Πίσω στην άσκηση}}
\end{frame}

\subsection{Άσκηση 6}
\begin{frame}[label=Λύση6]
 \begin{enumerate}
  \item Είναι γνησίως αύξουσα άρα $(f(+\infty),f(-\infty))$
  \item Προφανώς $[f(0),f(1)]$...
 \end{enumerate}

 \hyperlink{Άσκηση6}{\beamerbutton{Πίσω στην άσκηση}}
\end{frame}

\end{document}
