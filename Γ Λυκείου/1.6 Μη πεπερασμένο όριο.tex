\documentclass{presentation}

\title{Συναρτήσεις}
\subtitle{Μη πεπερασμένο όριο στο $x_0$}
\author[Λόλας]{Κωνσταντίνος Λόλας}
\institute[$10^ο$ ΓΕΛ]{$10^ο$ ΓΕΛ Θεσσαλονίκης}
\date{}

\begin{document}

\begin{frame}
  \titlepage
\end{frame}

\section{Θεωρία}
\begin{frame}{Στο άπειρο λοιπόν...}
  \centering
  \includegraphics[height=0.6\columnwidth]{images/buzz}
\end{frame}

\begin{frame}{Λάθος συλλογισμός}
  Το άπειρο ΔΕΝ ΕΙΝΑΙ ΑΡΙΘΜΟΣ! \pause
  \begin{block}{Ορισμός απείρου}
    Αν για κάθε $k\in\mathbb{R}$ μπορώ να βρώ $m\in Α$ ώστε $m>k$, τότε λέμε ότι το $Α$ έχει οσοδήποτε μεγάλους αριθμούς.
  \end{block} \pause
  άρα
  \begin{block}{Ορισμός μη πεπερασμένου ορίου}
    Έστω συνάρτηση $f:Α\to\mathbb{R}$. Αν για κάθε $k\in\mathbb{R}$ υπάρχει $x_0\in Α$ ώστε για κάθε $x$ σε κατάλληλη περιοχή γύρω από το $x_0$ να ισχύει $f(x)>k$
  \end{block}
\end{frame}

\begin{frame}{Ελληνικά!}
  \begin{block}{Ορισμός μη πεπερασμένου ορίου}
    Έστω συνάρτηση $f:Α\to\mathbb{R}$. Θα λέμε ότι τείνει στο άπειρο αν μεγαλώνει συνεχώς όταν πλησιάζουμε στο $x_0$. Τότε θα γράφουμε
    $$\lim\limits_{x \to x_0}{ f(x) }=+\infty$$
  \end{block} \pause
  ΜΟΝΟ ΕΓΩ θα επιτρέπεται να γράφω σκέτο $\infty$ και θα εννοώ $+\infty$ και εννοείται επειδή ξεχνάω!
\end{frame}

\begin{frame}{Το άλλο άπειρο?}
  \begin{block}{Ορισμός μη πεπερασμένου ορίου}
    Έστω συνάρτηση $f:Α\to\mathbb{R}$. Θα λέμε ότι τείνει στο μείον άπειρο αν μικραίνει συνεχώς όταν πλησιάζουμε στο $x_0$. Τότε θα γράφουμε
    $$\lim\limits_{x \to x_0}{ f(x) }=-\infty$$
  \end{block} \pause
  Αυτό δεν μπορώ να το παραβλέψω και αναγκαστικά το γράφω και εγώ!
\end{frame}

\begin{frame}{Πάμε στα γνωστά}
  Συναρτήσεις που πηγαίνουν στο $+\infty$. \pause Πάμε... \pause
  \begin{itemize}
    \item $\dfrac{1}{x}$
    \item $\dfrac{1}{x^2}$
    \item $\dfrac{1}{x^{2k}}$
    \item $\dfrac{1}{x^{2k+1}}$
    \item $\ln{x}$
    \item $εφ(x)$
  \end{itemize}
\end{frame}

\begin{frame}{Το άπειρο δεν είναι παιχνίδι (part 1)}
  Γρίφος time!
  \begin{itemize}
    \item Υπάρχει ένα ξενοδοχείο με άπειρα δωμάτια.
    \item Έρχεται ένας ταλαιπωρημένος οδηπόρος και ζητάει δωμάτιο!!!!!
    \item Ο ξενοδόχος του λέει ότι όλα τα δωμάτια είναι κατελημένα και δεν έχει ελεύθερο.
    \item Επειδή ο οδηπόρος είστε εσείς και κάνετε μαθηματικά με τον Λόλα, του δίνετε την λύση και τελικά παίρνετε το δωμάτιο 4.
    \item Προτείνετε μία λύση
  \end{itemize}
\end{frame}

\begin{frame}{Το άπειρο δεν είναι παιχνίδι (part 2)}
  Μπορώ πολύ εύκολα να αποδείξω ότι $1+2+3+4+\cdots = -\dfrac{1}{12}$

  \centering
  \includegraphics[height=0.4\columnwidth]{images/qiev6}
\end{frame}

\begin{frame}{Τι θα ήταν τα μαθηματικά χωρίς πράξεις}
  Μα, μα, μα... Είπαμε δεν είναι αριθμός!\pause, αλλά, αν μπορώ να μεγαλώνω συνέχεια και...
  \begin{itemize}
    \item προσθέσω έναν αριθμό? \pause
    \item αφαιρέσω έναν αριθμό? \pause
    \item πολλαπλασιάσω με αριθμό πάνω από 1? \pause
    \item διαιρέσω με αριθμό? \pause
    \item υψώσω σε δύναμη? \pause
    \item πολλαπλασιάσω με άλλο τόσο? \pause
    \item αφαιρέσω άλλο τόσο? \pause
    \item πολλαπλασιάσω με 0? \pause
    \item διαιρέσω με άλλο τόσο? \pause
  \end{itemize}
  Άρα προσοχή σε όσα δεν ορίζονται!
\end{frame}

\begin{frame}{Ας τα δούμε ΟΛΑ}
  όπου $a\in\mathbb{R}$ και δεν προκύπτει από όριο, δηλαδή είναι αριθμός
  \begin{multicols}{2}
    \begin{itemize}
      \item $\pm\infty + a=$ \pause $\pm\infty$ \pause
      \item $\pm\infty - a=$ \pause $\pm\infty$ \pause
      \item $\pm\infty \cdot a=$ \pause $\begin{cases} \pm\infty, & a>0 \\ \mp\infty, & a<0 \\ ?, & a=0\end{cases}$ \pause
      \item $\dfrac{\pm\infty}{a}=$ \pause $\begin{cases} \pm\infty, & a>0 \\ \mp\infty, & a<0\end{cases}$ \pause
      \item $\dfrac{a}{\pm\infty}=$ \pause $0$ \pause
      \item $(+\infty)^a=$ \pause $\begin{cases} +\infty, & a>0 \\ 0, & a<0 \\ ?, & a=0\end{cases}$ \pause
      \item $a^{+\infty}=$ \pause $\begin{cases} 0, & 0\le a <1 \\ ?, & a=1 \\ +\infty, & a>1\end{cases}$ \pause
      \item $a^{-\infty}=$ \pause $\begin{cases} +\infty, & 0\le a <1 \\ ?, & a=1 \\ 0, & a>1\end{cases}$ \pause
    \end{itemize}
  \end{multicols}
  Κρατάμε τα $\infty\cdot 0$, $\infty^0$, $1^{\pm\infty}$
\end{frame}

\begin{frame}{Και παιχνίδι με τα $\pm\infty$}
  \begin{multicols}{2}
    \begin{itemize}
      \item $+\infty + +\infty=$ \pause $+\infty$ \pause
      \item $-\infty + (-\infty)=$ \pause $-\infty$ \pause
      \item $+\infty + (-\infty)=$ \pause $?$ \pause
      \item $\pm\infty \cdot \pm\infty=$ \pause $\pm\infty$ \pause
      \item $\dfrac{\pm\infty}{\pm\infty}=$ \pause $?$ \pause
      \item $(+\infty)^{+\infty}=$ \pause $+\infty$ \pause
      \item $(+\infty)^{-\infty}=$ \pause $0$ \pause
    \end{itemize}
  \end{multicols}
  Κρατάμε τα $\infty\cdot 0$, $\infty^0$, $1^{\pm\infty}$, $+\infty + (-\infty)$, $\dfrac{\pm\infty}{\pm\infty}$, $\dfrac{0}{0}$
\end{frame}

\section{Ασκήσεις}
\begin{askisi}
  Στο σχήμα
  \href{https://www.geogebra.org/m/xzc6usbm}{\beamergotobutton{Geogebra}}
  φαίνεται η γραφική παράσταση μιας συνάρτησης $f(x)$. Να υπολογίσετε τα παρακάτω όρια (εφόσον υπάρχουν):
  \begin{itemize}
    \item $\lim\limits_{x \to 1}{ f(x) }$, $\lim\limits_{x \to 1}{ |f(x)| }$, $\lim\limits_{x \to 1}{ \sqrt{f(x)} }$ και $\lim\limits_{x \to 1}{ \dfrac{1}{f(x)} }$ \pause
    \item $\lim\limits_{x \to 0}{ f(x) }$, $\lim\limits_{x \to 0}{ |f(x)| }$ και $\lim\limits_{x \to 0}{ \dfrac{1}{f(x)} }$ \pause
    \item $\lim\limits_{x \to 3}{ f(x) }$, $\lim\limits_{x \to 3}{ |f(x)| }$ και $\lim\limits_{x \to 3}{ \dfrac{1}{f(x)} }$ \pause
    \item $\lim\limits_{x \to 4}{ \dfrac{1}{f(x)} }$, $\lim\limits_{x \to 6}{ \dfrac{1}{f(x)-3} }$ και $\lim\limits_{x \to 7}{ \dfrac{1}{f(x)} }$
  \end{itemize}
\end{askisi}

\begin{askisi}
  Να βρείτε τα όρια (αν υπάρχουν)
  \begin{enumerate}
    \item $\lim\limits_{x \to 3}{ \dfrac{1}{|x-3|} }$ \pause
    \item $\lim\limits_{x \to 1}{ \dfrac{x-3}{(x-1)^2} }$ \pause
    \item $\lim\limits_{x \to 2}{ \dfrac{2x+1}{x-2} }$ \pause
    \item $\lim\limits_{x \to 0}{ \dfrac{1+\sqrt{x}}{x}}$
  \end{enumerate}
\end{askisi}

\begin{askisi}
  Να βρείτε, (αν υπάρχει) το $\lim\limits_{x \to 1}{ \dfrac{3x+2}{x^2-1} }$
\end{askisi}

\begin{askisi}
  Για τις διάφορες τιμές του $λ$ να βρείτε το $\lim\limits_{x \to 2}{ \dfrac{x^2-λx+λ}{(x-2)^2} }$
\end{askisi}

\begin{askisi}
  Να βρείτε την τιμή του $α\in\mathbb{R}$ για την οποία το $\lim\limits_{x \to 1}{\dfrac{αx^2+x-2}{x^2-x}  }$ είναι πραγματικός αριθμός
\end{askisi}

\begin{askisi}
  Έστω μια συνάρτηση $f:(0,+\infty)\to\mathbb{R}$ για την οποία ισχύει:
  $$f(x)\le x-\dfrac{1}{x} \text{ για κάθε } x>0$$
  Να βρείτε τα όρια:
  \begin{enumerate}
    \item $\lim\limits_{x \to 0}{ f(x) }$ \pause
    \item $\lim\limits_{x \to 0}{ \dfrac{|f(x)-3|}{f^2(x)-3f(x)} }$
  \end{enumerate}
\end{askisi}

\begin{askisi}
  Αν για μια συνάρτηση ισχύει:
  $$|x-2|f(x)\ge x-1 \text{ για κάθε } x\ne 2$$
  Να βρείτε τα όρια:
  \begin{enumerate}
    \item $\lim\limits_{x \to 2}{ f(x) }$ \pause
    \item $\lim\limits_{x \to 2}{ f(x)ημ\dfrac{1}{f(x)} }$
  \end{enumerate}
\end{askisi}

\begin{askisi}
  Έστω $f:\mathbb{R}\to\mathbb{R}$ μια συνάρτηση, για την οποία ισχύει $\lim\limits_{x \to 0}{ \left( x^2f(x)  \right)  }=1$.
  Να βρείτε τα όρια
  \begin{enumerate}
    \item $\lim\limits_{x \to 0}{ f(x) }$ \pause
    \item $\lim\limits_{x \to 0}{ \dfrac{x-1}{f(x)} }$
  \end{enumerate}
\end{askisi}

\begin{askisi}
  Να βρείτε (αν υπάρχουν) τα παρακάτω όρια.
  \begin{enumerate}
    \item $\lim\limits_{x \to 0}{ \dfrac{3x+2}{|ημx|-|x|} }$ \pause
    \item $\lim\limits_{x \to 0}{ \left( \dfrac{1}{|x|}-\dfrac{1}{x}  \right)  }$
  \end{enumerate}
\end{askisi}

\section{}
\begin{frame}
  Στο moodle θα βρείτε τις ασκήσεις που πρέπει να κάνετε, όπως και αυτή τη παρουσίαση
\end{frame}

\end{document}
