\documentclass{presentation}

\title{Συναρτήσεις}
\subtitle{Ασύμπτωτες}
\author[Λόλας]{Κωνσταντίνος Λόλας}
\institute[$10^ο$ ΓΕΛ]{$10^ο$ ΓΕΛ Θεσσαλονίκης}
\date{}

\begin{document}

\begin{frame}
    \titlepage
\end{frame}

\section{Θεωρία}
\begin{frame}{Ναι αλλά "καταλήγουμε" κάπου?}
    Σχεδόν τελειώσαμε την σχεδίαση. Έμεινε να δούμε, αν πλησιάζουμε σε ευθείες και πότε!
\end{frame}

\begin{frame}{Ζωγραφική 1 από 3}
    Φτιάξτε συνάρτηση που να τείνει να γίνει η ευθεία $x=1$

    \onslide<2-> Τι παρατηρείτε για την συνάρτηση όσο $x\to 1$?
\end{frame}

\begin{frame}{Κατακόρυφη ασύμπτωτη}
    \begin{block}{Ορισμός}
        Η $x=x_0$ είναι \emph{κατακόρυφη ασύμπτωτη} της $C_f$ αν ένα τουλάχιστον από τα όρια $\lim\limits_{x \to x_0^+}{ f(x) }$ ή $\lim\limits_{x \to x_0^-}{ f(x) }$ είναι $+\infty$ ή $-\infty$.
    \end{block}
\end{frame}

\begin{frame}{Ζωγραφική 2 από 3}
    Φτιάξτε συνάρτηση που \emph{δεξιά} να τείνει να γίνει η ευθεία $y=1$

    \onslide<2-> Τι παρατηρείτε για την συνάρτηση όσο $x\to +\infty$?
\end{frame}

\begin{frame}{Οριζόντια ασύμπτωτη}
    \begin{block}{Ορισμός}
        Η $y=a$ είναι \emph{οριζόντια ασύμπτωτη} της $C_f$ στο $+\infty$ αν $\lim\limits_{x \to +\infty}{ f(x) }=a$
    \end{block}
    και αντίστοιχα
    \begin{block}{Ορισμός}
        Η $y=a$ είναι \emph{οριζόντια ασύμπτωτη} της $C_f$ στο $-\infty$ αν $\lim\limits_{x \to -\infty}{ f(x) }=a$
    \end{block}
\end{frame}

\begin{frame}{Ζωγραφική 3 από 3}
    Φτιάξτε συνάρτηση που \emph{δεξιά} να τείνει να γίνει η ευθεία $y=2x+1$

    \onslide<2-> Προσπαθήστε να ορίσετε συνθήκη για να είναι μία ευθεία ασύμπτωτη της $f(x)$
\end{frame}

\begin{frame}{Πλάγια ασύμπτωτη}
    \begin{block}{Ορισμός}
        Η $y=ax+b$ είναι \emph{ασύμπτωτη} της $C_f$ στο $+\infty$ αν $\lim\limits_{x \to +\infty}{ [f(x)-(ax+b)] }=0$
    \end{block}
    και αντίστοιχα
    \begin{block}{Ορισμός}
        Η $y=ax+b$ είναι \emph{ασύμπτωτη} της $C_f$ στο $-\infty$ αν $\lim\limits_{x \to -\infty}{ [f(x)-(ax+b)] }=0$
    \end{block}
\end{frame}

\begin{frame}{Μην μπερδευτούμε μόνο}
    \begin{itemize}
        \item<1-> η ασύμπτωτη με $a=0$ ονομάζεται οριζόντια
        \item<2-> η ασύμπτωτη με $a\ne 0$ ονομάζεται πλάγια
        \item<3-> η ασύμπτωτη που δεν ορίζεται το $a$ ονομάζεται κατακόρυφη
    \end{itemize}
\end{frame}

\begin{frame}{Και λίγοι υπολογισμοί}
    Ξέροντας ότι
    $$\lim\limits_{x \to +\infty}{ [f(x)-(ax+b)] }=0$$
    να βρείτε τα $a$ και $b$.
    \begin{block}{Πλάγια ασύμπτωτη}<2->
        Η ευθεία $y=ax+b$ λέγεται πλάγια ασύμπτωτη της $C_f$ στο $+\infty$ αν και μόνο αν
        $$\lim\limits_{x \to +\infty}{ \dfrac{f(x)}{x} }=a\in\mathbb{R}$$
        και
        $$\lim\limits_{x \to +\infty}{ (f(x)-ax) }=b\in\mathbb{R}$$
    \end{block}
\end{frame}

\begin{frame}{Παρατηρήσεις}

    Για το βιβλίο σχόλια, για εμάς ασκήσεις

    \begin{itemize}
        \item<1-> Ποιό είναι τα μοναδικα πολυώνυμα που έχουν ασύμπτωτες και ποιές?
        \item<2-> Τι πρέπει να ισχύει για τις ρητές συναρτήσεις ώστε να έχουν πλάγιες ασύμπτωτες?
        \item<3-> Ποιές συναρτήσεις έχουν κατακόρυφες ασύμπτωτες?
        \item<4-> Πού ψάχνουμε κατακόρυφες ασύμπτωτες?
    \end{itemize}

\end{frame}

\section{Ασκήσεις}
\begin{askisi}
    Να βρείτε τις κατακόρυφες ασύμπτωτες των γραφικών παραστάσεων των παρακάτω συναρτήσεων
    \begin{enumerate}
        \item<1-> $f(x)=\dfrac{\ln x}{x}$
        \item<2-> $f(x)=\dfrac{x}{x-2}$
        \item<3-> $f(x)=εφ x$
    \end{enumerate}

    % \hyperlink{Λύση1}{\beamerbutton{Λύση}}
\end{askisi}

\begin{askisi}
    Να βρείτε τις οριζόντιες και τις κατακόρυφες ασύμπτωτες των γραφικών παραστάσεων των συναρτήσεων:
    \begin{enumerate}
        \item<1-> $f(x)=\dfrac{x}{x^2+1}$
        \item<2-> $f(x)=\dfrac{e^x}{1+e^x}$
        \item<3-> $f(x)=\dfrac{ημx}{x}$
        \item<4-> $f(x)=e^{\frac{1}{x}}$
    \end{enumerate}

    % \hyperlink{Λύση2}{\beamerbutton{Λύση}}
\end{askisi}

\begin{askisi}
    Δίνεται η συνάρτηση $f(x)=\sqrt{x^2+1}$
    \begin{enumerate}
        \item<1-> Να βρείτε στο $+\infty$ και στο $-\infty$ τις ασύμπτωτες $ε_1$ και $ε_2$ αντίστοιχα της $C_f$
        \item<2-> Να δείξετε ότι η $C_f$ βρίσκεται πάνω από την $ε_1$ κοντά στο $+\infty$ και πάνω από την $ε_2$ κοντά στο $-\infty$
    \end{enumerate}

    % \hyperlink{Λύση3}{\beamerbutton{Λύση}}
\end{askisi}

\begin{askisi}
    Έστω $f$, $g:\mathbb{R}\to\mathbb{R}$ δύο συναρτήσεις για τις οποίες ισχύει:
    $$g(x)=f(x)-2x+\dfrac{x}{x^2+1} \text{, } x\in\mathbb{R}$$
    και η ευθεία $y=3x-2$ η οποία είναι ασύμπτωτη της $C_f$ στο $+\infty$
    \begin{enumerate}
        \item<1-> Να βρείτε την ασύμπτωτη της $C_g$ στο $+\infty$
        \item<2-> Να βρείτε τις τιμές του $λ$, για τις οποίες ισχύει:
            $$\lim\limits_{x \to +\infty}{ \dfrac{xf(x)-3x^2+λx-1}{λf(x)-4x+5} }=1$$
    \end{enumerate}

    % \hyperlink{Λύση4}{\beamerbutton{Λύση}}
\end{askisi}

\begin{askisi}
    Να δείξετε ότι η ευθεία $y=x$ είναι πλάγια ασύμπτωτη της γραφικής παράστασης της συνάρτησης $f(x)=\dfrac{x^2-x+1}{x-1}$ στο $+\infty$

    % \hyperlink{Λύση5}{\beamerbutton{Λύση}}
\end{askisi}

\begin{askisi}
    Να βρείτε τις πλάγιες ή οριζόντιες ασύμπτωτες στο $+\infty$ των γραφικών παραστάσεων των παρακάτω συναρτήσεων
    \begin{enumerate}
        \item<1-> $f(x)=x-1+\dfrac{1}{x}$
        \item<2-> $f(x)=2+\dfrac{1}{x+1}$
    \end{enumerate}

    % \hyperlink{Λύση6}{\beamerbutton{Λύση}}
\end{askisi}

\begin{askisi}
    Έστω η συνάρτηση $f(x)=\dfrac{x^2+x+2a}{x-a^2}$. Να βρείτε τις τιμές του $α\in\mathbb{R}$, για τις οποίες η ευθεία $ε:x=1$ είναι ασύμπτωτη της $C_f$

    %\hyperlink{Λύση7}{\beamerbutton{Λύση}}
\end{askisi}

\begin{askisi}
    Δίνεται η συνάρτηση $f(x)=\dfrac{a^2x^n+5x+1}{x^2+1}$. Να βρείτε τις τιμές των $a\in\mathbb{R}^*$ και $n\in\mathbb{N}-{0,1}$ για τις οποίες η ευθεία $ε:y=1$ είναι οριζόντια ασύμπτωτη της $C_f$ στο $+\infty$

    %\hyperlink{Λύση8}{\beamerbutton{Λύση}}
\end{askisi}

\begin{askisi}
    Να βρείτε τις τιμές των $α$ και $β\in\mathbb{R}$, ώστε
    $$\lim\limits_{x \to +\infty}{ \left(   \dfrac{αx^2+βx+3}{x-1}-x \right)}=2$$

    %\hyperlink{Λύση9}{\beamerbutton{Λύση}}
\end{askisi}

\begin{frame}
    Στο moodle θα βρείτε τις ασκήσεις που πρέπει να κάνετε, όπως και αυτή τη παρουσίαση
\end{frame}


\appendix

\section{Αποδείξεις}
\begin{frame}[label=Απόδειξη1]{Απόδειξη σημείο καμπής}
    \onslide<1-> Έστω ότι η $f$ έχει σημείο καμπής στο $x_0$ με κυρτή αριστερά και κοίλη δεξιά του σημείου.

    Άρα $f'(x)< f'(x_0)$ για κάθε $x<x_0$ και $f'(x)<f'(x_0)$ για κάθε $x>x_0$

    \onslide<2-> Αφού $f'$ παραγωγίσιμη, θα υπάρχει το όριο

    $$f''(x_0)=\lim\limits_{x \to x_0^-}{ \dfrac{f'(x)-f'(x_0)}{x-x_0} }\ge 0$$

    \onslide<3-> όμοια
    $$f''(x_0)=\lim\limits_{x \to x_0^+}{ \dfrac{f'(x)-f'(x_0)}{x-x_0} } \le 0$$

    \onslide<4-> Άρα $f''(x_0)=0$ \hyperlink{Θεώρημα1}{\beamerbutton{Πίσω στη θεωρία}}
\end{frame}


% \section{Λύσεις Ασκήσεων}
% \begin{frame}
%  \tableofcontents
% \end{frame}
%
% \begin{askisi}
%  Με θεώρημα ενδιαμέσων τιμών. Η συνάρτηση είναι συνεχής στο $[10,11]$ με $f(10)=1024$ και $f(11)=2048$. Αφού $2023\in (1024,2048)$ υπάρχει $x_0$...
%
%  \hyperlink{Άσκηση1}{\beamerbutton{Πίσω στην άσκηση}}
% \end{frame}
%
% \begin{askisi}
%  Με Bolzano ή με μέγιστης ελάχιστης τιμής και ΘΕΤ.
%
%  \begin{gather*}
%   f(3)<f(2)<f(1) \\
%   3f(3)<f(1)+f(2)+f(3)<3f(1) \\
%   f(3)<\dfrac{f(1)+f(2)+f(3)}{3}<f(1)
%  \end{gather*}
%
%  \hyperlink{Άσκηση2}{\beamerbutton{Πίσω στην άσκηση}}
% \end{frame}
%
% \begin{askisi}
%  Προφανές ελάχιστο στα $x_1=1$ και $x_2=3$. Ως συνεχής στο $[1,3]$ έχει σίγουρα ΚΑΙ μέγιστο στο $(1,3)$
%
%  \hyperlink{Άσκηση3}{\beamerbutton{Πίσω στην άσκηση}}
% \end{frame}
%
% \begin{askisi}
%  Η συνάρτηση `απόστασης` $f(x)-x$ είναι ορισμένη στο κλειστό διάστημα και έχει σίγουρα μέγιστο
%
%  \hyperlink{Άσκηση4}{\beamerbutton{Πίσω στην άσκηση}}
% \end{frame}
%
% \begin{askisi}
%  Όμοια με την Άσκηση 2
%
%  \hyperlink{Άσκηση5}{\beamerbutton{Πίσω στην άσκηση}}
% \end{frame}
%
% \begin{askisi}
%  \begin{enumerate}
%   \item Είναι γνησίως αύξουσα άρα $(f(+\infty),f(-\infty))$
%   \item Προφανώς $[f(0),f(1)]$...
%  \end{enumerate}
%
%  \hyperlink{Άσκηση6}{\beamerbutton{Πίσω στην άσκηση}}
% \end{frame}

\end{document}
