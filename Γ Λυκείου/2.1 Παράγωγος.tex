\documentclass{presentation}

\title{Συναρτήσεις}
\subtitle{Παράγωγος}
\author[Λόλας]{Κωνσταντίνος Λόλας}
\institute[$10^ο$ ΓΕΛ]{$10^ο$ ΓΕΛ Θεσσαλονίκης}
\date{}

\begin{document}

\begin{frame}
    \titlepage
\end{frame}

\section{Θεωρία}
\begin{frame}{Μαγεία}
    Ξέρετε τι είναι η κλίση...
    \begin{itemize}
        \item<1-> ευθείας
        \item<2-> καμπύλης?
    \end{itemize}
\end{frame}

\begin{frame}{Τι θα μάθουμε}
    \begin{itemize}
        \item<1-> Τι είναι η κλίση μιας οποιαδήποτε συνάρτηση στο $x_0$ της (παράγωγος)
        \item<2-> Από κλίση στο $x_0$ θα πάμε στο $x\in D_f$ (παράγωγος συνάρτηση)
        \item<3-> Από παράγωγο συνάρτησης, μονοτονία και τα συναφή (ακρότατα, Σ.Τ.)
        \item<4-> Από παράγωγο παραγώγου, κυρτότητα
        \item<5-> Νέα θεωρήματα (Rolle, ΘΜΤ)
        \item<6-> Υπολογισμός ορίων που αφήσαμε πιο πίσω ($\dfrac{\infty}{\infty}$, $\dfrac{0}{0}$)
        \item<7-> Μελέτη συνάρτησης (Γραφικά)
        \item<8-> Το αγαπημένο μου \onslide<9> διαφορικές εξισώσεις
    \end{itemize}
\end{frame}

\begin{frame}{Κλίση σε σημείο = Παράγωγος}
    Ας παίξουμε
    \href{https://www.geogebra.org/m/v6g6qvfw}{Geogebra}
\end{frame}

\begin{frame}{Ορισμός}
    \begin{block}{Παράγωγος}
        Έστω μια συνάρτηση $f$. Λέμε ότι η $f$ είναι παραγωγίσιμη στο $x_0\in D_f$ και γράφουμε $f'(x_0)$ αν υπάρχει το όριο:
        $$f'(x_0)=\lim\limits_{x \to x_0}{ \frac{f(x)-f(x_0)}{x-x_0} }$$
    \end{block}

    \onslide<2>Με αντικατάσταση $x=x_0+h$
    \begin{block}{Άλλος τύπος}
        $$f'(x_0)=\lim\limits_{h \to 0}{ \frac{f(x_0+h)-f(x_0)}{h} }$$
    \end{block}
\end{frame}

\begin{frame}{Άμεσες απορίες - παρατηρήσεις}
    \begin{itemize}
        \item<1-> Τι σημαίνει λοιπόν υπάρχει $f'(x_0)$
        \item<2-> Πότε δεν θα υπάρχει?
        \item<3-> Γραφικά πώς θα είναι η συνάρτηση που είναι (δεν είναι) παραγωγίσιμη
        \item<4-> Πάλι όρια!
        \item<5-> Με την συνέχεια τι έγινε?
    \end{itemize}
\end{frame}

\begin{frame}{Θεώρημα}
    \begin{block}{Παράγωγος $\to$ Συνέχεια}
        Αν μία συνάρτηση είναι παραγωγίσιμη σε ένα σημείο, τότε είναι και συνεχής στο σημείο αυτό
    \end{block}
    \onslide<2->
    Έχω $f'(x_0)=\lim\limits_{x \to x_0}{ \dfrac{f(x)-f(x_0)}{x-x_0} }$. Θέτω $g(x)=\dfrac{f(x)-f(x_0)}{x-x_0}$
    $$f(x)=g(x)(x-x_0)+f(x_0)$$
    \begin{align*}
        \lim\limits_{x \to x_0}{ f(x) } & =\lim\limits_{x \to x_0}{ g(x)(x-x_0)+f(x_0) } \\
                                        & =f'(x_0)\cdot (x_0-x_0)+f(x_0)                 \\
                                        & =f(x_0)
    \end{align*}
\end{frame}

\begin{frame}{Θεώρημα}
    \begin{block}{Παράγωγος $\to$ Συνέχεια}
        Αν μία συνάρτηση είναι παραγωγίσιμη σε ένα σημείο, τότε είναι και συνεχής στο σημείο αυτό
    \end{block}
    \onslide<2-> Φτιάξτε συνάρτηση (γραφικά) που ενώ είναι συνεχής σε ένα σημείο, δεν είναι παραγωγίσιμη στο σημείο αυτό.

    \onslide<3-> Άρα Συνέχεια $\nrightarrow$ Παράγωγος
\end{frame}

\begin{frame}{Συμβολισμοί}
    \begin{itemize}
        \item<1-> Lagrange $f'(x)$
        \item<2-> Leibniz $\dfrac{df}{dx}$
        \item<3-> Euler $f_x(x)$
    \end{itemize}
\end{frame}


\begin{frame}[noframenumbering]
    Στο moodle θα βρείτε τις ασκήσεις που πρέπει να κάνετε, όπως και αυτή τη παρουσίαση
\end{frame}

\section{Ασκήσεις}

\begin{frame}[noframenumbering]
    \vfill
    \centering
    \begin{beamercolorbox}[sep=8pt,center,shadow=true,rounded=true]{title}
        \usebeamerfont{title}Ασκήσεις
    \end{beamercolorbox}
    \vfill
\end{frame}

\begin{askisi}
    Να βρείτε την παράγωγο των παρακάτω συναρτήσεων στο $x_0$ εφόσον υπάρχει
    \begin{enumerate}
        \item<1-> $f(x)=1+ημx$, $x_0=0$
        \item<2-> $f(x)=\sqrt{x-1}$, $x_0=1$
    \end{enumerate}

    % \hyperlink{Λύση1}{\beamerbutton{Λύση}}
\end{askisi}

\begin{askisi}
    Να βρείτε την παράγωγο της συνάρτησης $f(x)=x+1-xημ|x|$, στο σημείο $x_0=0$.

    % \hyperlink{Λύση2}{\beamerbutton{Λύση}}
\end{askisi}

\begin{askisi}
    Να βρείτε την παράγωγο της συνάρτησης $f$ στο σημείο $x_0=0$, όταν
    \begin{enumerate}
        \item<1-> $\begin{cases}
                      x^2,    & x<0    \\
                      συνx-1, & x\ge 0
                  \end{cases}$
        \item<2-> $\begin{cases}
                      x^2ημ\dfrac{1}{x}, & x\ne 0 \\
                      0,                 & x= 0
                  \end{cases}$
    \end{enumerate}

    % \hyperlink{Λύση3}{\beamerbutton{Λύση}}
\end{askisi}

\begin{askisi}
    Αν $x+1\le f(x) \le x^2+x+1$ για κάθε $x\in\mathbb{R}$, να βρείτε την
    $$\frac{df(0)}{dx}$$

    % \hyperlink{Λύση4}{\beamerbutton{Λύση}}
\end{askisi}

\begin{askisi}
    Αν για μια συνάρτηση $f:\mathbb{R}\to\mathbb{R}$ ισχύει
    $$f(3+h)=2+h^2+ημh \text{, για κάθε } h\in\mathbb{R}$$
    Να αποδείξετε ότι $f(3)=2$ και να βρείτε την $f'(3)$.

    % \hyperlink{Λύση5}{\beamerbutton{Λύση}}
\end{askisi}

\begin{askisi}
    Αν η συνάρτηση $f$ είναι συνεχής στο $0$, να αποδείξετε ότι η συνάρτηση $g(x)=f(x)ημ^2x$ είναι παραγωγίσιμη στο $0$.

    % \hyperlink{Λύση6}{\beamerbutton{Λύση}}
\end{askisi}

\begin{askisi}
    Αφού μελετήσετε ως προς τη συνέχεις στο $x_0$ τις παρακάτω συναρτήσεις, να εξετάσετε αν είναι παραγωγίσιμες στο σημείο αυτό.
    \begin{enumerate}
        \item<1-> $f(x)=\begin{cases}
                      e^x, & x<0    \\
                      x^2, & x\ge 0
                  \end{cases}$, αν $x_0=0$
        \item<2-> $f(x)=|x-1|+3x-2$, αν $x_0=1$
    \end{enumerate}
\end{askisi}

\begin{askisi}
    Να βρείτε τις τιμές των $α$ και $β$, για τις οποίες η συνάρτηση
    $f(x)=\begin{cases}
            αx^3+1, & x\le 1 \\
            βx+3,   & x >1
        \end{cases}$, είναι παραγωγίσιμη στο $x_0=1$
\end{askisi}

\begin{askisi}
    Έστω η συνάρτηση $f$ με $f(1)=2$ και $f'(1)=-1$. Να βρείτε τα όρια:
    \begin{enumerate}
        \item<1-> $\lim\limits_{x \to 1}{ \dfrac{f(x)-2x}{x^2-x}}$
        \item<2-> $\lim\limits_{x \to 1}{ \dfrac{f^2(x)-2f(x)}{x^2-1}}$
        \item<3-> $\lim\limits_{x \to 1}{ \dfrac{xf(x)-2}{x-1}}$
    \end{enumerate}

    %\hyperlink{Λύση9}{\beamerbutton{Λύση}}
\end{askisi}

\begin{askisi}
    Έστω $f:\mathbb{R}\to\mathbb{R}$ μία συνάρτηση με $f(3)=0$ και $f'(3)=5$. Να βρείτε το $\lim\limits_{x \to 2}{ \dfrac{f(2x-1)}{x-2} }$

    %\hyperlink{Λύση10}{\beamerbutton{Λύση}}
\end{askisi}

\begin{askisi}
    Έστω μία συνάρτηση $f$ η οποία είναι παραγωγίσιμη στο $x_0=1$. Να αποδείξετε ότι:
    \begin{enumerate}
        \item<1-> $\lim\limits_{h \to 0}{ \dfrac{f(1+2h)-f(1)}{h} }=2f'(1)$
        \item<2-> $\lim\limits_{h \to 0}{ \dfrac{f(1+h)-f(1-h)}{h} }=2f'(1)$
        \item<3-> $\lim\limits_{x \to +\infty}{ xf\left( 1+\dfrac{1}{x} \right) }=f'(1)$, αν $f(1)=0$
    \end{enumerate}

    %\hyperlink{Λύση11}{\beamerbutton{Λύση}}
\end{askisi}

\begin{askisi}
    Έστω $f:\mathbb{R}\to\mathbb{R}$ μία συνάρτηση η οποία είναι συνεχής στο $1$. Να βρείτε τις τιμές $f(1)$ και $f'(1)$, όταν:
    \begin{enumerate}
        \item<1-> $\lim\limits_{x \to 1}{ \dfrac{f(x)-2}{x-1} }=4$
        \item<2-> $\lim\limits_{h \to 0}{ \dfrac{f(1+2h)-2}{h} }=8$
        \item<3-> $\lim\limits_{x \to +\infty}{ \left[ xf\left( \dfrac{x+1}{x} \right)-2x   \right]=4  }$
    \end{enumerate}

    %\hyperlink{Λύση12}{\beamerbutton{Λύση}}
\end{askisi}

\begin{askisi}
    Έστω $f:\mathbb{R}\to\mathbb{R}$ μία συνάρτηση με
    $$f^3(x)+f(x)+1=x^3 \text{, } x\in\mathbb{R}$$
    Να δείξετε ότι:
    \begin{enumerate}
        \item<1-> Η $f$ είναι συνεχής στο $x_0=1$
        \item<2-> $f'(1)=3$
        \item<3-> $\lim\limits_{x \to 1}{ \dfrac{f(2x^2-x)}{x-1}  }$
    \end{enumerate}

    %\hyperlink{Λύση13}{\beamerbutton{Λύση}}
\end{askisi}

\begin{askisi}
    Έστω $f:\mathbb{R}\to\mathbb{R}$ μία συνάρτηση η οποία είναι παραγωγίσιμη στο $0$ με $f'(0)=1$ και ισχύει:
    $$f(x+y)=f(x)+f(y)+xy \text{, για κάθε } x,y\in\mathbb{R}$$
    Να αποδείξετε ότι η $f$ είναι παραγωγίσιμη σε κάθε $x_0\in\mathbb{R}$

    %\hyperlink{Λύση14}{\beamerbutton{Λύση}}
\end{askisi}

\end{document}
