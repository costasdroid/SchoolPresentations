\documentclass[greek]{beamer}
\usepackage{amsmath,amsthm} % needed for mathematics
\usepackage{unicode-math}
\usepackage{xltxtra}
\usepackage{graphicx}
\usetheme{CambridgeUS}
\usecolortheme{seagull}
\usepackage{hyperref}
\usepackage{ulem} % underline words package
\usepackage{xgreek}

\usepackage{pgfpages} 
\usepackage{tikz} % package for shapes and more
%\setbeameroption{show notes on second screen}
%\setbeameroption{show only notes}

\setsansfont{Calibri} % it is said that Calibri is the proper font for reading difficulties

\usepackage{multicol} % package for two or more columns

\usepackage{appendixnumberbeamer} % remove page numbering in appendix

\usepackage{polynom} % polynomial divisions package

\usepackage{pgffor} % macros

\setbeamercovered{highly dynamic}
\setbeamertemplate{navigation symbols}{}

\newcounter{askisi} % enviroment for exercises
\newenvironment{askisi}
{
  \refstepcounter{askisi}\par
  \subsection{Άσκηση \theaskisi}
  \begin{frame}[label=Άσκηση\theaskisi,t]{Εξάσκηση \theaskisi}
}{
  \end{frame}
}

\newcounter{lisi} % enviroment for solutions
\newenvironment{lisi}
{
  \refstepcounter{lisi}\par
  \subsection{Άσκηση \thelisi}
  \begin{frame}[label=Λύση\thelisi,t]{Λύση \thelisi}
}{
  \end{frame}
}

\title{Συναρτήσεις}
\subtitle{Συνέχεια Συνάρτησης}
\author[Λόλας]{Κωνσταντίνος Λόλας}
\institute[$10^ο$ ΓΕΛ]{$10^ο$ ΓΕΛ Θεσσαλονίκης}
\date{}

\begin{document}

\begin{frame}
  \titlepage
\end{frame}

\section{Θεωρία}
\begin{frame}{Όταν εμείς το υπολογίζαμε...}
  Μέχρι στιγμής πλησιάζαμε. Ήρθε ο καιρός να φτάσουμε!
\end{frame}

\begin{frame}{Συνέχεια 1}
  \begin{block}{Συνέχεια σε σημείο}
    Μία συνάρτηση είναι συνεχής στο $x_0$ αν $\lim\limits_{x \to x_0}{ f(x) }=f(x_0)$
  \end{block}
\end{frame}

\begin{frame}{Συνέχεια 2}
  \begin{block}{Συνέχεια σε διάστημα}
    Μία συνάρτηση είναι συνεχής στο $[α,β]$ όταν:
    \begin{itemize}
      \item $\lim\limits_{x \to x_0}{ f(x) }=f(x_0)$ για κάθε $x\in (α,β)$
      \item $\lim\limits_{x \to α^+}{ f(x) }=f(α)$
      \item $\lim\limits_{x \to β^-}{ f(x) }=f(β)$
    \end{itemize}
  \end{block}
\end{frame}

\begin{frame}{Συνέχεια 3}
  \begin{block}{Συνεχής συνάρτηση}
    Μία συνάρτηση είναι συνεχής όταν είναι συνεχής σε κάθε σημείο του πεδίου ορισμού της.
  \end{block}
\end{frame}

\begin{frame}{Ας γνωριστούμε}
  Γνωστές συνεχείς συναρτήσεις:
  \begin{itemize}[<+->]
    \item Πολυωνυμικές
    \item Εκθετικές
    \item Λογαριθμικές
    \item Τριγωνομετρικές
  \end{itemize}
\end{frame}

\begin{frame}{Και πράξεις αυτών}
  Αν $f$ και $g$ συνεχείς τότε συνεχής
  \begin{itemize}[<+->]
    \item $f+g$
    \item $f-g$
    \item $f\cdot g$
    \item $\frac{f}{g}$
    \item $f\circ g$
    \item ΟΛΕΣ ΟΙ ΓΝΩΣΤΕΣ
  \end{itemize}
\end{frame}

\begin{frame}{Το μέλλον...}
  \begin{itemize}[<+->]
    \item Αντί να υπολογίζουμε όρια, θα υπολογίζουμε τιμές
    \item Αν δεν μπορούμε να υπολογίζουμε τιμές, θα υπολογίζουμε όρια
    \item Αφού η συνάρτηση δεν "διακόπτεται" βγάζουμε ωραία θεωρήματα
  \end{itemize}
\end{frame}

\begin{frame}[noframenumbering]
  Στο moodle θα βρείτε τις ασκήσεις που πρέπει να κάνετε, όπως και αυτή τη παρουσίαση
\end{frame}

\section{Ασκήσεις}

\begin{frame}[noframenumbering]
  \vfill
  \centering
  \begin{beamercolorbox}[sep=8pt,center,shadow=true,rounded=true]{title}
    \usebeamerfont{title}Ασκήσεις
  \end{beamercolorbox}
  \vfill
\end{frame}

\begin{askisi}
  Να εξετάσετε, αν καθεμιά από τις παρακάτω συναρτήσεις είναι συνεχής στο $x_0$:
  \begin{enumerate}[<+->]
    \item $f(x)=\begin{cases}
              \frac{x^2-1}{x-1}, & x\ne 1 \\
              2,                 & x=1
            \end{cases}$, $x_0=1$
    \item $f(x)=\begin{cases}
              \frac{ημx}{x}, & x<0    \\
              2x+1,          & x\ge 0
            \end{cases}$, $x_0=0$
  \end{enumerate}

  %\hyperlink{Λύση1}{\beamerbutton{Λύση}}
\end{askisi}

\begin{askisi}
  Να μελετήσετε τη συνάρτηση $f(x)=e^x+\ln (x+1)$ ως προς τη συνέχεια και να βρείτε το $\lim\limits_{x \to 0}{ f(x) }$.


  %\hyperlink{Λύση2}{\beamerbutton{Λύση}}
\end{askisi}

\begin{askisi}
  Δίνεται η συνάρτηση $f(x)=\begin{cases}
      e^x+ημx,             & x<0 \\
      1,                   & x=0 \\
      συνx\cdot \ln (x+1), & x>0
    \end{cases}$
  \begin{enumerate}[<+->]
    \item Να μελετήσετε τη συνάρτηση $f$ ως προς τη συνέχεια.
    \item Να αποδείξετε ότι η $f$ είναι συνεχής στο διάστημα $[-π,0]$.
  \end{enumerate}

  %\hyperlink{Λύση3}{\beamerbutton{Λύση}}
\end{askisi}


\begin{askisi}
  Δίνεται η συνάρτηση $f(x)=\begin{cases}
      4αe^x+βσυνx, & x<0         \\
      x+2,         & 0\le x\le 1 \\
      \ln x+αx-β,  & x>1
    \end{cases}$

  Να βρείτε τις τιμές των $α$ και $β$ για τις οποίες η $f$ είναι συνεχής.

  %\hyperlink{Λύση4}{\beamerbutton{Λύση}}
\end{askisi}

\begin{askisi}
  Έστω $f:\mathbb{R}\to\mathbb{R}$ μία συνεχής συνάρτηση για την οποία ισχύει
  $$xf(x)=x^2+ημx \text{, για κάθε } x\in\mathbb{R}$$
  Να βρείτε τον τύπο της συνάρτησης $f$

  %\hyperlink{Λύση5}{\beamerbutton{Λύση}}
\end{askisi}

\begin{askisi}
  Έστω $f:\mathbb{R}\to\mathbb{R}$ μία συνάρτηση η οποία είναι συνεχής στο $x_0=1$. Αν $\lim\limits_{x \to 1}{ \frac{f(x)-2}{x-1} }=3$, να δείξετε ότι $f(1)=2$

  %\hyperlink{Λύση6}{\beamerbutton{Λύση}}
\end{askisi}

\begin{askisi}
  Έστω $f:(0,+\infty)\to\mathbb{R}$ μία συνάρτηση για την οποία ισχύει
  $$f^3(x)+f(x)=\ln x \text{, για κάθε } x>0$$
  Να δείξετε ότι η $f$ είναι συνεχής στο $x_0=1$

  %\hyperlink{Λύση7}{\beamerbutton{Λύση}}
\end{askisi}

\begin{askisi}
  Έστω $f:\mathbb{R}\to\mathbb{R}$ μία συνάρτηση για την οποία ισχύει
  $$2f(x)=x+ημf(x) \text{, για κάθε } x\in\mathbb{R}$$
  Να δείξετε ότι:
  \begin{enumerate}[<+->]
    \item $|f(x)|\le |x|$, για κάθε $x\in\mathbb{R}$
    \item Η $f$ είναι συνεχής στο $x_0=0$
  \end{enumerate}

  %\hyperlink{Λύση8}{\beamerbutton{Λύση}}
\end{askisi}

\begin{askisi}
  Έστω $f:(0,+\infty)\to\mathbb{R}$ μία συνάρτηση για την οποία ισχύει
  $$f(xy)=f(x)+f(y) \text{, για κάθε } x,y\in(0,+\infty)$$
  Να δείξετε ότι αν η $f$ είναι συνεχής στο $x=1$, τότε η συνάρτηση είναι συνεχής στο $(0,+\infty)$

  %\hyperlink{Λύση9}{\beamerbutton{Λύση}}
\end{askisi}

\end{document}
