\documentclass{presentation}

\title{Συναρτήσεις}
\subtitle{Όρια συνάρτησης στο άπειρο}
\author[Λόλας]{Κωνσταντίνος Λόλας}
\institute[$10^ο$ ΓΕΛ]{$10^ο$ ΓΕΛ Θεσσαλονίκης}

\begin{document}

\begin{frame}
  \titlepage
\end{frame}

\section{Θεωρία}
\begin{frame}{Όταν πάμε εμείς στο άπειρο λοιπόν!}
  \centering
  \includegraphics[height=0.6\columnwidth]{images/buzz}
\end{frame}

\begin{frame}{Στο προηγούμενο επεισόδιο...}
  \begin{enumerate}
    \item $\lim\limits_{x \to 0^+}{ \dfrac{1}{x} } = +\infty\implies$ \pause $\lim\limits_{x \to +\infty}{ \dfrac{1}{x} }=0$ \pause
    \item $\lim\limits_{x \to +\infty}{ x^n } = +\infty$ \pause
    \item $\lim\limits_{x \to +\infty}{ \dfrac{1}{x^n} } = 0$ \pause
    \item $\lim\limits_{x \to -\infty}{ x^n } = \begin{cases}
              +\infty & n \text{ άρτιος}   \\
              -\infty & n \text{ περιττός} \\
            \end{cases}$ \pause
    \item $\lim\limits_{x \to -\infty}{ \dfrac{1}{x^n} } = 0$ \pause
    \item με $a>1$, $\lim\limits_{x \to +\infty}{ a^x } = +\infty$ \pause
    \item με $a>1$, $\lim\limits_{x \to -\infty}{ a^x } = 0$ \pause
    \item με $0<a<1$, $\lim\limits_{x \to +\infty}{ a^x } = 0$ \pause
    \item με $0<a<1$, $\lim\limits_{x \to -\infty}{ a^x } = +\infty$ \pause
    \item με $\lim\limits_{x \to +\infty}{ lnx } = +\infty$
  \end{enumerate}
\end{frame}

\begin{frame}{Συμπέρασμα}
  Όπως και με τα κανονικά όρια:
  \begin{enumerate}
    \item $\lim\limits_{x \to \pm\infty}{ f(x) }=k\in\mathbb{R}$ ή
    \item $\lim\limits_{x \to \pm\infty}{ f(x) }=\pm\infty$ ή
    \item $\lim\limits_{x \to \pm\infty}{ f(x) }$ δεν θα υπάρχει
  \end{enumerate}
\end{frame}

\begin{frame}{Μόνο 2 περιπτώσεις}
  Ασχολούμαστε μόνο με
  \begin{enumerate}
    \item $\dfrac{\pm\infty}{\pm\infty}$
    \item $+\infty-\infty$
  \end{enumerate}
\end{frame}

\begin{frame}{Δεν είναι όλα τα άπειρα ίδια}
  \begin{itemize}
    \item $x$ vs $x^2$ \pause
    \item $x^2$ vs $x^5$ \pause
    \item Πολυώνυμο vs εκθετική \pause
    \item Πολυώνυμο vs λογαριθμική
  \end{itemize}
\end{frame}

\begin{frame}{Πιο άπειρο είναι μεγαλύτερο κάνει κουμάντο}
  \begin{itemize}
    \item Υπάρχει μεγαλύτερο? το βγάζω κοινό παράγοντα
    \item Είναι ίσα? κάνω πράξεις και τα διώχνω
  \end{itemize}
\end{frame}

\begin{frame}{Δύο έτοιμα όρια}
  Έστω $P(x)=a_nx^n+a_{n-1}x^{n-1}+\cdots +a_1x+a_0$ και $Q(x)=b_kx^k+b_{k-1}x^{k-1}+\cdots +b_1x+b_0$
  \begin{itemize}
    \item $\lim\limits_{x \to \pm\infty}{ P(x) }=\lim\limits_{x \to \pm\infty}{ a_nx^n }$
    \item $\lim\limits_{x \to \pm\infty}{ \dfrac{P(x)}{Q(x)} }=\lim\limits_{x \to \pm\infty}{ \dfrac{a_nx^n}{b_kx^k} }$
  \end{itemize}
\end{frame}

\section{Ασκήσεις}
\begin{askisi}
  Στο σχήμα
  \href{https://www.geogebra.org/m/p9xmedm8}{\beamergotobutton{Geogebra}}
  φαίνεται η γραφική παράσταση μιας συνάρτησης $f(x)$. Να υπολογίσετε τα παρακάτω όρια (εφόσον υπάρχουν):
  \begin{itemize}
    \item $\lim\limits_{x \to -\infty}{ f(x) }$ \pause
    \item $\lim\limits_{x \to +\infty}{ f(x) }$ \pause
    \item $\lim\limits_{x \to -\infty}{ \dfrac{1}{f(x)-1} }$ \pause
    \item $\lim\limits_{x \to +\infty}{ \dfrac{1}{f(x)} }$
  \end{itemize}
\end{askisi}

\begin{askisi}
  Να βρείτε τα όρια:
  \begin{itemize}
    \item $\lim\limits_{x \to +\infty}{ 3x^2-x-1 }$ \pause
    \item $\lim\limits_{x \to +\infty}{ -2x^2+3x-1 }$ \pause
    \item $\lim\limits_{x \to -\infty}{ -3x^2+5x-1 }$ \pause
    \item $\lim\limits_{x \to -\infty}{ -2x^3+1 }$
  \end{itemize}
\end{askisi}

\begin{askisi}
  Να βρείτε τα όρια:
  \begin{itemize}
    \item $\lim\limits_{x \to +\infty}{ \dfrac{2x^3-x+1}{3x^3+x^2+1} }$ \pause
    \item $\lim\limits_{x \to -\infty}{ \dfrac{x^2-x+2}{2x^3+x+1} }$ \pause
    \item $\lim\limits_{x \to +\infty}{ \left( \dfrac{x^2}{x-1}-x  \right)  }$
  \end{itemize}
\end{askisi}

\begin{askisi}
  Να βρείτε το όριο $\lim\limits_{x \to +\infty}{ \left( 2x-|x^3-x-1| \right)  }$
\end{askisi}

\begin{askisi}
  Να βρείτε τα όρια:
  \begin{itemize}
    \item $\lim\limits_{x \to +\infty}{ \sqrt{4x^2-2x+1} }$ \pause
    \item $\lim\limits_{x \to -\infty}{ \left( \sqrt{x^2+5} -x \right)  }$
  \end{itemize}
\end{askisi}

\begin{askisi}
  Να βρείτε τα όρια:
  \begin{itemize}
    \item $\lim\limits_{x \to +\infty}{ \left( \sqrt{4x^2+2x+1}-2x \right)  }$ \pause
    \item $\lim\limits_{x \to -\infty}{ \left( x+ \sqrt{x^2+1} \right)  }$
  \end{itemize}
\end{askisi}

\begin{askisi}
  Να βρείτε το όριο $\lim\limits_{x \to +\infty}{ \left( (a-1)x^3-2x+1 \right)  }$, για τις διάφορες τιμές του $a\in\mathbb{R}$
\end{askisi}

\begin{askisi}
  Να βρείτε τις τιμές του $μ\in\mathbb{R}$, για τις οποίες το $\lim\limits_{x \to +\infty}{ \dfrac{(μ-1)x^3+μx^2-2}{(μ-2)x^2+3x+1}  }$, είναι πραγματικός αριθμός
\end{askisi}

\begin{askisi}
  Για τις διάφορες πραγματικές τιμές του $μ$, να υπολογίσετε το $\lim\limits_{x \to -\infty}{ \left( \sqrt{4x^2+1}+μx \right)  }$
\end{askisi}

\begin{askisi}
  Δίνεται η συνάρτηση $f(x)=\dfrac{x^n+x-1}{x^2+1}$, $n\in\mathbb{N}^*$. Να βρείτε το $\lim\limits_{x \to +\infty}{ f(x) } $ για τις διάφορες τιμές του $n\in\mathbb{N}^*$.
\end{askisi}

\begin{askisi}
  Έστω $f:\mathbb{R}\to\mathbb{R}$ μια συνάρτηση, για την οποία ισχύει $\lim\limits_{x \to +\infty}{ \left( xf\left( \dfrac{x-1}{x} \right)  \right)  }=2$, να υπολογίσετε το $\lim\limits_{x \to 1}{ \dfrac{f(x)}{x-1} }$.
\end{askisi}

\begin{askisi}
  Έστω $f:\mathbb{R}\to\mathbb{R}$ μια συνάρτηση, για την οποία ισχύει $\lim\limits_{x \to 1}{ f(x)  }=-\infty$, να υπολογίσετε τα όρια
  \begin{enumerate}
    \item $\lim\limits_{x \to 1}{ \dfrac{2f^2(x)+f(x)-1}{f^3(x)-f(x)-2} }$ \pause
    \item $\lim\limits_{x \to 1}{ \left( \sqrt{f^2(x)+1}+f(x) \right)  }$
  \end{enumerate}
\end{askisi}

\begin{askisi}
  Έστω $f:(-\infty,0)\to\mathbb{R}$ μια συνάρτηση, για την οποία ισχύει $$\lim\limits_{x \to -\infty}{ \dfrac{xf(x)-2x+3}{x+2}  }=1$$
  \begin{enumerate}[<+->]
    \item να βρείτε τα όρια:
          \begin{enumerate}
            \item $\lim\limits_{x \to -\infty}{ f(x) }$
            \item $\lim\limits_{x \to -\infty}{ \dfrac{3x^2f(x)-x^2+1}{xf(x)+3}  }$
          \end{enumerate}
    \item Αν επιπλέον ισχύει $f\left( (-\infty,0) \right)=(3,+\infty) $, να βρείτε το $\lim\limits_{x \to -\infty}{ \dfrac{x}{f(x)-3} }$
  \end{enumerate}
\end{askisi}

\begin{askisi}
  Έστω $f:(0,+\infty)\to\mathbb{R}$ μια συνάρτηση, για την οποία ισχύουν
  $$\lim\limits_{x \to +\infty}{ \dfrac{f(x)}{x} }=5 \text{ και } \lim\limits_{x \to +\infty}{ (f(x)-5x) }=2$$
  Να βρείτε το $λ\in\mathbb{R}$, ώστε
  $$\lim\limits_{x \to +\infty}{ \dfrac{3f(x)+λx-2}{xf(x)-5x^2+1} }=3$$
\end{askisi}

\begin{askisi}
  Να βρείτε τα όρια:
  \begin{enumerate}
    \item $\lim\limits_{x \to +\infty}{ ημ\dfrac{2x-1}{x^2+1} }$ \pause
    \item $\lim\limits_{x \to +\infty}{ \dfrac{x}{x^2+1}συνx  }$
  \end{enumerate}
\end{askisi}

\begin{askisi}
  Να αποδείξετε ότι:
  \begin{enumerate}
    \item $\lim\limits_{x \to -\infty}{ xημ\dfrac{1}{x} }=1$ \pause
    \item $\lim\limits_{x \to +\infty}{ \dfrac{ημx}{x}  }=0$  \pause
    \item $\lim\limits_{x \to +\infty}{ ημx\cdot ημ\dfrac{1}{x}  }=0$  \pause
    \item $\lim\limits_{x \to +\infty}{ \dfrac{x-ημx}{x-1}  }=1$
  \end{enumerate}
\end{askisi}

\begin{askisi}
  Να βρείτε τα όρια:
  \begin{enumerate}
    \item $\lim\limits_{x \to +\infty}{ (x+ημx) }$ \pause
    \item $\lim\limits_{x \to +\infty}{ \dfrac{x}{2-ημx}  }$
  \end{enumerate}
\end{askisi}

\begin{askisi}
  Να βρείτε τα όρια:
  \begin{enumerate}
    \item $\lim\limits_{x \to +\infty}{ \dfrac{e^x-2^x+1}{3^x-5^x-2} }$ \pause
    \item $\lim\limits_{x \to -\infty}{ \dfrac{3^x-5^x}{3^x-2^x}  }$
  \end{enumerate}
\end{askisi}

\begin{askisi}
  Να βρείτε το $\lim\limits_{x \to +\infty}{ \dfrac{2^x-a^x}{2^x+3a^x}  }$, $a>0$
\end{askisi}

\begin{askisi}
  Να βρείτε τα όρια:
  \begin{enumerate}
    \item $\lim\limits_{x \to +\infty}{ e^{-x^2+1} }$ \pause
    \item $\lim\limits_{x \to 0^-}{ e^{-\dfrac{1}{x}} }$\pause
    \item $\lim\limits_{x \to 0}{ \dfrac{1}{e^{x^2}-1}}$
  \end{enumerate}
\end{askisi}

\begin{askisi}
  Να βρείτε τα όρια:
  \begin{enumerate}
    \item $\lim\limits_{x \to 0}{ \dfrac{1}{x}-\ln x }$ \pause
    \item $\lim\limits_{x \to 0}{ \dfrac{x}{\ln x} }$\pause
    \item $\lim\limits_{x \to 1}{ \dfrac{1+\sqrt{x-1}}{\ln x}}$\pause
    \item $\lim\limits_{x \to 0}{ \dfrac{\ln x}{ημx}}$
  \end{enumerate}
\end{askisi}

\begin{askisi}
  Να βρείτε τα όρια:
  \begin{enumerate}
    \item $\lim\limits_{x \to +\infty}{\left(    \ln x + e^{-\dfrac{1}{x}} \right)}$ \pause
    \item $\lim\limits_{x \to 1}{ \ln\dfrac{x}{x-1} }$\pause
    \item $\lim\limits_{x \to +\infty}{\left( \ln (1+e^{2x})-x \right)}$
  \end{enumerate}
\end{askisi}

\begin{askisi}
  Να βρείτε τα όρια:
  \begin{enumerate}
    \item $\lim\limits_{x \to +\infty}{\left( \ln x+συνx \right)}$ \pause
    \item $\lim\limits_{x \to +\infty}{ \dfrac{συνx}{\ln x} }$
  \end{enumerate}
\end{askisi}

\begin{askisi}
  Δίνεται η συνάρτηση $f(x)=\ln x + \sqrt{x-1}$ με σύνολο τιμών το $[0,+\infty)$
  \begin{enumerate}
    \item Να δείξετε ότι η $f$ αντιστρέφεται \pause
    \item Να βρείτε το $\lim\limits_{x \to +\infty}{ \left( x^2f^{-1}(x) \right)  }$
  \end{enumerate}
\end{askisi}

\begin{askisi}
  Δίνεται η συνάρτηση $f(x)=x^x$, $x>0$. Να βρείτε το $\lim\limits_{x \to +\infty}{ f(x)  }$
\end{askisi}

\section{}
\begin{frame}
  Στο moodle θα βρείτε τις ασκήσεις που πρέπει να κάνετε, όπως και αυτή τη παρουσίαση
\end{frame}

\end{document}
