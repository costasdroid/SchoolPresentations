\documentclass{presentation}

\title{Συναρτήσεις}
\subtitle{Αντίστροφη}
\author[Λόλας]{Κωνσταντίνος Λόλας }
\institute[$10^ο$ ΓΕΛ]{$10^ο$ ΓΕΛ Θεσσαλονίκης}
\date{}

\begin{document}

\begin{frame}
  \titlepage
\end{frame}

\section{Θεωρία}
\begin{frame}{Αντίστροφη}
  \begin{block}{Ορισμός}
    Έστω συνάρτηση  $f:Α\to B$ που είναι 1-1. Η αντίστροφή της $f^{-1}:Β\to Α$ ορίζεται η συνάρτηση που για κάθε $x\in f(Α)$ αντιστοιχεί ένα $y\in Α$ ώστε:
    $$f^{-1}(x)=y \iff f(y)=x$$
  \end{block} \pause
  Και επειδή συνήθως το $x$ αφορά το $D_f$
\end{frame}

\begin{frame}{Αντίστροφη}
  \begin{block}{Ορισμός}
    Έστω συνάρτηση  $f:Α\to B$ που είναι 1-1. Η αντίστροφή της $f^{-1}:Β\to Α$ ορίζεται η συνάρτηση που για κάθε $y\in f(Α)$ αντιστοιχεί ένα $x\in Α$ ώστε:
    $$f^{-1}(y)=x \iff f(x)=y$$
  \end{block}

  \centering
  \includegraphics[width=0.35\textwidth]{"images/1.3.4 Μονοτονία.png"}
\end{frame}

\begin{frame}{Μήπως τα έχετε ξαναδεί?}
  \begin{itemize}
    \item $f(x)=x+3$ \pause
    \item $f(x)=2x$ \pause
    \item $f(x)=\sqrt{x}$ \pause
    \item $f(x)=e^x$ \pause
    \item $f(x)=x^2$!!! \pause
    \item Πιο σύνθετες?
  \end{itemize}
\end{frame}

\begin{frame}{Ικανότητες?}
  Τι προσπαθούμε να κάνουμε? \pause
  \begin{itemize}
    \item Να βρίσκουμε από πού ήρθαμε, το $x$! \pause
    \item Σύνολο τιμών
  \end{itemize}
\end{frame}

\begin{frame}{Πώς θα το κάνουμε?}
  Έχετε $y$ και ζητάτε $x$ \pause
  \begin{itemize}
    \item $y=x+3$ \pause
    \item $y=2x$ \pause
    \item $y=\sqrt{x}$ \pause
    \item $y=e^x$ \pause
    \item $y=x^2$!!! \pause
    \item Πιο σύνθετες?
  \end{itemize}
\end{frame}

\begin{frame}{Μπορώ?}
  Σχεδιάστε γραφικά μια 1-1 συνάρτηση. \pause

  Ξέροντας ότι τα $x$ γίνονται $y$ σχηματίστε την $f^{-1}$ \pause

  Άρα:
  \begin{itemize}
    \item Η $f^{-1}$ είναι συμμετρική της $f$ ως προς την ευθεία $y=x$ \pause
    \item Αν η $f$ περνά από την ευθεία $y=x$ τότε και η $f^{-1}$ περνά, και αντίστροφα.
  \end{itemize} \pause
  \begin{alertblock}{Προσοχή στα spam}
    Σχεδιάστε συνάρτηση που δεν έχει μόνο στην $y=x$ κοινά σημεία με την αντίστροφή της.
  \end{alertblock}
\end{frame}

\begin{frame}{Βασική Ιδιότητα}
  \begin{exampleblock}{Κρατηθείτε!}
    \begin{itemize}
      \item $f\left(f^{-1}(x)\right)=x$, για κάθε $x\in f(D_f)$ \pause
      \item $f^{-1}\left(f(x)\right)=x$, για κάθε $x\in D_f$
    \end{itemize}
  \end{exampleblock}
\end{frame}

\section{Ασκήσεις}
\begin{frame}{Εξάσκηση}
  Δίνεται η συνάρτηση $f(x)=e^x-1$
  \begin{enumerate}
    \item Να δείξετε ότι είναι 1-1. \pause
    \item Να δείξετε ότι αντιστρέφεται και να βρείτε την $f^{-1}$
  \end{enumerate}
\end{frame}

\begin{frame}{Εξάσκηση}
  Δίνεται η συνάρτηση $f(x)=\frac{x+4}{x+1}$.
  \begin{enumerate}
    \item Να δείξετε ότι η $f$ είναι συνάρτηση 1-1 \pause
    \item Να βρείτε την $f^{-1}$
    \item Να βρείτε τα κοινά σημεία των $C_f$ και $C_{f^{-1}}$ με τον άξονα συμμετρίας τους.
  \end{enumerate}
\end{frame}

\begin{frame}{Εξάσκηση}
  Δίνεται η συνάρτηση $f(x)=\frac{e^x-1}{e^x+1}$.
  \begin{enumerate}
    \item Να δείξετε ότι η $f$ αντιστρέφεται\pause
    \item Να βρείτε την $f^{-1}$
  \end{enumerate}
\end{frame}

\begin{frame}{Εξάσκηση}
  Έστω $f:[1,+\infty)\to \mathbb{R}$ μία συνάρτηση με $f(x)=(x-1)^2+2$.
  \begin{enumerate}
    \item Να δείξετε ότι η $f$ αντιστρέφεται \pause
    \item Να βρείτε την αντίστροφη της $f$ \pause
    \item Να σχεδιάσετε τις $C_f$ και $C_{f^{-1}}$ στο ίδιο σύστημα αξόνων\pause
    \item Για κάθε $x\ge 1$ θεωρούμε τα σημεία $Α(x,f(x))$ και $Β(f(x),x)$ των $C_f$ και $C_{f^{-1}}$ αντίστοιχα. Να βρείτε την ελάχιστη απόσταση $d$ των σημείων $Α$ και $Β$.
  \end{enumerate}
\end{frame}

\begin{frame}{Εξάσκηση}
  Δίνεται η συνάρτηση $f(x)=x^3$. Να δείξετε ότι η $f$ αντιστρέφεται και να βρείτε την αντίστροφή της.
\end{frame}

\begin{frame}{Εξάσκηση}
  Δίνεται η συνάρτηση $f(x)=\begin{cases}
      \frac{1}{x}, & x<0    \\
      x^2,         & x\ge 0
    \end{cases}$.

  Να δείξετε ότι η $f$ αντιστρέφεται και να βρείτε την $f^{-1}$
\end{frame}

\begin{frame}{Εξάσκηση}
  Έστω $f:\mathbb{R}\to\mathbb{R}$ μία συνάρτηση, με $f(\mathbb{R})=\mathbb{R}$, η οποία ικανοποιεί την σχέση
  $$f^3(x)+f(x)-x-1=0\text{, για κάθε }x\in \mathbb{R}$$
  \begin{enumerate}
    \item Να δείξετε ότι η $f$ αντιστρέφεται και να βρείτε τη συνάρτηση $f^{-1}$ \pause
    \item Να βρείτε τα κοινά σημεία της $C_f$ και της ευθείας $y=x$
  \end{enumerate}
\end{frame}

\begin{frame}{Εξάσκηση}
  Δίνεται η συνάρτηση $f(x)=x^5+x$, με $f(\mathbb{R})=\mathbb{R}$
  \begin{enumerate}
    \item Να δείξετε ότι η $f$ αντιστρέφεται \pause
    \item Να βρείτε τα κοινά σημεία των $C_f$ και $C_{f^{-1}}$
  \end{enumerate}
\end{frame}

\begin{frame}{Εξάσκηση}
  Έστω $f:\mathbb{R}\to\mathbb{R}$ μία συνάρτηση, με $f(\mathbb{R})=\mathbb{R}$, η οποία είναι γνησίως φθίνουσα και $f(0)=1$, $f(1)=-2$.
  \begin{enumerate}
    \item Να δείξετε ότι η $f$ αντιστρέφεται \pause
    \item Να βρείτε τις ρίζες και το πρόσημο της $f^{-1}$ \pause
    \item Να λύσετε την εξίσωση $f\left(f^{-1}(3x+4)-f^{-1}(-2)\right)=1$ \pause
    \item Να λύσετε την ανίσωση $f^{-1}\left(3+f(\ln x)\right)>0$
  \end{enumerate}
\end{frame}

\begin{frame}{Εξάσκηση}
  Έστω $f:\mathbb{R}\to\mathbb{R}$ μία συνάρτηση, με $f(\mathbb{R})=\mathbb{R}$, η οποία είναι γνησίως αύξουσα.
  \begin{enumerate}
    \item Να δείξετε ότι η $f$ αντιστρέφεται και $f^{-1}\uparrow$\pause
    \item Αν η $f$ είναι περιττή, να αποδείξετε ότι και η $f^{-1}$ είναι περιττή \pause
    \item Αν ισχύει $f(x)>x$ για κάθε $x\in\mathbb{R}$, να δείξετε ότι
          $$f^{-1}(x)<x\text{, για κάθε }x\in\mathbb{R}$$
  \end{enumerate}
\end{frame}

\begin{frame}{Εξάσκηση}
  Έστω $f:\mathbb{R}\to\mathbb{R}$ μία συνάρτηση, με $f(\mathbb{R})=\mathbb{R}$, η οποία είναι γνησίως φθίνουσα και $f(0)=1$.
  \begin{enumerate}
    \item Να δείξετε ότι η $f$ αντιστρέφεται \pause
    \item Να λύσετε την ανίσωση $f(x)-f^{-1}(1-x)<x+1$
  \end{enumerate}
\end{frame}

\begin{frame}
  Στο moodle θα βρείτε τις ασκήσεις που πρέπει να κάνετε, όπως και αυτή τη παρουσίαση
\end{frame}

\end{document}
