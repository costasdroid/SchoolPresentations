\documentclass[greek]{beamer}
%\usepackage{fontspec}
\usepackage{amsmath,amsthm}
\usepackage{unicode-math}
\usepackage{xltxtra}
\usepackage{graphicx}
\usetheme{CambridgeUS}
\usecolortheme{seagull}
\usepackage{hyperref}
\usepackage{ulem}
\usepackage{xgreek}

\usepackage{pgfpages}
\usepackage{tikz}
\usepackage{tkz-tab}
%\setbeameroption{show notes on second screen}
%\setbeameroption{show only notes}

\setsansfont{Noto Serif}

\usepackage{multicol}

\usepackage{appendixnumberbeamer}

\setbeamercovered{transparent}
\beamertemplatenavigationsymbolsempty

\title{Συναρτήσεις}
\subtitle{Παράγωγος}
\author[Λόλας]{Κωνσταντίνος Λόλας}
\date{}

\begin{document}

\begin{frame}
  \titlepage
\end{frame}

\section{Θεωρία}
\begin{frame}{Αν είσαι τεμπέλης!}
  Γιατί να ψάχνουμε την κλίση σε κάθε σημείο ξεχωριστά? \pause

  Ας το βρούμε για όλα και να κάνουμε αντικατάσταση
\end{frame}

\begin{frame}{Συνάρτηση παράγωγος}
  \begin{block}{Παράγωγος}
    Έστω μια συνάρτηση $f$. Η συνάρτηση παράγωγος της $f$ θα είναι η συνάρτηση που απεικονίζει το $x_0$ στο $f'(x_0)$
  \end{block}
\end{frame}

\begin{frame}{Παράδειγμα}
  Ας παίξουμε:
  \begin{block}{$f(x)=c$}
    $c'=0$
  \end{block} \pause

  \begin{block}{$f(x)=x$}
    $x'=1$
  \end{block} \pause

  \begin{block}{$f(x)=x^2$}
    $(x^2)'=2x$
  \end{block}
\end{frame}

\begin{frame}{Αποδείξεις}
  \begin{block}{$f+g$}
    $(f(x)+g(x))'=f'(x)+g'(x)$
  \end{block} \pause

  \begin{block}{$f-g$}
    $(f(x)-g(x))'=f'(x)-g'(x)$
  \end{block} \pause

  \begin{block}{$f\cdot g$}
    $(f(x)\cdot g(x))'=f'(x)g(x)+f(x)g'(x)$
  \end{block} \pause

  \begin{block}{$f/g$}
    $(f(x)/g(x))'=\frac{f'(x)g(x)-f(x)g'(x)}{g^2(x)}$
  \end{block} \pause

  \begin{block}{$f(g)$}
    $(f(g(x)))'=f'(g(x))g'(x)$
  \end{block}
\end{frame}

\section{Ασκήσεις}
\subsection{Άσκηση 1}
\begin{frame}[label=Άσκηση1]{Εξάσκηση 1}
  Να βρείτε την παράγωγο της συνάρτησης $f$ στο $x_0$, όταν:
  \begin{enumerate}
    \item<1-> $f(x)=x^5$, $x_0=-1$
    \item<2-> $f(x)=συνx$, $x_0=\frac{3π}{4}$
  \end{enumerate}

  % \hyperlink{Λύση1}{\beamerbutton{Λύση}}
\end{frame}

\subsection{Άσκηση 2}
\begin{frame}[label=Άσκηση2]{Εξάσκηση 2}
  Να βρείτε την παράγωγο των συναρτήσεων:
  \begin{enumerate}
    \item<1-> $f(x)=e^x+x+συνx$
    \item<2-> $f(x)=\ln x+\sqrt{x}+α^3$
    \item<3-> $f(x)=x^3+ημx+\ln 2$
  \end{enumerate}

  % \hyperlink{Λύση2}{\beamerbutton{Λύση}}
\end{frame}

\subsection{Άσκηση 3}
\begin{frame}[label=Άσκηση3]{Εξάσκηση 3}
  Να βρείτε την παράγωγο των συναρτήσεων:
  \begin{enumerate}
    \item<1-> $f(x)=2\ln x$
    \item<2-> $f(x)=4x^3$
    \item<3-> $f(x)=-\frac{5}{3}x^3+\frac{1}{2}x^2+2x-3$
    \item<4-> $f(x)=\frac{3}{4}x^4-α\ln x-β$
    \item<5-> $f(x)=x^3(2x^2-5)$
    \item<6-> $f(x)=\ln \frac{e^x}{x}+\ln \frac{1}{x}+e^{\ln x}$
  \end{enumerate}

  % \hyperlink{Λύση3}{\beamerbutton{Λύση}}
\end{frame}

\subsection{Άσκηση 4}
\begin{frame}[label=Άσκηση4]{Εξάσκηση 4}
  Να βρείτε την παράγωγο των συναρτήσεων:
  \begin{enumerate}
    \item<1-> $f(x)=e^xσυνx$
    \item<2-> $f(x)=3x^2\ln x$
    \item<3-> $f(x)=(x^2+1)e^x$
    \item<4-> $f(x)=xe^xημx$
  \end{enumerate}

  % \hyperlink{Λύση4}{\beamerbutton{Λύση}}
\end{frame}

\subsection{Άσκηση 5}
\begin{frame}[label=Άσκηση5]{Εξάσκηση 5}
  Να βρείτε την παράγωγο της συνάρτησης $f(x)=\sqrt{x}ημx$

  % \hyperlink{Λύση5}{\beamerbutton{Λύση}}
\end{frame}

\subsection{Άσκηση 6}
\begin{frame}[label=Άσκηση6]{Εξάσκηση 6}
  Έστω $f$,$g:\mathbb{R}\to\mathbb{R}$ δύο συναρτήσεις οι οποίες είναι παραγωγίσιμες στο $0$ με $f(0)=g(0)=1$ και $f'(0)=2$, $g'(0)=3$.

  \begin{enumerate}
    \item<1-> Να βρείτε την $(f\cdot g)'(0)$
    \item<2-> Αν $h(x)=ημx \cdot f(x)$, $x\in\mathbb{R}$, να βρείτε την $h'(0)$
  \end{enumerate}

  % \hyperlink{Λύση6}{\beamerbutton{Λύση}}
\end{frame}

\subsection{Άσκηση 7}
\begin{frame}[label=Άσκηση7]{Εξάσκηση 7}
  Να βρείτε την παράγωγο των συναρτήσεων:
  \begin{enumerate}
    \item<1-> $\frac{\ln x}{x}$
    \item<2-> $\frac{x}{x^2+1}$
    \item<3-> $\frac{x}{e^x}$
    \item<4-> $\frac{ημx}{1+συνx}$
    \item<5-> $εφx-x$
  \end{enumerate}

  %\hyperlink{Λύση7}{\beamerbutton{Λύση}}
\end{frame}

\subsection{Άσκηση 8}
\begin{frame}[label=Άσκηση8]{Εξάσκηση 8}
  Να βρείτε την παράγωγο των συναρτήσεων:
  \begin{enumerate}
    \item<1-> $\frac{1}{x^2}$
    \item<2-> $\frac{1}{2\ln x}$
    \item<3-> $\frac{x^2+2x-3}{x}$
  \end{enumerate}

  %\hyperlink{Λύση8}{\beamerbutton{Λύση}}
\end{frame}

\subsection{Άσκηση 9}
\begin{frame}[label=Άσκηση9]{Εξάσκηση 9}
  Να βρείτε την παράγωγο των συναρτήσεων:
  \begin{enumerate}
    \item<1-> $ημ(2x-5)$
    \item<2-> $συν(2x)$
    \item<3-> $e^{-x}$
    \item<4-> $e^{\frac{1}{x}}$
    \item<5-> $2\sqrt{\ln x}$
  \end{enumerate}

  %\hyperlink{Λύση9}{\beamerbutton{Λύση}}
\end{frame}

\subsection{Άσκηση 10}
\begin{frame}[label=Άσκηση10]{Εξάσκηση 10}
  Να βρείτε την παράγωγο των συναρτήσεων:
  \begin{enumerate}
    \item<1-> $\ln \sqrt{x^2+1}$
    \item<2-> $\ln(\sqrt{x^2+1})-x$
  \end{enumerate}

  %\hyperlink{Λύση10}{\beamerbutton{Λύση}}
\end{frame}

\subsection{Άσκηση 11}
\begin{frame}[label=Άσκηση11]{Εξάσκηση 11}
  Να βρείτε την παράγωγο των συναρτήσεων:
  \begin{enumerate}
    \item<1-> $(x^2+2)^3$
    \item<2-> $ημ^3x$
    \item<3-> $\ln^2(x^2+2)$
    \item<4-> $ημ^23x$
  \end{enumerate}

  %\hyperlink{Λύση11}{\beamerbutton{Λύση}}
\end{frame}

\subsection{Άσκηση 12}
\begin{frame}[label=Άσκηση12]{Εξάσκηση 12}
  Να βρείτε την παράγωγο της συνάρτησης:
  $$f(x)=x^{\frac{1}{x}}$$

  %\hyperlink{Λύση12}{\beamerbutton{Λύση}}
\end{frame}

\subsection{Άσκηση 13}
\begin{frame}[label=Άσκηση13]{Εξάσκηση 13}
  Έστω $f:\mathbb{R}\to\mathbb{R}$ μία συνάρτηση που είναι παραγωγίσιμη. Να βρείτε την παράγωγο της συνάρτησης $g$ όταν:
  \begin{enumerate}
    \item<1-> $g(x)=f(x+ημx)$
    \item<2-> $g(x)=f^2(-x)$
  \end{enumerate}

  %\hyperlink{Λύση13}{\beamerbutton{Λύση}}
\end{frame}

\subsection{Άσκηση 14}
\begin{frame}[label=Άσκηση14]{Εξάσκηση 14}
  Να βρείτε την παράγωγο των συναρτήσεων:
  \begin{enumerate}
    \item<1-> $f(x)=x^{\frac{2}{3}}$
    \item<2-> $f(x)=\sqrt[4]{x^5}$
    \item<3-> $f(x)=\sqrt[3]{x^2}$
  \end{enumerate}

  %\hyperlink{Λύση14}{\beamerbutton{Λύση}}
\end{frame}

\subsection{Άσκηση 15}
\begin{frame}[label=Άσκηση15]{Εξάσκηση 15}
  Να βρείτε την δεύτερη παράγωγο των συναρτήσεων:
  \begin{enumerate}
    \item<1-> $f(x)=x^3+5x^2-3x+1$
    \item<2-> $f(x)=\frac{1}{x^2+1}$
  \end{enumerate}

  %\hyperlink{Λύση15}{\beamerbutton{Λύση}}
\end{frame}

\subsection{Άσκηση 16}
\begin{frame}[label=Άσκηση16]{Εξάσκηση 16}
  Δίνεται η συνάρτηση
  $f(x)=\begin{cases}
      x^3, & x\le 0 \\
      x^2, & x>0
    \end{cases}$

  Να βρείτε την $f''(x)$

  %\hyperlink{Λύση16}{\beamerbutton{Λύση}}
\end{frame}

\subsection{Άσκηση 17}
\begin{frame}[label=Άσκηση17]{Εξάσκηση 17}
  Έστω $x$, $y$, $θ:[0,+\infty)\to \mathbb{R}$ τρεις συναρτήσεις με μεταβλητή το χρόνο $t$, οι οποίες είναι παραγωγίσιμες. Να βρείτε τις παραγώγους των συναρτήσεων:
  \begin{enumerate}
    \item<1-> $f(t)=t^2+x(t)y(t)$
    \item<2-> $f(t)=\ln x(t) +x^2(t)$
    \item<3-> $f(t)=εφθ(t)$
  \end{enumerate}

  %\hyperlink{Λύση17}{\beamerbutton{Λύση}}
\end{frame}

\subsection{Άσκηση 18}
\begin{frame}[label=Άσκηση18]{Εξάσκηση 18}
  Αν η συνάρτηση $x(t)$ είναι παραγωγίσιμη στο $[0,+\infty)$ και ισχύουν $y(t)=x^2(t)$, $y'(t)=2x'(t)$ και $x'(t)>0$, για κάθε $t\ge 0$, να δείξετε ότι $x(t)=1$ για κάθε $t\ge 0$.

  %\hyperlink{Λύση18}{\beamerbutton{Λύση}}
\end{frame}

\subsection{Άσκηση 19}
\begin{frame}[label=Άσκηση19]{Εξάσκηση 19}
  Έστω οι παραγωγίσιμες συναρτήσεις $x$, $y:[0,+\infty)\to\mathbb{R}$ με μεταβλητή το χρόνο $t$, για τις οποίες ισχύει $y^2(t)=3+x^2(t)$, για κάθε $t\in [0,+\infty )$. Αν τη χρονική στιγμή $t_0=1$ είναι $x(1)=1$, $x'(1)=4$ και $y(1)>0$, να βρείτε το $y'(1)$.

  %\hyperlink{Λύση19}{\beamerbutton{Λύση}}
\end{frame}

\subsection{Άσκηση 20}
\begin{frame}[label=Άσκηση20]{Εξάσκηση 20}
  \begin{enumerate}
    \item<1->  Να βρείτε πολυώνυμο $f(x)$ δευτέρου βαθμού, για το οποίο ισχύουν $f(0)=1$, $f'(2)=7$ και $f''(2016)=6$
    \item<2-> Να βρείτε πολυώνυμο $P(x)$, για το οποίο ισχύουν: $P(0)=4$ και $8P(x)=\left( P'(x)\cdot P''(x) \right) $, για κάθε $x\in\mathbb{R}$
  \end{enumerate}

  %\hyperlink{Λύση20}{\beamerbutton{Λύση}}
\end{frame}

\subsection{Άσκηση 21}
\begin{frame}[label=Άσκηση21]{Εξάσκηση 21}
  Έστω $f:\mathbb{R}\to\mathbb{R}$ μία συνάρτηση. Αν η $f$ είναι παραγωγίσιμη, να δείξετε ότι:
  \begin{enumerate}
    \item<1-> $\lim\limits_{h \to 0}{ \frac{f(x+ah)-f(x)}{h} }=af'(x)$, $a\in\mathbb{R}^*$
    \item<2-> $\lim\limits_{h \to 0}{ \frac{f(x+h)-f(x-h)}{h} }=2f'(x)$
  \end{enumerate}

  %\hyperlink{Λύση21}{\beamerbutton{Λύση}}
\end{frame}

\subsection{Άσκηση 22}
\begin{frame}[label=Άσκηση22]{Εξάσκηση 22}
  Έστω $f:\mathbb{R}\to\mathbb{R}$ μία παραγωγίσιμη συνάρτηση, για την οποία ισχύει
  $$f(x)+e^{f(x)}=x \text{, για κάθε } x\in\mathbb{R}$$
  \begin{enumerate}
    \item<1-> Να δείξετε ότι η $f$ είναι δύο φορές παραγωγίσιμη
    \item<2-> Να δείξετε ότι $f'(x)<1$ για κάθε $x\in\mathbb{R}$
  \end{enumerate}

  %\hyperlink{Λύση22}{\beamerbutton{Λύση}}
\end{frame}

\subsection{Άσκηση 23}
\begin{frame}[label=Άσκηση23]{Εξάσκηση 23}
  Έστω $f:\mathbb{R}\to\mathbb{R}$ μία παραγωγίσιμη συνάρτηση, για την οποία ισχύουν $f'(0)=1$
  $$f(x)\cdot f'(-x)=1 \text{, για κάθε } x\in\mathbb{R}$$
  \begin{enumerate}
    \item<1-> Να δείξετε ότι η παράγωγος της συνάρτησης $f$ είναι συνεχής
    \item<2-> Να δείξετε ότι $f'(x)>0$ για κάθε $x\in\mathbb{R}$
    \item<3-> Αν $g(x)=f(x)\cdot f(-x)$, για κάθε $x\in\mathbb{R}$, να δείξετε ότι $g'(x)=x$, για κα´θε $x\in\mathbb{R}$
  \end{enumerate}

  %\hyperlink{Λύση23}{\beamerbutton{Λύση}}
\end{frame}

\subsection{Άσκηση 24}
\begin{frame}[label=Άσκηση24]{Εξάσκηση 24}
  Έστω $f:Δ\to\mathbb{R}$ μία συνάρτηση με $f(Δ)\subseteq Δ$, για την οποία ορίζεται η συνάρτηση $f^{-1}:f(Δ)\to\mathbb{R}$ με $f'(x)\ne 0$, $x\in Δ$.

  Αν θεωρήσουμε γνωστό ότι η $f^{-1}$ είναι παραγωγίσιμη στο $f(Δ)$, να δείξετε ότι:
  \begin{enumerate}
    \item<1-> $(f^{-1})'(x)=\frac{1}{f'(f^{-1}(x))}$
    \item<2-> $(f^{-1})'(f(x))=\frac{1}{f'(x)}$
  \end{enumerate}

  %\hyperlink{Λύση24}{\beamerbutton{Λύση}}
\end{frame}

\subsection{Άσκηση 25}
\begin{frame}[label=Άσκηση25]{Εξάσκηση 25}
  Δίνεται η συνάρτηση $f(x)=x^5+x^3$
  \begin{enumerate}
    \item<1-> Να βρείτε το σύνολο τιμών της $f$
    \item<2-> Να δείξετε ότι υπάρχει η συνάρτηση $f^{-1}$ και να βρείτε το πεδίο ορισμού της
    \item<3-> Να δείξετε ότι η $f^{-1}$ δεν παραγωγίζεται στο $x_0=0$
  \end{enumerate}

  %\hyperlink{Λύση25}{\beamerbutton{Λύση}}
\end{frame}

\subsection{Άσκηση 26}
\begin{frame}[label=Άσκηση26]{Εξάσκηση 26}
  Δίνεται η συνάρτηση $f(x)=e^x+x$, $x\in\mathbb{R}$
  \begin{enumerate}
    \item<1-> Να δείξετε ότι υπάρχει η συνάρτηση $f^{-1}$
    \item<2-> Αν θεωρήσουμε γνωστό ότι η $f^{-1}$ είναι παραγωγίσιμη στο $f(\mathbb{R})=\mathbb{R}$, να βρείτε την $(f^{-1})'(1)$
  \end{enumerate}

  %\hyperlink{Λύση26}{\beamerbutton{Λύση}}
\end{frame}

\subsection{Άσκηση 27}
\begin{frame}[label=Άσκηση27]{Εξάσκηση 27}
  Έστω $f:(0,+\infty)\to\mathbb{R}$ μία συνάρτηση που είναι παραγωγίσιμη και ισχύει

  $$f(x\cdot y)=yf(x)+xf(y) \text{, } x \text{, } y>0$$

  Να δείξετε ότι $f'(x)=\frac{f(x)}{x}+f'(1)$, $x>0$

  %\hyperlink{Λύση27}{\beamerbutton{Λύση}}
\end{frame}
%
% \appendix
% \section{Λύσεις Ασκήσεων}
% \begin{frame}
%  \tableofcontents
% \end{frame}
%
% \subsection{Άσκηση 1}
% \begin{frame}[label=Λύση1]
%  Με θεώρημα ενδιαμέσων τιμών. Η συνάρτηση είναι συνεχής στο $[10,11]$ με $f(10)=1024$ και $f(11)=2048$. Αφού $2023\in (1024,2048)$ υπάρχει $x_0$...
%
%  \hyperlink{Άσκηση1}{\beamerbutton{Πίσω στην άσκηση}}
% \end{frame}
%
% \subsection{Άσκηση 2}
% \begin{frame}[label=Λύση2]
%  Με Bolzano ή με μέγιστης ελάχιστης τιμής και ΘΕΤ.
%
%  \begin{gather*}
%   f(3)<f(2)<f(1) \\
%   3f(3)<f(1)+f(2)+f(3)<3f(1) \\
%   f(3)<\frac{f(1)+f(2)+f(3)}{3}<f(1)
%  \end{gather*}
%
%  \hyperlink{Άσκηση2}{\beamerbutton{Πίσω στην άσκηση}}
% \end{frame}
%
% \subsection{Άσκηση 3}
% \begin{frame}[label=Λύση3]
%  Προφανές ελάχιστο στα $x_1=1$ και $x_2=3$. Ως συνεχής στο $[1,3]$ έχει σίγουρα ΚΑΙ μέγιστο στο $(1,3)$
%
%  \hyperlink{Άσκηση3}{\beamerbutton{Πίσω στην άσκηση}}
% \end{frame}
%
% \subsection{Άσκηση 4}
% \begin{frame}[label=Λύση4]
%  Η συνάρτηση `απόστασης` $f(x)-x$ είναι ορισμένη στο κλειστό διάστημα και έχει σίγουρα μέγιστο
%
%  \hyperlink{Άσκηση4}{\beamerbutton{Πίσω στην άσκηση}}
% \end{frame}
%
% \subsection{Άσκηση 5}
% \begin{frame}[label=Λύση5]
%  Όμοια με την Άσκηση 2
%
%  \hyperlink{Άσκηση5}{\beamerbutton{Πίσω στην άσκηση}}
% \end{frame}
%
% \subsection{Άσκηση 6}
% \begin{frame}[label=Λύση6]
%  \begin{enumerate}
%   \item Είναι γνησίως αύξουσα άρα $(f(+\infty),f(-\infty))$
%   \item Προφανώς $[f(0),f(1)]$...
%  \end{enumerate}
%
%  \hyperlink{Άσκηση6}{\beamerbutton{Πίσω στην άσκηση}}
% \end{frame}

\end{document}
