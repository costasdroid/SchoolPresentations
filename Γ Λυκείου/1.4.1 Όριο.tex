\documentclass{presentation}

\title{Συναρτήσεις}
\subtitle{Όριο Συνάρτησης στο $x_0\in\mathbb{R}$}
\author[Λόλας]{Κωνσταντίνος Λόλας }
\institute[$10^ο$ ΓΕΛ]{$10^ο$ ΓΕΛ Θεσσαλονίκης}
\date{}

\begin{document}

\begin{frame}
  \titlepage
\end{frame}

\section{Θεωρία}
\begin{frame}{Όριο}
  \centering
  \includegraphics[width=0.6\columnwidth]{images/6r7k74}
\end{frame}

\begin{frame}{Όριο}
  \begin{block}{Το αστέρι μας}
    \centering
    $\lim\limits_{x \to x_0}{ f(x) }$
  \end{block}

  \onslide<2->Διαβάζεται ως:
  \begin{itemize}
    \item<2-> Το όριο της εφ όταν το χι τείνει στο χιμηδεν
    \item<3-> Το όριο της $f$ στο $x_0$
    \item<4-> Όταν το $x$ πάει στο $x_0$, πού πάει η $f$...
  \end{itemize}
\end{frame}

\begin{frame}{Ξεπηδούν οι απορίες}
  \begin{itemize}
    \item<1-> Τι σημαίνει πλησιάζω στο $x_0$
          \begin{itemize}
            \item<2-> Δημιουργήστε την γραμμή των πραγματικών αριθμών και πλησιάστε στο $x=2$
            \item<3-> Με πόσους τρόπους μπορείτε να πλησιάσετε
          \end{itemize}
    \item<4-> Τι σημαίνει η $f$ πλησιάζει στο $l$
    \item<5-> Τι σημαίνει οσοδήποτε κοντά
  \end{itemize}
\end{frame}

\begin{frame}{Ας γίνουμε νονοί}
  \begin{block}{Αριστερό πλευρικό όριο}
    \begin{center}
      $\lim\limits_{x \to x_0^-}{ f(x) }$
    \end{center}

    Για μια συνάρτηση που ορίζεται σε διάστημα της μορφής $(α,x_0)$ για κατάλληλο $α$
  \end{block}
\end{frame}

\begin{frame}{Ας γίνουμε νονοί}
  \begin{block}{Αριστερό πλευρικό όριο}
    \begin{center}
      $\lim\limits_{x \to x_0^-}{ f(x) }$
    \end{center}
  \end{block}
  \begin{block}{Δεξί πλευρικό όριο}
    \begin{center}
      $\lim\limits_{x \to x_0^+}{ f(x) }$
    \end{center}

    Για μια συνάρτηση που ορίζεται σε διάστημα της μορφής $(x_0,α)$ για κατάλληλο $α$
  \end{block}
\end{frame}

\begin{frame}{Άρα}
  \begin{block}{Ύπαρξη ορίου}
    \begin{equation*}
      \lim\limits_{x \to x_0}{ f(x) }=λ\iff
      \begin{cases}
        \lim\limits_{x \to x_0^-}{ f(x) }=λ\in\mathbb{R} \\
        \lim\limits_{x \to x_0^+}{ f(x) }=λ\in\mathbb{R} \\
        \lim\limits_{x \to x_0^-}{ f(x) }=\lim\limits_{x \to x_0^+}{ f(x) }
      \end{cases}
    \end{equation*}
  \end{block}
\end{frame}

\begin{frame}{Περιπτωσάρα}
  Αν $f(x)=\sqrt{x}$?, ή $f(x)=\ln (-x)$?\pause
  \begin{block}{ }
    Αν μια συνάρτηση ορίζεται μόνο σε διάστημα της μορφής $(α,x_0)$ τότε $\lim\limits_{x \to x_0}{ f(x) }=\lim\limits_{x \to x_0^-}{ f(x) }$
  \end{block}\pause
  Όμοια για $\lim\limits_{x \to x_0}{ f(x) }=\lim\limits_{x \to x_0^+}{ f(x) }$
\end{frame}

\begin{frame}{Ιδιότητες}
  \begin{itemize}
    \item Το όριο στην περίπτωση που υπάρχει είναι μοναδικό \pause
    \item $\lim\limits_{x \to x_0}{ f(x) }=k\iff\lim\limits_{x \to x_0}{ \left( f(x)-k \right) }=0$ \pause
    \item $\lim\limits_{x \to x_0}{ f(x) }=k\iff\lim\limits_{h \to 0}{ f(h+x_0) }=k$
  \end{itemize}
\end{frame}

\begin{frame}{Άρα τι θα κάνουμε?}
  \begin{itemize}
    \item Θα περιγράφουμε
    \item<2->Θα υπολογίζουμε (χωρίς να ξέρουμε γιατί)
    \item<3-> Θα χρησιμοποιούμε ιδιότητες και τεχνικές
    \item<4-> αλλά και πάλι δεν θα καταλαβαίνουμε
  \end{itemize}
  \onslide<5->Ουσιαστικά τα όρια θα τα υπολογίζουμε εντελώς μηχανικά
\end{frame}

\begin{frame}{Επίδειξη}
  Στο διάλλειμα όποιος θέλει μπορεί να μάθει τον υπέρτατο ορισμό του ορίου. Ιδού:
  \begin{block}{Ορισμός ορίου}
    Έστω μια συνάρτηση ορισμένη σε διάστημα της μορφής $(α,x_0)\cup (x_0,β)$. Λέμε ότι η συνάρτηση τείνει στο $λ\in\mathbb{R}$ καθώς το $x$ τείνει στο $x_0$ όταν:
    \newline
    \newline
    Για κάθε $\epsilon>0$ υπάρχει $δ>0$ ώστε, για κάθε $x\in (α,x_0)\cup (x_0,β)$ με $0<|x-x_0|<δ$ να ισχύει $|f(x)-λ|<\epsilon$
  \end{block}

  \begin{tikzpicture}[remember picture, overlay]
    \node[yshift=-2cm,opacity=0.7] at (current page.center) {\includegraphics[width=.8\textwidth]{images/6r7k74}};
  \end{tikzpicture}
\end{frame}

\begin{frame}[noframenumbering]
  Στο moodle θα βρείτε τις ασκήσεις που πρέπει να κάνετε, όπως και αυτή τη παρουσίαση
\end{frame}

\section{Ασκήσεις}

\begin{frame}[noframenumbering]
  \vfill
  \centering
  \begin{beamercolorbox}[sep=8pt,center,shadow=true,rounded=true]{title}
    \usebeamerfont{title}Ασκήσεις
  \end{beamercolorbox}
  \vfill
\end{frame}

\begin{askisi}
  Μόνο από το βιβλίο, μόνο γραφικά!
\end{askisi}

\section{}
\begin{frame}
  Στο moodle θα βρείτε τις ασκήσεις που πρέπει να κάνετε, όπως και αυτή τη παρουσίαση
\end{frame}

\end{document}
