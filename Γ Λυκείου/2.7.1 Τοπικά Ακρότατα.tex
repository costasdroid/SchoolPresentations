\documentclass{presentation}

\title{Συναρτήσεις}
\subtitle{Fermat, Κρίσιμα Σημεία}
\author[Λόλας]{Κωνσταντίνος Λόλας}
\institute[$10^ο$ ΓΕΛ]{$10^ο$ ΓΕΛ Θεσσαλονίκης}
\date{}

\begin{document}

\begin{frame}
    \titlepage
\end{frame}

\section{Θεωρία}
\begin{frame}{Λίγη Γεωγραφία?}
    \begin{enumerate}
        \item<1-> Το ψηλότερο σημείο στη γη
        \item<2-> Το ψηλότερο σημείο στην Ελλάδα
        \item<3-> Το ψηλότερο σημείο στη διαδρομή Θεσσαλονίκη - Γιάννινα
    \end{enumerate}
\end{frame}

\begin{frame}{Τοπικά Ακρότατα}
    \begin{block}{Ορισμός}
        Μία συνάρτηση $f$, με πεδίο ορισμού $Α$, θα λέμε ότι παρουσιάζει στο $x_0\in Α$ \emph{τοπικό μέγιστο}, όταν υπάρχει $δ>0$ ώστε
        $$f(x)\le f(x_0) \text{ για κάθε } x\in Α\cap (x_0-δ,x_0+δ)$$
        Το $x_0$ λέγεται \emph{θέση} ή \emph{σημείο τοπικού ακροτάτου}, ενώ το $f(x_0)$ \emph{τοπικό μέγιστο} της $f$
    \end{block}
    \onslide<2->Άρα \emph{ΣΤΟ} $x_0$, \emph{ΤΟ} $f(x_0)$
\end{frame}

\begin{frame}{Συγκρίσεις παντού}
    \begin{block}{Ορισμός}
        Μία συνάρτηση $f$, με πεδίο ορισμού $Α$, θα λέμε ότι παρουσιάζει στο $x_0\in Α$ \emph{μέγιστο}, όταν
        $$f(x)\le f(x_0) \text{ για κάθε } x\in Α$$
    \end{block}
    \begin{block}{Ορισμός}
        Μία συνάρτηση $f$, με πεδίο ορισμού $Α$, θα λέμε ότι παρουσιάζει στο $x_0\in Α$ \emph{τοπικό μέγιστο}, όταν υπάρχει $δ>0$ ώστε
        $$f(x)\le f(x_0) \text{ για κάθε } x\in Α\cap (x_0-δ,x_0+δ)$$
        Το $x_0$ λέγεται \emph{θέση} ή \emph{σημείο τοπικού ακροτάτου}, ενώ το $f(x_0)$ \emph{τοπικό μέγιστο} της $f$
    \end{block}
\end{frame}

\begin{frame}{Σιγά μην τα ξεκαθαρίσαμε}
    \begin{enumerate}
        \item<1-> Το μέγιστο είναι και τοπικό \only<2> {\emph{ΣΩΣΤΟ}}
        \item<3-> Το τοπικό μέγιστο είναι το μέγιστο \only<4> {\emph{ΛΑΘΟΣ!!!!!!!!!!}}
        \item<5-> Το μεγαλύτερο από τα τοπικά μέγιστα είναι το μέγιστο \only<6> {\emph{ΛΑΘΟΣ!!!!!!!!!!}}
        \item<7-> Αν δεν έχει μέγιστο, δεν έχει και τοπικά μέγιστα \only<8> {\emph{ΛΑΘΟΣ!!!!!!!!!!}}
        \item<9-> Υπάρχει συνάρτηση με άπειρα τοπικά μέγιστα \only<10> {\emph{ΣΩΣΤΟ}}
    \end{enumerate}
\end{frame}

\begin{frame}{Ζωγραφική}
    \begin{enumerate}
        \item<1-> Φτιάξτε συνάρτηση με τοπικό ελάχιστο στο δεξί άκρο ενός διαστήματος
        \item<2-> Συμπέρασμα για το $f'(x_0)$?
        \item<3-> Φτιάξτε συνάρτηση με τοπικό ελάχιστο στο εσωτερικό ενός διαστήματος
        \item<4-> Συμπέρασμα για το $f'(x_0)$?
        \item<5-> Φτιάξτε συνάρτηση με τοπικό ελάχιστο στο εσωτερικό ενός διαστήματος και να υπάρχει το $f'(x_0)$
        \item<6-> Συμπέρασμα για το $f'(x_0)$?
    \end{enumerate}
\end{frame}

\begin{frame}[label=Θεώρημα1]{Θεώρημα Fermat}
    \begin{block}{Ορισμός}
        Έστω μια συνάρτηση $f$ ορισμένη σ’ ένα διάστημα $Δ$ και $x_0$ ένα εσωτερικό σημείο του $Δ$. Αν η $f$ παρουσιάζει τοπικό ακρότατο στο $x_0$ και είναι παραγωγίσιμη στο σημείο αυτό, τότε:
        $f'(x_0) = 0$
    \end{block}

    \hyperlink{Απόδειξη1}{\beamerbutton{Απόδειξη}}
\end{frame}

\begin{frame}{Όλα μαζί}
    \begin{enumerate}
        \item<1-> Αν στο εσωτερικό δεν ισχύει $f'=0$ τότε δεν έχω τ.ακρότατο
        \item<2-> Αν στο εσωτερικό δεν υπάρχει $f'$ τότε \emph{μπορεί} να έχω
        \item<3-> Και μένουν τα άκρα (προσοχή, δεν είναι πάντα ακρότατα)
    \end{enumerate}
\end{frame}

\begin{frame}{Πιθανά Τοπικά Ακρότατα}
    \begin{enumerate}
        \item<1-> Αν στο εσωτερικό δεν ισχύει $f'=0$ τότε δεν έχω τ.ακρότατο
        \item<2-> Αν στο εσωτερικό δεν υπάρχει $f'$ τότε \emph{μπορεί} να έχω
        \item<3-> Και μένουν τα άκρα (προσοχή, δεν είναι πάντα ακρότατα)
    \end{enumerate}

    \onslide<4->Άρα
    \begin{block}{Πιθανές θέσεις ακροτάτων}
        \begin{itemize}
            \item Τα εσωτερικά που $f'=0$
            \item Τα εσωτερικά που δεν ορίζεται η $f'$
            \item Τα άκρα

        \end{itemize}
    \end{block}

    Κρίσιμα σημεία είναι οι 2 πρώτες περιπτώσεις

\end{frame}

\begin{frame}{Ναι, αλλά πότε τα ``πιθανά'' είναι και ``σίγουρα''}
    Να βρείτε συνθήκη για την $f$ ώστε ένα σημείο της να είναι τοπικό μέγιστο
    \begin{block}{Έλεγχος πιθανών ακροτάτων}<2->
        Έστω μια συνάρτηση $f$ παραγωγίσιμη σ’ ένα διάστημα $(α, β)$, με εξαίρεση ίσως ένα σημείο του $x_0$, στο οποίο όμως η $f$ είναι συνεχής.

        \begin{itemize}
            \item Αν $f'(x) > 0$ στο $(α, x_0)$ και $f'(x) < 0$ στο $(x_0, β)$, τότε το $f(x_0)$ είναι τοπικό μέγιστο της $f$
            \item Αν $f'(x)< 0$ στο $(α, x_0)$ και $f'(x) > 0$ στο $(x_0, β)$, τότε το $f(x_0)$ είναι τοπικό ελάχιστο της $f$
            \item Aν η $f'(x)$ διατηρεί πρόσημο στο $(α,x_0)\cup(β,x_0)$ τότε το $f(x_0)$ δεν είναι τοπικό ακρότατο και η $f$ είναι γνησίως μονότονη στο $(α, β)$

        \end{itemize}
    \end{block}

\end{frame}

\section{Ασκήσεις}
\begin{askisi}
    Έστω η συνάρτηση $f(x)=2α\ln x-\dfrac{β}{x}+3α$, όπου $α$, $β\in\mathbb{R}$. Αν η $f$ παρουσιάζει ακρότατο στο $1$ το $5$, να βρείτε τα $α$ και $β$

    % \hyperlink{Λύση1}{\beamerbutton{Λύση}}
\end{askisi}

\begin{askisi}
    Δίνεται η συνάρτηση $f(x)=e^x-αx$, για την οποία ισχύει
    $$f(x)\ge 1 \text{ για κάθε } x\in\mathbb{R}$$

    Να αποδείξετε ότι $α=1$

    % \hyperlink{Λύση2}{\beamerbutton{Λύση}}
\end{askisi}

\begin{askisi}
    Αν για κάθε $x>0$ ισχύει
    $$α\ln x\le x-1,α\in\mathbb{R}$$

    να βρείτε την τιμή του $α$

    % \hyperlink{Λύση3}{\beamerbutton{Λύση}}
\end{askisi}

\begin{askisi}
    Έστω $f:\mathbb{R}\to\mathbb{R}$ μία παραγωγίσιμη συνάρτηση με $f(0)=1$ και ισχύει
    $$f(x)\ge 2e^x-x-1 \text{ για κάθε } x\in\mathbb{R}$$
    \begin{enumerate}
        \item<1-> Να βρείτε την εφαπτομένη της $C_f$ στο $x_0=0$
        \item<2-> Να υπολογίσετε το $\lim\limits_{x \to +\infty}{ f(x) }$
    \end{enumerate}

    % \hyperlink{Λύση4}{\beamerbutton{Λύση}}
\end{askisi}

\begin{askisi}
    Έστω $f:\mathbb{R}\to\mathbb{R}$ μία παραγωγίσιμη συνάρτηση με $f(0)=1$ η οποία είναι δύο φορές παραγωγίσιμη και ισχύουν:
    \begin{itemize}
        \item $f(x)\ge 1$ για κάθε $x\in\mathbb{R}$
        \item $f''(x)>0 $ για κάθε $x\in\mathbb{R}$
    \end{itemize}
    Να μελετήσετε τη συνάρτηση $f$ ως προς τη μονοτονία


    % \hyperlink{Λύση5}{\beamerbutton{Λύση}}
\end{askisi}

\begin{askisi}
    Δίνεται η συνάρτηση $f(x)=\begin{cases}
            x^3     & ,-1\le x<1               \\
            (x-2)^2 & , 1\le x\le \dfrac{5}{2}
        \end{cases}$.
    Να βρείτε
    \begin{enumerate}
        \item<1-> Τα κρίσιμα σημεία της $f$
        \item<2-> Τις πιθανές θέσεις ακροτάτων της $f$
        \item<3-> Το σύνολο τιμών της $f$
    \end{enumerate}

    % \hyperlink{Λύση6}{\beamerbutton{Λύση}}
\end{askisi}

\begin{askisi}
    Έστω $f:\mathbb{R}\to\mathbb{R}$ μία συνάρτηση η οποία είναι παραγωγίσιμη και ισχύει:

    $$f^3(x)+3f(x)=x^3+x \text{ για κάθε } x\in\mathbb{R}$$
    Να δείξετε ότι η $f$ δεν έχει ακρότατα

    %\hyperlink{Λύση7}{\beamerbutton{Λύση}}
\end{askisi}

\begin{askisi}
    Έστω $f:\mathbb{R}\to\mathbb{R}$ μία συνάρτηση η οποία είναι παραγωγίσιμη με $f'(0)=1$ και ισχύει:

    $$f^3(x)+e^x=f(f(x))+x \text{ για κάθε } x\in\mathbb{R}$$
    Να δείξετε ότι η $f$ δεν έχει ακρότατα

    %\hyperlink{Λύση8}{\beamerbutton{Λύση}}
\end{askisi}

\begin{askisi}
    Έστω $f:\mathbb{R}\to\mathbb{R}$ μία συνάρτηση η οποία είναι παραγωγίσιμη με $f(1)=1$ η οποία είναι δύο φορές παραγωγίσιμη και ισχύουν:
    \begin{itemize}
        \item $f(x)\ge x$ για κάθε $x\in\mathbb{R}$
        \item $\left( f^2(x) \right)'\ne 0$ για κάθε $x\in\mathbb{R}$
    \end{itemize}

    \begin{enumerate}
        \item<1-> Να βρείτε την εφαπτομένη της $C_f$ στο $x_0=1$
        \item<2-> Να αποδείξετε ότι η $f$ δεν έχει ακρότατα και είναι γνησίως αύξουσα
        \item<3-> Να βρείτε το $\lim\limits_{x \to 0^+}{ f\left( \dfrac{1}{x} \right) }$
    \end{enumerate}

    %\hyperlink{Λύση9}{\beamerbutton{Λύση}}
\end{askisi}

\begin{askisi}
    Δίνεται η συνάρτηση $f(x)=|e^x+αx-1|$, $x\in\mathbb{R}$ η οποία είναι παραγωγίσιμη.
    \begin{enumerate}
        \item<1-> Να αποδείξετε ότι η $f$ παρουσιάζει ελάχιστο και στη συνέχεια ότι
            $$f'(0)=0$$
        \item<2-> Να βρείτε την τιμή του $α$ και να δείξετε ότι
            $$f(x)=e^x-x-1,x\in\mathbb{R}$$
        \item<3-> Αν η $f$ είναι ορισμένη στο $Β=[-1,1]$, να βρείτε το $f(Β)$
    \end{enumerate}

    %\hyperlink{Λύση10}{\beamerbutton{Λύση}}
\end{askisi}

\begin{askisi}
    Έστω $f:[0,2]\to\mathbb{R}$ μια συνάρτηση με $f(0)=1$, $f(1)=0$, $f(2)=3$ η οποία είναι παραγωγίσιμη. Αν $f'\uparrow (0,2)$, να δείξετε ότι υπάρχει μοναδικό $x_0\in (0,2)$ τέτοιο ώστε $f'(x_0)=0$

    %\hyperlink{Λύση11}{\beamerbutton{Λύση}}
\end{askisi}

\begin{askisi}
    Έστω $f,g:\mathbb{R}\to\mathbb{R}$ δύο συναρτήσεις παραγωγίσιμες που έχουν κοινά σημεία τα $(α,f(α))$ και $(β,f(β))$ και η $C_f$ είναι πάνω από τη $C_g$ στο διάστημα $(α,β)$. Να δείξετε ότι:
    \begin{enumerate}
        \item<1-> Υπάρχει $ξ\in (α,β)$, τέτοιο ώστε η κατακόρυφη απόσταση των σημείων με τετμημένη $ξ$ των $C_f$ και $C_g$, να γίνεται μέγιστη
        \item<2-> Οι εφαπτόμενες των $C_f$ και $C_g$ στα σημεία $(ξ,f(ξ))$ και $(ξ,g(ξ))$ είναι παράλληλες
    \end{enumerate}

    %\hyperlink{Λύση12}{\beamerbutton{Λύση}}
\end{askisi}


\appendix

\section{Αποδείξεις}
\begin{frame}[label=Απόδειξη1]{Απόδειξη Fermat}
    \onslide<1-> Έστω ότι η $f$ έχει τοπικό μέγιστο στο $x_0$.

    Άρα $f(x)\le f(x_0)$ για κάθε $x$ γύρω από το $x_0$.

    \onslide<2-> Αφού $f$ παραγωγίσιμη, θα υπάρχει το όριο

    $$f'(x_0)=\lim\limits_{x \to x_0}{ \dfrac{f(x)-f(x_0)}{x-x_0} }=k\in\mathbb{R}$$

    \onslide<3-> Για $x<x_0\implies \dfrac{f(x)-f(x_0)}{x-x_0} >0$ άρα

    $$\lim\limits_{x \to x_0^-}{ \dfrac{f(x)-f(x_0)}{x-x_0} }\ge 0$$

    \onslide<4-> Για $x>x_0\implies \dfrac{f(x)-f(x_0)}{x-x_0} <0$ άρα

    $$\lim\limits_{x \to x_0^+}{ \dfrac{f(x)-f(x_0)}{x-x_0} }\le 0$$

    \onslide<5-> Άρα $0\le k \le 0$, δηλαδή $f'(x_0)=0$ \hyperlink{Θεώρημα1}{\beamerbutton{Πίσω στη θεωρία}}
\end{frame}


% \section{Λύσεις Ασκήσεων}
% \begin{frame}
%  \tableofcontents
% \end{frame}
%
% \begin{askisi}
%  Με θεώρημα ενδιαμέσων τιμών. Η συνάρτηση είναι συνεχής στο $[10,11]$ με $f(10)=1024$ και $f(11)=2048$. Αφού $2023\in (1024,2048)$ υπάρχει $x_0$...
%
%  \hyperlink{Άσκηση1}{\beamerbutton{Πίσω στην άσκηση}}
% \end{frame}
%
% \begin{askisi}
%  Με Bolzano ή με μέγιστης ελάχιστης τιμής και ΘΕΤ.
%
%  \begin{gather*}
%   f(3)<f(2)<f(1) \\
%   3f(3)<f(1)+f(2)+f(3)<3f(1) \\
%   f(3)<\dfrac{f(1)+f(2)+f(3)}{3}<f(1)
%  \end{gather*}
%
%  \hyperlink{Άσκηση2}{\beamerbutton{Πίσω στην άσκηση}}
% \end{frame}
%
% \begin{askisi}
%  Προφανές ελάχιστο στα $x_1=1$ και $x_2=3$. Ως συνεχής στο $[1,3]$ έχει σίγουρα ΚΑΙ μέγιστο στο $(1,3)$
%
%  \hyperlink{Άσκηση3}{\beamerbutton{Πίσω στην άσκηση}}
% \end{frame}
%
% \begin{askisi}
%  Η συνάρτηση `απόστασης` $f(x)-x$ είναι ορισμένη στο κλειστό διάστημα και έχει σίγουρα μέγιστο
%
%  \hyperlink{Άσκηση4}{\beamerbutton{Πίσω στην άσκηση}}
% \end{frame}
%
% \begin{askisi}
%  Όμοια με την Άσκηση 2
%
%  \hyperlink{Άσκηση5}{\beamerbutton{Πίσω στην άσκηση}}
% \end{frame}
%
% \begin{askisi}
%  \begin{enumerate}
%   \item Είναι γνησίως αύξουσα άρα $(f(+\infty),f(-\infty))$
%   \item Προφανώς $[f(0),f(1)]$...
%  \end{enumerate}
%
%  \hyperlink{Άσκηση6}{\beamerbutton{Πίσω στην άσκηση}}
% \end{frame}

\end{document}
