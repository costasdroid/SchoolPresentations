\documentclass{../presentation}

\title{Συναρτήσεις}
\subtitle{Ολοκληρώματα και Ανισότητες}
\author[Λόλας]{Κωνσταντίνος Λόλας}
\institute[$10^ο$ ΓΕΛ]{$10^ο$ ΓΕΛ Θεσσαλονίκης}
\date{}

\begin{document}

\begin{frame}
  \titlepage
\end{frame}

\section{Θεωρία}

\begin{frame}[noframenumbering]
  Στο moodle θα βρείτε τις ασκήσεις που πρέπει να κάνετε, όπως και αυτή τη παρουσίαση
\end{frame}

\section{Ασκήσεις}

\exercises

\begin{askisi}
  Δίνεται η συνάρτηση $f(x)=\sqrt{1+e^x}$. Να δείξετε ότι:
  \begin{enumerate}[<+->]
    \item Η συνάρτηση $f$ είναι γνησίως αύξουσα
    \item $\sqrt{2}<f(x)<\sqrt{1+e}$, για κάθε $x\in [0,1]$
    \item $\sqrt{2}<\int_{0}^{1} f(x)\, dx<\sqrt{1+e}$
  \end{enumerate}
\end{askisi}

\begin{askisi}
  Δίνεται η συνάρτηση $f(x)=\dfrac{e^x}{x}$, $x>0$
  \begin{enumerate}[<+->]
    \item Να μελετήσετε την $f$ ως προς τα ακρότατα
    \item Να δείξετε ότι $\int_{\frac{1}{2}}^{2}f(x)\, dx>\dfrac{3e}{2}$
  \end{enumerate}
\end{askisi}

\begin{askisi}
  Δίνεται η συνάρτηση $f(x)=\ln (e^x+1)$
  \begin{enumerate}[<+->]
    \item Να δείξετε ότι η $f$ είναι κυρτή
    \item Να βρείτε την εφαπτόμενη της $C_f$ στο $x_0=0$
    \item Να δείξετε ότι:
          \begin{enumerate}[<+->]
            \item $\int_{0}^{1}f(x)\, dx>\dfrac{1}{4}+\ln 2$
            \item $\int_{1}^{2}xf(x)\, dx>\dfrac{7+9\ln 2}{6}$
          \end{enumerate}
  \end{enumerate}
\end{askisi}

\begin{askisi}
  Δίνεται η συνάρτηση $f(x)=\sqrt{x^2+2x+2}$. Να δείξετε ότι:
  \begin{enumerate}[<+->]
    \item Η ευθεία $y=x+1$ είναι ασύμπτωτη της $C_f$ για $x\to +\infty$ και είναι κάτω από την $C_f$ στο $[0,+\infty)$
    \item $\int_{0}^{1}f(x)\, dx>\dfrac{3}{2}$
  \end{enumerate}
\end{askisi}

\begin{askisi}
  Δίνεται η συνάρτηση $f(x)=\ln (1+x^2)$. Να δείξετε ότι:
  \begin{enumerate}[<+->]
    \item $f(x)\le x^2$, για κάθε $x\in \mathbb{R}$
    \item $\int_{0}^{1}f(x)\, dx<\dfrac{1}{3}$
  \end{enumerate}
\end{askisi}

\begin{askisi}
  Έστω $f:\mathbb{R}\to \mathbb{R}$ μια συνάρτηση η οποία είναι συνεχής και γνησίως αύξουσα. Να δείξετε ότι:
  \begin{enumerate}[<+->]
    \item $0<\int_{0}^{\frac{\pi}{2}}συνx\cdot f(ημx)\, dx<1$, όταν $f(0)=0$ και $f(1)=1$
    \item $\int_{0}^{1}f(e^x)\, dx<2$, όταν $f(e)=2$
    \item Αν $f(1)=2$, τότε
          \begin{enumerate}[<+->]
            \item $\int_{1}^{2}\dfrac{f(x)}{x}\, dx>2\ln 2$
            \item $\int_{1}^{2}f^2(x)\, dx>4$
          \end{enumerate}
  \end{enumerate}
\end{askisi}

\begin{askisi}
  Δίνεται η συνάρτηση $f(x)=2e^x-x^2-x-1$.
  \begin{enumerate}[<+->]
    \item Να μελετήσετε την $f$ ως προς την κυρτότητα και τα σημεία καμπής
    \item Να δείξετε ότι $f(x)\ge x+1$, για κάθε $x\ge 0$
    \item Να δείξετε ότι:
          \begin{enumerate}[<+->]
            \item $\int_{1}^{e}\frac{f(x)}{x}\, dx>e$
            \item $\int_{1}^{2}xf(x)\, dx>\dfrac{23}{6}$
            \item $\int_{0}^{1}xf(e^x)\, dx>\dfrac{3}{2}$
          \end{enumerate}
  \end{enumerate}
\end{askisi}

\begin{askisi}
  Έστω $f:[1,e]\to \mathbb{R}$ μια συνάρτηση η οποία είναι συνεχής και ισχύουν:
  $$f(x)\ge \frac{1}{x}+1 \text{, για κάθε } x\in [1,e] \text{ και } \int_{1}^{e}f(x)\, dx=e$$
  Να βρείτε τη συνάρτηση $f$.
\end{askisi}

\begin{askisi}
  Να αποδείξετε ότι $$\frac{4}{3}<\int_{0}^{1}e^{x^2}\, dx<e$$
\end{askisi}

\begin{askisi}
  Έστω $f:[0,2]\to \mathbb{R}$ μια συνάρτηση η οποία είναι συνεχής με ελάχιστη τιμή $1$ και μέγιστη τιμή $3$. Να δείξετε ότι:
  $$\frac{4}{3}<\int_{0}^{2}f(x)\, dx\cdot \int_{0}^{2}\frac{1}{f(x)}\, dx<12$$
\end{askisi}

\begin{askisi}
  Έστω $f:\mathbb{R}\to \mathbb{R}$ μια συνάρτηση η οποία είναι θετική και συνεχής. Να δείξετε ότι
  $$\int_{2}^{4}f(\frac{x}{2})\, dx > \int_{1}^{2}xf(x)\, dx$$
\end{askisi}

\begin{askisi}
  Έστω $f:\mathbb{R}\to \mathbb{R}$ μια συνάρτηση η οποία είναι συνεχής. Αν $\int_{0}^{1}f^2(x)\, dx=1$ να δείξετε ότι $$2+3\int_{0}^{1}f(x)\, dx \ge 0$$
\end{askisi}

\begin{askisi}
  Έστω $f:[0, +\infty)\to \mathbb{R}$ μια συνάρτηση η οποία είναι συνεχής και κυρτή. Να δείξετε ότι:
  \begin{enumerate}[<+->]
    \item $f(2x)-f(x)\ge xf'(x)$, για κάθε $x\ge 0$
    \item $\int_{0}^{1}f(2x)\, dx=\dfrac{1}{2}\int_{0}^{2}f(x)\, dx$
    \item $\int_{0}^{2}f(x)\, dx\ge 2f(1)$.
  \end{enumerate}
\end{askisi}

\begin{askisi}
  Δίνεται η συνάρτηση $f(x)=\begin{cases}
      \dfrac{\ln x}{x-1}, & 0<x\ne 1 \\
      1,                  & x=1
    \end{cases}$.
  \begin{enumerate}[<+->]
    \item Να δείξετε ότι η $f$ είναι συνεχής.
    \item Έστω $α$ ένας σταθερός αριθμός με $α\in (0,1)$. Να δείξετε ότι:
          $$\int_{1}^{α}f(x)\, dx<\frac{1}{α^2} \int_{α}^{α^2}xf(x)\, dx$$
  \end{enumerate}
\end{askisi}

\begin{askisi}
  Έστω $f:[1,+\infty)\to \mathbb{R}$ μια συνεχής συνάρτηση με $f(1)=0$ και $xf'(x)>1$, για κάθε $x>1$.
  \begin{enumerate}[<+->]
    \item Να δείξετε ότι $f(x)\ge \ln x$, για κάθε $x\ge 1$
    \item Να δείξετε ότι η εξίωση $(x-1)\int_{1}^{e}\dfrac{f(x)}{e-1}\, dx=1$ έχει μία τουλάχιστη λύση στο $(1,e)$
  \end{enumerate}
\end{askisi}

\begin{askisi}
  Έστω $f:\mathbb{R}\to \mathbb{R}$ μια συνάρτηση με $f(0)=0$, η οποία είναι παραγωγίσιμη και ισχύει $f'(x)>2 \left( xf(x)+e^{x^2} \right)$, για κάθε $x\in \mathbb{R}$. Να δείξετε ότι:
  \begin{enumerate}[<+->]
    \item $f(x)\ge 2xe^{x^2}$, για κάθε $x\in [0,1]$
    \item $\int_{0}^{1}f(x)\, dx>e-1$
    \item Η εξίσωση $x\int_{0}^{x}f(x)\, dx=e-x$ έχει ακριβώς μία ρίζα στο $(0,1)$
  \end{enumerate}
\end{askisi}

\begin{askisi}
  Έστω $f:\mathbb{R}\to \mathbb{R}$ μια συνάρτηση με $f(0)=0$, η οποία είναι κυρτή με $f'(1)=3$. Να δείξετε ότι:
  \begin{enumerate}[<+->]
    \item $f(x)\le 3x$, για κάθε $x\in [0,1]$
    \item Υπάρχει μοναδικό $α\in (0,1)$ τέτοιο ώστε $α\int_{0}^{1}f(x^2)\, dx=2α-1$
  \end{enumerate}
\end{askisi}

\begin{askisi}
  Δίνεται η συνάρτηση $f(x)=\ln (1+e^x)$. Να δείξετε ότι:
  \begin{enumerate}[<+->]
    \item Η συνάρτηση $f$ είναι γνησίως αύξουσα και κυρτή
    \item $\dfrac{1}{4}+\ln 2<\int_{0}^{1}f(x)\, dx<\ln (1+e)$
    \item Υπάρχει μοναδικό $θ\in (0,1)$ τέτοιο ώστε $4\int_{0}^{θ}f(x)\, dx=4f(θ)-θ+1$
  \end{enumerate}
\end{askisi}

\end{document}