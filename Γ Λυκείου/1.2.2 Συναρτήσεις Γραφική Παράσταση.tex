\documentclass{presentation}

\title{Συναρτήσεις}
\subtitle{Γραφική Παραστάση}
\author[Λόλας]{Κωνσταντίνος Λόλας }
\institute[$10^ο$ ΓΕΛ]{$10^ο$ ΓΕΛ Θεσσαλονίκης}
\date{}

\begin{document}

\begin{frame}
      \titlepage
\end{frame}

\begin{frame}{Ορισμός}
      \begin{block}{Ορισμός}
            Γραφική παράσταση μιας συνάρτησης είναι το σύνολο των σημείων $A(x,f(x))$, $x\in D_f$, και συμβολίζεται με $C_f$
      \end{block}
\end{frame}

\begin{frame}{Τι βλέπω}
      \begin{itemize}[<+->]
            \item Είναι γραφική παράσταση?
            \item Πεδίο Ορισμού
            \item Σύνολο τιμών
            \item Ρίζες
            \item Πρόσημο
            \item Κοινά σημεία
            \item Κατακόρυφη απόσταση
            \item Σχετική θέση
      \end{itemize}
\end{frame}

\begin{frame}{Γνωστές Γραφικές}
      \begin{itemize}[<+->]
            \item $y=a$
            \item $y=ax+b$
            \item $y=x^2$, $y=ax^2+bx+c$
            \item $y=ax^3$
            \item $y=\frac{a}{x}$
            \item $y=|x|$
            \item $y=ημ x$, $y=συν x$, $y=εφ x$
            \item $y=a^x$, $y=e^x$
            \item $y=\ln x$
            \item Μετατοπίσεις
      \end{itemize}
\end{frame}

\begin{frame}{Μετατοπίσεις}
      $$y=f(x)$$
      \begin{itemize}[<+->]
            \item $y=f(x)+c$
            \item $y=f(x+c)$
            \item $a\cdot f(x)$
            \item $y=f(a\cdot x)$
            \item $y=-f(x)$
            \item $y=|f(x)|$
            \item $y=f(-x)$
      \end{itemize}
\end{frame}

\section{Ασκήσεις}

\begin{askisi}
      \href{https://www.geogebra.org/m/jmmx7bp8}{\beamergotobutton{Άσκηση Geogebra}}
      \begin{enumerate}[<+->]
            \item Να βρείτε το πεδίο ορισμού και το σύνολο τιμών
            \item Να βρείτε τις τιμές: $f(2)$ και $f(f(0))$
            \item Να λύσετε γραφικά την $f(x)=0$
            \item Να λύσετε γραφικά την $f(x)<0$
            \item Να βρείτε το πεδίο ορισμού της συνάρτησης $g(x)=\ln f(x)$
      \end{enumerate}
      %\hyperlink{Λύση1}{\beamerbutton{Λύση}}
\end{askisi}

\begin{askisi}
      \href{https://www.geogebra.org/m/td6m58hw}{\beamergotobutton{Άσκηση Geogebra}}
      \begin{enumerate}[<+->]
            \item Να βρείτε τα κοινά σημεία των $C_f$ και $C_g$
            \item Να λύσετε την $f(x)=g(x)$
            \item Να λύσετε τις ανισώσεις:
                  \begin{enumerate}[<+->]
                        \item $f(x)>g(x)$
                        \item $f(x)<g(x)$
                  \end{enumerate}
            \item Να λύσετε την εξίσωση $2g(x)=f(g(0))$
            \item Να βρείτε την κατακόρυφη απόσταση των συναρτήσεων στο $x_0=0$
      \end{enumerate}
      %\hyperlink{Λύση2}{\beamerbutton{Λύση}}
\end{askisi}

\begin{askisi}
      Δίνεται η συνάρτηση $f(x)=ax^2-5a+1$, της οποίας η γραφική παράσταση διέρχεται από το σημείο $Α(3,5)$. Να βρείτε:
      \begin{enumerate}[<+->]
            \item την τιμή του $a$
            \item τα κοινά σημεία της $C_f$ με τους άξονες $y'y$ και $x'x$
            \item τα διαστήματα του $x$ που η $C_f$ βρίσκεται πάνω από τον άξονα $x'x$
      \end{enumerate}
      %\hyperlink{Λύση3}{\beamerbutton{Λύση}}
\end{askisi}

\begin{askisi}
      Δίνονται οι συναρτήσεις $f(x)=\frac{1}{x}$ και $g(x)=1$. Να βρείτε:
      \begin{enumerate}[<+->]
            \item τα κοινά τους σημεία
            \item την σχετική τους θέση
      \end{enumerate}
      %\hyperlink{Λύση4}{\beamerbutton{Λύση}}
\end{askisi}

\begin{askisi}
      Να σχεδιάσετε τις γραφικές παραστάσεις των συναρτήσεων:
      \begin{enumerate}[<+->]
            \item
                  $f(x)=
                        \begin{cases}
                              \frac{1}{x}, & x<0    \\
                              x^2,         & x\ge 0
                        \end{cases}$

            \item $f(x)=
                        \begin{cases}
                              e^{-x}, & x<0    \\
                              -συν x, & x\ge 0
                        \end{cases}$
      \end{enumerate}
      Από τη γραφική παράσταση να προσδιορίσετε το σύνολο τιμών σε καθεμία περίπτωση
      %\hyperlink{Λύση5}{\beamerbutton{Λύση}}
\end{askisi}

\begin{askisi}
      Στο ίδιο σύστημα αξόνων να σχεδιάσετε τις γραφικές παραστάσεις των συναρτήσεων $e^x$, $ημ x$ για $x>0$, να βρείτε τη σχετική τους θέση και να επιβεβαιώσετε αλγεβρικά την ανισώτητα:
      $$e^x>ημ x \text{, για κάθε } x>0$$
      %\hyperlink{Λύση6}{\beamerbutton{Λύση}}
\end{askisi}

\begin{askisi}
      Να σχεδιάσετε τις γραφικές παραστάσεις των παρακάτω συναρτήσεων στο ίδιο σύστημα αξόνων
      $$f(x)=(x-1)^2+1\text{, } x\ge 1 \text{ και } g(x)=1+\sqrt{x-1}$$
      %\hyperlink{Λύση7}{\beamerbutton{Λύση}}
\end{askisi}

\begin{askisi}
      \href{https://www.geogebra.org/m/euy2uhma}{\beamergotobutton{Άσκηση Geogebra}}
      \begin{enumerate}[<+->]
            \item Να λύσετε την εξίσωση $f(x)=2$
            \item Να βρείτε πεδίο ορισμού της $$g(x)=\frac{1}{f(x)-1}$$
            \item Να Βρείτε το πλήθος ριζών των εξισώσεων
                  \begin{enumerate}[<+->]
                        \item $f(x)=5/2$
                        \item $2f(x)-1=0$
                        \item $f(x)=a^2+1$, $a\ne 0$
                  \end{enumerate}
      \end{enumerate}
      %\hyperlink{Λύση8}{\beamerbutton{Λύση}}
\end{askisi}

\begin{askisi}
      \href{https://www.geogebra.org/m/dvzdm7bw}{\beamergotobutton{Άσκηση Geogebra}}
      \begin{enumerate}[<+->]
            \item Να βρείτε το πλήθος των λύσεων της εξίσωσης $f(x)=a$, για τις διάφορες τιμές του $a\in\mathbb{R}$
            \item Να δείξετε ότι η εξίσωση $f(x)=3ημ a - 5$ είναι αδύνατη, για κάθε $a\in\mathbb{R}$
      \end{enumerate}
      %\hyperlink{Λύση9}{\beamerbutton{Λύση}}
\end{askisi}

\begin{askisi}
      Να εξετάσετε
      \begin{enumerate}[<+->]
            \item αν ο αριθμός $2$ ανήκει στο σύνολο τιμών της συνάρτησης $f(x)=1+\sqrt{x}$
            \item αν ο αριθμός $0$ ανήκει στο σύνολο τιμών της συνάρτησης $f(x)=\frac{e^x-1}{x}$
      \end{enumerate}
      %\hyperlink{Λύση10}{\beamerbutton{Λύση}}
\end{askisi}

\begin{askisi}
      Έστω $f:A\to\mathbb{R}$ μία συνάρτηση με $A=\mathbb{R}$ και $f(A)=(1,+\infty)$.
      \begin{enumerate}[<+->]
            \item Να δείξετε ότι η εξίσωση $f(x)=2023$ έχει μία τουλάχιστον λύση
            \item Να δείξετε ότι η εξίσωση $f(x)=a^2+1$ έχει μία τουλάχιστον λύση, για κάθε $a\in\mathbb{R^*}$
            \item Να εξετάσετε αν υπάρχει $x_0\ge 0$ τέτοιο ώστε $f(x)=e^{x_0}$
      \end{enumerate}
      %\hyperlink{Λύση11}{\beamerbutton{Λύση}}
\end{askisi}

\end{document}
