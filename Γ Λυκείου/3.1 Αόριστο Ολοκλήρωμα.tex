\documentclass{../presentation}

\title{Συναρτήσεις}
\subtitle{Αόριστο Ολοκλήρωμα}
\author[Λόλας]{Κωνσταντίνος Λόλας}
\institute[$10^ο$ ΓΕΛ]{$10^ο$ ΓΕΛ Θεσσαλονίκης}
\date{}

\begin{document}

\begin{frame}
  \titlepage
\end{frame}

\section{Θεωρία}
\begin{frame}{Καιρός να τελειώνουμε!}
  Απλά, θα μάθουμε να αντιπαραγωγίζουμε! Επειδή:
  \begin{itemize}[<+->]
    \item μπορούμε
    \item θα λύνουμε προβλήματα ρυθμού μεταβολής (ροπή αδράνειας, έργο δύναμης, κέντρο μάζας...)
    \item θα βρίσκουμε εμβαδά. Whattttttttttttt????????
  \end{itemize}
\end{frame}

\begin{frame}{Ποιό Σύμβολο?}
  \begin{block}{Ορισμός Αόριστου Ολοκληρώματος}
    Έστω $f(x)$ συνάτηση ορισμένη σε διάστημα $Δ$ και $F(x)$ μία συνάρτηση με $F'(x)=f(x)$. Τότε
    $$\int f(x) \, dx=F(x)+c$$
  \end{block}
  H $F$ έχει πολλά ονόματα: Αρχική, Αντιπαράγωγος, Παράγουσα...
\end{frame}

\begin{frame}{Πόσες αρχικές υπάρχουν μίας $f$}
  \begin{block}{Θεώρημα}
    Έστω $F(x)$ μία αρχική της $f(x)$. Τότε:
    \begin{itemize}
      \item κάθε $F(x)+c$ είναι επίσης αρχική
      \item κάθε αρχική είναι της μορφής $F(x)+c$
    \end{itemize}
  \end{block}
\end{frame}

\begin{frame}{Σαν έτοιμοι από καιρό, σαν θαρραλέοι...}
  Γνωστές αρχικές:
  \begin{multicols}{2}
    \begin{itemize}[<+->]
      \item $\int 0 \, dx=c$
      \item $\int a \, dx=ax+c$
      \item $\int x^a \, dx=\dfrac{x^{a+1}}{a+1}+c$, $a\ne -1$
      \item $\int \dfrac{1}{x} \, dx=\ln|x|+c$ !!!!!!!!!
      \item $\int ημx \, dx=-συνx+c$
      \item $\int συνx \, dx=ημx+c$
      \item $\int \dfrac{1}{συν^2x} \, dx=εφx+c$
      \item $\int \dfrac{1}{ημ^2x} \, dx=-σφx+c$
      \item $\int e^x \, dx=e^x+c$
      \item $\int a^x \, dx=\dfrac{a^x}{\ln a}+c$...
    \end{itemize}
  \end{multicols}
\end{frame}

\begin{frame}{Άλλες Αρχικές?}
  Γνωστές αρχικές 2:
  \begin{itemize}[<+->]
    \item $\int f'(x) \, dx=f(x)+c$
    \item $\int af'(x) \, dx=af(x)+c$
    \item $\int f'(x)g(x)+f(x)g'(x) \, dx=f(x)g(x)+c$
    \item $\int \dfrac{f'(x)g(x)-f(x)g'(x)}{g^2(x)} \, dx=\dfrac{f(x)}{g(x)}+c$
    \item $\int f'(g(x))g'(x) \, dx=f(g(x))+c$
  \end{itemize}
\end{frame}

\begin{frame}{Το "τέλειο" των ολοκληρωμάτων...}
  Δεν έχουν όλες οι συναρτήσεις "αλγεβρικό τύπο" για αρχική!
  \begin{itemize}[<+->]
    \item $e^{x^2}$
    \item $\dfrac{e^x}{x}$
    \item $ημx^2$...
  \end{itemize}
\end{frame}

\begin{frame}{Τι γίνεται με την ύπαρξη?}
  Εκτός ύλης! Να σας αρκεί ότι αν είναι συνεχής τότε έχει αρχική...
  \newline
  \only<2>{Πάντως, εννοείται υπάρχουν και μη συνεχείς που έχουν αρχική!}
\end{frame}

\begin{frame}[noframenumbering]
  Στο moodle θα βρείτε τις ασκήσεις που πρέπει να κάνετε, όπως και αυτή τη παρουσίαση
\end{frame}

\section{Ασκήσεις}

\exercises

\begin{askisi}
  Δίνεται η συνάρτηση $f(x)=ημx$. Να βρείτε:
  \begin{enumerate}[<+->]
    \item Μια παράγουσα της $f$
    \item Όλες τις παράγουσες της $f$
    \item Την παράγουσα της $f$ της οποίας η γραφική παράσταση διέρχεται από το σημείο $A(0,1)$
  \end{enumerate}
\end{askisi}

\begin{askisi}
  Να βρείτε μία παράγουσα των συναρτήσεων:
  \begin{enumerate}[<+->]
    \item $f(x)=x^4$
    \item $f(x)=\dfrac{1}{x^5}$, $x>0$
    \item $f(x)=\sqrt[3]{x}$
    \item $\dfrac{1}{\sqrt[3]{x^2}}$, $x>0$
  \end{enumerate}
\end{askisi}

\begin{askisi}
  Να βρείτε μία παράγουσα των συναρτήσεων:
  \begin{enumerate}[<+->]
    \item $f(x)=ημx+e^x-\dfrac{1}{x}$, $x>0$
    \item $f(x)=x^2+x-\dfrac{1}{x^2}$, $x<0$
  \end{enumerate}
\end{askisi}

\begin{askisi}
  Να βρείτε μια παράγουσα των συναρτήσεων:
  \begin{enumerate}[<+->]
    \item $f(x)=5x^2$
    \item $f(x)=4x^3$
    \item $f(x)=\dfrac{3}{2x}$, $x>0$
    \item $f(x)=\dfrac{1}{\sqrt{x}}$
    \item $f(x)=3x^2+ax+b$
    \item $f(x)=\dfrac{a}{x}-bσυνx$, $x>0$
  \end{enumerate}
\end{askisi}

\begin{askisi}
  Να βρείτε μια παράγουσα των συναρτήσεων:
  \begin{enumerate}[<+->]
    \item $f(x)=\dfrac{(2x-1)^2}{x^2}$, $x<0$
    \item $f(x)=\dfrac{1}{ημ^2x\cdot συν^2x}$, $x\in \left( 0,\dfrac{π}{2} \right) $
  \end{enumerate}
\end{askisi}

\begin{askisi}
  Να βρείτε μια παράγουσα των συναρτήσεων:
  \begin{enumerate}[<+->]
    \item $f(x)=συνx-xημx$
    \item $f(x)=\dfrac{συνx-ημx}{e^x}$
  \end{enumerate}
\end{askisi}

\begin{askisi}
  Να βρείτε μια παράγουσα των συναρτήσεων:
  \begin{enumerate}[<+->]
    \item $f(x)=e^{ημx}συνx$
    \item $f(x)=2xe^{x^2}$
    \item $f(x)=\dfrac{e^x}{\sqrt{1+e^x}}$
    \item $f(x)=\dfrac{x}{\sqrt{x^2+1}}$
  \end{enumerate}
\end{askisi}

\begin{askisi}
  Να βρείτε μια παράγουσα των συναρτήσεων:
  \begin{enumerate}[<+->]
    \item $f(x)=\dfrac{2x+1}{x^2+x+1}$
    \item $f(x)=\dfrac{1}{x-1}$, $x>1$
    \item $f(x)=εφx$, $x\in \left( 0,\dfrac{π}{2} \right) $
    \item $f(x)=\dfrac{1}{x\ln x}$, $x>1$
    \item $f(x)=\dfrac{(x-1)^2}{x^2+1}$
    \item $f(x)=\dfrac{1}{1+e^{-x}}$
  \end{enumerate}
\end{askisi}

\begin{askisi}
  Να βρείτε όλες τις παράγουσες των συναρτήσεων:
  \begin{enumerate}[<+->]
    \item $f(x)=(2x-3)(x^2-3x+1)^2$
    \item $f(x)=ημx\cdot συν^2x$
    \item $f(x)=\dfrac{\ln x}{x}$
    \item $f(x)=2x\sqrt{x^2+1}$
    \item $f(x)=εφ^3x+εφ^5x$, $x\in \left( 0,\dfrac{π}{2} \right)$
  \end{enumerate}
\end{askisi}

\begin{askisi}
  Να βρείτε μια παράγουσα των συναρτήσεων:
  \begin{enumerate}[<+->]
    \item $f(x)=e^{2x}$
    \item $f(x)=\dfrac{1}{3x+2}$, $x>0$
    \item $f(x)=συν2x+e^{-x}$
  \end{enumerate}
\end{askisi}

\begin{askisi}
  Να βρείτε της αρχικές της συνάρτησης $f(x)=2|x|+1$, $x\in\mathbb{R}$
\end{askisi}

\begin{askisi}
  Να βρείτε τον τύπο της συνάρτησης $f$ για κάθε μία από τις παρακάτω περιπτώσεις:
  \begin{enumerate}[<+->]
    \item $f'(x)=e^x-ημx$, $x\in\mathbb{R}$ και $f(0)=2$
    \item $f'(x)=\dfrac{1}{x}+\dfrac{2}{x^2}-1$, $x>0$ και $f(1)=1$
  \end{enumerate}
\end{askisi}

\begin{askisi}
  Να βρείτε τον τύπο της συνάρτησης $f$ για κάθε μία από τις παρακάτω περιπτώσεις:
  \begin{enumerate}[<+->]
    \item $f'(x)=e^{-x}+\dfrac{1}{1+x}$, $x>-1$ και $f(0)=-1$
    \item $f'(x)=\dfrac{x}{\sqrt{x^2+1}}$, $x\in\mathbb{R}$ και $f(1)=\sqrt{2}$
  \end{enumerate}
\end{askisi}

\begin{askisi}
  Να βρείτε τον τύπο της συνάρτησης $f$ για κάθε μία από τις παρακάτω περιπτώσεις:
  \begin{enumerate}[<+->]
    \item $x^2f'(x)=x^2+x-1$, $x>0$ και $f(1)=0$
    \item $f'(x)=\dfrac{(x+1)^2}{x^2+1}$, $x\in\mathbb{R}$ και $f(0)=0$
  \end{enumerate}
\end{askisi}

\begin{askisi}
  Να βρείτε τον τύπο της συνάρτησης $f$ για κάθε μία από τις παρακάτω περιπτώσεις:
  \begin{enumerate}[<+->]
    \item $f'(x)=x(2συνx-xημx)$, $x\in\mathbb{R}$ και $f(0)=1$
    \item $f'(x)=\dfrac{1-\ln x}{x^2}$, $x>0$ και $f(1)=0$
  \end{enumerate}
\end{askisi}

\begin{askisi}
  Να βρείτε τον τύπο της συνάρτησης $f$ όταν ισχύουν $f''(x)=e^x-ημx$, $x\in\mathbb{R}$ και $f'(0)=f(0)=1$
\end{askisi}

\begin{askisi}
  Έστω $f$,$g:\mathbb{R}\to\mathbb{R}$ δύο συναρτήσεις, για τις οποίες ισχύει $f''(x)=g''(x)+2$, για κάθε $x\in\mathbb{R}$. Αν οι $C_f$ και $C_g$ τέμνονται πάνω στον άξονα $y'y$ και οι εφαπτομένες των $C_f$ και $C_g$ στο $x_0=1$ είναι παράληλες, να λύσετε την ανίσωση
  $$f(x)<g(x)$$
\end{askisi}

\begin{askisi}
  Έστω $f:\mathbb{R}\to\mathbb{R}$ μία συνάρτηση με $f(π)=0$, η οποία είναι συνεχής και ισχύει
  $$f'(x)=\begin{cases}
      3x^2 & ,x<0 \\
      συνx & ,x>0
    \end{cases}$$
  Να βρείτε τον τύπο της $f(x)$
\end{askisi}

\begin{askisi}
  Έστω $f:\mathbb{R}\to\mathbb{R}$ μία συνάρτηση, για την οποία ισχύει $f(0)=2$ και
  $$(x-1)f'(x)=2x^2-x-1$$
  για κάθε $x\in\mathbb{R}$. Να βρείτε την $f(x)$
\end{askisi}

\begin{askisi}
  Να βρείτε τον τύπο της συνάρτησης $f$ για κάθε μία από τις παρακάτω περιπτώσεις:
  \begin{enumerate}[<+->]
    \item $(x^2+1)f'(x)=1-2xf(x)$, $x\in\mathbb{R}$ και $f(0)=1$
    \item $f'(x)ημx=ημ^2x+f(x)συνx$, $x\in \left( 0,π \right) $ και $f(\frac{π}{2})=\frac{π}{2}$
  \end{enumerate}
\end{askisi}

\begin{askisi}
  Να βρείτε τον τύπο της συνάρτησης $f$ για κάθε μία από τις παρακάτω περιπτώσεις:
  \begin{enumerate}[<+->]
    \item $f'(x)=e^{x-f(x)}$, $x\in\mathbb{R}$ και $f(0)=\ln 2$
    \item $\dfrac{f'(x)}{f(x)}=2x+\dfrac{1}{x}$, $x>0$ και $f(1)=e$ και $f(x)\ne 0$ για κάθε $x>0$
  \end{enumerate}
\end{askisi}

\begin{askisi}
  Έστω $f:(0,+\infty)\to\mathbb{R}$ μία συνάρτηση, για την οποία ισχύει $xf'(x)+(1-x)f(x)=0$ για κάθε $x>0$ και $f(1)=e$. Να βρείτε την $f(x)$
\end{askisi}

\begin{askisi}
  Να βρείτε τον τύπο της συνάρτησης $f$ για κάθε μία από τις παρακάτω περιπτώσεις:
  \begin{enumerate}[<+->]
    \item $xf'(x)=συνx-f(x)$, $x\in\mathbb{R}$
    \item $f'(x)-\dfrac{x}{f(x)}=0$, $x\in\mathbb{R}$ και $f(0)=2$
  \end{enumerate}
\end{askisi}

\begin{askisi}
  Από την πώληση ενός προϊόντος μιας εταιρείας, διαπιστώθηκε ότι ο ρυθμός μεταβολής του κόστους $Κ(t)$ του προϊόντος είναι $1000-0.8t$ (σε ευρώ την ημέρα), ενώ ο ρυθμός είπσραξης $Ε(t)$ στο τέλος των $t$ ημερών δίνεται από τον τύπο $Ε'(t)=1300+0.4t$ (σε ευρώ την ημέρα). Να βρείτε το συνολικό κέρδος της εταιρείας από την τρίτη έως και την έβδομη ημέρα.
\end{askisi}

\end{document}