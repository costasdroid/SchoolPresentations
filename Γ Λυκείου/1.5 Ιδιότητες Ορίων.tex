\documentclass[greek]{beamer}
%\usepackage{fontspec}
\usepackage{amsmath,amsthm}
\usepackage{unicode-math}
\usepackage{xltxtra}
\usepackage{graphicx}
\usetheme{CambridgeUS}
\usecolortheme{seagull}
\usepackage{hyperref}
\usepackage{ulem}
\usepackage{xgreek}
\usepackage{pgfpages}
\usepackage{tikz}
%\setbeameroption{show notes on second screen}
%\setbeameroption{show only notes}

\setsansfont{Noto Serif}

% \newtheorem{definition}{Ορισμός}

\title{Συναρτήσεις}
\subtitle{Πράξεις ορίων}
\author[Λόλας]{Κωνσταντίνος Λόλας}
\date{}

\begin{document}

\begin{frame}
 \titlepage
\end{frame}

\section{Θεωρία}
\begin{frame}{Υπολογισμοί ορίων}
 \centering
 \includegraphics[height=0.6\columnwidth]{images/oria}
\end{frame}

\begin{frame}{Πρόσημο}
 \begin{block}{Θεώρημα 1ο}
  \begin{itemize}
   \item Αν $\lim\limits_{x \to x_0}{ f(x) }>0$, τότε \pause $f(x)>0$ κοντά στο $x_0$ \pause
   \item Αν $\lim\limits_{x \to x_0}{ f(x) }<0$, τότε $f(x)<0$ κοντά στο $x_0$
  \end{itemize}
 \end{block}
\end{frame}

\begin{frame}{Διάταξη}
 \begin{block}{Θεώρημα 2ο}
  Αν οι συναρτήσεις $f$, $g$ έχουν όριο στο $x_0$ και ισχύει $f(x)\le g(x)$ κοντά στο $x_0$, τότε
  $$\lim\limits_{x \to x_0}{ f(x) }\le \lim\limits_{x \to x_0}{ g(x) }$$
 \end{block}
\end{frame}

\begin{frame}{Από κάπου να ποιαστούμε?}
 \begin{itemize}
  \item $\lim\limits_{x \to x_0}{ x }=x_0$
  \item $\lim\limits_{x \to x_0}{ c }=c$
 \end{itemize}
\end{frame}

\begin{frame}{Και υπολογίζουμε...}
 ΜΟΝΟ αν υπάρχουν τα όρια των $f$ και $g$ τότε
 \begin{itemize}
  \item $\lim\limits_{x \to x_0}{ f(x)+g(x) }=\lim\limits_{x \to x_0}{ f(x) }+\lim\limits_{x \to x_0}{ g(x) }$
  \item $\lim\limits_{x \to x_0}{ k\cdot f(x) }=k\cdot \lim\limits_{x \to x_0}{ f(x) }$
  \item $\lim\limits_{x \to x_0}{ f(x)\cdot g(x) }=\lim\limits_{x \to x_0}{ f(x) }\cdot\lim\limits_{x \to x_0}{ g(x) }$
  \item $\lim\limits_{x \to x_0}{ \frac{f(x)}{g(x)} }=\frac{\lim\limits_{x \to x_0}{ f(x) }}{\lim\limits_{x \to x_0}{ g(x) }}$ εννοείται ότι $\lim\limits_{x \to x_0}{ g(x) }\ne 0$
  \item $\lim\limits_{x \to x_0}{ |f(x)| }=|\lim\limits_{x \to x_0}{ f(x) }|$
  \item $\lim\limits_{x \to x_0}{ \sqrt[n]{f(x)} }=\sqrt[n]{\lim\limits_{x \to x_0}{ f(x) }}$, εννοείται ότι $f(x)>0$ κοντά στο $x_0$
 \end{itemize}
\end{frame}

\begin{frame}{Τα προφανή όρια!}
 \begin{block}{Από τα προηγούμενα...}
  \begin{itemize}
   \item $\lim\limits_{x \to x_0}{ x^n }=x_0^n$, για $n\in \mathbb{N}^*$ \pause
   \item $\lim\limits_{x \to x_0}{ P(x) }=P(x_0)$ \pause, η πρώτη σας απόδειξη!!! \pause
   \item $\lim\limits_{x \to x_0}{ \frac{P(x)}{Q(x)} }=\frac{P(x_0)}{Q(x_0)}$ \pause με λίγη προσοχή \pause
   \item $\lim\limits_{x \to x_0}{ ημ(x) }=ημ(x_0)$ \pause
   \item $\lim\limits_{x \to x_0}{ συν(x) }=συν(x_0)$ \pause
   \item $\lim\limits_{x \to x_0}{ εφ(x) }=εφ(x_0)$
  \end{itemize}
 \end{block}
\end{frame}

\begin{frame}{Θεώρημα 3ο}
 \begin{block}{Sandwich, Παρεμβολής...}
  Έστω οι συναρτήσεις $f$, $g$ και $h$. Αν
  \begin{itemize}
   \item $h(x)\le f(x) \le g(x)$, κοντά στο $x_0$ \pause
   \item $\lim\limits_{x \to x_0}{ h(x) }=\lim\limits_{x \to x_0}{ g(x) }=k \in\mathbb{R}$, \pause
  \end{itemize}
  τότε $\lim\limits_{x \to x_0}{ f(x) }=k$
 \end{block}
\end{frame}

\begin{frame}{Σχεδόν τελειώσαμε}
 \begin{block}{Και λίγα άγνωστα όρια}
  \begin{itemize}
   \item $\lim\limits_{x \to 0}{ \frac{ημx}{x} }=1$, \pause
   \item $\lim\limits_{x \to 0}{ \frac{1-συνx}{x} }=0$
  \end{itemize}
 \end{block}
\end{frame}

\begin{frame}{Όλα τα είχε η Μαριωρή...}
 Τι γίνεται με τη σύνθεση  $\lim\limits_{x \to x_0}{ f(g(x))}$? \pause
 \begin{enumerate}
  \item Θέτουμε $u=g(x)$ \pause
  \item Υπολογίζουμε (αν υπάρχει!) το $u_0=\lim\limits_{x \to x_0}{ g(x)}$ \pause
  \item Υπολογίζουμε (αν υπάρχει!) το $k=\lim\limits_{u \to u_0}{ f(u)}$ \pause
 \end{enumerate}
 Τότε αν $g(x)\ne u_0$ κοντά στο $x_0$, τότε προφανώς $\lim\limits_{x \to x_0}{ f(g(x))}=k$
\end{frame}

\section{Ασκήσεις}
\begin{frame}{Εξάσκηση}
 Αν για τις συναρτήσεις $f$, $g$ ισχύουν
 $$\lim\limits_{x \to 2}{ f(x) }=1 \text{ και } \lim\limits_{ x \to 2}{ g(x) }=2$$
 να υπολογίσετε το $\lim\limits_{x \to 2}{ (3f(x)+f(x)\cdot g(x)) }$
\end{frame}

\begin{frame}{Εξάσκηση}
 Να υπολογίσετε τα όρια
 \begin{enumerate}
  \item $\lim\limits_{x \to 3}{ \frac{x^2-3x}{x^2-9} }$ \pause
  \item $\lim\limits_{x \to 1}{ \frac{2x^3-2x}{2x^2-5x+3} }$ \pause
  \item $\lim\limits_{x \to 2}{ \frac{x^3-8}{x^3-7x+6} }$ \pause
  \item $\lim\limits_{x \to 1}{ \left( \frac{x}{x-1}+\frac{x-2}{x^2-x}  \right)  }$
 \end{enumerate}
\end{frame}

\begin{frame}{Εξάσκηση}
 Να υπολογίσετε τα όρια
 \begin{enumerate}
  \item $\lim\limits_{x \to 4}{ \frac{x-4}{\sqrt{x}-2} }$ \pause
  \item $\lim\limits_{x \to -1}{ \frac{\sqrt{x+5}-2}{x^2+x} }$ \pause
  \item $\lim\limits_{x \to 1}{ \frac{\sqrt{x^2+3}-3x+1}{x^2-1} }$ \pause
  \item $\lim\limits_{x \to 1}{ \frac{2\sqrt{2x-1}-\sqrt{x+3}}{\sqrt{x}-1} }$
 \end{enumerate}
\end{frame}

\begin{frame}{Εξάσκηση}
 Έστω η συνάρτηση $f(x)=\begin{cases}
   x-2,  & 0<x\le 1 \\
   3x-4, & 1<x<2    \\
   2x-1, & x>2
  \end{cases}$.
 Να βρείτε (αν υπάρχουν) τα όρια:
 \begin{enumerate}
  \item $\lim\limits_{x \to 3}{ f(x) }$ \pause
  \item $\lim\limits_{x \to 1}{ f(x) }$ \pause
  \item $\lim\limits_{x \to 2}{ f(x) }$
 \end{enumerate}
\end{frame}

\begin{frame}{Εξάσκηση}
 Έστω η συνάρτηση $f(x)=\begin{cases}
   \frac{x^2-x-2}{2x-4}, & x\ne 2 \\
   a,                    & x=2
  \end{cases}$.
 \begin{enumerate}
  \item Να βρείτε το $\lim\limits_{x \to 2}{ f(x) }$ \pause
  \item Να βρείτε το $a$ ώστε $\lim\limits_{x \to 2}{ f(x) }=f(2)$
 \end{enumerate}
\end{frame}

\begin{frame}{Εξάσκηση}
 Έστω η συνάρτηση $f(x)=\begin{cases}
   ασυνx+ημx-β,    & x<0    \\
   α\sqrt{x+1}+2β, & x\ge 0
  \end{cases}$.
 Να βρείτε τα $α$ και $β\in\mathbb{R}$ ώστε $\lim\limits_{x \to 0}{ f(x) }=1$
\end{frame}

\begin{frame}{Εξάσκηση}
 Να βρείτε αν υπάρχουν τα όρια:
 \begin{enumerate}
  \item $\lim\limits_{x \to 2}{ \frac{|x^3-x-1|-|x-7|}{x^2-4} }$ \pause
  \item $\lim\limits_{x \to 2}{ \frac{|x^2-4|-x+2}{x^2-2x} }$
 \end{enumerate}
\end{frame}

\begin{frame}{Εξάσκηση}
 Έστω $f:\mathbb{R}\to\mathbb{R}$ μια συνάρτηση, για την οποία:
 \begin{enumerate}
  \item Αν ισχύει $3x-2-x^2\le f(x) \le x^2-x$, για κάθε $x\in\mathbb{R}$, να βρείτε το $\lim\limits_{x \to 1}{ f(x) }$ \pause
  \item Αν ισχύει $|f(x)-2|\le x^2$, για κάθε $x\in\mathbb{R}$, να βρείτε το $\lim\limits_{x \to 0}{ f(x) }$ \pause
  \item Αν ισχύει $f(\mathbb{R})=(0,1)$, να βρείτε το $\lim\limits_{x \to 0}{ \left(x^2f(x)  \right) }$
 \end{enumerate}
\end{frame}

\begin{frame}{Εξάσκηση}
 Να βρείτε το $\lim\limits_{x \to 0}{ \frac{f(x)}{x} }$, όταν ισχύει:
 $$2x-3x^2\le f(x) \le x^4+2x$$
\end{frame}

\begin{frame}{Εξάσκηση}
 Να βρείτε το $\lim\limits_{x \to 0}{ xσυν\frac{1}{x} }$
\end{frame}

\begin{frame}{Εξάσκηση}
 Να υπολογίσετε τα όρια
 \begin{enumerate}
  \item $\lim\limits_{x \to 0}{ \frac{ημx}{x}+συνx^2 }$ \pause
  \item $\lim\limits_{x \to 0}{ \frac{συνx-1}{x}-εφx }$ \pause
  \item $\lim\limits_{x \to 0}{ \left( xημ\frac{1}{x}+\frac{ημ^2x}{x^2} \right)}$ \pause
  \item $\lim\limits_{x \to 0}{ \frac{1-συν^2x}{x} }$
 \end{enumerate}
\end{frame}

\begin{frame}{Εξάσκηση}
 Να υπολογίσετε τα όρια
 \begin{enumerate}
  \item $\lim\limits_{x \to 0}{ \frac{x}{ημx} }$ \pause
  \item $\lim\limits_{x \to 0}{ \frac{εφx}{x} }$ \pause
  \item $\lim\limits_{x \to 0}{ \frac{συνx-1}{ημx} }$ \pause
  \item $\lim\limits_{x \to 0}{ \left( ημx\cdot ημ\frac{1}{x} \right) }$
 \end{enumerate}
\end{frame}

\begin{frame}{Εξάσκηση}
 Να υπολογίσετε τα όρια
 \begin{enumerate}
  \item $\lim\limits_{x \to 0}{ \frac{x+ημx}{x} }$ \pause
  \item $\lim\limits_{x \to 0}{ \frac{ημ(π-x)}{x^2+x} }$ \pause
  \item $\lim\limits_{x \to 0}{ \frac{ημx+3x}{2x-ημx} }$ \pause
  \item $\lim\limits_{x \to 0}{ \frac{ημ^2x}{\sqrt{x^2+1}-1} }$
 \end{enumerate}
\end{frame}

\begin{frame}{Εξάσκηση}
 Να υπολογίσετε τα όρια
 \begin{enumerate}
  \item $\lim\limits_{x \to 0}{ \frac{ημ5x}{x} }$ \pause
  \item $\lim\limits_{x \to 0}{ \frac{συν3x-1}{x} }$ \pause
  \item $\lim\limits_{x \to π}{ \frac{ημx}{π-x} }$ \pause
  \item $\lim\limits_{x \to 0}{ \frac{ημx^2}{ημ3x} }$
 \end{enumerate}
\end{frame}

\begin{frame}{Εξάσκηση}
 Να υπολογίσετε τα όρια
 \begin{enumerate}
  \item $\lim\limits_{x \to 1}{ \frac{x\sqrt{x}-1}{x^2-\sqrt{x}} }$ \pause
  \item $\lim\limits_{x \to 2}{ \frac{\sqrt{x-1}+\sqrt[3]{x-1}-2}{x-2} }$
 \end{enumerate}
\end{frame}

\begin{frame}{Εξάσκηση}
 Αν $\lim\limits_{x \to 1}{ f(x) }=0$, να βρείτε το $\lim\limits_{x \to 1}{ \frac{f(x)-2ημf(x)}{f(x)+1-συνf(x)} }$
\end{frame}

\begin{frame}{Εξάσκηση}
 Έστω $f:\mathbb{R}\to\mathbb{R}$ μία συνάρτηση για την οποία ισχύει $\lim\limits_{x \to 0}{ \frac{f(x)-1}{x} }=2$. Να βρείτε τα όρια:
 \begin{enumerate}
  \item $\lim\limits_{x \to 0}{ \frac{f(5x)-1}{x} }$ \pause
  \item $\lim\limits_{x \to 0}{ \frac{f(3x)-f(2x)}{x} }$
 \end{enumerate}
\end{frame}

\begin{frame}{Εξάσκηση}
 Έστω $f:\mathbb{R}\to\mathbb{R}$ μία συνάρτηση για την οποία ισχύει $\lim\limits_{h \to 0}{ \frac{f(1+h)-1}{h} }=2$. Να βρείτε τα όρια:
 \begin{enumerate}
  \item $\lim\limits_{h \to 0}{ \frac{f(1-2h)-1}{h} }$ \pause
  \item $\lim\limits_{h \to 0}{ \frac{f(1+2h)-f(1-h)}{h} }$
 \end{enumerate}
\end{frame}

\begin{frame}{Εξάσκηση}
 Αν $\lim\limits_{x \to 1}{ \frac{f(x)-3}{x-1} }=2$, να βρείτε το $\lim\limits_{x \to 1}{ \frac{f(x^2-x+1)-3}{x-1} }$
\end{frame}

\begin{frame}{Εξάσκηση}
 Έστω $f:\mathbb{R}\to\mathbb{R}$ μία συνάρτηση. Να βρείτε το $\lim\limits_{x \to x_0}{ f(x) }$ όταν:
 \begin{enumerate}
  \item $\lim\limits_{x \to x_0}{\left(f(x)-x^2+3x  \right)=3  }$ και $x_0=2$ \pause
  \item $\lim\limits_{x \to x_0}{\frac{f(x)-1}{x-2}=3  }$ και $x_0=2$ \pause
  \item $\lim\limits_{x \to x_0}{\frac{f(x)+ημx}{x}=2  }$ και $x_0=0$
 \end{enumerate}
\end{frame}

\begin{frame}{Εξάσκηση}
 Έστω $f:\mathbb{R}\to\mathbb{R}$ μία συνάρτηση για την οποία ισχύει $\lim\limits_{x \to 0}{ \frac{f(x)}{x} }=1$.
 \begin{enumerate}
  \item Να βρείτε το $\lim\limits_{x \to 0}{f(x)}$ \pause
  \item Να βρείτε το $\lim\limits_{x \to 0}{\frac{f^2(x)+xf(x)+xημx}{f^2(x)+x^2+ημ^2x}}$
 \end{enumerate}
\end{frame}

\begin{frame}{Εξάσκηση}
 Να βρείτε το πεδίο ορισμού των συναρτήσεων:
 \begin{enumerate}
  \item $f(x)=\sqrt{|x|-|ημx|}$ \pause
  \item $f(x)=\frac{1}{ημ^2x-x^2}$
 \end{enumerate}
\end{frame}

\begin{frame}{Εξάσκηση}
 Να λύσετε την εξίσωση $|ημx|=|π-x|$
\end{frame}

\begin{frame}{Εξάσκηση}
 \begin{enumerate}
  \item Να λύσετε την εξίσωση $ημ(x^2+x)-x^2=x$ \pause
  \item Δίνεται η συνάρτηση $f(x)=2x+3\frac{ημx}{x}$. Να αποδείξετε ότι
        $$2x-3<f(x)<2x+3 \text{ για κάθε } x\ne 0$$
 \end{enumerate}
\end{frame}

\begin{frame}{Εξάσκηση}
 Να υπολογίσετε τα όρια
 \begin{enumerate}
  \item $\lim\limits_{x \to 0}{ \frac{x}{\sqrt[3]{x}} }$ \pause
  \item $\lim\limits_{x \to 0}{ \frac{x}{\sqrt[3]{x^2}} }$
 \end{enumerate}
\end{frame}

\section{}
\begin{frame}
 Στο moodle θα βρείτε τις ασκήσεις που πρέπει να κάνετε, όπως και αυτή τη παρουσίαση
\end{frame}

\end{document}
