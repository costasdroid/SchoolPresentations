\documentclass[greek]{beamer}
%\usepackage{fontspec}
\usepackage{amsmath,amsthm}
\usepackage{unicode-math}
\usepackage{xltxtra}
\usepackage{graphicx}
\usetheme{CambridgeUS}
\usecolortheme{seagull}
\usepackage{hyperref}
\usepackage{ulem}
\usepackage{xgreek}

\usepackage{pgfpages}
\usepackage{tikz}
\usepackage{tkz-tab}
%\setbeameroption{show notes on second screen}
%\setbeameroption{show only notes}

\setsansfont{Noto Serif}

\usepackage{multicol}

\usepackage{appendixnumberbeamer}

\setbeamercovered{transparent}
\beamertemplatenavigationsymbolsempty

\title{Συναρτήσεις}
\subtitle{Ρυθμός Μεταβολής}
\author[Λόλας]{Κωνσταντίνος Λόλας}
\date{}

\begin{document}

\begin{frame}
 \titlepage
\end{frame}

\section{Θεωρία}
\begin{frame}{Καλώς ορίσατε στα Oscar των μαθηματικών}
$x_1^2$
 Τα μαθηματικά είναι ωραία γιατί:
 \begin{enumerate}
  \item<1-> ακολουθούν κανόνες
  \item<2-> είναι σαφώς ορισμένα
  \item<3-> δεν δίνουν διαφορετικές ερμηνίες
  \item<4-> δεν είναι για όλους, αλλά κυρίως
  \item<5-> ενώ θα ήταν βαρετά από μόνα τους, εφαρμόζουν ΠΛΗΡΩΣ στη φυσική
 \end{enumerate}
 \onslide<6-> Γι αυτό μάλιστα δεν αρέσουν σε όλους
\end{frame}

\begin{frame}{Ρυθμός μεταβολής}
 \begin{block}{Ρυθμός μεταβολής του μεγέθους $A$}
  Είναι το πηλίκο
  $$\frac{ΔA}{Δt}$$
 \end{block}
\end{frame}

\begin{frame}{Ερμηνίες}
 Καλός ο ορισμός αλλά μιλάει για μεταβολή ή πιο σωστά για μέση μεταβολή, π.χ.
 \begin{enumerate}
  \item<1-> μέση ταχύτητα $v=\frac{Δx}{Δt}=\frac{x(t)-x(t_0)}{t-t_0}$
  \item<2-> επιτάχυνση $α=\frac{Δv}{Δt}$
  \item<3-> δύναμη $F=\frac{ΔP}{Δt}$
 \end{enumerate}
 \onslide<4-> Τι γίνεται με τη στιγμιαία ταχύτητα, επιτάχυνση, δύναμη κτλ?
\end{frame}

\begin{frame}{Στιγμιαία λοιπόν}
 \begin{enumerate}
  \item<1-> ταχύτητα $v=\frac{dx}{dt}=\lim\limits_{Δt \to 0}{ \frac{x(t_0+Δt) - x(t_0)}{Δt} }$
  \item<2-> επιτάχυνση $α=\frac{dv}{dt}$
  \item<3-> δύναμη $F=\frac{dP}{dt}$
 \end{enumerate}
 \onslide<4-> ή ποιό elegant
 \begin{enumerate}
  \item<5-> ταχύτητα $v(t)=x'(t)=\frac{dx}{dt}$
  \item<6-> επιτάχυνση $a(t)=v'(t)=\frac{dv}{dt}$
  \item<7-> δύναμη $F=P'(t)$
 \end{enumerate}
\end{frame}

\begin{frame}{Επιστροφή στα μαθηματικά!}
 Αν και δεν μου αρέσει, στα μαθηματικά ορίζεται
 \begin{block}{Ρυθμός μεταβολής του μεγέθους $A$ ως προς την μεταβλητή $B$}
  Είναι το πηλίκο
  $$Α'(Β)=\frac{dA}{dΒ}$$
 \end{block}
 ή αλλιώς η παράγωγος του $Α$ ως προς το $Β$
\end{frame}

\begin{frame}{One only rule}
 Σαν πολλά μας τα 'παν. Ρυθμός μεταβολής είναι η παράγωγος! ΤΕΛΟΣ
 \begin{itemize}
  \item<2-> Άρα κάθε συνάρτηση αφού έχει παράγωγο έχει και ρυθμό μεταβολής
  \item<3-> Δεν υπάρχει άλλη μεταβλητή πέρα από αυτή που παραγωγίζουμε
  \item<4-> Άρα όλα είναι συναρτήσεις εκτός από ΤΗΝ \emph{μεταβλητή}
   \begin{itemize}
    \item<5-> $x'=x'$
    \item<6-> $(x^2)'=2xx'$
    \item<7-> $(xy+y^3\ln x)'=x'y+xy'+3y^2y'\ln x+y^3\frac{1}{x}x'$
   \end{itemize}
 \end{itemize}
\end{frame}

\begin{frame}{Κύριε μας μπερδεύει ο συμβολισμός}
 Πολλοί, για να είναι σίγουροι κρατάνε το $x'(t)=\frac{dx(t)}{dt}$ και γράφουνε πάντα τις συναρτήσεις. π.χ.
 $$(x^2(t))'=\frac{dx^2(t)}{dt}=\frac{dx^2(t)}{dx(t)}\frac{dx(t)}{dt}=2x(t)x'(t)$$
\end{frame}

\section{Ασκήσεις}
\subsection{Άσκηση 1}
\begin{frame}[label=Άσκηση1]{Εξάσκηση 1}
 Δίνεται η συνάρτηση $f(x)=x^3-3x^2+1$
 \begin{enumerate}
  \item<1-> Να βρείτε το ρυθμό μεταβολής της $f$ ως προς το $x$ στο σημείο με $x=1$
  \item<2-> Να βρείτε τις τιμές του $x$, που ο ρυθμός μεταβολής της $f$ ως προς το $x$ είναι αρνητικός
 \end{enumerate}

 % \hyperlink{Λύση1}{\beamerbutton{Λύση}}
\end{frame}

\subsection{Άσκηση 2}
\begin{frame}[label=Άσκηση2]{Εξάσκηση 2}
 Το εμβαδό $Ε$ ενός τετραγώνου αυξάνει. Η πλευρά του $α$ σε cm, που αυξάνει, δίνεται από τον τύπο $α=3t+2$, όπου $t$ ο χρόνος σε sec.
 \begin{enumerate}
  \item<1-> Να αποδείξετε ότι $Ε=Ε(t)=(3t+2)^2$
  \item<2-> Να βρείτε το ρυθμό μεταβολής του εμβαδού $Ε$ του τετραγώνου, όταν $t=2$ sec.
 \end{enumerate}

 % \hyperlink{Λύση2}{\beamerbutton{Λύση}}
\end{frame}

\subsection{Άσκηση 3}
\begin{frame}[label=Άσκηση3]{Εξάσκηση 3}
 Δύο κινητά $Α$ και $Β$ ξεκινούν συγχρόνως από την αρχή των αξόνων $Ο$. Το $Α$ κινείται στον ημιάξονα $Οx$ με ταχύτητα $6cm/sec$ και το $Β$ στον ημιάξονα $Οy$ με ταχύτητα $8cm/sec$.
 \begin{enumerate}
  \item<1-> Να βρείτε τις συναρτήσεις θέσεως των $Α$ και $Β$
  \item<2-> Να βρείτε τη χρονική στιγμή που η απόσταση των $Α$ και $Β$ είναι $50cm$
  \item<3-> Να αποδείξετε ότι η απόσταση $d=(ΑΒ)$ των δύο κινητών αυξάνεται με σταθερό ρυθμό τον οποίο και να προσδιορίσετε.
 \end{enumerate}

 % \hyperlink{Λύση3}{\beamerbutton{Λύση}}
\end{frame}

\subsection{Άσκηση 4}
\begin{frame}[label=Άσκηση4]{Εξάσκηση 4}
 Ένα κινητό $Μ$ κινείται κατά μήκος της καμπύλης $y=\sqrt{x}$ ξεκινώντας από το $Ο$ και η τετμημένη του $x$ αυξάνεται με ρυθμό $4cm/sec$
 \begin{enumerate}
  \item<1-> Να αποδείξετε ότι η τετμημένη του κινητού για κάθε χρονική στιγμή $t$, $t\ge 0$ δίνεται από τον τύπο $x(t)=4t$.
  \item<2-> Να βρείτε το χρόνο που χρειάζεται το κινητό να φθάσει στο σημείο $(4,2)$
  \item<3-> Να βρείτε το ρυθμό μεταβολής της τεταγμένης του $Μ$ καθώς περνάει από το σημείο $Β(16,4)$
 \end{enumerate}

 % \hyperlink{Λύση4}{\beamerbutton{Λύση}}
\end{frame}

\subsection{Άσκηση 5}
\begin{frame}[label=Άσκηση5]{Εξάσκηση 5}
 Οι διαστάσεις $x$ και $y$ ενός ορθογωνίου μεταβάλλονται. Το $x$ αυξάνει με ρυθμό $2cm/sec$ και το $y$ ελαττώνεται με ρυθμό $3cm/sec$. Να βρείτε το ρυθμό μεταβολής:
 \begin{enumerate}
  \item<1-> Της περιμέτρου
  \item<2-> Του εμβαδού $Ε$ του ορθογωνίου τη χρονική στιγμή που είναι $x=10cm$ και $y=12cm$
 \end{enumerate}

 % \hyperlink{Λύση5}{\beamerbutton{Λύση}}
\end{frame}

\subsection{Άσκηση 6}
\begin{frame}[label=Άσκηση6]{Εξάσκηση 6}
 Έστω $Ε$ το εμβαδό του τριγώνου $ΟΑΜ$ που περικλείεται από την ευθεία $ε:y=x$, το άξονα $x'x$ και την ευθεία $x=λ$, $λ>0$.
 \begin{enumerate}
  \item<1-> Να αποδείξετε ότι $Ε=\frac{1}{2}λ^2$
  \item<2-> Αν το $λ$ αυξάνεται με ρυθμό $3cm/s$, να βρείτε το ρυθμό μεταβολής του εμβαδού $Ε$, όταν $λ=2cm$
 \end{enumerate}

 % \hyperlink{Λύση6}{\beamerbutton{Λύση}}
\end{frame}

\subsection{Άσκηση 7}
\begin{frame}[label=Άσκηση7]{Εξάσκηση 7}
 Ένα σημείο $Μ$ κινείται κατά μήκος της καμπύλης $y=x^2$, $x\ge0$ ξεκινώντας από την αρχή των αξόνων $Ο$.
 \begin{enumerate}
  \item<1-> Αν ο ρυθμός μεταβολής $x'(t)$ της τετμημένης του σημείου $Μ$ είναι $2cm/s$, να βρείτε το χρόνο που θα χρειαστεί για να φτάσει στο σημείο $Β(4,16)$
  \item<2-> Να βρείτε σε ποιο σημείο της καμπύλης ο ρυθμός μεταβολής της τεταγμένης $y$ του $Μ$ είναι διπλάσιος του ρυθμού μεταβολής της τετμημένης του $x$ αν υποτεθεί ότι $x'(t)>0$, για κάθε $t\ge0$
  \item<3-> Καθώς το $Μ$ περνάει από το $Α(2,4)$, η τετμημένη του ελαττώνεται με ρυθμό $3cm/s$. Να βρείτε το ρυθμό μεταβολής της τεταγμένης $y$ του $Μ$ τη χρονική στιγμή που περνάει από το $Α$
 \end{enumerate}

 %\hyperlink{Λύση7}{\beamerbutton{Λύση}}
\end{frame}

\subsection{Άσκηση 8}
\begin{frame}[label=Άσκηση8]{Εξάσκηση 8}
 Ένα κινητό κινείται σε ελλειπτική τροχιά με εξίσωση $4x^2+y^2=4$. Καθώς περνάει από το σημείο $Α(\frac{1}{2},\sqrt{3})$ η τετμημένη του $x$ ελαττώνεται με ρυθμό $2$ μονάδες το δευτερόλεπτο. Να βρείτε το ρυθμό μεταβολής της τεταγμένης του $y$ τη χρονική στιγμή που το κινητό περνάει από το $Α$.

 %\hyperlink{Λύση8}{\beamerbutton{Λύση}}
\end{frame}

\subsection{Άσκηση 9}
\begin{frame}[label=Άσκηση9]{Εξάσκηση 9}
 Ένα κινητό κινείται στη καμπύλη $C:y=e^x$. Καθώς το $Μ$ περνάει από το σημείο $Α(0,1)$, η τετμημένη του $x$ αυξάνει με ρυθμό $3$ μονάδες το δευτερόλεπτο. Να βρείτε το ρυθμό μεταβολής της απόστασης $l=(ΟΜ)$ τη χρονική στιγμή που το κινητό περνάει από το $Α$.

 %\hyperlink{Λύση9}{\beamerbutton{Λύση}}
\end{frame}

\subsection{Άσκηση 10}
\begin{frame}[label=Άσκηση10]{Εξάσκηση 10}
 Ένα κινητο $Μ$ κινείται στην καμπύλη $C:y=x^3$. Καθώς το $Μ$ περνάει από το σημείο $Α(1,1)$, η τετμημένη του $x$ ελαττώνεται με ρυθμό $2$ μονάδες το δευτερόλεπτο. Να βρείτε το ρυθμό μεταβολής της γωνίας $θ=\hat{ΜΟx}$ τη χρονική στιγμή που το κινητό περνάει από το $Α$.

 %\hyperlink{Λύση10}{\beamerbutton{Λύση}}
\end{frame}

\subsection{Άσκηση 11}
\begin{frame}[label=Άσκηση11]{Εξάσκηση 11}
 Μία σκάλα μήκους $5m$ είναι τοποθετημένη σ' έναν τοίχο. Το κάτω μέρος της σκάλας $Β$ γλιστράει στο δάπεδο με σταθερό ρυθμό $0,3m/s$. Τη χρονική στιγμή $t_0$ που η κορυφή της σκάλας απέχει από το δάπεδο $3m$, να βρείτε τη ταχύτητα με την οποία πέφτει η κορυφή $Α$ της σκάλας.

 %\hyperlink{Λύση11}{\beamerbutton{Λύση}}
\end{frame}

\subsection{Άσκηση 12}
\begin{frame}[label=Άσκηση12]{Εξάσκηση 12}
 Μία γυναίκα ύψους $2m$ απομακρύνεται από τη βάση ενός φανοστάτη ύψους $10cm$ με ταχύτητα $0,5m/s$. Με ποια ταχύτητα αυξάνεται ο ίσκιος της?

 %\hyperlink{Λύση12}{\beamerbutton{Λύση}}
\end{frame}

\subsection{Άσκηση 13}
\begin{frame}[label=Άσκηση13]{Εξάσκηση 13}
 Δίνεται η συνάρτηση $f(x)=x^2$, $x\le 0$.
 \begin{enumerate}
  \item<1-> Να βρείτε την τετμημένη του σημείο τομής $Μ$ της εφαπτομένης της $C_f$ στο σημείο της $Α(a,f(a))$, $a\ne 0$ με τον άξονα $x'x$.
  \item<2-> Έστω ότι το σημείο $Α$ κινείται κατά μήκος της $C_f$ και ο ρυθμός μεταβολής του $a(t)$ δίνεται από τον τύπο $a'(t)=2a(t)$. Να βρείτε το ρυθμό μεταβολής της τετμημένης του σημείου $Μ$ του προηγούμενου ερωτήματος τη χρονική στιγμή που το $Α$ έχει τετμημένη $-2$
  \item<3-> Να βρείτε το ρυθμό μεταβολής της γωνίας $θ$ που σχηματίζει η εφαπτομένη της $C_f$ στο $Α$ με τον $x'x$ την ίδια χρονική στιγμή με το 2. ερώτημα
 \end{enumerate}

 %\hyperlink{Λύση13}{\beamerbutton{Λύση}}
\end{frame}


%
% \appendix
% \section{Λύσεις Ασκήσεων}
% \begin{frame}
%  \tableofcontents
% \end{frame}
%
% \subsection{Άσκηση 1}
% \begin{frame}[label=Λύση1]
%  Με θεώρημα ενδιαμέσων τιμών. Η συνάρτηση είναι συνεχής στο $[10,11]$ με $f(10)=1024$ και $f(11)=2048$. Αφού $2023\in (1024,2048)$ υπάρχει $x_0$...
%
%  \hyperlink{Άσκηση1}{\beamerbutton{Πίσω στην άσκηση}}
% \end{frame}
%
% \subsection{Άσκηση 2}
% \begin{frame}[label=Λύση2]
%  Με Bolzano ή με μέγιστης ελάχιστης τιμής και ΘΕΤ.
%
%  \begin{gather*}
%   f(3)<f(2)<f(1) \\
%   3f(3)<f(1)+f(2)+f(3)<3f(1) \\
%   f(3)<\frac{f(1)+f(2)+f(3)}{3}<f(1)
%  \end{gather*}
%
%  \hyperlink{Άσκηση2}{\beamerbutton{Πίσω στην άσκηση}}
% \end{frame}
%
% \subsection{Άσκηση 3}
% \begin{frame}[label=Λύση3]
%  Προφανές ελάχιστο στα $x_1=1$ και $x_2=3$. Ως συνεχής στο $[1,3]$ έχει σίγουρα ΚΑΙ μέγιστο στο $(1,3)$
%
%  \hyperlink{Άσκηση3}{\beamerbutton{Πίσω στην άσκηση}}
% \end{frame}
%
% \subsection{Άσκηση 4}
% \begin{frame}[label=Λύση4]
%  Η συνάρτηση `απόστασης` $f(x)-x$ είναι ορισμένη στο κλειστό διάστημα και έχει σίγουρα μέγιστο
%
%  \hyperlink{Άσκηση4}{\beamerbutton{Πίσω στην άσκηση}}
% \end{frame}
%
% \subsection{Άσκηση 5}
% \begin{frame}[label=Λύση5]
%  Όμοια με την Άσκηση 2
%
%  \hyperlink{Άσκηση5}{\beamerbutton{Πίσω στην άσκηση}}
% \end{frame}
%
% \subsection{Άσκηση 6}
% \begin{frame}[label=Λύση6]
%  \begin{enumerate}
%   \item Είναι γνησίως αύξουσα άρα $(f(+\infty),f(-\infty))$
%   \item Προφανώς $[f(0),f(1)]$...
%  \end{enumerate}
%
%  \hyperlink{Άσκηση6}{\beamerbutton{Πίσω στην άσκηση}}
% \end{frame}

\end{document}
