\documentclass[greek]{beamer}
\usepackage{amsmath,amsthm} % needed for mathematics
\usepackage{unicode-math}
\usepackage{xltxtra}
\usepackage{graphicx}
\usetheme{CambridgeUS}
\usecolortheme{seagull}
\usepackage{hyperref}
\usepackage{ulem} % underline words package
\usepackage{xgreek}

\usepackage{pgfpages} 
\usepackage{tikz} % package for shapes and more
%\setbeameroption{show notes on second screen}
%\setbeameroption{show only notes}

\setsansfont{Calibri} % it is said that Calibri is the proper font for reading difficulties

\usepackage{multicol} % package for two or more columns

\usepackage{appendixnumberbeamer} % remove page numbering in appendix

\usepackage{polynom} % polynomial divisions package

\usepackage{pgffor} % macros

\setbeamercovered{highly dynamic}
\setbeamertemplate{navigation symbols}{}

\newcounter{askisi} % enviroment for exercises
\newenvironment{askisi}
{
  \refstepcounter{askisi}\par
  \subsection{Άσκηση \theaskisi}
  \begin{frame}[label=Άσκηση\theaskisi,t]{Εξάσκηση \theaskisi}
}{
  \end{frame}
}

\newcounter{lisi} % enviroment for solutions
\newenvironment{lisi}
{
  \refstepcounter{lisi}\par
  \subsection{Άσκηση \thelisi}
  \begin{frame}[label=Λύση\thelisi,t]{Λύση \thelisi}
}{
  \end{frame}
}

\title{Συναρτήσεις}
\subtitle{Συνέπειες Bolzano 1 (Διατήρηση Προσήμου)}
\author[Λόλας]{Κωνσταντίνος Λόλας}
\institute[$10^ο$ ΓΕΛ]{$10^ο$ ΓΕΛ Θεσσαλονίκης}
\date{}

\begin{document}

\begin{frame}
  \titlepage
\end{frame}

\section{Θεωρία}
\begin{frame}{Ένα μάθημα μόνο για το πρόσημο?}
  \begin{itemize}[<+->]
    \item Φτιάξτε άξονες
    \item Σχεδιάστε όσες συναρτήσεις μπορείτε που ισχύει $f^2(x)=1$ για κάθε $x\in \mathbb{R}$
  \end{itemize}
  \onslide<3>Συμπέρασμα...
\end{frame}

\begin{frame}{Ένα μάθημα μόνο για το πρόσημο?}
  \begin{itemize}[<+->]
    \item Φτιάξτε άξονες
    \item Σχεδιάστε όσες συνεχείς στο $\mathbb{R}$ συναρτήσεις μπορείτε που ισχύει $f^2(x)=1$ για κάθε $x\in \mathbb{R}$
  \end{itemize}
  \onslide<3>Συμπέρασμα...
\end{frame}

\begin{frame}{Θεώρημα 1}
  \begin{block}{Θεώρημα σταθερού προσήμου}
    Έστω μια συνάρτηση $f$ συνεχής στο διάστημα $Δ$. Αν $f(x)\ne 0$ για κάθε $x\in Δ$ τότε η $f$ διατηρεί το πρόσημο της σε όλο το $Δ$
  \end{block}
\end{frame}

\begin{frame}{Θεώρημα 2}
  \begin{block}{Θεώρημα σταθερού προσήμου (γενίκευση)}
    Μια συνεχής συνάρτηση $f$ διατηρεί το πρόσημό της μεταξύ δύο διαδοχικών της ριζών.
  \end{block}
\end{frame}

\begin{frame}[noframenumbering]
  Στο moodle θα βρείτε τις ασκήσεις που πρέπει να κάνετε, όπως και αυτή τη παρουσίαση
\end{frame}

\section{Ασκήσεις}

\begin{frame}[noframenumbering]
  \vfill
  \centering
  \begin{beamercolorbox}[sep=8pt,center,shadow=true,rounded=true]{title}
    \usebeamerfont{title}Ασκήσεις
  \end{beamercolorbox}
  \vfill
\end{frame}

\begin{askisi}
  Έστω $f:\mathbb{R}\to\mathbb{R}$ μία συνάρτηση με $f(0)=1$ η οποία είναι συνεχής και ισχύει $f(x)\ne 0$ για κάθε $x\in\mathbb{R}$. Να βρείτε το πεδίο ορισμού της συνάρτησης $g(x)=\ln f(x)$
\end{askisi}

\begin{askisi}
  Έστω $f:\mathbb{R}\to\mathbb{R}$ μία συνάρτηση η οποία είναι συνεχής και ισχύει $f^2(x)>0$ για κάθε $x\in\mathbb{R}$. Να βρείτε το
  $$\lim\limits_{x \to +\infty}{ \frac{f(1)x^2+1}{f(0)x+2} }$$
\end{askisi}

\begin{askisi}
  Έστω $f:\mathbb{R}\to\mathbb{R}$ μία συνάρτηση η οποία είναι συνεχής. Αν $f(3)=-2$ και $x_1=1$ και $x_2=4$ είναι διαδοχικές ρίζες της εξίσωσης $f(x)=0$, να βρείτε το
  $$\lim\limits_{x \to +\infty}{ f(2)x^3-x+1 }$$
\end{askisi}

\begin{askisi}
  Να βρείτε το πρόσημο των συναρτήσεων
  \begin{itemize}[<+->]
    \item $f(x)=2x^3-x-1$
    \item $f(x)=x-ημx$
    \item $f(x)=\sqrt{x^2+1}-x$
  \end{itemize}
\end{askisi}

\begin{askisi}
  Να βρείτε το πρόσημο της συνάρτησης $f(x)=2ημx-1$, $x\in [0,π]$.
\end{askisi}

\begin{askisi}
  Έστω $f:\mathbb{R}\to\mathbb{R}$ μία συνάρτηση η οποία είναι συνεχής και ισχύει
  $$|f(x)|=e^x \text{ για κάθε } x\in\mathbb{R}$$
  \begin{itemize}[<+->]
    \item Να αποδείξετε ότι $f(x)\ne 0$, για κάθε $x\in\mathbb{R}$
    \item Αν $f(0)=-1$ να βρείτε τον τύπο της $f$
  \end{itemize}
\end{askisi}

\begin{askisi}
  Έστω $f:\mathbb{R}\to\mathbb{R}$ μία συνάρτηση με $f(0)=1$ η οποία είναι συνεχής και ισχύει $f^2(x)=x^2+1$ για κάθε $x\in\mathbb{R}$. Να βρείτε τον τύπο της $f$.
\end{askisi}

\begin{askisi}
  Να βρείτε τη συνεχή συνάρτηση $f$ με $f(0)=1$ για την οποία ισχύει $f^2(x)=1+2xf(x)$ για κάθε $x\in\mathbb{R}$
\end{askisi}

\begin{askisi}
  Έστω $f:\mathbb{R}\to\mathbb{R}$ μία συνάρτηση με $f(2)=2$ η οποία είναι συνεχής και ισχύει $f^2(x)+2=x+2f(x)$ για κάθε $x\in [1,+\infty]$. Να βρείτε τον τύπο της $f$.
\end{askisi}

\begin{askisi}
  Έστω $f:[-1,1]\to\mathbb{R}$ μία συνάρτηση με $f(0)=-1$ η οποία είναι συνεχής και ισχύει $x^2+f^2(x)=1$, $x\in [-1,1]$. Να βρείτε τον τύπο της $f$.
\end{askisi}

\begin{askisi}
  Έστω $f:[-1,1]\to\mathbb{R}$ μία συνεχής συνάρτηση για την οποία ισχύει $4x^2+f^2(x)=4$ για κάθε $x\in [-1,1]$
  \begin{itemize}[<+->]
    \item Να βρείτε τις ρίζες της εξίσωσης $f(x)=0$
    \item Να δείξετε ότι η $f$ διατηρεί το πρόσημό της στο $(-1,1)$
    \item Ποιος μπορεί να είναι ο τύπος της $f$;
    \item Αν $f(0)=2$, να βρείτε την $f$
  \end{itemize}
\end{askisi}

\begin{askisi}
  Να βρείτε όλες τις συνεχείς συναρτήσεις $f:\mathbb{R}\to\mathbb{R}$ που ικανοποιούν τη σχέση
  $$f^2(x)+2x=x^2+1 \text{, } x\in\mathbb{R}$$
\end{askisi}

\end{document}
