\documentclass{../presentation}

\title{Συναρτήσεις}
\subtitle{Εμβαδό Επίπεδου Χωρίου}
\author[Λόλας]{Κωνσταντίνος Λόλας}
\institute[$10^ο$ ΓΕΛ]{$10^ο$ ΓΕΛ Θεσσαλονίκης}


\begin{document}

\begin{frame}
  \titlepage
\end{frame}

\section{Θεωρία}

\begin{frame}[noframenumbering]
  Στο moodle θα βρείτε τις ασκήσεις που πρέπει να κάνετε, όπως και αυτή τη παρουσίαση
\end{frame}

\section{Ασκήσεις}

\exercises

\begin{askisi}
  Να υπολογίσετε το εμβαδόν E του χωρίου που περικλείεται από τον άξονα $x'x$, τη γραφική παράσταση της συνάρτησης:
  \begin{itemize}[<+->]
    \item $f(x)=3x^2$ και τις ευθείες $x=1$, $x=2$.
    \item $f(x)=ημx$ και τις ευθείες $x=-\frac{π}{2}$, $x=-\frac{π}{4}$.
  \end{itemize}
\end{askisi}

\begin{askisi}
  Δίνεται η συνάρτηση $f(x)=x^2-2x$. Να υπολογίσετε το εμβαδόν $E$ του χωρίου που περικλείεται από τη γραφική παράσταση της $f$, τον άξονα $x'x$ και τις ευθείες:
  \begin{itemize}[<+->]
    \item $x=1$, $x=2$
    \item $x=2$, $x=3$
    \item $x=1$, $x=3$
  \end{itemize}
\end{askisi}

\begin{askisi}
  Δίνεται η συνάρτηση $f(x)=x^2-2x$. Να υπολογίσετε το εμβαδόν $E$ του χωρίου που περικλείεται από τη γραφική παράσταση της $f$
  \begin{itemize}[<+->]
    \item και τον άξονα $x'x$.
    \item τον άξονα $x'x$ και τις ευθείες $x=0$, $x=3$.
  \end{itemize}
\end{askisi}

\begin{askisi}
  Δίνεται η συνάρτηση $f(x)=e^x-1$. Να βρείτε το εμβαδόν $E$ του χωρίου που περικλείεται από τη γραφική παράσταση της $f$, τον άξονα $x'x$ και την ευθεία $x=1$.
\end{askisi}

\begin{askisi}
  Να υπολογίσετε το εμβαδόν του χωρίου $E$ που περικλείεται από τον άξονα $x'x$, τη γραφική παράσταση της συνάρτησης:
  \begin{itemize}[<+->]
    \item $f(x)=x^3-x^2+2x-1$ και τις ευθείες $x=1$, $x=2$.
    \item $f(x)=ημx-2x$ και την ευθεία $x=π$.
  \end{itemize}
\end{askisi}


\begin{askisi}
  Δίνεται η συνάρτηση $f(x)=\begin{cases}
      3x^2     & x \leq 0 \\
      e^{-x}-1 & x>0
    \end{cases}$
  \begin{itemize}[<+->]
    \item Να βρείτε το εμβαδόν $E$ του χωρίου $Ω$ που περικλείεται από τη γραφική παράσταση της $f$, τον άξονα $x'x$ και τις ευθείες $x=-1$, $x=1$.
    \item Να βρείτε την ευθεία $x=a$ που χωρίζει το χωρίο $Ω$ σε δύο ισεµβαδικά χωρία.
  \end{itemize}
\end{askisi}

\begin{askisi}
  Να βρείτε το εμβαδόν του χωρίου που περικλείεται από τις γραφικές παραστάσεις των συναρτήσεων $f(x)=\ln x$ και $g(x)=e^x$ και τις ευθείες $x=1$, $x=e$.
\end{askisi}


\begin{askisi}
  Δίνονται οι συναρτήσεις $f(x)=x^2$ και $g(x)=x+2$. Να υπολογίσετε το εμβαδόν του $E$ του χωρίου που περικλείεται από:
  \begin{itemize}[<+->]
    \item Τις γραφικές παραστάσεις των συναρτήσεων $f$ και $g$.
    \item Τις $C_f$, $C_g$, και τις ευθείες:
          \begin{itemize}[<+->]
            \item $x=2$, $x=3$
            \item $x=-1$, $x=3$
          \end{itemize}
  \end{itemize}
\end{askisi}


\begin{askisi}
  Δίνεται η συνάρτηση $f(x)=4-x^2$. Να βρείτε την τιµή του $a$, ώστε το χωρίο που περικλείεται από τη $C_f$ και την ευθεία $y=2$, να χωρίζεται από την ευθεία $y=4-a^2$, $0<a<\sqrt{2}$ σε δύο ισεµβαδικά χωρία.
\end{askisi}

\begin{askisi}
  Να βρείτε το εμβαδόν του χωρίου $E$ που περικλείεται από τη γραφική παράσταση της συνάρτησης $f(x)=ημx$, την εφαπτομένη $ε$ της $C_f$ στο $x_0=0$ και την ευθεία $x=\frac{π}{2}$.
\end{askisi}

\begin{askisi}
  Δίνεται η συνάρτηση $f(x)=x- \dfrac{\ln x}{x^2}$. Να βρείτε το εμβαδόν $E$ του χωρίου που περικλείεται την πλάγια ασύμπτωτη της $C_f$ στο $+\infty$ και την ευθεία $x=2$.
\end{askisi}

\begin{askisi}
  Δίνεται η συνάρτηση $f(x)=e^x$. Να βρείτε το εμβαδόν $E$ του χωρίου που περικλείεται από τη γραφική παράσταση της $f$, την εφαπτομένη $ε$ της $C_f$ που διέρχεται από την αρχή των αξόνων και την ευθεία $x=-1$.
\end{askisi}

\begin{askisi}
  Δίνεται η συνάρτηση $f(x) = -x^2 + 2x$.
  \begin{enumerate}[<+->]
    \item Να βρείτε τις εφαπτόμενες της $C_f$ που διέρχονται από το σημείο $A(1, 2)$.
    \item Αν $O$, $Β$ είναι τα κοινά σημεία της $C_f$ με τον άξονα $x'x$, να αποδείξετε ότι η $C_f$ χωρίζει το τρίγωνο $AOB$ σε δύο χωρία, ώστε που λόγος των εμβαδών τους να είναι $2:1$.
  \end{enumerate}
\end{askisi}

\begin{askisi}
  Δίνεται η συνάρτηση $f(x) = e^x +x- 1$.
  \begin{enumerate}[<+->]
    \item Να δείξετε ότι η $f$ είναι αντιστρέψιμη και να βρείτε το πρόσημο της $f^{-1}$.
    \item Να βρείτε το εμβαδόν του χωρίου $Ω$ που περικλείεται από τη γραφική παράσταση της $f^{-1}$, τον άξονα $x'x$ και τις ευθείες $x=0$, $x=e$.
  \end{enumerate}
\end{askisi}


\begin{askisi}
  Δίνεται η συνάρτηση $f(x)=x^3-x^2+x$.
  \begin{enumerate}[<+->]
    \item Να δείξετε ότι η $f$ αντιστρέφεται.
    \item Να βρείτε το εμβαδόν του χωρίου $Ω$ που περικλείεται από τις $C_f$ και $C_{f^{-1}}$.
  \end{enumerate}
\end{askisi}

\begin{askisi}
  Δίνεται η συνάρτηση $f(x) = \frac{x+2-e^x}{1+e^x}$.
  \begin{enumerate}[<+->]
    \item Να δείξετε ότι η ευθεία $ε: y = x+2$ είναι ασύμπτωτη της $C_f$ στο $-\infty$.
    \item Να βρείτε το εμβαδόν $E(a)$ του χωρίου $Ω$ που περικλείεται από τη $C_f$, την $ε$, τον άξονα $y'y$ και την ευθεία $x=a$, $a<0$.
    \item Να βρείτε το όριο $\lim_{a \to -\infty} E(a)$.
    \item Αν το $a$ ελαττώνεται με ρυθμό $2 \, \text{μον}/\text{sec}$, να βρείτε το ρυθμό μεταβολής του εμβαδού $E(a)$ τη χρονική στιγμή που είναι $a = -\ln 2$.
  \end{enumerate}
\end{askisi}

\begin{askisi}
  Δίνεται η συνάρτηση $f(x) = \ln(x+\sqrt{x^2+1})$, $x \in \mathbb{R}$.
  \begin{enumerate}[<+->]
    \item Να δείξετε ότι η $f$ είναι αντιστρέψιμη.
    \item Να βρείτε το σημείο καμπής της $C_f$.
    \item Να υπολογίσετε το εμβαδόν του χωρίου $Ω$ που περικλείεται από την $C_f$, τον άξονα συμμετρίας των $C_f$ και $C_{f^{-1}}$, και την ευθεία $x=1$.
  \end{enumerate}
\end{askisi}

\begin{askisi}
  Έστω $f:[-1,2] \to \mathbb{R}$ μία συνάρτηση, η οποία είναι παραγωγίσιμη με συνεχή παράγωγο και η γραφική παράσταση της $f'$ φαίνεται στο διπλανό σχήμα. Αν ισχύουν $f(-1)=0$ και $E(Ω_1) = 2E(Ω_2) = 2E(Ω_3) = 2$.
  \begin{enumerate}[<+->]
    \item Να βρείτε τις τιμές $f(0)$, $f(1)$ και $f(2)$.
    \item Να μελετήσετε τη συνάρτηση $f$ ως προς τη μονοτονία και τα ακρότατα.
    \item Να μελετήσετε τη συνάρτηση $f$ ως προς την κυρτότητα και να βρείτε τα σημεία καμπής της γραφικής παράστασης της $f$.
    \item Να υπολογίσετε το όριο $\lim_{x \to -1} \frac{ημ(x+1)}{f(x)}$.
  \end{enumerate}
\end{askisi}

\end{document}