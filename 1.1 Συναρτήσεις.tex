\documentclass[greek]{beamer}
%\usepackage{fontspec}
\usepackage{amsmath,amsthm}
\usepackage{unicode-math}
\usepackage{xltxtra}
\usepackage{graphicx}
\usetheme{Warsaw}
\usecolortheme{seahorse}
\usepackage{hyperref}
\usepackage{ulem}
\usepackage{xgreek}
\usepackage{pgfpages}
%\setbeameroption{show notes on second screen}
%\setbeameroption{show only notes}

\setsansfont{DejaVu Sans}

% \newtheorem{definition}{Ορισμός}

\title{Συναρτήσεις}
\author[Λόλας]{Κωνσταντίνος. Λόλας}

\begin{document}

\begin{frame}
 \titlepage
\end{frame}
\begin{frame}{Ορισμός}
 \begin{block}{Ορισμός}
  Έστω $Α$ ένα υποσύνολο του $\mathbb{R}$. Ονομάζουμε πραγματική συνάρτηση με πεδίο ορισμού το $Α$ μια διαδικασία (κανόνα) $f$, με την οποία κάθε στοιχείο $x\in A$ αντιστοιχίζεται σε ένα μόνο πραγματικό αριθμό $y$. Το $y$ ονομάζεται τιμή της $f$ στο $x$ και συμβολίζεται με $f(x)$.
 \end{block}
\end{frame}

\begin{frame}{Τρόπος ορισμού}
 \begin{itemize}
  \item<1-> Απλός τύπος
        $$f(x)=x^2+2,x\in\mathbb{R}\quad g(a)=\frac{2}{ημa}, a\le1$$
  \item<2-> Ορισμένη κατά "κλάδους"
        $$
         f(x)=
         \begin{cases}
          x^2+2,         & x<2 \\
          \frac{2}{ημx}, & x>5 \\
          -\sqrt{2},     & x=3
         \end{cases}
        $$
  \item<3-> Περιγραφικά
 \end{itemize}
\end{frame}

\begin{frame}{Συμβολισμοί}
 \begin{itemize}
  \item $f:A\to\mathbb{R}$
  \item $x\to f(x)$
  \item $f(A)=\{y| y=f(x)\text{ για κάποιο }x\in A\}$
 \end{itemize}
\end{frame}

\begin{frame}{Πεδίο Ορισμού}
 \begin{itemize}
  \item<1-> Διαίρεση
        $$\frac{a}{b},b\ne 0$$
  \item<2-> Ρίζες
        $$\sqrt[n]{a},a\ge 0$$
  \item<3-> Λογάριθμοι
        $$\ln a,a > 0$$
  \item<4-> Κρυφά
        $$εφx,\quad x^x\ldots$$
 \end{itemize}
\end{frame}

\begin{frame}{Προαπαιτούμενα}
 Πρέπει να γνωρίζετε πολύ καλά
 \begin{itemize}
  \item Εξισώσεις
  \item Ανισώσεις
 \end{itemize}
\end{frame}

\begin{frame}
  Στο moodle θα βρείτε τις ασκήσεις που πρέπει να κάνετε, όπως και αυτή τη παρουσίαση
\end{frame}

\begin{frame}{Εξάσκηση}
  Δίνεται η συνάρτηση $f(x)=x^3-x+a$ με $f(-1)=1$
  \begin{enumerate}
    \item<1-> Να βρείτε την τιμή του $a$.
    \item<2-> Να λύσετε την εξίσωση $f(x)=1$.
  \end{enumerate}
\end{frame}

\begin{frame}{Εξάσκηση}
  Για τη συνάρτηση $f(x)=\begin{cases}
    2x+a^2, & x<3 \\
    x-3+b^2, & \ge3
  \end{cases}$, ισχύει $f(0)+f(3)=0$.
  \begin{enumerate}
    \item<1-> Να βρείτε το πεδίο ορισμού της.
    \item<2-> Να υπολογίσετε τα $a$ και $b$.
    \item<3-> Να βρείτε τις τιμές $f(\pi)$ και $f(e)$.
  \end{enumerate}
\end{frame}

\begin{frame}{Εξάσκηση}
  Να βρείτε το πεδίο ορισμού των συναρτήσεων
  \begin{enumerate}
    \item<1-> $\frac{1}{x-1}$
    \item<2-> $\frac{2x}{x^2-3x+2}$
    \item<3-> $\sqrt{x-1}$
    \item<4-> $\ln (x-1)$
    \item<5-> $\sqrt{x-1}-\sqrt[3]{2-x}$
    \item<6-> $\frac{\sqrt{1-x^2}}{x}$
    \item<7-> $\frac{\ln x}{x-1}$
    \item<8-> $\frac{x-1}{\ln x}$
    \item<9-> $\sqrt{\ln x}$
    \item<10-> $\frac{1}{\sqrt{x}-1}$
    \item<11-> $\ln \left(\sqrt{x-1}-x+3\right)$
  \end{enumerate}
\end{frame}

\end{document}
