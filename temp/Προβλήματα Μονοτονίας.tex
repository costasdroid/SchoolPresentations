\documentclass{../presentation}

\title{Συναρτήσεις}
\subtitle{Προβλήματα Μονοτονίας}
\author[Λόλας]{Κωνσταντίνος Λόλας}
\institute[$10^ο$ ΓΕΛ]{$10^ο$ ΓΕΛ Θεσσαλονίκης}
\date{}

\begin{document}

\begin{frame}
  \titlepage
\end{frame}

\section{Θεωρία}
\begin{frame}{Και κάτι ενδιαφέρον!}
  Φυσική:
  \begin{itemize}[<+->]
    \item ταχύτητα
    \item επιτάχυνση
    \item ρυθμός μεταβολής (ορμής, θερμοκρασίας, κλπ)
  \end{itemize}
  Οικονομία:
  \begin{itemize}[<+->]
    \item Κέρδος
    \item Κόστος
    \item Παραγωγή
  \end{itemize}
  new era:
  \begin{itemize}[<+->]
    \item Κοινωνική αποδοχή
    \item Πολιτική επιρροή
    \item Μετοχές
    \item Bitcoin
  \end{itemize}
  Μαθηματικά???
\end{frame}

\begin{askisi}{Μαθηματικά Λοιπόν}
  Με σύρμα $α$ μέτρων φτιάχνετε ένα ορθογώνιο.
  \begin{enumerate}[<+->]
    \item Τι τιμές μπορεί να πάρει το εμβαδό του;
    \item Ποια είναι η μέγιστη τιμή του εμβαδού;
    \item Τι διαστάσεις έχει;
  \end{enumerate}
\end{askisi}

\begin{askisi}{Επαγγελματίες!}
  Έχετε ένα ταξιδιωτικό γραφείο και σας παίρνει το 10ο ΓΕΛ για εκδρομή στην Βουδαπέστη. Το κόστος της εκδρομής είναι $500\euro$ το άτομο μέχρι τα 50 άτομα. Για κάθε άτομο που ξεπερνάει τα 50, το κόστος μειώνεται κατά $5\euro$ το άτομο. Πόσα άτομα πρέπει να δηλώσουν για να έχετε τα μέγιστα έσοδα;
\end{askisi}

\begin{askisi}{Εργολάβοι!}
  Έχετε ένα εργοστάσιο από την μία μεριά ενός ποταμού πλάτους $300$ μέτρων. Θέλετε να μεταφέρετε τα εμπορεύματα σας στην αποθήκη στην άλλη μεριά και σε απόσταση $500$ μέτρων. Χρειάζεται να κατασκευάσετε γέφυρα για να διασχύσετε το ποτάμι και δρόμο για να φτάσετε την αποθήκη. Η κατασκευή της γέφυρας κοστίζει $1000\euro$ το μέτρο και η κατασκευή του δρόμου $500\euro$ το μέτρο. Πόσα μέτρα πρέπει να είναι η γέφυρα και πόσα ο δρόμος για να είναι το κόστος της κατασκευής τους το ελάχιστο δυνατό;
\end{askisi}

\begin{frame}{Και πολλά άλλα!}

\end{frame}


\end{document}
