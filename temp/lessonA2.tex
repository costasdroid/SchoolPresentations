\documentclass[a4paper,12pt]{article}
\usepackage{amsmath, amssymb}
\usepackage{pgffor}
\usepackage{polynom}

\usepackage{amsmath}
\usepackage{unicode-math}
\usepackage{xltxtra}
\usepackage{xgreek}

\usepackage{polynom}

\setmainfont{Liberation Serif}

\usepackage{tabularx}

\pagestyle{empty}

\usepackage{geometry}
 \geometry{a4paper, total={190mm,280mm}, left=10mm, top=10mm}

 \usepackage{graphicx}
 \graphicspath{ {images/} }

\usepackage{wrapfig}
\usepackage{hyperref}

 \usepackage{float}

\usepackage{geometry}
\geometry{margin=1in}

\title{\textbf{Φύλο εργασίας: Σχήμα Horner}}
\author{Άλγεβρα Β' Λυκείου}
\date{}

\begin{document}
\maketitle

\section{Η Ανάγκη}
\begin{itemize}
  \item Υπολογίστε την τιμή του πολυωνύμου $f(x) = 2x^3 - 3x^2 + 4x - 5$ για $x = 2$.

        \vspace{1cm}
        \dotfill
  \item Θα κάνατε το ίδιο για $x = 2,53$?
\end{itemize}

\section*{Ο νέος Αλγόριθμος - Σχήμα Horner}
\begin{enumerate}
  \item Γράφουμε τους συντελεστές του πολυωνύμου σε μία λίστα.
  \item Σημειώνουμε το $ρ$ στα δεξιά της λίστας.
  \item Κατεβάζουμε τον πρώτο συντελεστή.
  \item Πολλαπλασιάζουμε το $ρ$ με το αποτέλεσμα και το προσθέτουμε στον συντελεστή.
  \item Επαναλαμβάνουμε το βήμα 4 μέχρι να φτάσουμε στον τελευταίο συντελεστή.
  \item Το τελευταίο αποτέλεσμα είναι η τιμή του πολυωνύμου - υπόλοιπο.
  \item Η τρίτη γραμμή είναι το πηλίκο της διαίρεσης.
\end{enumerate}

\section*{Παράδειγμα ${2x^3-5x^2+2x-1}$ για $x = 3$}

\polyhornerscheme[tutor=true, x=3, stage=1]{2x^3-5x^2+2x-1} \hspace{1cm}
\polyhornerscheme[tutor=true, x=3, stage=2]{2x^3-5x^2+2x-1} \hspace{1cm}
\polyhornerscheme[tutor=true, x=3, stage=3]{2x^3-5x^2+2x-1} \hspace{1cm} \vspace{1cm}

\polyhornerscheme[tutor=true, x=3, stage=4]{2x^3-5x^2+2x-1} \hspace{1cm}
\polyhornerscheme[tutor=true, x=3, stage=5]{2x^3-5x^2+2x-1}\hspace{1cm}
\polyhornerscheme[tutor=true, x=3, stage=6]{2x^3-5x^2+2x-1}\hspace{1cm}\vspace{1cm}

\polyhornerscheme[tutor=true, x=3, stage=7]{2x^3-5x^2+2x-1}\hspace{1cm}
\polyhornerscheme[tutor=true, x=3, stage=8]{2x^3-5x^2+2x-1}\hspace{1cm}
\polyhornerscheme[x=3, resultleftrule=true, resultrightrule=true, resultbottomrule=true]{2x^3-5x^2+2x-1}


\section{Εξάσκηση}

\begin{enumerate}
  \item Να κάνετε το σχήμα Horner για το πολυώνυμο $3x^3-2x^2+5x-1$ για $x = 2$.
        \vspace{3cm}
  \item Να γράψετε το πηλίκο και το υπόλοιπο της διαίρεσης.

        \vspace{1cm}
        \dotfill
  \item Είναι ο αριθμός 2 είναι ρίζα του πολυωνύμου;
  \item Το $x-2$ διαιρεί το πολυώνυμο;
\end{enumerate}

\section{Δοκιμασία}

\begin{enumerate}
  \item Άσκηση 10 Α Ομάδας. Να κάνετε την διαίρεση $(3x^2 - 2αx - 8α^2): (x - 2α)$
        \begin{enumerate}
          \item Με το σχήμα Horner.
          \item Με κάθετη διαίρεση.
          \item Ποιον τρόπο προτιμάτε και γιατί;
        \end{enumerate}
  \item Άσκηση 5 Α Ομάδας. Αν $P(x) = -2x^3 - 2x^2 - x + 2409$
        \begin{enumerate}
          \item Πώς θα βρίσκατε το $P(1)$ με Horner ή αντικατάσταση;
          \item Πώς θα βρίσκατε το $P(-11)$ με Horner ή αντικατάσταση;
          \item Ποιον τρόπο προτιμάτε και γιατί;
        \end{enumerate}
\end{enumerate}

\section{Διερεύνηση}

\begin{enumerate}
  \item Ποιές δυσκολίες μπορεί να αντιμετωπίσει κάποιος στη διαίρεση με το σχήμα Horner;
  \item Ποια είναι τα πλεονεκτήματα του σχήματος Horner;
  \item \textbf{Πώς λειτουργεί ο αλγόριθμος; Να γράψετε ένα πολυώνυμο σε μορφή που να φαίνονται τα βήματα του αλγόριθμου.}
  \item \textbf{Λειτουργεί ο αλγόριθμος σε διαίρεση με οποιοδήποτε πολυώνυμο πρώτου βαθμού π.χ. $3x-2$;}
\end{enumerate}

\section{Καθήκοντα}

Στο moodle υπάρχουν οι ασκήσεις που θα έχετε για το σπίτι και υποστηρικτικό υλικό.

\end{document}
