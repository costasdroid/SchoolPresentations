\documentclass[a4paper,12pt,greek]{article}
\usepackage{amsmath}
\usepackage{unicode-math}
\usepackage{xltxtra}
\usepackage{xgreek}

\setmainfont{Liberation Serif}

\usepackage{tabularx}

\pagestyle{empty}

\usepackage{geometry}
 \geometry{a4paper, total={190mm,280mm}, left=10mm, top=10mm}

 \usepackage{graphicx}
 \graphicspath{ {images/} }

 \usepackage{wrapfig}

 \usepackage{float}

\usepackage{geometry}
\geometry{margin=1in}

\title{\textbf{Φύλλο Εργασίας: Ρίζες Πολυωνύμων \\ και το Θεώρημα Υπολοίπου}}
\author{Μαθηματικά Β' Λυκείου}
\date{}

\begin{document}
\maketitle

\section*{Οδηγίες}
Λύστε προσεκτικά τα παρακάτω προβλήματα. Δείξτε όλα τα απαραίτητα βήματα και δικαιολογήστε τις απαντήσεις σας.

\section*{Μέρος 1: Κατανόηση των Ριζών των Πολυωνύμων}
\begin{enumerate}
  \item Προσδιορίστε εάν οι παρακάτω αριθμοί είναι ρίζες των πολυωνύμων χρησιμοποιώντας το Θεώρημα Υπολοίπου.
        \begin{enumerate}
          \item Είναι το $x = -1$ ρίζα του $f(x) = x^3 + x^2 - x - 1$;
          \item Είναι το $x = 3$ ρίζα του $f(x) = x^4 - 5x^3 + 6x^2 - 2x + 1$;
        \end{enumerate}
  \item Χρησιμοποιώντας τη διαίρεση πολυωνύμων, ελέγξτε αν το $x - 2$ είναι παράγοντας του $f(x) = x^3 - 5x + 6$.
\end{enumerate}

\section*{Μέρος 2: Ανάλυση Λαθών}
\begin{enumerate}
  \item Ένας μαθητής υπολογίζει το $f(2)$ για $f(x) = x^3 - 3x^2 + x - 3$ και βρίσκει $-5$. Επαληθεύστε αν η απάντησή του είναι σωστή.
  \item Εξηγήστε γιατί το Θεώρημα Υπολοίπου είναι χρήσιμο για τον προσδιορισμό των ριζών ενός πολυωνύμου αντί της παραγοντοποίησης.
\end{enumerate}

\section*{Μέρος 3: Γρήγορος Έλεγχος (Σωστό ή Λάθος)}
\begin{enumerate}
  \item Αν $f(a) = 0$, τότε το $x-a$ είναι παράγοντας του $f(x)$. \textbf{(Σωστό/Λάθος)}
  \item Όταν διαιρούμε το $f(x) = x^3 - 4x + 3$ με το $x - 1$, το υπόλοιπο είναι πάντα μηδέν. \textbf{(Σωστό/Λάθος)}
\end{enumerate}

\vfill
\textbf{Πρόσθετη Πρόκληση:} Βρείτε ένα πολυώνυμο $f(x)$ τρίτου βαθμού που να έχει ρίζες $x = 1$ και $x = -2$.

\end{document}
