\documentclass[a4paper,12pt]{article}
\usepackage{amsmath, amssymb}

\usepackage{amsmath}
\usepackage{unicode-math}
\usepackage{xltxtra}
\usepackage{xgreek}

\setmainfont{Liberation Serif}

\usepackage{tabularx}

\pagestyle{empty}

\usepackage{geometry}
 \geometry{a4paper, total={190mm,280mm}, left=10mm, top=10mm}

 \usepackage{graphicx}
 \graphicspath{ {images/} }

 \usepackage{wrapfig}

 \usepackage{float}

\usepackage{geometry}
\geometry{margin=1in}

\title{\textbf{Σχέδιο Μαθήματος: Ρίζες Πολυωνύμων \ και το Θεώρημα Υπολοίπου}}
\author{Μαθηματικά Β' Λυκείου}
\date{}

\begin{document}
\maketitle

\section*{Στόχοι Μαθήματος}
Με το τέλος του μαθήματος, οι μαθητές θα μπορούν:
\begin{itemize}
  \item Να κατανοούν την έννοια των ριζών των πολυωνύμων.
  \item Να εφαρμόζουν το Θεώρημα Υπολοίπου για να ελέγχουν αν ένας αριθμός είναι ρίζα.
  \item Να εργάζονται ομαδικά για την επίλυση προβλημάτων πολυωνύμων.
  \item Να αξιολογούν την κατανόησή τους μέσω σύντομης αξιολόγησης.
\end{itemize}

\section*{Δομή Μαθήματος (40 λεπτά)}

\textbf{0-5 λεπτά: Προθέρμανση και Εισαγωγή}
\begin{itemize}
  \item Συζήτηση: "Τι είναι ένα πολυώνυμο;" (Παραδείγματα στον πίνακα).
  \item Παρουσίαση στόχων μαθήματος.
\end{itemize}

\textbf{5-10 λεπτά: Παρουσίαση Θεωρίας (Διαδραστικός Πίνακας)}
\begin{itemize}
  \item Ορισμός ριζών πολυωνύμου και σχέση τους με τους παράγοντες.
  \item Παρουσίαση του Θεωρήματος Υπολοίπου και ενός απλού παραδείγματος.
\end{itemize}

\textbf{10-20 λεπτά: Ομαδική Δραστηριότητα (Εργασία σε Φύλλα)}
\begin{itemize}
  \item Οι μαθητές χωρίζονται σε ομάδες και λαμβάνουν διαφορετικά πολυώνυμα.
  \item Κάθε ομάδα χρησιμοποιεί το Θεώρημα Υπολοίπου για να ελέγξει αν ένας αριθμός είναι ρίζα.
\end{itemize}

\textbf{20-30 λεπτά: Συζήτηση και Ανάλυση Λαθών}
\begin{itemize}
  \item Οι ομάδες παρουσιάζουν τις λύσεις τους στον διαδραστικό πίνακα.
  \item Διόρθωση λαθών και συζήτηση κοινών παρανοήσεων.
\end{itemize}

\textbf{30-35 λεπτά: Ατομική Αξιολόγηση}
\begin{itemize}
  \item Σύντομο φύλλο εργασίας με 3-4 ερωτήσεις.
  \item Ερώτηση σωστού/λάθους και αναγνώριση λάθους σε λύση.
\end{itemize}

\textbf{35-40 λεπτά: Ανασκόπηση και Συμπεράσματα}
\begin{itemize}
  \item Συζήτηση απαντήσεων και αποσαφήνιση αποριών.
  \item Ερώτηση εξόδου: "Γιατί είναι χρήσιμο το Θεώρημα Υπολοίπου;"
\end{itemize}

\section*{Αξιολόγηση}
\begin{itemize}
  \item Παρατήρηση ομαδικής εργασίας.
  \item Έλεγχος απαντήσεων στη συζήτηση.
  \item Αξιολόγηση σύντομης ατομικής εργασίας.
\end{itemize}

\end{document}
