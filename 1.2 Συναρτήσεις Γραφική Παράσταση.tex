\documentclass[greek]{beamer}
%\usepackage{fontspec}
\usepackage{amsmath,amsthm}
\usepackage{unicode-math}
\usepackage{xltxtra}
\usepackage{graphicx}
\usetheme{Warsaw}
\usecolortheme{seahorse}
\usepackage{hyperref}
\usepackage{ulem}
\usepackage{xgreek}
\usepackage{pgfpages}
%\setbeameroption{show notes on second screen}
%\setbeameroption{show only notes}

\setsansfont{DejaVu Sans}

% \newtheorem{definition}{Ορισμός}

\title{Συναρτήσεις, Γραφικές Παραστάσεις}
\author[Λόλας]{Κωνσταντίνος. Λόλας}
\date{}

\begin{document}

\begin{frame}
 \titlepage
\end{frame}
\begin{frame}{Ορισμός}
 \begin{block}{Ορισμός}
  Γραφική παράσταση μιας συνάρτησης είναι το σύνολο των σημείων $A(x,f(x))$, $x\in D_f$, και συμβολίζεται με $C_f$
 \end{block}
\end{frame}

\begin{frame}{Τι βλέπω}
 \begin{itemize}
  \item<1-> Είναι γραφική παράσταση?
  \item<2-> Πεδίο Ορισμού
  \item<3-> Σύνολο τιμών
  \item<4-> Ρίζες
  \item<5-> Πρόσημο
  \item<6-> Κοινά σημεία
  \item<7-> Κατακόρυφη απόσταση
  \item<8-> Σχετική θέση
 \end{itemize}
\end{frame}

\begin{frame}{Γνωστές Γραφικές}
 \begin{itemize}
  \item<1-> $y=a$
  \item<2-> $y=ax+b$
  \item<3-> $y=x^2$, $y=ax^2+bx+c$
  \item<4-> $y=ax^3$
  \item<5-> $y=\frac{a}{x}$
  \item<6-> $y=|x|$
  \item<7-> $y=ημ x$, $y=συν x$, $y=εφ x$
  \item<8-> $y=a^x$, $y=e^x$
  \item<9-> $y=\ln x$
  \item<10-> Μετατοπίσεις
 \end{itemize}
\end{frame}

\begin{frame}{Μετατοπίσεις}
 $$y=f(x)$$
 \begin{itemize}
  \item<1-> $y=f(x)+c$
  \item<2-> $y=f(x+c)$
  \item<3-> $a\cdot f(x)$
  \item<4-> $y=f(a\cdot x)$
  \item<5-> $y=-f(x)$
  \item<6-> $y=|f(x)|$
  \item<7-> $y=f(-x)$
 \end{itemize}
\end{frame}

\begin{frame}{Εξάσκηση}
 \href{https://www.geogebra.org/m/jmmx7bp8}{\beamergotobutton{Άσκηση Geogebra}}
 \begin{enumerate}
  \item<1-> Να βρείτε το πεδίο ορισμού και το σύνολο τιμών
  \item Να βρείτε τις τιμές: $f(2)$ και $f(f(0))$
  \item<2-> Να λύσετε γραφικά την $f(x)=0$
  \item<3-> Να λύσετε γραφικά την $f(x)<0$
  \item<4-> Να βρείτε το πεδίο ορισμού της συνάρτησης $g(x)=\ln x$
 \end{enumerate}
\end{frame}

\begin{frame}{Εξάσκηση}
 \href{https://www.geogebra.org/m/td6m58hw}{\beamergotobutton{Άσκηση Geogebra}}
 \begin{enumerate}
  \item<1-> Να βρείτε τα κοινά σημεία των $C_f$ και $C_g$
  \item<2-> Να λύσετε την $f(x)=g(x)$
  \item<3-> Να λύσετε τις ανισώσεις:
        \begin{enumerate}
         \item<4-> $f(x)>g(x)$
         \item<5-> $f(x)<g(x)$
        \end{enumerate}
  \item<6-> Να λύσετε την εξίσωση $2g(x)=f(g(0))$
  \item<7-> Να βρείτε την κατακόρυφη απόσταση των συναρτήσεων στο $x_0=0$
 \end{enumerate}
\end{frame}

\begin{frame}{Εξάσκηση}
 Δίνεται η συνάρτηση $f(x)=ax^2-5a+1$, της οποίας η γραφική παράσταση διέρχεται από το σημείο $Α(3,5)$. Να βρείτε:
 \begin{enumerate}
  \item<1-> την τιμή του $a$
  \item<2-> τα κοινά σημεία της $C_f$ με τους άξονες $y'y$ και $x'x$
  \item<3-> τα διαστήματα του $x$ που η $C_f$ βρίσκεται πάνω από τον άξονα $x'x$
 \end{enumerate}
\end{frame}

\begin{frame}{Εξάσκηση}
 Δίνονται οι συναρτήσεις $f(x)=\frac{1}{x}$ και $g(x)=1$. Να βρείτε:
 \begin{enumerate}
  \item<1-> τα κοινά τους σημεία
  \item<2-> την σχετική τους θέση
 \end{enumerate}
\end{frame}

\begin{frame}{Εξάσκηση}
 Να σχεδιάσετε τις γραφικές παραστάσεις των συναρτήσεων:
 \begin{enumerate}
  \item<1->
        $f(x)=
         \begin{cases}
          \frac{1}{x}, & x<0    \\
          x^2,         & x\ge 0
         \end{cases}$

  \item<2-> $f(x)=
         \begin{cases}
          e^{-x}, & x<0    \\
          -συν x, & x\ge 0
         \end{cases}$
 \end{enumerate}
 Από τη γραφική παράσταση να προσδιορίσετε το σύνολο τιμών σε καθεμία περίπτωση
\end{frame}

\begin{frame}{Εξάσκηση}
 Στο ίδιο σύστημα αξόνων να σχεδιάσετε τις γραφικές παραστάσεις των συναρτήσεων $e^x$, $ημ x$ για $x>0$, να βρείτε τη σχετική τους θέση και να επιβεβαιώσετε αλγεβρικά την ανισώτητα:
 $$e^x>ημ x \text{, για κάθε } x>0$$
\end{frame}

\begin{frame}{Εξάσκηση}
 Να σχεδιάσετε τις γραφικές παραστάσεις των παρακάτω συναρτήσεων στο ίδιο σύστημα αξόνων
 $$f(x)=(x-1)^2+1\text{, } x\ge 1 \text{ και } g(x)=1+\sqrt{x-1}$$
\end{frame}

\begin{frame}{Εξάσκηση}
 \href{https://www.geogebra.org/m/euy2uhma}{\beamergotobutton{Άσκηση Geogebra}}
 \begin{enumerate}
  \item<1-> Να λύσετε την εξίσωση $f(x)=2$
  \item<2-> Να βρείτε πεδίο ορισμού της $$g(x)=\frac{1}{f(x)-1}$$
  \item<3-> Να Βρείτε το πλήθος ριζών των εξισώσεων
        \begin{enumerate}
         \item<4-> $f(x)=5/2$
         \item<5-> $2f(x)-1=0$
         \item<6-> $f(x)=a^2+1$, $a\ne 0$
        \end{enumerate}
 \end{enumerate}
\end{frame}

\begin{frame}{Εξάσκηση}
 \href{https://www.geogebra.org/m/dvzdm7bw}{\beamergotobutton{Άσκηση Geogebra}}
 \begin{enumerate}
  \item<1-> Να βρείτε το πλήθος των λύσεων της εξίσωσης $f(x)=a$, για τις διάφορες τιμές του $a\in\mathbb{R}$
  \item<2-> Να δείξετε ότι η εξίσωση $f(x)=3ημ a - 5$ είναι αδύνατη, για κάθε $a\in\mathbb{R}$
 \end{enumerate}
\end{frame}

\begin{frame}{Εξάσκηση}
 Να εξετάσετε
 \begin{enumerate}
  \item<1-> αν ο αριθμός $2$ ανήκει στο σύνολο τιμών της συνάρτησης $f(x)=1+\sqrt{x}$
  \item<2-> αν ο αριθμός $0$ ανήκει στο σύνολο τιμών της συνάρτησης $f(x)=\frac{e^x-1}{x}$
 \end{enumerate}
\end{frame}

\begin{frame}{Εξάσκηση}
 Έστω $f:A\to\mathbb{R}$ μία συνάρτηση με $A=\mathbb{R}$ και $f(A)=(1,+\infty)$.
 \begin{enumerate}
  \item<1-> Να δείξετε ότι η εξίσωση $f(x)=2023$ έχει μία τουλάχιστον λύση
  \item<2-> Να δείξετε ότι η εξίσωση $f(x)=a^2+1$ έχει μία τουλάχιστον λύση, για κάθε $a\in\mathbb{R^*}$
  \item<3-> Να εξετάσετε αν υπάρχει $x_0\ge 0$ τέτοιο ώστε $f(x)=e^{x_0}$
 \end{enumerate}
\end{frame}

\begin{frame}
 Στο moodle θα βρείτε τις ασκήσεις που πρέπει να κάνετε, όπως και αυτή τη παρουσίαση
\end{frame}

\end{document}
