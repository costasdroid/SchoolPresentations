\documentclass{presentation}

\title{Άλγεβρα - Ανισώσεις}
\subtitle{Α Βαθμού, Απόλυτα}
\author[Λόλας]{Κωνσταντίνος Λόλας }

\begin{document}

\begin{frame}
  \titlepage
\end{frame}

\section{Θεωρία}

\begin{frame}{Γιατί ανισώσεις; Γιατί όχι μόνο εξισώσεις;}
  \begin{itemize}[<+->]
    \item Οι εξισώσεις θέλουν "μία ή μερικές" λύσεις – αυστηρές!
    \item Οι ανισώσεις είναι πιο ανοιχτόμυαλες: πολλές λύσεις, περισσότερη ελευθερία!
    \item Χρήσιμες όταν η ζωή δεν είναι... ίση (και συνήθως δεν είναι!)
  \end{itemize}
\end{frame}

\begin{frame}{Αν αλλάξω μεριά... θα αλλάξω και ζωή;}
  \begin{itemize}[<+->]
    \item Στις εξισώσεις: αλλάζεις μέλη, όλα καλά!
    \item Στις ανισώσεις: αλλάζεις μέλη, αλλάζει και η φορά – προσοχή, μην μπερδευτείς!
    \item Πολλαπλασιάζεις με αρνητικό; Η ανίσωση κάνει τούμπα!
  \end{itemize}
\end{frame}

\begin{frame}{Διάταξη: Ο ξάδερφος που αποδεικνύει, όχι που λύνει}
  \begin{itemize}[<+->]
    \item Ίδια σύμβολα ($<, >, \leq, \geq$), άλλη φάση
    \item Στις διατάξεις: "Αποδεικνύω!" – Στις ανισώσεις: "Λύνω!"
    \item Εδώ ψάχνουμε όλα τα $x$ που κάνουν την ανίσωση χαρούμενη
  \end{itemize}
\end{frame}

\begin{frame}{Απόλυτα: Η παγίδα του "ό,τι και να γίνει"}
  \begin{itemize}[<+->]
    \item $|x|<a$ $\Rightarrow$ $-a<x<a$ (σαν αγκαλιά)
    \item $|x|>a$ $\Rightarrow$ $x<-a$ ή $x>a$ (τρέχει μακριά!)
    \item Πάντα δύο περιπτώσεις – διπλή διασκέδαση
  \end{itemize}
\end{frame}

\begin{frame}{Γινόμενα: Ο πίνακας προσήμων σώζει ζωές}
  \begin{itemize}[<+->]
    \item Βρίσκω ρίζες, φτιάχνω πινακάκι (σαν sudoku για μαθηματικούς)
    \item Κοιτάω πού το γινόμενο είναι $+$ ή $-$ – και διαλέγω στρατόπεδο
  \end{itemize}
\end{frame}

\moodle

\section{Ασκήσεις}

\exercises

\begin{askisi}
  Να λύσετε τις ανισώσεις:
  \begin{enumerate}[<+->]
    \item $x-2(x-1)>3x+5$
    \item $3x<3(x+2)$
    \item $4-2(x-1)<5-2x$
    \item $3-(2x+1) \le 2(1-x)$
  \end{enumerate}
\end{askisi}

\begin{askisi}
  Να βρείτε το μεγαλύτερο ακέραιο, για τον οποίο αληθεύει η ανίσωση
  $$1-\frac{2x-1}{10}\ge x-\frac{2-3x}{5}$$
\end{askisi}

\begin{askisi}
  Να βρείτε τις κοινές λύσεις των ανισώσεων
  \begin{enumerate}[<+->]
    \item $3-2(x+1)\ge 5$ και $3x-1>2(x-5)$
    \item $2x+3>3$ και $1-3x<0$
    \item $2x\le 2x+1\le 11$
    \item $3x+1>x-(x-1)$ και $x-2<1-(5-x)$
    \item $-1\le \dfrac{3x}{2}-1<2$, όταν $x\in \mathbb{Z}$
  \end{enumerate}
\end{askisi}

\begin{askisi}
  Να κάνετε τον πίνακα προσήμων των τιμών των παραστάσεων:
  \begin{enumerate}[<+->]
    \item $P(x)=(x-1)(x-2)$
    \item $A=1-\dfrac{1}{x}$
  \end{enumerate}
\end{askisi}

\begin{askisi}
  Να λύσετε την ανίσωση $λ(x-λ)<2x-4$, για τις διάφορες τιμές του $λ\in \mathbb{R}$.
\end{askisi}

\begin{askisi}
  Να λύσετε τις ανισώσεις:
  \begin{enumerate}[<+->]
    \item $|x|<2$
    \item $3|x|-1\le 0$
    \item $|x|>5$
    \item $1-2d(x,0)\le 0$
  \end{enumerate}
\end{askisi}

\begin{askisi}
  Να λύσετε τις ανισώσεις:
  \begin{enumerate}[<+->]
    \item $|x|+1<0$
    \item $|x|+2>0$
    \item $|x|\le 0$
    \item $|x|>0$
  \end{enumerate}
\end{askisi}

\begin{askisi}
  Να λύσετε τις ανισώσεις:
  \begin{enumerate}[<+->]
    \item $|2x-3|<6$
    \item $\sqrt{9x^2-6x+1}\ge 2$
  \end{enumerate}
\end{askisi}

\begin{askisi}
  Να λύσετε τις ανισώσεις:
  \begin{enumerate}[<+->]
    \item $1<|x|\le 3$
    \item $2\le |3x-1| <5$
  \end{enumerate}
\end{askisi}

\begin{askisi}
  Να λύσετε τις ανισώσεις
  \begin{enumerate}[<+->]
    \item $2(|x|-1)-(|-x|-2)<|3x|$
    \item $|2x-1|-\left|\dfrac{1}{2}-x\right|>1$
  \end{enumerate}
\end{askisi}

\begin{askisi}
  Να λύσετε τις ανισώσεις
  \begin{enumerate}[<+->]
    \item $x^2<1$
    \item $x^4-81>0$
    \item $(x-1)^2-2<0$
  \end{enumerate}
\end{askisi}

\begin{askisi}
  Να λύσετε την ανίσωση $|2x-3|<|2x+1|$
\end{askisi}

\begin{askisi}
  Να λύσετε την ανίσωση $|x+2|-|x-1|<x+2$
\end{askisi}

\begin{askisi}
  Να λύσετε:
  \begin{enumerate}[<+->]
    \item την ανίσωση $\left||x-1|-2\right|<2$
    \item την ανίσωση $\left||2x-1|-3\right|+|2x-1|=3$
  \end{enumerate}
\end{askisi}

\begin{askisi}
  Να λύσετε ανισώσεις:
  \begin{enumerate}[<+->]
    \item $\dfrac{|x|+2}{|x|+1}<2$
    \item $\left|\dfrac{1}{|x|+2}\right|<\dfrac{1}{2}$
    \item $\dfrac{2}{|x-1|}>1$
  \end{enumerate}
\end{askisi}

\begin{askisi}
  Να βρείτε τις τιμές του $λ$, για τις οποίες η εξίσωση $x^2-(λ-1)x+3=0$ έχει πραγματικές ρίζες
\end{askisi}

\begin{askisi}
  Δίνεται η συνάρτηση $y=\dfrac{\sqrt{2-|x|}}{x}$
  \begin{enumerate}[<+->]
    \item Να βρείτε τις τιμές του $x$, για τις οποίες ορίζεται η συνάρτηση
    \item Να λύσετε την εξίσωση $y=1$
  \end{enumerate}
\end{askisi}

\end{document}