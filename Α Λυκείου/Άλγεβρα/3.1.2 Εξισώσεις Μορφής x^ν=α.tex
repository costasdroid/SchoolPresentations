\documentclass{presentation}

\title{Άλγεβρα - Εξισώσεις}
\subtitle{Εξισώσεις της μορφής $x^ν = α$}
\author[Λόλας]{Κωνσταντίνος Λόλας }

\begin{document}

\begin{frame}
  \titlepage
\end{frame}

\section{Θεωρία}

\begin{frame}{Όταν το $x$ κάνει γυμναστική!}
  Μέχρι τώρα είχαμε απλές εξισώσεις... τώρα το $x$ ανέβηκε σε άλλο level!
  \begin{itemize}[<+->]
    \item $x=5$ (παιδάκι)
    \item $2x+3=7$ (έφηβος)
    \item $x^2=9$ (ενήλικας με δύναμη!)
    \item $x^3=8$ (bodybuilder!)
    \item $x^ν=α$ (Hulk mode!)
  \end{itemize}
\end{frame}

\begin{frame}{Η βασική ιδέα}
  \begin{block}{Το πρόβλημα}
    Πώς λύνουμε την εξίσωση $x^ν=α$;
  \end{block}
  \pause

  \begin{itemize}[<+->]
    \item Εξαρτάται από το $ν$ (τον εκθέτη)
    \item Εξαρτάται από το $α$ (τον αριθμό)
    \item Το $ν$ καθορίζει πόσες λύσεις θα έχουμε
    \item Το $α$ καθορίζει αν έχουμε λύσεις!
  \end{itemize}
\end{frame}

\begin{frame}{Όταν ο εκθέτης είναι περιττός}
  \begin{block}{Περιττός εκθέτης: $ν=2κ+1$}
    Για την εξίσωση $x^ν=α$ με $ν$ περιττό:
    \begin{itemize}[<+->]
      \item Αν $α>0$: $x=\sqrt[ν]{α}$ (μία θετική λύση)
      \item Αν $α=0$: $x=0$ (μία λύση)
      \item Αν $α<0$: $x=-\sqrt[ν]{-α}$ (μία αρνητική λύση)
    \end{itemize}
  \end{block}
  \pause

  \begin{exampleblock}{Παραδείγματα}
    \begin{itemize}[<+->]
      \item $x^3=8 \Rightarrow x=\sqrt[3]{8}=2$
      \item $x^3=-8 \Rightarrow x=-\sqrt[3]{8}=-2$
      \item $x^5=32 \Rightarrow x=\sqrt[5]{32}=2$
      \item $x^5=-32 \Rightarrow x=-\sqrt[5]{32}=-2$
    \end{itemize}
  \end{exampleblock}
\end{frame}

\begin{frame}{Γιατί οι περιττοί χρειάζονται δύο περιπτώσεις;}
  \begin{itemize}[<+->]
    \item Η ρίζα ορίζεται \textbf{μόνο} για θετικούς αριθμούς!
    \item Για $x^3=8$: Απλά $x=\sqrt[3]{8}=2$ \quad (OK!)
    \item Για $x^3=-8$: Δεν μπορούμε να γράψουμε $\sqrt[3]{-8}$!

          \textbf{Λύση}: Γράφουμε $x^3=-8$ ως $x^3=-1\cdot 8$

          \begin{itemize}[<+->]
            \item Αν $x^3=-8$, τότε $(-x)^3=8$ (αλλαγή προσήμου!)
            \item Άρα $-x=\sqrt[3]{8}=2 \Rightarrow x=-2$
            \item Γενικά: $x^ν=α$ με $α<0$ γίνεται $x=-\sqrt[ν]{-α}$
          \end{itemize}
  \end{itemize}
\end{frame}

\begin{frame}{Όταν ο εκθέτης είναι άρτιος}
  \begin{block}{Άρτιος εκθέτης: $ν=2κ$}
    Για την εξίσωση $x^ν=α$ με $ν$ άρτιο:
    \begin{itemize}[<+->]
      \item Αν $α<0$: \textbf{Καμία λύση} (αδύνατη!)
      \item Αν $α=0$: $x=0$ (μία λύση)
      \item Αν $α>0$: $x=\pm\sqrt[ν]{α}$ (δύο λύσεις!)
    \end{itemize}
  \end{block}
\end{frame}

\begin{frame}{Γιατί οι άρτιοι είναι τόσο καβγαδιάρηδες;}
  \begin{itemize}[<+->]
    \item Οι άρτιες δυνάμεις \textbf{χάνουν} το πρόσημο!
    \item $(-2)^2=4$ (έγινε θετικό!)
    \item $(2)^2=4$ (ήταν θετικό!)
    \item Άρα $x^{2κ}\ge 0$ πάντα!
    \item Δεν μπορούμε να φτάσουμε σε αρνητικό αριθμό
    \item Αλλά από δύο δρόμους φτάνουμε στον ίδιο θετικό: $x$ και $-x$
  \end{itemize}
\end{frame}

\begin{frame}{Παραδείγματα με άρτιο εκθέτη}
  \begin{exampleblock}{Με λύσεις}
    \begin{itemize}[<+->]
      \item $x^2=9 \Rightarrow x=\pm 3$ (δύο λύσεις)
      \item $x^4=16 \Rightarrow x=\pm 2$ (δύο λύσεις)
      \item $x^6=64 \Rightarrow x=\pm 2$ (δύο λύσεις)
    \end{itemize}
  \end{exampleblock}
  \pause

  \begin{alertblock}{Χωρίς λύσεις}
    \begin{itemize}[<+->]
      \item $x^2=-4$ (αδύνατη!)
      \item $x^4=-16$ (αδύνατη!)
      \item $x^{100}=-1$ (αδύνατη!)
    \end{itemize}
  \end{alertblock}
\end{frame}

\begin{frame}{Σύνοψη - Ο χάρτης επιβίωσης}
  \begin{center}
    \begin{tabular}{|c|c|c|c|}
      \hline
      \textbf{Εκθέτης} & \textbf{Αριθμός} & \textbf{Λύση}      & \textbf{Πλήθος} \\
      \hline
      \hline
      Περιττός         & $α>0$            & $x=\sqrt[ν]{α}$    & 1               \\
      $ν=2κ+1$         & $α=0$            & $x=0$              & 1               \\
                       & $α<0$            & $x=-\sqrt[ν]{-α}$  & 1               \\
      \hline
      \hline
      Άρτιος           & $α>0$            & $x=\pm\sqrt[ν]{α}$ & 2               \\
      $ν=2κ$           & $α=0$            & $x=0$              & 1               \\
                       & $α<0$            & Αδύνατη            & 0               \\
      \hline
    \end{tabular}
  \end{center}
\end{frame}

\begin{frame}{Προσοχή στις παγίδες!}
  \begin{alertblock}{Σωστό ή Λάθος;}
    \begin{itemize}
      \item<1-> $x^2=9 \Rightarrow x=3$ \onslide<2->{\quad (\textbf{Λάθος!} Ξέχασες το $-3$)}
      \item<3-> $x^3=-8 \Rightarrow \sqrt[3]{-8}$ \onslide<4->{\quad (\textbf{Λάθος!} Δεν ορίζεται!)}
      \item<5-> $x^4=-1 \Rightarrow x=\pm 1$ \onslide<6->{\quad (\textbf{Λάθος!} Είναι αδύνατη!)}
      \item<7-> $\sqrt{x^2}=x$ \onslide<8->{\quad (\textbf{Λάθος!} Είναι $\sqrt{x^2}=|x|$)}
    \end{itemize}
  \end{alertblock}
\end{frame}

\begin{frame}{Ειδική περίπτωση: $x^2=α^2$}
  \begin{block}{Προσοχή!}
    Αν έχουμε $x^2=α^2$, τότε:
    $$x^2-α^2=0 \Rightarrow (x-α)(x+α)=0$$
    $$x=α \text{ ή } x=-α$$
    Δηλαδή: $x=\pm α$
  \end{block}
  \pause

  \begin{exampleblock}{Παράδειγμα}
    $x^2=(2x-1)^2 \Rightarrow x=\pm(2x-1)$
    \begin{itemize}
      \item $x=2x-1 \Rightarrow x=1$
      \item $x=-(2x-1) \Rightarrow x=-2x+1 \Rightarrow 3x=1 \Rightarrow x=\frac{1}{3}$
    \end{itemize}
  \end{exampleblock}
\end{frame}

\moodle

\section{Ασκήσεις}

\exercises

\begin{askisi}
  Λύστε τις εξισώσεις:
  \begin{enumerate}[<+->]
    \item $x^3-8=0$
    \item $x^3+1=0$
    \item $x^3+5=0$
    \item $x^2-4=0$
    \item $x^4-2=0$
    \item $x^4+1=0$
  \end{enumerate}
\end{askisi}

\begin{askisi}
  Λύστε τις εξισώσεις:
  \begin{enumerate}[<+->]
    \item $x^3=5x$
    \item $x^5-32x^2=0$
  \end{enumerate}
\end{askisi}

\begin{askisi}
  Να λύσετε τις εξισώσεις:
  \begin{enumerate}[<+->]
    \item $(x-1)^4-81=0$
    \item $(3x-1)^5+32=0$
    \item $8x^3-(x-1)^3=0$
  \end{enumerate}
\end{askisi}

\begin{askisi}
  Να λύσετε την εξίσωση $(x-2)^4-|x-2|=0$.
\end{askisi}

\begin{askisi}
  Να λύσετε την εξίσωση $26x^3+3x^2-3x+1=0$.
\end{askisi}

\end{document}