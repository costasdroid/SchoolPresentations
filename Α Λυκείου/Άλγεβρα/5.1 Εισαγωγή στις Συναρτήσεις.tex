\documentclass{../../presentation}

\title{Άλγεβρα Α Λυκείου}
\subtitle{Εισαγωγή στις Συναρτήσεις}
\author[Λόλας]{Κωνσταντίνος Λόλας}
\date{\today}

\begin{document}

\frame{\titlepage}

\section{Θεωρία}
\subsection{Συναρτήσεις}

\begin{frame}{Συναρτήσεις Λοιπόν}

\end{frame}

\begin{frame}{Ορισμός Συνάρτησης}

  \begin{block}{Ορισμός}
    Συνάρτηση από ένα σύνολο $Α$ σε ένα σύνολο $Β$ λέγεται μια διαδικασία (κανόνας) με την οποία κάθε στοιχείο του συνόλου $Α$ αντιστοιχίζεται σε ένα ακριβώς στοιχείο του συνόλου $Β$.
  \end{block}
\end{frame}

\begin{frame}
  Λέξεις κλειδιά
  \begin{itemize}[<+->]
    \item Πεδίο Ορισμού (σύνολο ορισμού)
    \item Τιμή της Συνάρτησης
    \item Ανεξάρτητη μεταβλητή
    \item Εξαρτημένη μεταβλητή
    \item Σύνολο τιμών
    \item Συμβολισμοί
          \begin{itemize}[<+->]
            \item Με $f$
            \item Με βέλος
            \item Με τελεία!
          \end{itemize}
    \item Τύπος (ορισμός)
          \begin{itemize}[<+->]
            \item Πράξεις
            \item Νοητά
            \item Αναδρομικά
          \end{itemize}
    \item Αναπαράσταση
          \begin{itemize}[<+->]
            \item Γραφική παράσταση
            \item Πίνακες τιμών
          \end{itemize}
  \end{itemize}
\end{frame}

\begin{frame}{Στο δια ταύτα}
  $$f(x)=\dots$$
  \begin{itemize}
    \item Πεδίο ορισμού\dots
    \item Τιμή της συνάρτησης\dots
    \item Σύνολο τιμών\dots
  \end{itemize}
\end{frame}

\moodle

\section{Ασκήσεις}

\exercises

\begin{askisi}
  Δίνεται η συνάρτηση $f(x)=x^2-3x-1$, $x\in\mathbb{R}$.
  \begin{enumerate}[<+->]
    \item Να γράψετε το πεδίο ορισμού της $f$.
    \item Να υπολογίσετε τις τιμές:
          $$f(0) \text{, } f(-1) \text{, }f(2α) \text{, }f(x+1) \text{, }f\left(f(0)\right) $$
  \end{enumerate}
\end{askisi}

\begin{askisi}
  Δίνεται η συνάρτηση $f(x)=αx^2+βx-1$. Να βρείτε τις τιμές των $α$ και $β$, για τις οποίες ισχύει $f(1)=0$ και $f(2)=3$.
\end{askisi}

\begin{askisi}
  Δίνεται η συνάρτηση $f(x)=
    \begin{cases}
      2x,  & x\in(-\infty,1] \\
      x^2, & x\in(1,+\infty)
    \end{cases}$.
  \begin{enumerate}[<+->]
    \item Να γράψετε το πεδίο ορισμού της $f$
    \item Να βρείτε τις τιμές $f(0)$, $f(1)$, $f(2)$
  \end{enumerate}
\end{askisi}

\begin{askisi}
  Δίνεται η συνάρτηση $f(x)=
    \begin{cases}
      αx^2-1, & -2<x<0      \\
      αx^3+β, & 0\le x\le 1
    \end{cases}$.
  \begin{enumerate}[<+->]
    \item Να γράψετε το πεδίο ορισμού της $f$
    \item Να βρείτε τις τιμές των $α$ και $β$ για τις οποίες ισχύουν $f(-1)=2$ και $f(1)=3$
  \end{enumerate}
\end{askisi}

\begin{askisi}
  Να βρείτε το πεδίο ορισμού των παρακάτω συναρτήσεων
  \begin{enumerate}[<+->]
    \item $f(x)=x^2+x-1$
    \item $f(x)=\dfrac{x}{x^2-1}$
    \item $f(x)=\dfrac{2}{5x^2-x-4}$
    \item $f(x)=\dfrac{x-1}{x^2-x+1}$
  \end{enumerate}
\end{askisi}

\begin{askisi}
  Να βρείτε το πεδίο ορισμού των παρακάτω συναρτήσεων
  \begin{enumerate}[<+->]
    \item $f(x)=\sqrt{x-1}$
    \item $f(x)=(x-1)^{2/3}$
    \item $f(x)=\dfrac{1}{\sqrt[3]{2-x}}$
    \item $f(x)=\dfrac{\sqrt{1-x^2}}{x}$
    \item $f(x)=\sqrt{2-|x|}+\sqrt{x-1}$
    \item $f(x)=\dfrac{x}{\sqrt{x}-1}$
  \end{enumerate}
\end{askisi}

\begin{askisi}
  Δίνεται η συνάρτηση $f(x)=x^2-x-2$.
  \begin{enumerate}[<+->]
    \item Να λύσετε την εξίσωση $f(x)=-2$
    \item Να λύσετε την ανίσωση $f(x+1)-f(2x)>x^2-x$
  \end{enumerate}
\end{askisi}

\begin{askisi}
  Δίνεται η συνάρτηση $f(x)=\dfrac{2x-2}{x^3-x}$.
  \begin{enumerate}[<+->]
    \item Να βρείτε το πεδίο ορισμού της $f$
    \item Να λύσετε την εξίσωση $f(x)=1$
  \end{enumerate}
\end{askisi}

\begin{askisi}
  Δίνεται η συνάρτηση $f(x)=\begin{cases}
      3x-2, & x\le 1 \\
      2x+1, & x>1
    \end{cases}$. Να λύσετε την εξίσωση $f(x)=4$
\end{askisi}

\begin{askisi}
  Δίνεται η συνάρτηση $f(x)=x^2-x$. Να εξετάσετε αν οι αριθμοί $2$ και $-1$ ανήκουν στο σύνολο τιμών της $f$.
\end{askisi}

\begin{askisi}
  Δίνεται η συνάρτηση $f(x)=\begin{cases}
      \dfrac{2}{x}, & x< 0    \\
      3x,           & x \ge 0
    \end{cases}$. Να βρείτε τις τιμές:
  \begin{enumerate}[<+->]
    \item $f(x^2)$
    \item $f(-x)$, αν $x>0$
    \item $f\left(1-\dfrac{1}{x}\right)$, αν $x\ge 1$
  \end{enumerate}
\end{askisi}

\begin{askisi}
  Δίνεται η συνάρτηση $f(x)=\begin{cases}
      -3x, & x\le 0 \\
      2x,  & x > 0
    \end{cases}$. Να λύσετε την εξίσωση $f(-x)=f(x^2+1)$
\end{askisi}

\begin{askisi}
  Δίνεται η συνάρτηση $f(x)=\begin{cases}
      \dfrac{2}{x}-1, & 0<x\le 1 \\
      2x,             & x > 1
    \end{cases}$. Να λύσετε
  \begin{enumerate}[<+->]
    \item την εξίσωση $f\left(\dfrac{1}{x}\right)=3$ στο διάστημα $[1,+\infty)$
    \item την ανίσωση $f(2-x)>3$ στο διάστημα $(0,1)$
  \end{enumerate}
\end{askisi}


\end{document}